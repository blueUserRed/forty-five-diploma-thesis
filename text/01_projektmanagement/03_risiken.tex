
\section{Risiken}\label{sec:risiken}

\renewcommand{\kapitelautor}{Autor: Nils} % todo: replace
%
Eine Risikoanalyse ist ein wichtiger Teil im Planungsprozess, um mögliche Herausforderungen zu berücksichtigen.
Mit einem klaren Überblick über bevorstehende Risiken lassen sich Maßnahmen treffen, um diese vorzubeugen oder eventuell zu vermeiden.\cite{AsanaRisiken}

Projekte sind immer stark von Risiken geprägt, so auch \ff. Darum gab es auch eine Planungsphase, in der sich das Team mit bevorstehenden Risiken befasste.
Es gibt klassische Risiken, wie eine zeitliche Verzögerung durch Ausfälle oder Schwierigkeiten beim Know-How, in dem Fall der Steam Release.
Dieses Diplomarbeitsprojekt hatte jedoch mit ganz anderen Herausforderungen zu kämpfen. Es ist die Kunst, andere zufrieden zu stellen und zu unterhalten, die eine Herausforderung darstellt.
Das Ziel ist es, Leute mit dem Videospiel zu unterhalten und dafür zu sorgen, dass es nicht eintönig oder gar langweilig wird.
Spieler sollen nicht nur Spaß am Spielen haben, sondern im besten Fall sogar ihren Freunden davon zu erzählen.
Das größte Risiko für ein Videospiel ist, dass es nicht unterhaltsam ist und keinen Zeitvertreib bietet. Darum wurde während des gesamten Projekts ein intensiver Fokus auf den Unterhaltungsfaktor gelegt.
Denn Videospieldesign besteht aus verschiedensten Aspekten, die alle Teil des Gamedesigns sind.
Dazu gehören zum einen ein grafisch ansprechendes Spiel, welche mit der Hilfe der Designguidelines erreicht wurde, welche in einem Vorprojekt entstanden sind.
Das der Unterhaltungsfaktor gegeben ist, wurde durch intensives Playtesting des Teams und durch Aussenstehende erreicht.
Außerdem wurden wie im Kapitel\ref{ch:sounds} auch Tests der Sounds hinzugefügt.
%

% resets author
\renewcommand{\kapitelautor}{}
