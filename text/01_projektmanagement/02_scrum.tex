
\section{Scrum}\label{sec:scrum}

\renewcommand{\kapitelautor}{Autor: Irgendwer} % todo: replace

\subsection{Rollen}\label{subsec:rollen}

%
\subsubsection{Scrum Master}\label{subsubsec:Scrum-Master}
%
Der Leiter eines Projektteams ist der Scrum Master. Er hat sich damit zu befassen, dass das Projekt kontinuierliche Fortschritte erzielt und das durchgehend Verbesserung und Optimierungen am Entwicklungsprozess geschehen.
Eine weitere Tätigkeit, mit der sich der Scrum Master befasst, ist die Einhaltung des Projektablaufs zum Beispiel Sprints und Meetings.\cite{AsanaScrumMaster}
Im Rahmen des Diplomarbeitsprojekts hat der Scrum Master auch die Aufgaben eines Projektleiters übernommen, wie die Organisation und Durchführung von Meetings und ist Ansprechperson für projektspezifische Angelegenheiten gewesen.
%
\subsubsection{Produkt Owner}\label{subsubsec:Product-Owner}
%
\italic{"Beim Product Owner handelt es sich um eine standardmäßige Rolle in Scrum-Teams, deren Fokus darauf gerichtet ist, das bestmögliche Produkt für Endnutzer abzuliefern.
Um dies zu erreichen, entwickelt der Product Owner eine Vision davon, wie das Produkt funktionieren soll, definiert spezifische Produktfunktionen und unterteilt diese in Product-Backlog-Elemente, an denen das Scrum-Team arbeiten kann.
Der Product Owner trägt die Verantwortung für das fertiggestellte Produkt."}\cite{AsanaProductOwner}

Im Rahmen des Diplomarbeitsprojekts hat der Product Owner auch die Aufgaben eines stellvertretenden Projektleiters übernommen, wie die Unterstützung und Hilfe bei Organisation von Meetings und interne Planung und Kommunikation.
%
\subsubsection{Development Team}\label{subsubsec:Development-Team}
%
Das Development Team ist mit der Produkt/Software Entwicklung beschäftigt. Es arbeitet nach Vorgaben des Product Owners, unter der Führung des Scrum Masters um ihr Ziel zu erreichen.
Im Rahmen des Diplomarbeitsprojekts war das Development Team aus einem Leiter, der sich auch mit Sound und Vertrieb beschäftigt hat, zwei Designern und 2 Programmierern zusammengesetzt.
%
\subsection{Product Backlog}\label{subsec:product-backlog}
%

Der Product Backlog wurde während der Sprints verwendet und vom Projektleiter gepflegt. Der Backlog zeigt eine Aufgabenübersicht zugeordnet zu den verantwortlichen Teammitgliedern.
Die Aufgaben für einen Sprint werden während des Sprintplannings in den jeweiligen Sprint Backlog gezogen, um diese nach und nach zu erledigen.
Das Ziel ist die Tasks so aufzuspalten, dass man sie in einem Sprint erledigen kann und sie nur bei keinem Erfolg im nächsten Sprint weiter behandelt werden müssen.
Er wurde in der Endphase durch Tasks ersetzt, da das Backlog System während des Testens nicht passend ist. Durch die lange Dauer eines Sprints, kann man nicht aktiv auf Anpassungen reagieren,
da diese oft unterschiedlich lang dauern und oftmals Abhängigkeiten von Teammitgliedern bestehen. Außerdem ist nicht immer klar, wie lange es dauert, um eine neue Demo zu testen.
%

\subsection{Sprints}\label{subsec:sprints}
%
Ein Sprint ist in der Projektmanagementmethode Scrum das Hauptwerkzeug für agiles Management. Sprints sind zeitlich vordefinierte Rahmen, in denen das Development Team Zeit hat, um ihr Sprintgoal zu erfüllen.
Dies startet mit einer Planungsphase, dem Sprintplanning, geht über in die Entwicklungsphase, dem Sprint, und endet mit einem Sprint Review, in dem die Erreichung des Sprint Goals geprüft wird.
Zuletzt wird ein Retrospective absolviert, um sich intern, über Positives und Negatives im Sprint, Feedback zu geben.
%

% resets author
\renewcommand{\kapitelautor}{}
