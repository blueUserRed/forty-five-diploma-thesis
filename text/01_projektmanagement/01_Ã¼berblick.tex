
\section{Überblick}\label{sec:ueberblick}

\renewcommand{\kapitelautor}{Autor: Nils} % todo: replace

%
Beim Arbeiten an einem Projekt ist das Projektmanagement ein ausschlaggebender Faktor, so auch bei einem Videospiel. Es gibt verschiedenste Projektmanagementmethoden, die man verwenden kann, um sein Projekt zum Erfolg zu führen.
In diesen Fall wurden die beiden Methoden Scrum und Kanban verwendet.

Scrum ist ein Agile-Framework, welches dem Projektteam ermöglicht ihre Arbeit zu Planen und innerhalb eines Teams zu managen.
Die Methode gibt Vorgaben mit Rollenverteilungen, Abläufen und Richtlinien an, um die Arbeit und Vorgehensweise von Projektteams durchgehend zu Verbessern und auf fortlaufende Änderung vorzubereiten.
Standardgemäß besteht Scrum aus zeitlich definierten Sprints, welche das Ziel verfolgen, nach jedem Sprint kleine Teilerfolge zu erzielen.


\italic{"Scrum is an empirical process, where decisions are based on observation, experience and experimentation.
Scrum has three pillars: transparency, inspection and adaptation. This supports the concept of working iteratively.
Think of Empiricism as working through small experiments, learning from that work and adapting both what you are doing and how you are doing it as needed."}\cite{Scrum}


Das Kanban Framework arbeitet mit einem Board, das eine Übersicht über alle Aufgaben verschafft und in drei Kategorien gegliedert ist.
Mit dem Status ToDo, in Progress, und Done wird der momentane Arbeitsstand der Tasks signalisiert. Das Board ist eine Art des visuellen Projektmanagements die einen einfacheren Überblick verschaffen soll.\cite{AsanaKanban}


Die Entscheidung ist auf diese beiden Methoden gefallen, da es ermöglicht das Projekt in zwei Phasen zu unterteilen.
Zuerst in eine große Entwicklungsphase, wo die Großteile der Spielsysteme entstanden sind, sei es das Backpacksystem oder der Shop, aber auch Sounds und die Steampage.
Danach ging es in eine Testphase über, wo ein größerer Fokus auf ständiges Testen und Anpassen des Spiels gelegt wurde.
Die Testphase teilte sich in Tasks auf, da es sich oft um schnelle Änderungen und Anpassungen im Spiel bezog, da laufend neue Testdemos entwickelt wurden.
Zusätzlich hat eine enge interne Kommunikation es ermöglicht schnellstmöglich auf jegliche Probleme beim Testen zu reagieren und ständig neue Anpassungen zu treffen.

%

% resets author
\renewcommand{\kapitelautor}{}
