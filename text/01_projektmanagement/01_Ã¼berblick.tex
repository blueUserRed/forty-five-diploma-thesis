
\section{Überblick}\label{sec:ueberblick}

\renewcommand{\kapitelautor}{Autor: Nils} % todo: replace

%
Beim Entwickeln eines Videospiels ist das Projektmanagement eine ausschlaggebende Rolle. Nicht nur sorgt man für eine zeitliche Planung, sondern man hat auch intern eine klare Struktur.
Dies kann einerseits die Herangehensweise, aber auch die Kommunikation im Projektteam immens beeinflussen. Es gibt verschiedenste Projektmanagementmethoden, die man verwenden kann, um sein Projekt zum Erfolg zu führen.
In unserem Fall haben wir uns zweier bedient.

Scrum ist ein Agile-Framework, das es Projekte und Teams bei einheitliche und geplanter Arbeit unterstützt und ihnen hilft, Aufgaben nach dringlichkeit zu priorisieren und zu erledigen.
Die Methode gibt Vorgaben mit Rollenverteilungen, Abläufen und Richtlinien an, um die Arbeit und Vorgangsweise von Projektteams durchgehend zu Verbessern und auf fortlaufende Änderung vorzubereiten.
Standardgemäß besteht Scrum aus zeitlich definierten Sprints, welche das Ziel verfolgen, nach jedem Sprint kleine Teilerfolge zu erzielen.

"Scrum is an empirical process, where decisions are based on observation, experience and experimentation. 
Scrum has three pillars: transparency, inspection and adaptation. This supports the concept of working iteratively. 
Think of Empiricism as working through small experiments, learning from that work and adapting both what you are doing and how you are doing it as needed."\cite{Scrum}


Das Framework arbeitet mit einem Board das eine Übersicht über alle Aufgaben verschafft und in drei Kategorien gegliedert ist.
Mit dem Status ToDo, in Progress, und Done wird der momentane Arbeitsstand der Tasks signalisiert. Das Board ist eine Art des visuellen Projektmanagements die einen einfacheren Überblick verschaffen soll.\cite{AsanaKanban}

 Die Entscheidung ist auf diese beiden Methoden, da es ermöglicht das Projekt in zwei Phasen zu unterteilen.
Zuerst in eine große Entwicklungsphase, wo die Großteile der Spielsysteme entstanden sind, sei es das Mapsystem, das Backpacksystem oder der Shop, aber auch Demo Sounds und die Steampage.
Als Zweites eine Testphase, wo ein größerer Fokus auf ständiges Testen und anpassen des Spiels gelegt wird.
Die Testphase lief auch primär in Tasks ab, da es sich oft um schnelle Änderungen und Anpassungen im Spiel bezog, da laufend neue Testdemos released werden.
Zusätzlich hat eine enge interne Kommunikation es ermöglicht schnellstmöglich auf jegliche Probleme beim Testen zu reagieren und ständig neue Anpassungen zu treffen.
%

% resets author
\renewcommand{\kapitelautor}{}
