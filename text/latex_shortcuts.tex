\newcommand{\bold}[1]{\textbf{#1}}
\newcommand{\italic}[1]{\textit{#1}}
\newcommand{\code}[1]{\texttt{#1}}

% einfaches "siehe ..." - das Ziel muss man markieren mit \label{name} -- macht pandoc automatisch
% einfache Variante
%\newcommand{\kap}[1]{Kapitel~\ref{#1}, Seite~\pageref{#1}}
%\newcommand{\siehe}[1]{siehe \kap{#1}}
%\newcommand{\abb}[1]{Abbildung~\ref{#1}, Seite~\pageref{#1}}
% bessere Variante - braucht varioref
\newcommand{\kap}[1]{Kapitel~\vref{#1}}
\newcommand{\siehe}[1]{siehe \kap{#1}}
\newcommand{\abb}[1]{Abbildung~\vref{#1}}


%% http://ieg.ifs.tuwien.ac.at/~aigner/download/tuwien.sty
%Div. Abkürzungen (in Anlehnung an Jochen Köpper, jkthesis):
%\RequirePackage{xspace}
\newcommand{\bzw}{bzw.\@\xspace}
\newcommand{\bzgl}{bzgl.\@\xspace}
\newcommand{\ca}{ca.\@\xspace}
\newcommand{\dah}{d.\thinspace{}h.\@\xspace}
\newcommand{\Dah}{D.\thinspace{}h.\@\xspace}
\newcommand{\ds}{d.\thinspace{}s.\@\xspace}
\newcommand{\evtl}{evtl.\@\xspace}
\newcommand{\ua}{u.\thinspace{}a.\@\xspace}
\newcommand{\Ua}{U.\thinspace{}a.\@\xspace}
\newcommand{\usw}{usw.\@\xspace}
\newcommand{\va}{v.\thinspace{}a.\@\xspace}
\newcommand{\vgl}{vgl.\@\xspace}
\newcommand{\zB}{z.\thinspace{}B.\@\xspace}
\newcommand{\ZB}{Zum Beispiel\xspace}
\newcommand{\FF}{.Forty-Five\@\xspace}
\newcommand{\ff}{.Forty-Five\@\xspace}

%% https://github.com/Digital-Media/HagenbergThesis
\newcommand{\latex}{La\-TeX\xspace} % kein schnoerkeliges LaTeX mehr
\newcommand{\tex}{TeX\xspace}       % kein schnoerkeliges TeX mehr
\newcommand{\bs}{\textbackslash}    % Backslash character
\newcommand{\obnh}{\hskip 0pt } %optional break without hyphen: e.g. PlugIn{\obnh}Filter

\newcommand{\sa}{s.\ auch\@\xspace}
\newcommand{\so}{s.\ oben\xspace}
\newcommand{\su}{s.\ unten\@\xspace}

\newcommand{\uae}{u.\thinspace{}\"A.\@\xspace}
\newcommand{\uva}{u.\thinspace{}v.\thinspace{}a.\@\xspace}
\newcommand{\uvm}{u.\thinspace{}v.\thinspace{}m.\@\xspace}

\newcommand{\inlineCode}[1]{\mintinline{text}{#1}}
\newcommand{\inlineKotlin}[1]{\mintinline{kotlin}{#1}}
\newcommand{\inlineOnj}[1]{\mintinline{onj}{#1}}
\newcommand{\inlineGlsl}[1]{\mintinline{glsl}{#1}}
\newcommand{\inlineJava}[1]{\mintinline{java}{#1}}
% \citeauthor \citeyear
%\newcommand{\zit}[1]{ (vgl. \cite{#1})}
\newcommand{\zit}[1]{ (vgl. \citeauthor{#1} \citeyear{#1})}
%\newcommand{\zitt}[2]{(\cite{#1, #2})}
%\newcommand{\zittt}[3]{(\cite{#1, #2, #3})}
%\newcommand{\zitttt}[4]{(\cite{#1, #2, #3, #4})}

\newcommand{\quoted}[1]{\frqq#1\flqq}

\newcommand{\codeblockCaption}[1]{#1} % might be changed later

\newenvironment{liste}{\begin{itemize}\setlength{\itemsep}{1pt}\setlength{\itemsep}{0pt}\setlength{\parsep}{0pt}}{\end{itemize}}

\newenvironment{infoBox}{\begin{awesomeblock}[blue]{3pt}{}{magenta}}{\end{awesomeblock}} % todo: make look good

\newenvironment{codeBlock}[2]
{\VerbatimEnvironment\begin{figure}[H]\def\myenvargumentII{#2}\centering\begin{minted}{#1}}
{\end{minted}\caption{\myenvargumentII}\end{figure}}
