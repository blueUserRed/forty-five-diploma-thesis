
\section{Hintergrundmusik}\label{sec:hintergrundmusik}

\renewcommand{\kapitelautor}{Autor: Nils Hubmann}

%
In \ff ist Hintergrundmusik ein wichtiges Thema.
Sie soll das Spielerlebnis anregen und den Spieler mit Emotionen passend zu der Spielsituation erfüllen.
Durch den Einsatz von Musik, die während des gesamten Spiels zu hören ist, wollen wir den Hörsinn der Spieler gezielt ansprechen und so ein deutlich umfassenderes Spielerlebnis kreieren.
Unsere Sounds sind auf die individuellen Teile des Spiels angepasst und sollen auch einerseits ruhige Stimmungen als auch Aufregung und Spannung vermitteln.
Es gibt drei Teile der Hintergrundmusik, die einerseits für den Titelbildschirm, für die Map, die der Spieler erkundet, als auch für den Kampf, in dem hitzige Gefechte stattfinden.

\subsubsection{Splice}\label{subsubsec:Splice}
Eine große Unterstützung im Projekt war die Plattform Splice.
Diese bietet Effekt- und Soundsamples an, welche erworben werden können.
Durch ein Abonnement wurden Credits gesammelt, die für passende Melodie Samples verwendet wurden.
Durch die große Auswahl an Sounds wurden passende Melodie Samples gesucht, die unseren Kriterien entsprachen.

\subsubsection{Entstehung der Musikstücke}\label{subsubsec:Musik-Enstehung}
Die Musik ist in Kooperation mit einer externen Person namens Nils Jandrasits entstanden.
Dieser hat das Team unterstützt und zusammen mit uns an der Hintergrundmusik gearbeitet.
Hierfür wurden die bereits erwähnten Samples verwendet und darauf aufgebaut.
Als Digital Audio Workstation wurde Fl Studio verwendet, da Nils Jandrasits bereits über umfangreiche Kenntnisse in diesem Programm verfügt.
Nils Hubmann und Nils Jandrasits arbeiteten in mehreren Sessions an den Musikstücken und passten diese mit verschiedensten Instrumenten im Programm wie Trommeln und Gitarren an.
Ein wichtiger Faktor war hierbei nicht nur das Wild West Thema zu treffen, sondern auch die passende Situation und Stimmung zu vermitteln.
Außerdem finden sich auch in den Musikstücken Effekt Sounds wieder.
Dazu gehören Geräusche wie das Traben von Pferden, die Rufe eines Uhus oder das Klappern einer Schlange.
Dies diente dazu gezielt Geräusche zu verwenden, die in der Natur und im Wilden Westen vertreten sind.

\subsubsection{Sound Testing}\label{subsubsec:Sound-Testing}
Das Testen der Hintergrundmusik war ein wichtiger Part.
Denn nicht nur die vermittelte Stimmung ist wichtig, sondern auch, dass die Musik nicht störend, bei mehrmaligem Hören, auf den Spieler wirkt.
Darum entschieden wir uns dazu lange unterschiedliche Musikparts zu designen, um den Abstand zwischen den gleichen Teilstücken der Melodie zu maximieren.
Außerdem wurden mehrmals Rhythmusänderungen in bestimmten Teilen vorgenommen, um die Spannung mancher Situation klarer definieren und steuern zu können.

%

% resets author
\renewcommand{\kapitelautor}{}
