
\section{UI-sounds}\label{sec:ui-sounds}

\renewcommand{\kapitelautor}{Autor: Nils Hubmann} % todo: replace

%
In \ff spielt nicht nur die Hintergrundmusik eine wichtige Rolle, sondern auch die Effektsounds, die das Spielerlebnis vertiefen und die Welt des Wilden Westens lebendig machen.
Diese Sounds reichen von kleinen Details wie dem Hovern über Karten und button Sounds bis hin zu markanten Geräuschen wie dem Schuss Sound, der eine wichtige Rolle spielt und auch in verschiedenen Varianten verfügbar ist.
Effektsounds sind ein weiterer Weg um die Sinne von Nutzern anzusprechen und ein aufregendes Spielerlebnis zu garantieren

\bold{Aufnahme und Nutzung von Sounds:}
Einige der Effektsounds wurden auf traditionelle Weise, durch Geräusche und Mikrofon aufgenommen. Indem wir tatsächliche Spielkarten benutzten und bestimmten Bewegungen durchführten sind Geräusche für das Spiel entstanden, so wie man es sich von Spielkarten erwartet.
Das Hovern über Karten wurde beispielsweise durch das vorsichtige Bewegen von Karten über einer Tischoberfläche erzeugt, während das Aufheben und Ablegen von Karten durch das Stapeln und Auseinandernehmen von Karten realisiert wurde.
Diese Aufnahmen verleihen den Sounds eine authentische Qualität und tragen dazu bei, die Spieler tiefer in das Spiel eintauchen zu lassen.

\bold{Integration von Splice-Sounds:}
Neben den selbst aufgenommenen Sounds haben wir auch auf die umfangreiche Bibliothek von Splice zurückgegriffen, um eine Vielzahl von Effektsounds zu erhalten.
Es bietet eine breite Palette von hochwertigen Sounds, die perfekt auf unsere Bedürfnisse zugeschnitten waren, darunter Schüsse, UI-Buttons und Hintergrundgeräusche wie Rascheln und Eulenlaute.
Durch die Integration dieser Sounds konnten wir unsere Klangpalette erweitern und eine noch detailliertere Klanglandschaft schaffen.
Trotzdem bestand die Schwierigkeit darin einerseits passende Sounds zu finden und andererseits gefundene Sounds anzupassen.
Es kam oftmals zu der Situation, dass Geräusche oftmals zu lang, zu kurz oder von der Tonhöhe nicht passend waren. Diese wurden dann einzeln in Audition angepasst.

\bold{Sound Testing}
Auch die Sounds müssen sich zum Spiel passend anhören. Wenn man viel Erfahrung mit Videospielen hat, dann erwartet man sich ganz bestimmte Geräusche, wie zum Beispiel, dass sich manche Knöpfe, je nach Aussehen, mechanisch anhören.
Das ist auch ein großes Problem bei der Soundauswahl, da sich beim Sammeln von Sounds schwer sagen lässt, ob diese im Spiel passend sind. Dadurch wurden auch Effekt Sounds  teamintern mehrmals getestet und angepasst.
Um diesen Vorgang zu erleichtern wurde oftmals eine Auswahl von Sounds beispielsweise Button Sounds gesucht, sodass man mehrere Optionen zum Testen hat.

\bold{Sounddesign und Effektintegration:}
Bei der Gestaltung von Effektsounds für das Spiel war es wichtig, dass sie nicht nur realistisch klingen, sondern auch funktional sind. Die Sounds mussten gut erkennbar sein und gleichzeitig nicht zu dominant, um die Hintergrundmusik oder die Dialoge zu überdecken.
Darüber hinaus war es entscheidend, die Sounds so zu gestalten, dass sie das Spielerlebnis unterstützen und die Atmosphäre des Wilden Westens verstärken.
%

% resets author
\renewcommand{\kapitelautor}{}
