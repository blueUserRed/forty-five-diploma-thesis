
\section{Social Media}\label{sec:social-media}

\renewcommand{\kapitelautor}{Autor: Nils Hubmann} % todo: replace


%
Social Media hat sich als unverzichtbares Werkzeug in der Vermarktung von Videospielen etabliert, und wir haben verschiedene Plattformen wie YouTube, Instagram und Reddit in Betracht gezogen, um die Reichweite unseres Spiels zu maximieren.
Besonders haben wir uns darauf konzentriert, mit YouTubern zusammenzuarbeiten, die durch ihre Videos unser Spiel einer breiten Zielgruppe vorstellen können.
Eine vielversprechende Entwicklung war die Kontaktaufnahme mit dem YouTuber Spieletrend, der positiv auf unsere Anfrage reagierte und sich bereit erklärte,
ein Video über unser Spiel zu erstellen und es ausführlich zu testen. Diese Zusammenarbeit verspricht nicht nur eine größere Sichtbarkeit, sondern auch authentische Einblicke und Bewertungen,
die das Interesse potenzieller Spieler wecken können. Zusätzlich haben wir unseren eigenen Content auf Social Media erstellt, um unsere Community zu engagieren und das Spiel bekannter zu machen.
Über unseren offiziellen Instagram-Account Microwavestudiosofficial wird Conent aus dem Spiel, so wie das Team und der Trailer präsentiert, um das Interesse der Spieler zu wecken und die Vorfreude auf das Spiel zu steigern.
Auch über den Schul-Instagram-Account htl\_rennweg haben wir Content veröffentlicht, um Mitschüler für unser Spiel zu begeistern.
Darüber hinaus haben wir unsere Steam-Seite aktiv beworben und einen Trailer auf YouTube veröffentlicht, um potenzielle Spieler anzusprechen und einen ersten Eindruck von unserem Spiel zu vermitteln.
Die Nutzung verschiedener Social-Media-Plattformen bietet uns die Möglichkeit, direkt mit unserer Zielgruppe zu interagieren, Feedback zu erhalten und eine engagierte Community aufzubauen.
Durch die gezielte Zusammenarbeit mit Influencern und die aktive Einbindung unserer Community können wir das Interesse an unserem Spiel steigern und eine treue Fangemeinde aufbauen, die unser Spiel unterstützt und weiterempfiehlt.
%

\renewcommand{\kapitelautor}{}
