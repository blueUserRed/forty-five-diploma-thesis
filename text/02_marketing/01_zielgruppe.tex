\usepackage{hyperref}
\section{Zielgruppe}\label{sec:zielgruppe}

\renewcommand{\kapitelautor}{Autor: Nils} % todo: replace

%
Die Einschätzung der Produktqualität und seines Erfolgs auf dem Markt hängt größtenteils vom Nutzen ab, den Kunden aus dem Produkt ziehen.
Es ist jedoch unerlässlich, dass das Produkt die Anforderungen, Wünsche und Erwartungen der Kunden optimal erfüllt, da sie letztendlich über den Erfolg eines Unternehmens entscheiden.
In diesem Zusammenhang sind Analysen wie die Definition der Zielgruppe äußerst hilfreich.
„Unter dem Begriff Zielgruppenanalyse versteht man alle Aktivitäten, die damit einhergehen, zu verstehen, was die Konsumenten von einem
Produkt erwarten, wie sie sich verhalten und welche Bedürfnisse sie haben.“\zit{Zielgruppe}


Die übliche Klassifizierung von Zielgruppen basiert oft auf äußeren Merkmalen wie sozio-demographischen Daten oder finanziellen Aspekten.
Jedoch geht es bei der Definition einer Zielgruppe weit darüber hinaus. Eine Zielgruppe zeichnet sich vor allem durch gemeinsame Wünsche,
Bedürfnisse und Probleme aus. Daher ist es wichtig, Ihre Zielgruppen primär über ihre Bedürfnisse und Probleme zu definieren.
Erst danach können weitere übliche Kriterien wie Kaufkraft, Religion oder Alter zur weiteren Eingrenzung herangezogen werden. \cite[vgl.]{Zielgruppedef}

Dabei sind wir auf folgende Ergebnisse gekommen.

\subsubsection{Demographische Merkmale}\label{subsubsec:Demographische-Merkmale}

Die demografischen Merkmale beziehen sich auf das Alter, Geschlecht, Berufsstand und weiteren wichtigen Aspekte des täglichen Lebens.\cite[vgl.]{DemographischeMerkmale}
Alter: 16-50 Jährige (Das Spiel hat eine Altersbeschränkung von 16 Jahren)
Geschlecht: Das Geschlecht ist nicht relevant
Bildung: mittelmäßiges Bildungsniveau
Beruf: Das Spiel ist für keine bestimmte Berufsgruppen interessant.
Einkommen: Da das Spiel gratis sein wird, ist das Einkommen der Zielgruppe nicht
relevant.

\subsubsection{Geographische Merkmale}\label{subsubsec:Geographische-Merkmale}

„Auch diese Art von Merkmalen wird recht häufig genutzt, um zu erfahren, wo ihre Kunden wohnen, um so eine Werbung zu machen, welche auch die richtigen Leute am richtigen Ort trifft.“\zit{GeographischeMerkmale}
Die geografische Aufteilung nach Bundesländern oder Bezirken ist für dieses Computerspiel von geringer Bedeutung, da das Spiel global betrachtet wird und die Spielerbasis weltweit anspricht.
Die Kommunikation erfolgt hauptsächlich in englischer Sprache, um eine weitreichende Verständigung zu ermöglichen, insbesondere in englischsprachigen Ländern wie den USA, Großbritannien, Kanada und anderen.
Diese Länder werden aufgrund ihrer Muttersprache als Zielregionen betrachtet.

\subsubsection{Psychographische Merkmale}\label{subsubsec:Psychographische-Merkmale}

„Psychografische Faktoren sind beispielsweise Verhaltensmerkmale, Werte, Vorlieben und Charaktereigenschaften Ihrer Kunden und Kundinnen. Welchen Lifestyle haben sie? Was be-
wegt sie und warum? Welchen Hobbys gehen sie in ihrer Freizeit nach? Ist ein ausgeprägtes Gesundheits- und Umweltbewusstsein vorhanden? Haben sie traditionelle Einstellungen oder
sind sie offen für Neues? All diese Fragen helfen Ihnen, sich besser in Ihre Zielgruppe hinein zu versetzen.“ \zit{PsychographischeMerkmale}

Persönlichkeit: Das Spiel ist hauptsächlich an Personen gerichtet die siche für Karten- und Strategiespiele interessieren.

Freundesstand: Da dieses Spiel allein spielbar ist, sind für die Nutzung keine weiteren Personen notwendig.

Hobby: Computerspielen

Folgende Genres, die zur Bestimmung der Verhaltensweise wichtig sind:
\begin{liste}
    \item WildWest
    \item Strategie
    \item Indie
    \item Einzelspieler
    \item Roguelite
    \item Kostenlos
\end{liste}


\subsubsection{Zusammenfassung}\label{subsubsec:Zusammenfassung}
Das Spiel ist hauptsächlich für Hobbyspieler interessant, die nicht viel Geld in Spiele investieren wollen. Dabei hat \ff einen Vorteil gegenüber anderen Indiegames einen Vorteil,
da diese zwar nicht so viel kosten aber trotzdem nicht gratis sind. Die meisten Indiegames kosten zwischen 5 und 25€ und sind daher Low Budget Games.\cite[.vgl]{IndiegamesPreis}
Des Weiteren wissen viele neue Spieler gar nicht, ob ihnen das Spiel gefällt, weshalb sie es, dadurch das es gratis ist, testen können. Das
kann wiederum dem Spiel helfen, mehr Aufmerksamkeit zu bekommen.


% resets author
\renewcommand{\kapitelautor}{}
