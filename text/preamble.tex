
\setcounter{secnumdepth}{3}

%\usepackage[T1]{fontenc}
\usepackage[utf8]{inputenc}
\usepackage[english, ngerman]{babel, varioref} % Deutsch muss letztes sein
\usepackage{lastpage}
\usepackage{listings}
\usepackage{blindtext}
\usepackage[inline]{enumitem} %% Aufzählungen nicht so weit einrücken

% Listen etwas wenige einrücken, erfordert enumitem
\setitemize{leftmargin=*}

\usepackage{lmodern}
\usepackage{xspace}
\usepackage{graphicx}
\graphicspath{ {./images/} }
\usepackage{float}

\usepackage[hyphens]{url}

\usepackage{makeidx}
\makeindex

\usepackage{natbib}

\PassOptionsToPackage{normalem}{ulem}

\usepackage{ulem}
\usepackage{needspace}

\setlength\partopsep{0.5ex} % schoenere Listen

\usepackage[bottom]{footmisc} % fussnote ganz unten

\usepackage[]{microtype}
\UseMicrotypeSet[protrusion]{basicmath} % disable protrusion for tt fonts

%\usepackage{multirow}   % Allows table elements to span several rows.
%\usepackage{booktabs}   % Improves the typesettings of tables.
%\usepackage{subcaption} % Allows the use of subfigures and enables their referencing.
\usepackage[ruled,linesnumbered,algochapter]{algorithm2e} % Enables the writing of pseudo code.
\usepackage[usenames,dvipsnames,table]{xcolor} % Allows the definition and use of colors. This package has to be included before tikz.
\usepackage{nag}       % Issues warnings when best practices in writing LaTeX documents are violated.
%\usepackage{todonotes} % Provides tooltip-like todo notes.
\usepackage{color}
%\usepackage[binary-units]{siunitx}

% Definiert einen Bundsteg von 1.5cm
% NUR BEI BEIDSEITIGEN DRUCK!!
\usepackage{geometry}
\geometry{
    left = 2cm,
    right = 2.5cm,
    bindingoffset = 1.5cm,
}

%  Kopf und Fußzeilen -- links und rechts verschieden
\newcommand{\kopfseitenummer}{{\bfseries \thepage}}
\newcommand{\kopfkapl}{{\bfseries\leftmark}}
\newcommand{\kopfkapr}{{\bfseries\rightmark}}
\newcommand{\kopfbild}{\voffset7mm\includegraphics[width=25mm]{HTL3RLogoRGB}}
\newcommand{\kopfHTL}{Höhere Technische Bundeslehranstalt Wien 3, \\Rennweg 	Abteilung für Informationstechnologie}

\usepackage{fontspec}
\usepackage{scalefnt}

% setzt Schriftart für Fließtext
% muss installiert sein
\setmainfont{Aptos}

% setzt Schriftart für Code-Blöcke
% muss installiert sein
\newfontfamily\codefont{JetBrainsMono-Regular.ttf}[NFSSFamily=JetBrainsMonoFamily]



\usepackage[automark,headsepline,footsepline,plainfootsepline]{scrlayer-scrpage}
\setkomafont{pageheadfoot}{\normalcolor\footnotesize\scshape}
\setkomafont{pagenumber}{\normalfont\normalsize}
\clearpairofpagestyles
\ihead{\voffset7mm\includegraphics[width=35mm]{logo_red}}
%\ihead{\headmark}
\ohead{\kopfbild}
\ifoot{\kapitelautor}
\ofoot{\pagemark}
\ModifyLayer[addvoffset=-.6ex]{scrheadings.foot.above.line}% Linie verschieben
\ModifyLayer[addvoffset=-.6ex]{plain.scrheadings.foot.above.line}% Linie verschieben
\setlength{\headheight}{32pt}

% alle Seiten mit Kopfzeile
\renewcommand{\chapterpagestyle}{scrheadings}


\usepackage{minted}
\usemintedstyle{forty_five_style}
% Konfig für Code-Blöcke
\setminted{
    frame=lines,
    framesep=2mm,
    breaklines=true,
    fontfamily=JetBrainsMonoFamily,
    fontsize=\scriptsize
}

%\usepackage{awesomebox}
%\setlength{\aweboxleftmargin}{1pt}

%\usepackage{scrhack}

%% glossar
% kann man löschen falls kein Glossar gebraucht
%\usepackage[acronym, toc]{glossaries}
%\makeglossaries
%\input{text/glossar.tex}

\usepackage{tcolorbox}
\tcbuselibrary{xparse,skins,breakable}
\definecolor{htl3red}{RGB}{255,51,0}
\newtcolorbox{TitlePageBox}{%
    breakable,
    blanker,
    left=1em,
    borderline west={0.15cm}{3pt}{htl3red},
}

\usepackage[unicode=true,
    bookmarks=true,bookmarksnumbered=false,bookmarksopen=false,
    breaklinks=true,pdfborder={0 0 0},backref=false,colorlinks=false]
{hyperref}
\hypersetup{
    pdftitle={.Forty-Five},
    pdfauthor={Markus Böheim, Nils Hubmann, Philip Jankovic, Marvin Kurka, Felix Zwickelstorfer},
    pdfsubject={Diplomarbeit},
    pdfkeywords={Forty-Five, Card-Game, card game, wild west, Wild-West, open source, western game, bullets, revolver}
}
\urlstyle{same} % don't use monospace font for urls

% Auch Fußnoten bündig ausrichten
\deffootnote[]{1em}{1em}{\textsuperscript{\thefootnotemark\ }}
\sloppy % weniger Meldungen
\voffset7mm % etwas nach unten

%% schöner: 10000 -- gar keine, 1000 als Mittelweg
\clubpenalty = 10000 % Schusterjungen verhindern
\widowpenalty = 10000 % Hurenkinder verhindern
\displaywidowpenalty = 10000
