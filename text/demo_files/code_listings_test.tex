
\chapter{Code-Blöcke Test}\label{ch:code-blocke-test}

\renewcommand{\kapitelautor}{Autor: Marvin}

Das in einem Wild-West Stil gehaltene Spiel ist ein neuartiger Spin von dem klassischen Genre des \quoted{Rogue Lites}.
Der Spieler erkundet eine Welt die aus verschiedenen Landschaften wie einem verzauberten Wald oder einer trockenen Wüste,
sammelt Karten in Form von Bullets, baut Decks und bekämpft Gegner in spannenden Kämpfen.
Die Karten/Bullets können in den Revolver geladen und abgefeuert werden.
Je nachdem, wie die Karten im Revolver platziert werden, können die einzigartigen Fähigkeiten jeder Karte sowie die
Drehung der Revolvertrommel dazu verwendet werden, immer stärkere Combos zu spielen und dem Gegner Effeckte und Schaden zuzufügen.
Die verschiedenen computergesteuerten Gegnertypen verfügen jedoch über ihre eigenen Tricks und setzen den Spieler mit Attacken unter Druck.

\begin{minted}{kotlin}
    /**
     * creates a new instance of the card named [name] and puts it in the hand of the player
     */
    @AllThreadsAllowed
    fun tryToPutCardsInHandTimeline(name: String, amount: Int = 1): Timeline = Timeline.timeline {
        var cardsToDraw = 0
        action {
            val maxCards = hardMaxCards - cardHand.cards.size
            cardsToDraw = min(maxCards, amount)
        }
        action {
            if (cardsToDraw == 0) return@action
            val cardProto = cardPrototypes
                .firstOrNull { it.name == name }
                ?: throw RuntimeException("unknown card: $name")
            repeat(cardsToDraw) {
                cardHand.addCard(cardProto.create())
            }
            FortyFiveLogger.debug(logTag, "card $name entered hand")
            checkCardMaximums()
        }
    }
\end{minted}

\begin{minted}{onj}
import "imports/colors.onj" as color;
import "screens/input_maps.onj" as inputMaps;
import "screens/shared_components.onj" as sharedComponents;

use Common;
use Screen;
use Style;

var worldWidth = 1600.0;
var worldHeight = 900.0;

options: {
    transitionAwayTime: 1.0,
    background: "hover_detail_background",
    inputMap: [
        ...(inputMaps.defaultInputMap),
        ...(inputMaps.healOrMaxInputMap),
        ...(inputMaps.addMaxHPInputMap),
    ],
    screenController: $HealOrMaxHPScreenController {
        addLifeActorName: "add_lives_option",
    }
},
styles: [
    {
        width: 40#percent,
        height: 83#percent,
        flexDirection: flexDirection.column,
        alignItems: align.center,
        justifyContent: justify.center,
        touchable: touchable.enabled,
        background: "heal_or_max_selector_background",
        marginRight: 0#percent,
        positionRight: 0#percent,
    },
    {
        style_priority: 2,
        background: "heal_or_max_selector_background_selected",
        width: 70#percent,
        height: 125#percent,
        style_condition: actorState("selected"),
        marginRight: -30#percent,
        positionRight: 22#percent,
    },
],
root $Box {
} chilren [

]
\end{minted}

\Blindtext

%%%%%%%%%%%%%%%%%%%%%%%%%%%%%%%%%%%%%%%%%%%%%%%%%%%%%%%%%%%%%%%%%%%%%%%%%%%%%%%%%%%%%%%%%%
% wer hat diese Kapitel geschrieben oder leer
\renewcommand{\kapitelautor}{}
