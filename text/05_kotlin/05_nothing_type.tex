
\section{Nothing Typ}\label{sec:nothing-type}

\renewcommand{\kapitelautor}{Autor: Marvin Kurka}

Der \inlineKotlin{kotlin.Nothing} Typ ist in der Kotlin Standard-Bibliothek definiert, er hat allerdings einige
spezielle Eigenschaften.
Zum Beispiel handelt sich bei Nothing um einen unified subtype (auch bottom Type genannt) für alle in Kotlin
definierten Typen.
Das heißt, eine Instanz von Nothing wäre auch eine Instanz jeder anderen Klasse, sei es \inlineKotlin{String},
\inlineKotlin{Int} oder \inlineKotlin{() -> Boolean}.
Da so ein Objekt einen logischen Widerspruch darstellt, handelt es sich bei Nothing auch um einen uninhabited Type.
Das heißt, dass niemals eine tatsächliche Instanz von Nothing existieren kann.

Auf den ersten Blick mag so ein Typ nicht sehr nützlich wirken, er erlaubt den Compiler aber einige nützliche
Annahmen über den Control Flow zu machen.
Sollte jemals eine Variable den Wert Nothing annehmen, was einen logischen Widerspruch darstellt, weiß der Compiler,
dass dieser Code niemals erreicht werden kann.\cite{kspeCFG,kspecNothing}

%! language = kotlin
\begin{minted}{kotlin}
fun getNothing(): Nothing {
    ...
}

fun test() {
    // Nothing ist ein Subtype von String, daher ist der folgende Code erlaubt
    val s: String = getNothing()

    // für diesen Code gibt der Compiler eine Warning aus, da er niemals erreicht werden kann
    println("Hello World")
}
\end{minted}
\codeblockCaption{
Dadurch das die Funktion \inlineKotlin{getNothing()} den Rückgabewert Nothing hat, der niemals existieren kann,
weiß der Compiler, dass diese Funktion niemals auf normale Art und Weise returnen kann und in einer Exception
resultieren muss.
}

Um den Nothing-Typen tatsächlich nützlich zu machen, sind in Kotlin Statements, die einen Control-Flow Transfer
verursachen eigentlich Expressions mit dem Rückgabewert Nothing.
Beispiele sind: \inlineKotlin{throw}, \inlineKotlin{return}, \inlineKotlin{break} und \inlineKotlin{continue}.
Da diese Expressions einen sofortigen Wechsel des Control-Flows verursachen hat das zur Folge, dass der nachfolgende
Code niemals ausgeführt wird.
Damit können diese Expressions den Wert Nothing haben.\cite{kspeCFG}

%! language = kotlin
\begin{minted}{kotlin}
operator fun Int?.plus(other: Int?): Int? {
    // return hat den Wert Nothing, was ein Subtype von Int ist,
    // daher ist die Verwendung nach dem elvis operator erlaubt
    val first = this ?: return null
    val second = other ?: return null
    return first + second
}
\end{minted}
\codeblockCaption{Praktisches Beispiel für die Verwendung des Nothing-Typen}

Dadurch das \inlineKotlin{return} in Kotlin eine Expression ist, ist auch der folgende Code gültiges Kotlin:

%! language = kotlin
\begin{minted}{kotlin}
fun test() {
    println("Hello")
    return return return return return
}
\end{minted}

Funktionen, die den Rückgabewert Nothing haben, müssen immer eine Exception werfen.

%! language = kotlin
\begin{minted}{kotlin}
fun crashProgram(cause: String): Nothing {
    MyLogger.log("Program crashed because of: $cause")
    throw RuntimeException(cause)
}

fun doSomething(input: String?) {
    otherFunction(input ?: crashProgram("input must not be null"))
}
\end{minted}

Ein Beispiel für eine Funktion aus der Kotlin Standard-Bibliothek die Nothing zurückgibt, ist die
\inlineKotlin{TODO()} Funktion.

%! language = kotlin
\begin{minted}{kotlin}
fun someFunction(): Int = TODO("not yet implemented")
\end{minted}

% resets author
\renewcommand{\kapitelautor}{}
