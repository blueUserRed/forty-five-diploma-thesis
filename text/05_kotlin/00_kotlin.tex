
\chapter{Kotlin}\label{ch:kotlin}

Kotlin ist eine objektorientierte Programmiersprache mit funktionalen Elementen, einem statischen Typensystem und
Multiplatform-Support.

Zu den besonderen Merkmalen von Kotlin gehören:
\begin{liste}
    \item Null-Safety, was die Fehleranfälligkeit reduziert,
    \item Java-Interoperabilität, was das stückweise Migrieren von Code erlaubt,
    \item Multiplatform-Support, was das Verwenden von Kotlin auf verschieden Plattformen erlaubt,
    \item Moderner Syntax, was Boilerplate reduziert und Code leichter lesbar macht
\end{liste}

2016 wurde die erste stabile Version von JetBrains offiziell released und seitdem ist Kotlins Userbase
stetig gewachsen und die Sprache hat besonderen Anklang in der Android-Entwicklung gefunden.
2017 hat Google angekündigt, das Kotlin als Sprache für Android-Entwicklung supported wird und seit 2019 ist Android
sogar Kotlin-First, was heißt, dass neue Features vermehrt mit/für Kotlin entwickelt werden.

Trotz Kotlins geringen Alters hat es bereits Popularität bei Entwicklern erlangt, und das nicht nur im
Android-Development.
In der Stackoverflow Developer Survey 2023 gaben 9\% der Entwickler an, Kotlin zu verwenden, und 60\% der
Umfrageteilnehmer, die die Sprache verwendet haben, wollen das im nächste Jahr wieder tun.

Außerdem erlaubt Kotlin Multiplatform das Teilen von Code über mehrere Plattformen hinweg.
Kotlin kompiliert nicht nur zu JVM-Bytecode, sondern andere Compilation-Targets inkludieren JavaScript, WASM, oder
nativer Binärcode.\cite{androidDevKotlinFirst, kspecIntroduction, soDevSurvey23, kMediaKit}


\section{Null Safety}\label{sec:null-safety}

\renewcommand{\kapitelautor}{Autor: Marvin Kurka}

% todo: quelle für basic infos zu null
\inlineCode{null}, auch unter den Namen \inlineCode{nil} oder \inlineCode{None} bekannt, wird in Programmiersprachen
verwendet, um einen nicht vorhandenen Wert zu repräsentieren.
Wird ein Zugriff auf \inlineCode{null} ausgeführt, sei es ein Funktionsaufruf oder Property-access, kommt es zu
einem Fehler.

\subsection{Situation in Java}
Ein Beispiel für den Einsatz eines Null-Werts ist die \inlineJava{File::getParent} Funktion, die, unter normalen
Umständen, die übergeordnete Datei zurückgibt.
Allerdings ist das nicht in allen Fällen möglich, und in diesen wird \inlineJava{null} stattdessen als Rückgabewert
verwendet.\cite{jdocFile}
Dieser Umstand ist aber nicht in der Funktionssignatur reflektiert, in dieser ist als Rückgabewert einfach
\inlineJava{String} angegeben.
Für den Programmierer ist es also nicht sofort offensichtlich, dass ein Wert auch null sein kann, ein Blick in die
Dokumentation ist hier notwendig.
Dieser Umstand erhöht nicht nur den Aufwand für den Programmierer, er macht auch eine statische Analyse unmöglich.
So kann ein Compiler oder Linter nicht aus den Typen lesen, ob er \inlineJava{null} sein kann, und so auch in den
meisten Fällen nicht vor wahrscheinlichen Programmierfehlern warnen.
Außerdem haben \inlineJava{null}-Werte die Eigenschaft durch das Programm zu propagieren, entsteht also ein
\inlineJava{null}-Wert am Beginn einer langen Funktionsaufrufskette, kann es zu Situationen kommen in der das Programm
an einer Stelle crasht, die mit dem eigentlichen Fehler nichts zu tun hat.
Aus diesen Gründen ist \inlineJava{null} eine häufige Fehlerursache in Java, laut Google sind
NullPointerExceptions sogar die häufigste Crash-Ursache von Android-Apps im Google Play Store.\cite{androidDevNPE}

\subsection{Lösungsansätze}
Wie im letzten Absatz argumentiert, ist \inlineJava{null}, wie es in Sprachen wie Java implementiert ist, schlechtes
Programmiersprachendesign.
So hat auch Tony Hoare, der im Jahr 1965 zuerst das Konzept eines \inlineJava{null}-Pointers in die Sprache ALGOL W
eingeführt hat, diese Entscheidung in einer berühmten Präsentation als seinen "Milliarden Dollar Fehler" bezeichnet.\cite{infoqNullRefs}
Deswegen haben über die Jahre Designer von Programmiersprachen nach Wegen gesucht, optionale Werte auf eine sinnvollere
und weniger fehleranfällige Weise zu repräsentieren.
So auch in der Sprache Java, die 2014 um die \inlineJava{Optional<T>} Klasse erweitert wurde.\cite{jdocOptional}
Diese kann als Wrapper um Werte verwendet werden und stellt Funktionen zur Verfügung, die auf Präsenz eines Wertes
testen und/oder abhängig davon Code ausführen.

\begin{codeBlock}{java}{Demo: \inlineJava{Optional<T>}}
getOptionalValue().ifPresentOrElse(
    value -> System.out.println("result was: " + value),
    () -> System.out.println("no value")
);
\end{codeBlock}

Trotz der höheren Sicherheit findet \inlineJava{Optional<T>} wenig Verwendung, da, durch die nachträgliche und relative
späte Einführung in die Sprache große Teile der Standard-Bibliothek selbst keine Optionals verwenden (Siehe
File-Beispiel).
Außerdem ist das Nachziehen von Optional in existierenden Code-Bases mit viel Arbeit verbunden, da sämtliche
Funktionssignaturen und die Usages der Funktionen angepasst werden müssten.
All das führt dazu, dass ein anderer Weg \inlineJava{null}-Werte zu markieren eine wesentliche höhere Verbreitung
erzielt hat.

Diese Alternative sind Annotations.\label{java-null-annotations}
Annotations können von dem Programmierer verwendet werden, um Metadaten in den Code einzufügen, so auch ob ein Wert null
sein kann, mittels der \inlineJava{@NotNull} und \inlineJava{@Nullable} Annotations.
Wichtig anzumerken ist, dass diese Annotations nichts an der tatsächlichen Funktion des Codes ändern und auch keine
Compiler-Fehler versuchen, sollten sie nicht respektiert werden.
Allerdings können sie von statischen Analyse-Tools verwendet werden, um Warnings bei verdächtigem Code auszugeben.
Der Fakt, dass diese Annotations den Code nicht beeinflussen, macht es einfacher diese in bereits existierende Projekte
einzubinden, auch wenn mit ihnen gegenüber Optionals einiges an Sicherheit verloren geht.
Da die Java-Standardbibliothek diese Annotations nicht enthält, sind mehrere konkurrierende Bibliotheken entstanden,
die diese zur Verfügung stellen.
Zum Beispiel:
\begin{itemize}
    \item Jetbrains Annotations \url{https://central.sonatype.com/artifact/org.jetbrains/annotations/24.0.1}
    \item The Checker Framework \url{https://checkerframework.org/}
    \item Eclipse \url{https://help.eclipse.org/latest/index.jsp?topic=%2Forg.eclipse.jdt.doc.user%2Ftasks%2Ftask-using_null_type_annotations.htm}
\end{itemize}

Die Programmiersprache Rust, die dafür bekannt ist, hohen Wert auf Sicherheit und fehlerfreien Code zu legen, erlaubt
gar keine \inlineJava{null}-Werte.
Stattdessen bietet sie ein \inlineCode{Option<T>} enum an, dass zwei mögliche Werte hat:
\inlineCode{None} und \inlineCode{Some(T)}.
In vielerlei Hinsicht ist diese Lösung ähnlich zu Javas Optionals, da Rust allerdings keine klassischen
\inlineJava{null}-Werte erlaubt und die Standard-Bibliothek mit diesem Enum im Hinterkopf designet wurde, funktioniert
diese Lösung in Rust besser als in Java.\cite{rustDocOptional}

\subsection{Nulls in Kotlin}
In Kotlin müssen Typen, die den Wert \inlineKotlin{null} annehmen dürfen, explizit als nullable markiert werden.
Das wird mittels eines Fragezeichens hinter dem Typen umgesetzt.
Der Compiler analysiert zur compile-time welche Werte null sein könnten und gibt einen Fehler aus, wenn mit
\inlineKotlin{null} nicht richtig umgegangen wird.\cite{kdocNullSafety}

%! language = Kotlin
\begin{codeBlock}{kotlin}{Demo: Null-Safety in Kotlin}
var a = "Hello"
a = null // Compile-time Fehler
var b: String? = "Hello"
b = null // erlaubt
\end{codeBlock}

%! language = kotlin
\begin{codeBlock}{kotlin}{Demo: Null-Check in Kotlin}
    fun someFunction(string: String?) {
        if (string != null) println(string.length) // erlaubt
        println(string.length) // Compile-time Fehler
    }
\end{codeBlock}

Diese Schutzmechanismen zwingen den Programmierer mit \inlineKotlin{null}-Werten sinnvoll umzugehen und helfen so dabei
Code zu schreiben, der weniger fehleranfällig ist.
Laut Google war das Team von Google Home durch den (teilweisen) Umstieg auf Kotlin in der Lage, die Zahl ihrer
NullPointerExceptions um 33\% Prozent zu reduzieren.\cite{androidDevGoogleHome}

Neben dem statischen Überprüfen auf mögliche \inlineKotlin{null}-Werte ist es auch wichtig, den notwendigen
Syntax und/oder die notwendigen Funktionen zur Verfügung zu stellen, die gebraucht werden, um mit ihnen sinnvoll
umzugehen.
Während if-Bedingungen für \inlineKotlin{null} Checks theoretisch ausreichen, führen sie zu langen und schlecht lesbaren
Code, weswegen Kotlin eigene Operatoren für \inlineKotlin{null}-Werte hat.\cite{kdocNullSafety}

%! language = kotlin
\begin{codeBlock}{kotlin}{Demo: Umgang mit \inlineKotlin{null}-Werten}
// Der safe call operator führt einen Zugriff nur aus, wenn der Wert nicht null ist
// Ist der Wert null, gibt auch der operator null zurück
nullableVariable?.nullableProperty?.someFunction()

// Der not null assertion operator wirft eine Exception, wenn auf einem null-Wert angewendet wird
nullableVariable!!.nullableProperty!!.someFunction() // kann NullPointerException werfen

// Der elvis operator überprüft, ob ein Wert null ist, und verwendet dann einen anderen Wert stattdessen
val result = nullableVariable?.nullableProperty?.someFunction() ?: "default Value"
\end{codeBlock}

\subsection{Interoperabilität mit Java}
Ein wichtiger Aspekt von Kotlin ist die gute Kompatibilität mit Java.
Diese erlaubt es Entwicklern Code stückweise von Java zu Kotlin zu migrieren und erlaubt auch die Verwendung von
Java-Bibliotheken in Kotlin-Projekten.
Da Java allerdings keine \inlineKotlin{null}-safe Sprache ist, stellt sich die Frage wie mit Rückgabewerten von
Java-Funktionen umgegangen wird, da an der Funktionssignatur nicht erkannt werden kann, ob dieser Wert
\inlineKotlin{null} sein kann.

In frühen Kotlin Versionen wurde dieses Problem so umgangen, dass der Kotlin-Compiler alle von Java kommenden Typen als
nullable interpretiert hat.
Diese Herangehensweise hat sich allerdings als nicht praktikabel herausgestellt, da absurde Mengen an \inlineKotlin{null}
Checks dadurch notwendig wurden.
Neue Kotlin Versionen suchen nach den vorher erwähnten Annotations und verwenden diese um herauszufinden, ob ein Type
nullable ist oder nicht.
Wird keine solche Annotation gefunden, verwendet Kotlin einen sogenannten Platform Type stattdessen.\cite{kdocNullInterop,kblogNullSafety}

Platform types werden mit einem Rufzeichen nach dem Typen gekennzeichnet: \inlineKotlin{String!}.
Diese Typen können nicht direkt definiert werden, sondern entstehen nur bei Interaktion mit Java-Code, wenn der Compiler
nicht aus Annotations schließen kann, ob der Typ nullable ist oder nicht.
Kotlin schreibt bei Platform Types keine \inlineKotlin{null} Checks vor, gibt aber auch keine Warnings aus, wenn
welche gemacht werden.
Wird ein Platform Typ einer Variable zugewiesen sollte explizit ein Typ angegeben werden, um die nullability zu
definieren.
Wird das nicht gemacht, gibt der Compiler eine Warning aus.\cite{kdocNullInterop}

% resets author
\renewcommand{\kapitelautor}{}


\section{Flow Typing}\label{sec:flow-typing}

\renewcommand{\kapitelautor}{Autor: Marvin Kurka}

Flow-Typing bedeutet, dass der Compiler Annahmen über den Typen von Variablen machen kann, basierend auf dem Control
Flow von dem Programm.

%! language = kotlin
\begin{codeBlock}{kotlin}{Beispiel für Flow-Typing in Kotlin}
fun test(s: Any?): Unit = when (s) {

    null -> println("s is null")

    // s hat hier den Typ String
    is String -> println("s is a string with length ${s.length}")

    // s hat hier den Typ List<*>
    is List<*> -> println("s is a list with ${s.distinct().size} distinct items")

    // s hat durch den anfänglichen null-Check hier den Typ Any
    else -> println("nothing matched with ${s::class.simpleName}")
}
\end{codeBlock}

Kotlin implementiert Flow-Typing mittels Smart-Casts, die automatisch in den Code eingefügt werden, wenn gewisse
Control-Flow Instruktionen verwendet werden.
Das inkludiert \zB whens, ifs, not-null-assertions, casts, is-checks, \usw

%! language = kotlin
\begin{codeBlock}{kotlin}{Mehr Beispiele für die Anwendung von Flow-Typing}
sealed class A {
    object B : A() {
        fun funInB() = 1
    }
    object C : A()
}

fun test(nullable: String?, a: A) {
    nullable!!
    // Ist nullable tatsächlich null, wäre die letzte Zeile gecrasht, daher ist nullable jetzt von Typ String
    println(nullable.length)

    if (a is A.C) {
        // a hat in diesem Scope den Typen A.C
    }

    if (a !is A.B) throw RuntimeException()
    // a hat jetzt Typ A.B
    a.funInB()
}
\end{codeBlock}

%! language = kotlin
\begin{codeBlock}{kotlin}{Demo: Smart-Cast durch Assignment}
fun test() {
    var a: Any? = ""
    a = 4
    a.inc() // .inc() Aufruf funktioniert, obwohl a als Any? deklariert wurde
}
\end{codeBlock}

Mittels Vereinfachung von komplexen Expressions \bzw desugaring baut der Kotlin-Compiler einen Control-Flow-Graph (CFG)
auf, der den Control Flow innerhalb einer Funktion modelliert und es dem Compiler erlaubt basierend auf diesem effizient
Analysen durchzuführen.\zit{kspeCFG}
Das Smart-Cast-System verwendet den CFG um an einer bestimmten Stelle im Programm die möglichen Typen einer Expression
(die sogenannte Smart-Cast Sink) zu ermitteln.
Eine Smart-Cast Sink kann zum Beispiel eine lokale Variable sein, oder eine beliebige Verkettung nicht-mutierbarer
(mit \inlineKotlin{val} deklarierter) Properties.

%! language = kotlin
\begin{codeBlock}{kotlin}{Demo: Mögliche Smart-Cast Sinks}
data class Container<T>(val value: T)
class A {
    val a = Container<Any>("Hello")
    var c: Any = "Hello"

    fun test() {
        // hier ist die smart cast sink a.value, eine Verkettung von zwei nicht-mutierbaren Properties
        a.value as String
        println(a.value.length)

        // c ist eine mutierbare Property (mit var deklariert) und kann daher nicht für Smart Casts verwendet werden
        c as String
        println(c.length) // compile time fehler
    }
}
\end{codeBlock}

Wie in dem vorherigen Code-Snippet ersichtlich, ist es nicht erlaubt eine mutierbare Property für Smart-Casts zu
verwenden.
Das ist der Fall, da sich der Wert der Property durch Code, der nicht im CFG erfasst ist, ändern kann (In diesem Fall
könnte ein anderer Thread den Wert überschreiben).
Solche Smart-Cast Sinks werden als instabil bezeichnet.
Andere instabile Smart-Cast Sinks sind \zB Properties mit custom Gettern, delegierte Properties oder Properties die
in anderen Modulen definiert werden.

%! language = kotlin
\begin{codeBlock}{kotlin}{Demo: Instabile Smart-Cast Sinks}
class A {

    // Pair ist eine Klasse aus Kotlins Standard-Bibliothek, die in einem anderen Modul definiert ist.
    val a: Pair<Any, Any> = "Hello" to "World"
    val b: Any by lazy { "Hello" }
    val c: Any get() = "Hello"

    fun test() {
        a.first as String
        println(a.first.length) // compile time Fehler

        b as String
        println(b.length) // compile time Fehler

        c as String
        println(c.length) // compile time Fehler
    }
}
\end{codeBlock}

In all diesen Fällen kann der Compiler nicht garantieren, das der Wert der Smart-Cast Sink sich nicht durch Code,
der nicht im CFG erfasst ist, ändert.
So enthalten \inlineKotlin{b} und \inlineKotlin{c} zum Beispiel custom Getter, die theoretisch bei jedem Aufruf einen
anderen Wert zurückgeben könnten.\zit{kspecSmartCasts}

% resets author
\renewcommand{\kapitelautor}{}


\section{Lambdas}\label{sec:lambdas}

\renewcommand{\kapitelautor}{Autor: Marvin Kurka}

\subsection{Lambdas in Java und Javascript}
Anders als Kotlin wurde Java ursprünglich ohne Lambdas entwickelt, diese wurden erst in der Version Java 8 hinzugefügt.
Durch diesen Umstand entstand eine wesentliche Limitation der Java Lambdas, nämlich den Fakt das Closures nicht
funktionieren wie man es von anderen Sprachen wie \zB Javascript erwartet.
Generell ist es so, das ein Lambda Werte, die in äußeren Scopes definiert wurden in einer sogenannten Closure captured
und so weiter sowohl lesend als auch schreibend auf diese zugreifen kann.
Java allerdings erlaubt nur den lesenden Zugriff, und nur wenn auch Code außerhalb des Lambdas niemals schreibend auf
die Variable zugreift.\zit{mdnDocsClosures, jarticleLambdas}

\begin{codeBlock}{javascript}{Demo: Closures in Javascript}
function createClosure() {
    let i = 0;
    const getter = () => i;
    const setter = value => { i = value; };
    return [getter, setter];
}
const [getter, setter] = createClosure();
setter(10);
console.log(getter()); // => 10
setter(getter() * 2);
console.log(getter()); // => 20
\end{codeBlock}

\begin{codeBlock}{java}{Demo: Limitierte Closures in Java}
int i = 0;
String effectivelyFinal = "hi";
AtomicInteger a = new AtomicInteger(0);
Runnable r = () -> {
    i++; // compile time fehler
    a.incrementAndGet(); // workaround mit AtomicInteger
    System.out.println(effectivelyFinal); // erlaubt, weil effectivelyFinal nie zugewiesen wird
};
\end{codeBlock}

Kotlin wurde anders als Java von Anfang an mit Lambdas entwickelt und weist daher keine solchen Limitationen auf.
Da Kotlin sich selbst als Sprache mit vielen Elementen aus der funktionalen Programmierung bezeichnet, ist es wichtig,
dass Lambdas und Closures so funktionieren wie erwartet und gut in die Sprache integriert sind.\zit{kspecIntroduction}

\subsection{Lambda Syntax in Kotlin}
Ein Aspekt der Kotlins Lambdas von Lambda-Implementation in anderen Sprachen besonders unterscheidet ist der Syntax.
Eine geschwungene Klammer, die frei im Code auftaucht, eröffnet nicht wie anderen Sprachen wie \zB Java oder Javascript
einen neuen Code-Block, sondern deklariert ein Lambda.
Lambda Parameter werden in den geschwungenen Klammern angegeben.\zit{kspecLambdaLiteral}

%! language = kotlin
\begin{codeBlock}{kotlin}{Demo: Lambda Syntax in Kotlin}
fun test() {
    {
        println("hi")
    }
}
\end{codeBlock}

Das hier gezeigte \inlineKotlin{println()} wird nie tatsächlich ausgeführt, das es in einem Lambda
definiert wurde, das selbst nie ausgeführt wird.

%! language = kotlin
\begin{codeBlock}{kotlin}{Demo: Lambda mit Parameter}
fun test() {
    val lambda: (String) -> Unit = { string -> println(string) }
    lambda("hello")
    lambda("world")
}
\end{codeBlock}

Zusätzlich stellt Kotlin einige syntaktische Vereinfachen zur Verfügung, wenn eine Funktion als letzten Parameter
ein Lambda nimmt.
In diesem Fall kann das Lambda nämlich außerhalb der Klammern des Funktionsaufrufs geschrieben werden.

%! language = kotlin
\begin{codeBlock}{kotlin}{Demo: Funktionsaufruf mit Lambda}
fun test() {
    // repeat ist keine Control-Flow-Struktur, sondern eine normale Funktion, die in der
    // Standard-Bibliothek definiert ist.
    repeat(5) {
        println("hi")
    }
}
\end{codeBlock}

Wenn die Funktion sonst keine Parameter nimmt, können die Klammern sogar komplett weggelassen werden.

%! language = kotlin
\begin{codeBlock}{kotlin}{Demo: Syntaktischer Zucker bei Funktionsaufrufen mit Lambdas}
fun test() {
    run {
    }
    // ist gleich
    run() {
    }
    // ist gleich
    run({ })
}
\end{codeBlock}

\inlineKotlin{run} ist eine Funktion aus der Standard-Bibliothek, die ein Lambda entgegennimmt und einmal ausführt.
Diese Funktion wird in der Praxis verwendet, um normale Code-Blöcke zu ersetzen.

Achtung!
Ist eine geschwungene Klammer Teil einer syntaktischen Struktur eröffnet sie einen normalen Code-Block, kein Lambda!

%! language = kotlin
\begin{codeBlock}{kotlin}{Demo: Code-Blöcke in Control-Flow Strukturen}
fun test() {
    if (someCondition) { // Diese geschwungene Klammer eröffnet kein Lambda!
    }
}
\end{codeBlock}

Eine weitere Vereinfachung ist, dass ein Lambda, das nur einen Parameter nimmt, diesen nicht explizit angeben muss.
Stattdessen wird implizit eine lokale Variable namens \inlineKotlin{it} erschaffen.
Das erleichtert vor allem lange Aufrufketten auf Iterables.

%! language = kotlin
\begin{codeBlock}{kotlin}{Beispiel: Anwendung von Lambdas mit Listen-Funktionen}
data class Country(val name: String, val population: Int)
fun test() {
    val countries = listOf<Country>()
    // erstellt einen String mit den Anfangsbuchstaben aller Länder mit mehr als 500 Einwohnern die mit "land" enden
    val result = countries
        .filter { it.population > 500 }
        .map { it.name }
        .filter { it.endsWith("land") }
        .map { it.first() }
        .joinToString()
}
\end{codeBlock}

Wie im letzten Beispiel ersichtlich, ist es nicht notwendig explizit eine \inlineKotlin{return} expression in einem
Lambda zu verwenden.
Stattdessen wird der Wert der letzten Expression im Block automatisch als return-Wert angenommen.

% resets author
\renewcommand{\kapitelautor}{}


\section{Nothing Typ}\label{sec:nothing-type}

\renewcommand{\kapitelautor}{Autor: Marvin Kurka}

Der \inlineKotlin{kotlin.Nothing} Typ ist in der Kotlin Standard-Bibliothek definiert, er hat allerdings einige
spezielle Eigenschaften.
Zum Beispiel handelt sich bei Nothing um einen unified subtype (auch bottom Type genannt) für alle in Kotlin
definierten Typen.
Das heißt, eine Instanz von Nothing wäre auch eine Instanz jeder anderen Klasse, sei es \inlineKotlin{String},
\inlineKotlin{Int} oder \inlineKotlin{() -> Boolean}.
Da so ein Objekt einen logischen Widerspruch darstellt, handelt es sich bei Nothing auch um einen uninhabited Type.
Das heißt, dass niemals eine tatsächliche Instanz von Nothing existieren kann.

Auf den ersten Blick mag so ein Typ nicht sehr nützlich wirken, er erlaubt den Compiler aber einige nützliche
Annahmen über den Control Flow zu machen.
Sollte jemals eine Variable den Wert Nothing annehmen, was einen logischen Widerspruch darstellt, weiß der Compiler,
dass dieser Code niemals erreicht werden kann.\cite{kspeCFG,kspecNothing}

%! language = kotlin
\begin{minted}{kotlin}
fun getNothing(): Nothing {
    ...
}

fun test() {
    // Nothing ist ein Subtype von String, daher ist der folgende Code erlaubt
    val s: String = getNothing()

    // für diesen Code gibt der Compiler eine Warning aus, da er niemals erreicht werden kann
    println("Hello World")
}
\end{minted}
\codeblockCaption{
Dadurch das die Funktion \inlineKotlin{getNothing()} den Rückgabewert Nothing hat, der niemals existieren kann,
weiß der Compiler, dass diese Funktion niemals auf normale Art und Weise returnen kann und in einer Exception
resultieren muss.
}

Um den Nothing-Typen tatsächlich nützlich zu machen, sind in Kotlin Statements, die einen Control-Flow Transfer
verursachen eigentlich Expressions mit dem Rückgabewert Nothing.
Beispiele sind: \inlineKotlin{throw}, \inlineKotlin{return}, \inlineKotlin{break} und \inlineKotlin{continue}.
Da diese Expressions einen sofortigen Wechsel des Control-Flows verursachen hat das zur Folge, dass der nachfolgende
Code niemals ausgeführt wird.
Damit können diese Expressions den Wert Nothing haben.\cite{kspeCFG}

%! language = kotlin
\begin{minted}{kotlin}
operator fun Int?.plus(other: Int?): Int? {
    // return hat den Wert Nothing, was ein Subtype von Int ist,
    // daher ist die Verwendung nach dem elvis operator erlaubt
    val first = this ?: return null
    val second = other ?: return null
    return first + second
}
\end{minted}
\codeblockCaption{Praktisches Beispiel für die Verwendung des Nothing-Typen}

Dadurch das \inlineKotlin{return} in Kotlin eine Expression ist, ist auch der folgende Code gültiges Kotlin:

%! language = kotlin
\begin{minted}{kotlin}
fun test() {
    println("Hello")
    return return return return return
}
\end{minted}

Funktionen, die den Rückgabewert Nothing haben, müssen immer eine Exception werfen.

%! language = kotlin
\begin{minted}{kotlin}
fun crashProgram(cause: String): Nothing {
    MyLogger.log("Program crashed because of: $cause")
    throw RuntimeException(cause)
}

fun doSomething(input: String?) {
    otherFunction(input ?: crashProgram("input must not be null"))
}
\end{minted}

Ein Beispiel für eine Funktion aus der Kotlin Standard-Bibliothek die Nothing zurückgibt, ist die
\inlineKotlin{TODO()} Funktion.

%! language = kotlin
\begin{minted}{kotlin}
fun someFunction(): Int = TODO("not yet implemented")
\end{minted}

Ein weiterer Anwendungsfall von Nothing ist als generischer Parameter, um \zB eine leere Liste zu signalisieren.
Eine \inlineKotlin{List<Nothing>} muss immer leer sein, da keine Instanz von Nothing existieren kann.

% resets author
\renewcommand{\kapitelautor}{}


\section{Extension Functions}\label{sec:extension-functions}

\renewcommand{\kapitelautor}{Autor: Marvin Kurka}

Kotlin erlaubt es Funktionen zu definieren, die auf einem Objekt aufgerufen werden können, ohne tatsächlich in der
Klasse dieses Objekts definiert worden zu sein.
Solche Funktionen nennt man Extension Functions.
Sie werden definiert, indem man vor dem Funktionsnamen einen sogenannten Receiver-Typen gefolgt von einem Punkt angibt.
Bei dem Aufruf einer Extension Function wird der Receiver als Parameter mitgegeben, und ist dann in der Funktion als
\inlineKotlin{this} verfügbar.
Das heißt auch, dass Extension Function statisch, also zur compile time, resolved werden.
Es finden also keine virtuellen Lookups statt.\zit{kdocExtensions}

%! language = kotlin
\begin{codeBlock}{kotlin}{Beispiel: Praxisnaher Einsatz einer Extension Function}
fun String.escapeNewLines(): String = this.replace("\n", "\\n")

fun test() {
    val s = "String\nwith\nNewlines"
    println(s.escapeNewLines())
}
\end{codeBlock}

Wenn Receiver-Typen generische Parameter haben, kann die Extension Function entweder selbst einen generischen Parameter
verwenden, oder auch einen konkreten Typen einsetzen.
Das erlaubt es Funktionen zu definieren, die nur verfügbar sind, wenn bestimmte Typen für die generischen Parameter
eingesetzt wurden, was mit normalen member Functions nicht möglich ist.\zit{kdocExtensions}

%! language = kotlin
\begin{codeBlock}{kotlin}{Demo: Extension Function mit konkreten Typen als generischen Parameter im Receiver}
fun Iterable<Int>.evenNumbers(): List<Int> = this.filter { it % 2 == 0 }
\end{codeBlock}

Wichtig zu erwähnen ist auch, dass Encapsulation nicht durch Extension Functions gebrochen wird.
Das heißt, dass Extension Functions nicht auf private Funktionen/Properties des Receivers zugreifen können.
So ein Zugriff wäre spätestens auf dem Level der JVM verboten, da Extension Functions nicht in der Klasse des Receivers
definiert sind.

Nicht nur Funktionen können Receiver angeben, sondern auch Lambdas.
Auch hier wird der Receiver-Typ mit einem Punkt vor dem Lambda-Typ geschrieben, und der Receiver ist als
\inlineKotlin{this} im Lambda verwendbar.\zit{kdocLambdas}

%! language = kotlin
\begin{codeBlock}{kotlin}{Demo: Lambda mit Receiver}
fun <T> createList(listInitializer: MutableList<T>.() -> Unit): List<T> =
    mutableListOf<T>().apply { listInitializer(this) }

fun test() {
    val list = createList<String> {
        // Aufrufe werden implizit auf 'this' ausgeführt
        add("Hi")
        add("Hello")
        add("World")
        removeFirst()
    }
}
\end{codeBlock}

% resets author
\renewcommand{\kapitelautor}{}


\section{Inline Functions}\label{sec:inline-functions}

\renewcommand{\kapitelautor}{Autor: Marvin Kurka}

\subsection{inline Funktionen in anderen Sprachen}
In C/C++ werden von den Compilern häufig kurze Funktionen geinlined.
Das heißt, der Funktionsaufruf wird mit dem Body der Funktion ersetzt, und kein tatsächlicher
Funktionsaufruf findet statt.
C und C++ haben den \inlineCode{inline} Modifier, der als Hinweis für den Compiler dient, dass diese Funktion
geinlined werden soll.
Der Compiler darf allerdings selber entscheiden, ob er das tut \bzw auch nicht mit \inlineCode{inline} markierte
Funktionen inlinen.
Der Grund für Funktions-inlining ist in erster Linie Performance, da dadurch ein Funktionsaufruf gespart wird.\cite{crefInline}

In Kotlin ist der Grund für Funktions-inlining explizit nicht um den Overhead eines Funktionsaufrufs zu sparen,
der Compiler warnt sogar, wenn der \inlineKotlin{inline} Modifier verwendet wird und kein sonstiger Grund für
inlining besteht.\cite{kdocInline}

\subsection{Effizientere Lambdas}
Dieser Abschnitt hat nur für Kotlin/JVM Gültigkeit.
Auf anderen Plattformen wie native, js oder wasm könnte sich der Compiler anders verhalten.

Eine Anwendung für inline-Funktionen sind higher-order-functions, also Funktionen, die andere Funktionen als Parameter
nehmen.
Im Normalfall muss der Compiler für das Lambda eine anonyme Klasse generieren, die da Lambda enthält, und alle Variablen,
die Teil der Closure sind, speichern.
Wenn eine solche Funktion als \inlineKotlin{inline} markiert ist, kann der Compiler nicht nur die Funktion, sondern auch
das Lambda inlinen, was diesen Overhead spart.\cite{kdocInline}

Beispiel für higher-order inline Funktionen:

%! language = kotlin
\begin{codeBlock}{kotlin}{Demo: Inline Funktion vs. Normale Funktion}
fun runLambda(lambda: () -> Unit) = lambda()
inline fun runLambdaInline(lambda: () -> Unit) = lambda()

fun test() {
    runLambda {
        println("runLambda")
    }
    runLambdaInline {
        println("runLambdaInline")
    }
}
\end{codeBlock}

Dieser Code wurde mit kotlinc-jvm 1.9.21 compiled und mit javap 14.0.2 disassembled.
Unten ist die Disassembly der test-Funktion zu sehen.

\begin{codeBlock}{text}{Disassembly der test-Funktion}
 0: getstatic     #34                 // Field TestKt$test$1.INSTANCE:LTestKt$test$1;
 3: checkcast     #18                 // class kotlin/jvm/functions/Function0
 6: invokestatic  #36                 // Method runLambda:(Lkotlin/jvm/functions/Function0;)V
 9: iconst_0
10: istore_0
11: iconst_0
12: istore_1
13: ldc           #37                 // String runLambdaInline
15: getstatic     #43                 // Field java/lang/System.out:Ljava/io/PrintStream;
18: swap
19: invokevirtual #49                 // Method java/io/PrintStream.println:(Ljava/lang/Object;)V
22: nop
23: nop
24: return
\end{codeBlock}

Wie in der Disassembly zu sehen ist, hat der Compiler für das erste Lambda (Offset 0--8) eine eigene Klasse
(TestKt\$test\$1) erstellt.
Da das Lambda in diesem Fall keine Variablen in einer Closure speichert, kann der Compiler ein Singleton generieren,
was zumindest den Konstruktor-Aufruf spart.
Wie von Offset 13--21 zu sehen ist, wird das \inlineKotlin{println} der zweiten Funktion direkt in der test-Funktion
aufgerufen, da das Lambda geinlined wurde.

\begin{infoBox}
In der Disassembly sind auch mehrere merkwürdige Instruktionen zu sehen, \zB von Offset 9--12 oder von 22--23.
Diese sind Anhang~\ref{ch:anhang-1} erklärt.
\end{infoBox}

\subsection{reified Generics}

Während Sprachen wie Java oder Kotlin Support für Generics haben, können auf der JVM nur konkrete Klassen oder
Interfaces als Typ angegeben werden.
Deshalb kommt es bei der Kompilation zur "Type-Erasure", bei der generische Typen mit konkreten Typen, in den meisten
Fällen \inlineKotlin{Object}, ersetzt werden.
Durch diesen Prozess geht Runtime-Information zu generischen Typen verloren, was \zB dazu führt, dass Casts zu
generischen Typen in Java nicht funktionieren wie erwartet.\cite{jdocTypeErasure}

\begin{codeBlock}{java}{Demo: Type-Erasure in Java}
public static <T> T test(Object arg) {
    boolean isInstance = arg instanceof T; // Compile-time Fehler
    return (T) arg; // Warning: Unchecked Cast
}
\end{codeBlock}

Der Compiler erlaubt den Cast zwar, kann ihn aber nicht tatsächlich durchführen, \dah die Funktion könnte einen Wert
zurückgeben, der nicht von Typ \inlineJava{T} ist.

In Kotlin bieten inline-Funktionen einen Workaround für dieses Problem.
Da der Typ an der Call-site vorhanden ist, kann eine inline Funktion einen generischen Parameter als
\inlineKotlin{reified} deklarieren, was Casts/is-checks möglich macht.\cite{kdocInline}

%! language = kotlin
\begin{codeBlock}{kotlin}{Demo: reified Generics}
inline fun <reified T> Iterable<Any>.findInstance(): T? = this.find { it is T } as T?
\end{codeBlock}

% resets author
\renewcommand{\kapitelautor}{}


\section{Kotlin vs. Java}\label{sec:kotlin-vs-java}

\renewcommand{\kapitelautor}{Autor: Marvin Kurka}

Die Wahl der Programmiersprache wurde bereits vor Beginn der Diplomarbeit, am Anfang des Projektes, getroffen.
Da das Projektteam als Framework LibGdx verwenden wollte, musste die Sprache JVM-basiert sein.
Die offensichtliche Wahl wäre Java gewesen -- mit dieser Sprache hatten alle Projektmitglieder Erfahrung und es hätte
keine Notwendigkeit gegeben, eine neue Sprache zu lernen.
Dennoch fiel die Wahl auf Kotlin, da angenommen wurde, dass die durch den Lernaufwand verlorene Zeit in der Entwicklung
wieder aufgeholt werden würde.
Ob sich diese Annahme bewahrheitet hat wissen wir nicht.
Da niemand in dem Team ein Spiel in vergleichbarer Größe entwickelt hat, fehlen die Vergleichswerte.
Allerdings nehmen wir aufgrund des moderneren Syntax, Erfahrungsberichten von \zB Google, und den zusätzlichen Features,
die in diesem Kapitel ausführlich beschrieben wurden, an, die richtige Entscheidung getroffen zu haben.

% resets author
\renewcommand{\kapitelautor}{}

