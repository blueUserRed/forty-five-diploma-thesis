
\section{Flow Typing}\label{sec:flow-typing}

\renewcommand{\kapitelautor}{Autor: Marvin Kurka}

Flow-Typing bedeutet, dass der Compiler Annahmen über den Typen von Variablen machen kann, basierend auf dem Control
Flow von dem Programm.

%! language = kotlin
\begin{codeBlock}{kotlin}{Beispiel für Flow-Typing in Kotlin}
fun test(s: Any?): Unit = when (s) {

    null -> println("s is null")

    // s hat hier den Typ String
    is String -> println("s is a string with length ${s.length}")

    // s hat hier den Typ List<*>
    is List<*> -> println("s is a list with ${s.distinct().size} distinct items")

    // s hat durch den anfänglichen null-Check hier den Typ Any
    else -> println("nothing matched with ${s::class.simpleName}")
}
\end{codeBlock}

Kotlin implementiert Flow-Typing mittels Smart-Casts, die automatisch in den Code eingefügt werden, wenn gewisse
Control-Flow Instruktionen verwendet werden.
Das inkludiert \zB whens, ifs, not-null-assertions, casts, is-checks, \usw

%! language = kotlin
\begin{codeBlock}{kotlin}{Mehr Beispiele für die Anwendung von Flow-Typing}
sealed class A {
    object B : A() {
        fun funInB() = 1
    }
    object C : A()
}

fun test(nullable: String?, a: A) {
    nullable!!
    // Ist nullable tatsächlich null, wäre die letzte Zeile gecrasht, daher ist nullable jetzt von Typ String
    println(nullable.length)

    if (a is A.C) {
        // a hat in diesem Scope den Typen A.C
    }

    if (a !is A.B) throw RuntimeException()
    // a hat jetzt Typ A.B
    a.funInB()
}
\end{codeBlock}

%! language = kotlin
\begin{codeBlock}{kotlin}{Demo: Smart-Cast durch Assignment}
fun test() {
    var a: Any? = ""
    a = 4
    a.inc() // .inc() Aufruf funktioniert, obwohl a als Any? deklariert wurde
}
\end{codeBlock}

Mittels Vereinfachung von komplexen Expressions \bzw desugaring baut der Kotlin-Compiler einen Control-Flow-Graph (CFG)
auf, der den Control Flow innerhalb einer Funktion modelliert und es dem Compiler erlaubt basierend auf diesem effizient
Analysen durchzuführen.\cite{kspeCFG}
Das Smart-Cast-System verwendet den CFG um an einer bestimmten Stelle im Programm die möglichen Typen einer Expression
(die sogenannte Smart-Cast Sink) zu ermitteln.
Eine Smart-Cast Sink kann zum Beispiel eine lokale Variable sein, oder eine beliebige Verkettung nicht-mutierbarer
(mit \inlineKotlin{val} deklarierter) Properties.

%! language = kotlin
\begin{codeBlock}{kotlin}{Demo: Mögliche Smart-Cast Sinks}
data class Container<T>(val value: T)
class A {

    val a = Container<Any>("Hello")

    var c: Any = "Hello"

    fun test() {
        // hier ist die smart cast sink a.value, eine Verkettung von zwei nicht-mutierbaren Properties
        a.value as String
        println(a.value.length)

        // c ist eine mutierbare Property (mit var deklariert) und kann daher nicht für Smart Casts verwendet werden
        c as String
        println(c.length) // compile time fehler
    }
}
\end{codeBlock}

Wie in dem vorherigen Code-Snippet ersichtlich, ist es nicht erlaubt eine mutierbare Property für Smart-Casts zu
verwenden.
Das ist der Fall, da sich der Wert der Property durch Code, der nicht im CFG erfasst ist, ändern kann (In diesem Fall
könnte ein anderer Thread den Wert überschreiben).
Solche Smart-Cast Sinks werden als instabil bezeichnet.
Andere instabile Smart-Cast Sinks sind \zB Properties mit custom Gettern, delegierte Properties oder Properties die
in anderen Modulen definiert werden.

%! language = kotlin
\begin{codeBlock}{kotlin}{Demo: Instabile Smart-Cast Sinks}
class A {

    // Pair ist eine Klasse aus Kotlins Standard-Bibliothek, die in einem anderen Modul definiert ist.
    val a: Pair<Any, Any> = "Hello" to "World"

    val b: Any by lazy {
        "Hello"
    }

    val c: Any
        get() = "Hello"

    fun test() {
        a.first as String
        println(a.first.length) // compile time Fehler

        b as String
        println(b.length) // compile time Fehler

        c as String
        println(c.length) // compile time Fehler
    }

}
\end{codeBlock}

In all diesen Fällen kann der Compiler nicht garantieren, das der Wert der Smart-Cast Sink sich nicht durch Code,
der nicht im CFG erfasst ist, ändert.
So enthalten \inlineKotlin{b} und \inlineKotlin{c} zum Beispiel custom Getter, die theoretisch bei jedem Aufruf einen
anderen Wert zurückgeben könnten.\cite{kspecSmartCasts}

% resets author
\renewcommand{\kapitelautor}{}
