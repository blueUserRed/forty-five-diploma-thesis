
\section{Inline Functions}\label{sec:inline-functions}

\renewcommand{\kapitelautor}{Autor: Marvin Kurka}

\subsection{inline Funktionen in anderen Sprachen}
In C/C++ werden von den Compilern häufig kurze Funktionen geinlined.
Das heißt, der Funktionsaufruf wird mit dem Body der Funktion ersetzt, und kein tatsächlicher
Funktionsaufruf findet statt.
C und C++ haben den \inlineCode{inline} Modifier, der als Hinweis für den Compiler dient, dass diese Funktion
geinlined werden soll.
Der Compiler darf allerdings selber entscheiden, ob er das tut \bzw auch nicht mit \inlineCode{inline} markierte
Funktionen inlinen.
Der Grund für Funktions-inlining ist in erster Linie Performance, da dadurch ein Funktionsaufruf gespart wird.\zit{crefInline}

In Kotlin ist der Grund für Funktions-inlining explizit nicht um den Overhead eines Funktionsaufrufs zu sparen,
der Compiler warnt sogar, wenn der \inlineKotlin{inline} Modifier verwendet wird und kein sonstiger Grund für
inlining besteht.\zit{kdocInline}

\subsection{Effizientere Lambdas}
Dieser Abschnitt hat nur für Kotlin/JVM Gültigkeit.
Auf anderen Plattformen wie native, js oder wasm könnte sich der Compiler anders verhalten.

Eine Anwendung für inline-Funktionen sind higher-order-functions, also Funktionen, die andere Funktionen als Parameter
nehmen.
Im Normalfall muss der Compiler für das Lambda eine anonyme Klasse generieren, die da Lambda enthält, und alle Variablen,
die Teil der Closure sind, speichern.
Wenn eine solche Funktion als \inlineKotlin{inline} markiert ist, kann der Compiler nicht nur die Funktion, sondern auch
das Lambda inlinen, was diesen Overhead spart.\zit{kdocInline}

Beispiel für higher-order inline Funktionen:

%! language = kotlin
\begin{codeBlock}{kotlin}{Demo: Inline Funktion vs. Normale Funktion}
fun runLambda(lambda: () -> Unit) = lambda()
inline fun runLambdaInline(lambda: () -> Unit) = lambda()

fun test() {
    runLambda {
        println("runLambda")
    }
    runLambdaInline {
        println("runLambdaInline")
    }
}
\end{codeBlock}

Dieser Code wurde mit kotlinc-jvm 1.9.21 compiled und mit javap 14.0.2 disassembled.
Unten ist die Disassembly der test-Funktion zu sehen.

\begin{codeBlock}{text}{Disassembly der test-Funktion}
 0: getstatic     #34                 // Field TestKt$test$1.INSTANCE:LTestKt$test$1;
 3: checkcast     #18                 // class kotlin/jvm/functions/Function0
 6: invokestatic  #36                 // Method runLambda:(Lkotlin/jvm/functions/Function0;)V
 9: iconst_0
10: istore_0
11: iconst_0
12: istore_1
13: ldc           #37                 // String runLambdaInline
15: getstatic     #43                 // Field java/lang/System.out:Ljava/io/PrintStream;
18: swap
19: invokevirtual #49                 // Method java/io/PrintStream.println:(Ljava/lang/Object;)V
22: nop
23: nop
24: return
\end{codeBlock}

Wie in der Disassembly zu sehen ist, hat der Compiler für das erste Lambda (Offset 0--8) eine eigene Klasse
(TestKt\$test\$1) erstellt.
Da das Lambda in diesem Fall keine Variablen in einer Closure speichert, kann der Compiler ein Singleton generieren,
was zumindest den Konstruktor-Aufruf spart.
Wie von Offset 13--21 zu sehen ist, wird das \inlineKotlin{println} der zweiten Funktion direkt in der test-Funktion
aufgerufen, da das Lambda geinlined wurde.

\begin{infoBox}
In der Disassembly sind auch mehrere merkwürdige Instruktionen zu sehen, \zB von Offset 9--12 oder von 22--23.
Diese sind Anhang~\ref{ch:anhang-1} erklärt.
\end{infoBox}

\subsection{reified Generics}

Während Sprachen wie Java oder Kotlin Support für Generics haben, können auf der JVM nur konkrete Klassen oder
Interfaces als Typ angegeben werden.
Deshalb kommt es bei der Kompilation zur \quoted{Type-Erasure}, bei der generische Typen mit konkreten Typen, in den meisten
Fällen \inlineKotlin{Object}, ersetzt werden.
Durch diesen Prozess geht Runtime-Information zu generischen Typen verloren, was \zB dazu führt, dass Casts zu
generischen Typen in Java nicht funktionieren wie erwartet.\zit{jdocTypeErasure}

\begin{codeBlock}{java}{Demo: Type-Erasure in Java}
public static <T> T test(Object arg) {
    boolean isInstance = arg instanceof T; // Compile-time Fehler
    return (T) arg; // Warning: Unchecked Cast
}
\end{codeBlock}

Der Compiler erlaubt den Cast zwar, kann ihn aber nicht tatsächlich durchführen, \dah die Funktion könnte einen Wert
zurückgeben, der nicht von Typ \inlineJava{T} ist.

In Kotlin bieten inline-Funktionen einen Workaround für dieses Problem.
Da der Typ an der Call-site vorhanden ist, kann eine inline Funktion einen generischen Parameter als
\inlineKotlin{reified} deklarieren, was Casts/is-checks möglich macht.\zit{kdocInline}

%! language = kotlin
\begin{codeBlock}{kotlin}{Demo: reified Generics}
inline fun <reified T> Iterable<Any>.findInstance(): T? = this.find { it is T } as T?
\end{codeBlock}

% resets author
\renewcommand{\kapitelautor}{}
