
\section{Null Safety}\label{sec:null-safety}

\renewcommand{\kapitelautor}{Autor: Marvin Kurka}

% todo: quelle für basic infos zu null
\inlineCode{null}, auch unter den Namen \inlineCode{nil} oder \inlineCode{None} bekannt, wird in Programmiersprachen
verwendet, um einen nicht vorhandenen Wert zu repräsentieren.
Wird ein Zugriff auf \inlineCode{null} ausgeführt, sei es ein Funktionsaufruf oder Property-access, kommt es zu
einem Fehler.

\subsection{Situation in Java}
Ein Beispiel für den Einsatz eines Null-Werts ist die \inlineJava{File::getParent} Funktion, die, unter normalen
Umständen, die übergeordnete Datei zurückgibt.
Allerdings ist das nicht in allen Fällen möglich, und in diesen wird \inlineJava{null} stattdessen als Rückgabewert
verwendet.\cite{jdocFile}
Dieser Umstand ist aber nicht in der Funktionssignatur reflektiert, in dieser ist als Rückgabewert einfach
\inlineJava{String} angegeben.
Für den Programmierer ist es also nicht sofort offensichtlich, dass ein Wert auch null sein kann, ein Blick in die
Dokumentation ist hier notwendig.
Dieser Umstand erhöht nicht nur den Aufwand für den Programmierer, er macht auch eine statische Analyse unmöglich.
So kann ein Compiler oder Linter nicht aus den Typen lesen, ob er \inlineJava{null} sein kann, und so auch in den
meisten Fällen nicht vor wahrscheinlichen Programmierfehlern warnen.
Außerdem haben \inlineJava{null}-Werte die Eigenschaft durch das Programm zu propagieren, entsteht also ein
\inlineJava{null}-Wert am Beginn einer langen Funktionsaufrufskette, kann es zu Situationen kommen in der das Programm
an einer Stelle crasht, die mit dem eigentlichen Fehler nichts zu tun hat.
Aus diesen Gründen ist \inlineJava{null} eine häufige Fehlerursache in Java, laut Google sind
NullPointerExceptions sogar die häufigste Crash-Ursache von Android-Apps im Google Play Store.\cite{androidDevNPE}

\subsection{Lösungsansätze}
Wie im letzten Absatz argumentiert, ist \inlineJava{null}, wie es in Sprachen wie Java implementiert ist, schlechtes
Programmiersprachendesign.
So hat auch Tony Hoare, der im Jahr 1965 zuerst das Konzept eines \inlineJava{null}-Pointers in die Sprache ALGOL W
eingeführt hat, diese Entscheidung in einer berühmten Präsentation als seinen "Milliarden Dollar Fehler" bezeichnet.\cite{infoqNullRefs}
Deswegen haben über die Jahre Designer von Programmiersprachen nach Wegen gesucht, optionale Werte auf eine sinnvollere
und weniger fehleranfällige Weise zu repräsentieren.
So auch in der Sprache Java, die 2014 um die \inlineJava{Optional<T>} Klasse erweitert wurde.\cite{jdocOptional}
Diese kann als Wrapper um Werte verwendet werden und stellt Funktionen zur Verfügung, die auf Präsenz eines Wertes
testen und/oder abhängig davon Code ausführen.

\begin{minted}{java}
getOptionalValue().ifPresentOrElse(
    value -> System.out.println("result was: " + value),
    () -> System.out.println("no value")
);
\end{minted}
\codeblockCaption{Beispiel für die Verwendung von \inlineJava{Optional<T>}}

Trotz der höheren Sicherheit findet \inlineJava{Optional<T>} wenig Verwendung, da, durch die nachträgliche und relative
späte Einführung in die Sprache große Teile der Standard-Bibliothek selbst keine Optionals verwenden (Siehe
File-Beispiel).
Außerdem ist das Nachziehen von Optional in existierenden Code-Bases mit viel Arbeit verbunden, da sämtliche
Funktionssignaturen und die Usages der Funktionen angepasst werden müssten.
All das führt dazu, dass ein anderer Weg \inlineJava{null}-Werte zu markieren eine wesentliche höhere Verbreitung
erzielt hat.

Diese Alternative sind Annotations.\label{java-null-annotations}
Annotations können von dem Programmierer verwendet werden, um Metadaten in den Code einzufügen, so auch ob ein Wert null
sein kann, mittels der \inlineJava{@NotNull} und \inlineJava{@Nullable} Annotations.
Wichtig anzumerken ist, dass diese Annotations nichts an der tatsächlichen Funktion des Codes ändern und auch keine
Compiler-Fehler versuchen, sollten sie nicht respektiert werden.
Allerdings können sie von statischen Analyse-Tools verwendet werden, um Warnings bei verdächtigem Code auszugeben.
Der Fakt, dass diese Annotations den Code nicht beeinflussen, macht es einfacher diese in bereits existierende Projekte
einzubinden, auch wenn mit ihnen gegenüber Optionals einiges an Sicherheit verloren geht.
Da die Java-Standardbibliothek diese Annotations nicht enthält, sind mehrere konkurrierende Bibliotheken entstanden,
die diese zur Verfügung stellen.
Zum Beispiel:
\begin{itemize}
    \item Jetbrains Annotations \url{https://central.sonatype.com/artifact/org.jetbrains/annotations/24.0.1}
    \item The Checker Framework \url{https://checkerframework.org/}
    \item Eclipse \url{https://help.eclipse.org/latest/index.jsp?topic=%2Forg.eclipse.jdt.doc.user%2Ftasks%2Ftask-using_null_type_annotations.htm}
\end{itemize}

Die Programmiersprache Rust, die dafür bekannt ist, hohen Wert auf Sicherheit und fehlerfreien Code zu legen, erlaubt
gar keine \inlineJava{null}-Werte.
Stattdessen bietet sie ein \inlineCode{Option<T>} enum an, dass zwei mögliche Werte hat:
\inlineCode{None} und \inlineCode{Some(T)}.
In vielerlei Hinsicht ist diese Lösung ähnlich zu Javas Optionals, da Rust allerdings keine klassischen
\inlineJava{null}-Werte erlaubt und die Standard-Bibliothek mit diesem Enum im Hinterkopf designet wurde, funktioniert
diese Lösung in Rust besser als in Java.\cite{rustDocOptional}

\subsection{Nulls in Kotlin}
In Kotlin müssen Typen, die den Wert \inlineKotlin{null} annehmen dürfen, explizit als nullable markiert werden.
Das wird mittels eines Fragezeichens hinter dem Typen umgesetzt.
Der Compiler analysiert zur compile-time welche Werte null sein könnten und gibt einen Fehler aus, wenn mit
\inlineKotlin{null} nicht richtig umgegangen wird.\cite{kdocNullSafety}

%! language = kotlin
\begin{minted}{kotlin}
    var a = "Hello"
    a = null // Compile-time Fehler

    var b: String? = "Hello"
    b = null // erlaubt
\end{minted}
\codeblockCaption{Variable a hat implizit den Typen \inlineKotlin{String}, der kein null erlaubt.}

%! language = kotlin
\begin{minted}{kotlin}
    fun someFunction(string: String?) {
        if (string != null) println(string.length) // erlaubt
        println(string.length) // Compile-time Fehler
    }
\end{minted}
\codeblockCaption{
Bevor Funktionen auf einer Variable, die \inlineKotlin{null} sein könnte aufgerufen werden, ist ein Check erforderlich.
}

Diese Schutzmechanismen zwingen den Programmierer mit \inlineKotlin{null}-Werten sinnvoll umzugehen und helfen so dabei
Code zu schreiben, der weniger fehleranfällig ist.
Laut Google war das Team von Google Home durch den (teilweisen) Umstieg auf Kotlin in der Lage, die Zahl ihrer
NullPointerExceptions um 33\% Prozent zu reduzieren.\cite{androidDevGoogleHome}

Neben dem statischen Überprüfen auf mögliche \inlineKotlin{null}-Werte ist es auch wichtig, den notwendigen
Syntax und/oder die notwendigen Funktionen zur Verfügung zu stellen, die gebraucht werden, um mit ihnen sinnvoll
umzugehen.
Während if-Bedingungen für \inlineKotlin{null} Checks theoretisch ausreichen, führen sie zu langen und schlecht lesbaren
Code, weswegen Kotlin eigene Operatoren für \inlineKotlin{null}-Werte hat.\cite{kdocNullSafety}

%! language = kotlin
\begin{minted}{kotlin}
// Der safe call operator führt einen Zugriff nur aus, wenn der Wert nicht null ist
// Ist der Wert null, gibt auch der operator null zurück
nullableVariable?.nullableProperty?.someFunction()

// Der not null assertion operator wirft eine Exception, wenn auf einem null-Wert angewendet wird
nullableVariable!!.nullableProperty!!.someFunction() // kann NullPointerException werfen

// Der elvis operator überprüft, ob ein Wert null ist, und verwendet dann einen anderen Wert stattdessen
val result = nullableVariable?.nullableProperty?.someFunction() ?: "default Value"
\end{minted}
\codeblockCaption{Demonstration vom Umgang mit \inlineKotlin{null}-Werten}

\subsection{Interoperabilität mit Java}
Ein wichtiger Aspekt von Kotlin ist die gute Kompatibilität mit Java.
Diese erlaubt es Entwicklern Code stückweise von Java zu Kotlin zu migrieren und erlaubt auch die Verwendung von
Java-Bibliotheken in Kotlin-Projekten.
Da Java allerdings keine \inlineKotlin{null}-safe Sprache ist, stellt sich die Frage wie mit Rückgabewerten von
Java-Funktionen umgegangen wird, da an der Funktionssignatur nicht erkannt werden kann, ob dieser Wert
\inlineKotlin{null} sein kann.

In frühen Kotlin Versionen wurde dieses Problem so umgangen, dass der Kotlin-Compiler alle von Java kommenden Typen als
nullable interpretiert hat.
Diese Herangehensweise hat sich allerdings als nicht praktikabel herausgestellt, da absurde Mengen an \inlineKotlin{null}
Checks dadurch notwendig wurden.
Neue Kotlin Versionen suchen nach den vorher erwähnten Annotations und verwenden diese um herauszufinden, ob ein Type
nullable ist oder nicht.
Wird keine solche Annotation gefunden, verwendet Kotlin einen sogenannten Platform Type stattdessen.\cite{kdocNullInterop,kblogNullSafety}

Platform types werden mit einem Rufzeichen nach dem Typen gekennzeichnet: \inlineKotlin{String!}.
Diese Typen können nicht direkt definiert werden, sondern entstehen nur bei Interaktion mit Java-Code, wenn der Compiler
nicht aus Annotations schließen kann, ob der Typ nullable ist oder nicht.
Kotlin schreibt bei Platform Types keine \inlineKotlin{null} Checks vor, gibt aber auch keine Warnings aus, wenn
welche gemacht werden.
Wird ein Platform Typ einer Variable zugewiesen sollte explizit ein Typ angegeben werden, um die nullability zu
definieren.
Wird das nicht gemacht, gibt der Compiler eine Warning aus.\cite{kdocNullInterop}

% resets author
\renewcommand{\kapitelautor}{}
