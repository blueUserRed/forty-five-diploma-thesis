
\section{Lambdas}\label{sec:lambdas}

\renewcommand{\kapitelautor}{Autor: Marvin Kurka}

\subsection{Lambdas in Java und Javascript}
Anders als Kotlin wurde Java ursprünglich ohne Lambdas entwickelt, diese wurden erst in der Version Java 8 hinzugefügt.
Durch diesen Umstand entstand eine wesentliche Limitation der Java Lambdas, nämlich den Fakt das Closures nicht
funktionieren wie man es von anderen Sprachen wie \zB Javascript erwartet.
Generell ist es so, das ein Lambda Werte, die in äußeren Scopes definiert wurden in einer sogenannten Closure captured
und so weiter sowohl lesend als auch schreibend auf diese zugreifen kann.
Java allerdings erlaubt nur den lesenden Zugriff, und nur wenn auch Code außerhalb des Lambdas niemals schreibend auf
die Variable zugreift.\cite{mdnDocsClosures, jarticleLambdas}

\begin{minted}{javascript}
function createClosure() {
    let i = 0;
    const getter = () => i;
    const setter = value => { i = value; };
    return [getter, setter];
}
const [getter, setter] = createClosure();
setter(10);
console.log(getter()); // => 10
setter(getter() * 2);
console.log(getter()); // => 20
\end{minted}
\codeblockCaption{Demonstration von closures in Javascript}

\begin{minted}{java}
int i = 0;
String effectivelyFinal = "hi";
AtomicInteger a = new AtomicInteger(0);
Runnable r = () -> {
    i++; // compile time fehler
    a.incrementAndGet(); // workaround mit AtomicInteger, funktioniert
    System.out.println(effectivelyFinal); // erlaubt, weil effectivelyFinal nie zugewiesen wird
};
\end{minted}
\codeblockCaption{Demonstration von limitierten closures in Java}

Kotlin wurde anders als Java von Anfang an mit Lambdas entwickelt und weist daher keine solchen Limitationen auf.
Da Kotlin sich selbst als Sprache mit vielen Elementen aus der funktionalen Programmierung bezeichnet, ist es wichtig,
dass Lambdas und Closures so funktionieren wie erwartet und gut in die Sprache integriert sind.\cite{kspecIntroduction}

\subsection{Lambda Syntax in Kotlin}
Ein Aspekt der Kotlins Lambdas von Lambda-Implementation in anderen Sprachen besonders unterscheidet ist der Syntax.
Eine geschwungene Klammer, die frei im Code auftaucht, eröffnet nicht wie anderen Sprachen wie \zB Java oder Javascript
einen neuen Code-Block, sondern deklariert ein Lambda.
Lambda Parameter werden in den geschwungenen Klammern angegeben.\cite{kspecLambdaLiteral}

%! language = kotlin
\begin{minted}{kotlin}
fun test() {
    {
        println("hi")
    }
}
\end{minted}
\codeblockCaption{
Das hier gezeigte \inlineKotlin{println()} wird nie tatsächlich ausgeführt, das es in einem Lambda
definiert wurde, das selbst nie ausgeführt wird
}

%! language = kotlin
\begin{minted}{kotlin}
fun test() {
    val lambda: (String) -> Unit = { string -> println(string) }
    lambda("hello")
    lambda("world")
}
\end{minted}
\codeblockCaption{Lambda-Literal mit einem Parameter}

Zusätzlich stellt Kotlin einige syntaktische Vereinfachen zur Verfügung, wenn eine Funktion als letzten Parameter
ein Lambda nimmt.
In diesem Fall kann das Lambda nämlich außerhalb der Klammern des Funktionsaufrufs geschrieben werden.

%! language = kotlin
\begin{minted}{kotlin}
fun test() {
    // repeat ist keine Control-Flow-Struktur, sondern eine normale Funktion, die in der
    // Standard-Bibliothek definiert ist.
    repeat(5) {
        println("hi")
    }
}
\end{minted}
\codeblockCaption{Demonstration von einem Funktionsaufruf mit Lambda}

Wenn die Funktion sonst keine Parameter nimmt, können die Klammern sogar komplett weggelassen werden.

%! language = kotlin
\begin{minted}{kotlin}
fun test() {
    run {
    }
    // ist gleich
    run() {
    }
    // ist gleich
    run({ })
}
\end{minted}

\inlineKotlin{run} ist eine Funktion aus der Standard-Bibliothek, die ein Lambda entgegennimmt und einmal ausführt.
Diese Funktion wird in der Praxis verwendet, um normal Code-Blöcke zu ersetzen.

Achtung!
Ist eine geschwungene Klammer Teil einer syntaktischen Struktur eröffnet sie einen normalen Code-Block, kein Lambda!

%! language = kotlin
\begin{minted}{kotlin}
fun test() {
    if (someCondition) { // Diese geschwungene Klammer eröffnet kein Lambda!
    }
}
\end{minted}

Eine weitere Vereinfachung ist, dass ein Lambda, das nur einen Parameter nimmt, diesen nicht explizit angeben muss.
Stattdessen wird implizit eine lokale Variable namens \inlineKotlin{it} erschaffen.
Das erleichtert vor allem lange Aufrufketten auf Iterables.

%! language = kotlin
\begin{minted}{kotlin}
data class Country(val name: String, val population: Int)
fun test() {
    val countries = listOf<Country>()
    // erstellt einen String mit den Anfangsbuchstaben aller Länder mit mehr als 500 Einwohnern die mit "land" enden
    val result = countries
        .filter { it.population > 500 }
        .map { it.name }
        .filter { it.endsWith("land") }
        .map { it.first() }
        .fold("") { acc, cur -> acc + cur }
}
\end{minted}
\codeblockCaption{Praxisnahes Beispiel für Lambdas}

Wie im letzten Beispiel ersichtlich, ist es nicht notwendig explizit eine \inlineKotlin{return} expression in einem
Lambda zu verwenden.
Stattdessen wird der Wert der letzten Expression im Block automatisch als return-Wert angenommen.

%! language = kotlin
\begin{minted}{kotlin}

enum class ColorMode { LightMode, DarkMode }
data class UserPrefs(val selectedColorMode: ColorMode?)
data class Defaults(val defaultColorMode: ColorMode = ColorMode.DarkMode)

fun test() {
    val colorPicker: (UserPrefs, Defaults) -> String = { prefs, defaults ->
        val colorMode = prefs.selectedColorMode ?: defaults.defaultColorMode
        when (colorMode) { // when ist die letzte Expression im Lambda, daher wird ihr Wert als return-Wert genommen
            ColorMode.LightMode -> "#ffffff"
            ColorMode.DarkMode -> "#000000"
        }
    }
}
\end{minted}

% resets author
\renewcommand{\kapitelautor}{}
