
\section{Inline Functions}\label{sec:inline-functions}

\renewcommand{\kapitelautor}{Autor: Marvin Kurka}

\subsection{inline Funktionen in anderen Sprachen}
In C/C++ werden von den Compilern häufig kurze Funktionen geinlined.
Das heißt, der Funktionsaufruf wird mit den tatsächlichen Body der Funktion ersetzt, und kein tatsächlicher
Funktionsaufruf findet statt.
C und C++ haben den \inlineCode{inline} Modifier, der als Hinweis für den Compiler dient, dass diese Funktion
geinlined werden soll.
Der Compiler darf allerdings selber entscheiden, ob er das tut \bzw auch nicht mit \inlineCode{inline} markierte
Funktionen inlinen.
Der Grund für Funktions-inlining ist in erster Linie Performance, da dadurch ein Funktionsaufruf gespart wird.\cite{crefInline}

In Kotlin ist der Grund für Funktions-inlining explizit nicht um den Overhead eines Funktionsaufrufs zu sparen,
der Compiler warnt sogar, wenn der \inlineKotlin{inline} Modifier verwendet wird und kein sonstiger Grund für
inlining besteht.

\subsection{Effizientere Lambdas und non-local returns}
Eine Anwendung für inline-Funktionen sind higher-order-functions, also Funktionen, die andere Funktionen als Parameter
nehmen.



% resets author
\renewcommand{\kapitelautor}{}
