
\section{Extension Functions}\label{sec:extension-functions}

\renewcommand{\kapitelautor}{Autor: Marvin Kurka}

Kotlin erlaubt es Funktionen zu definieren, die auf einem Objekt aufgerufen werden können, ohne tatsächlich in der
Klasse dieses Objekts definiert worden zu sein.
Solche Funktionen nennt man Extension Functions.
Sie werden definiert, indem man vor dem Funktionsnamen einen sogenannten Receiver-Typen gefolgt von einem Punkt angibt.
Bei dem Aufruf einer Extension Function wird der Receiver als Parameter mitgegeben, und ist dann in der Funktion als
\inlineKotlin{this} verfügbar.
Das heißt auch, dass Extension Function statisch, also zur compile time, resolved werden.
Es finden also keine virtuellen Lookups statt.\cite{kdocExtensions}

%! language = kotlin
\begin{minted}{kotlin}
fun String.escapeNewLines(): String = this.replace("\n", "\\n")

fun test() {
    val s = "String\nwith\nNewlines"
    println(s.escapeNewLines())
}
\end{minted}

Wenn Receiver-Typen generische Parameter haben, kann die Extension Function entweder selbst einen generischen Parameter
verwenden, oder auch einen konkreten Typen einsetzen.
Das erlaubt es Funktionen zu definieren, die nur verfügbar sind, wenn bestimmte Typen für die generischen Parameter
eingesetzt wurden, was mit normalen member Functions nicht möglich ist.\cite{kdocExtensions}

%! language = kotlin
\begin{minted}{kotlin}
fun Iterable<Int>.evenNumbers(): List<Int> = this.filter { it % 2 == 0 }
\end{minted}

Wichtig zu erwähnen ist auch, dass Encapsulation nicht durch Extension Functions gebrochen wird.
Das heißt, dass Extension Functions nicht auf private Funktionen/Properties des Receivers zugreifen können.
So ein Zugriff wäre spätestens auf dem Level der JVM verboten, da Extension Functions nicht in der Klasse des Receivers
definiert sind.

Nicht nur Funktionen können Receiver angeben, sondern auch Lambdas.
Auch hier wird der Receiver-Typ mit einem Punkt vor dem Lambda-Typ geschrieben, und der Receiver ist als
\inlineKotlin{this} im Lambda verwendbar.\cite{kdocLambdas}

%! language = kotlin
\begin{minted}{kotlin}
fun <T> createList(listInitializer: MutableList<T>.() -> Unit): List<T> =
    mutableListOf<T>().apply { listInitializer(this) }

fun test() {
    val list = createList<String> {
        // Aufrufe werden implizit auf 'this' ausgeführt
        add("Hi")
        add("Hello")
        add("World")
        removeFirst()
    }
}
\end{minted}

% resets author
\renewcommand{\kapitelautor}{}
