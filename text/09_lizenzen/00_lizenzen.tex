
\chapter{Linzenzen}\label{ch:lizenzen}
\renewcommand{\kapitelautor}{Autor: Nils} % todo: replace

%

Indem wir sicherstellen, dass wir alle benötigten Lizenzen besitzen beziehungsweise dokumentieren von welchen Seiten die Elemente stammen können wir sicherstellen, dass unser Spiel frei von Urheberrechtsverletzungen ist.
Das ist wichtig, um rechtliche Probleme zu vermeiden, die zu hohen Geldstrafen führen könnten und sogar die gesamte Existenz des Spiels gefährden könnten.
Diese Vorsichtsmaßnahme ermöglicht es uns, kreative Freiheit zu genießen und hochwertige Ressourcen zu nutzen, ohne uns um etwaige rechtliche Konsequenzen sorgen zu müssen.
Darüber hinaus trägt die ordnungsgemäße Lizenzierung und Nutzung kommerziell nutzbarer Ressourcen dazu bei, das professionelle Image unseres Spiels zu stärken und das Vertrauen unserer Spieler zu gewinnen.
Indem wir zeigen, dass wir uns um die Einhaltung von Urheberrechten und Lizenzbestimmungen kümmern, zeigen wir auch unser Engagement für Qualität und Integrität.
Insgesamt ist es von entscheidender Bedeutung, dass wir bei unserem Spiel darauf achten, dass alle Elemente ordnungsgemäß lizenziert sind und dass alle externen Ressourcen,
die wir nutzen, kommerziell nutzbar sind. Dies ist nicht nur eine grundlegende Voraussetzung für die rechtliche Sicherheit unseres Spiels, sondern auch ein wichtiger Schritt,
um das Vertrauen unserer Spieler zu gewinnen und die Qualität und Integrität unseres Spiels zu demonstrieren.
%
\subsubsection{Open Source}\label{subsubsec:Open-Source}
%
\italic{Open Source ist ein Begriff, der ursprünglich auf Open Source-Software (OSS) zurückgeht. Es handelt sich dabei um Code, der der Öffentlichkeit zugänglich ist, das heißt, jeder kann ihn anzeigen sowie nach Belieben verändern und verteilen.

Open Source Software wird dezentral und kollaborativ entwickelt und stützt sich auf Peer-Review und Community-Produktion. Diese Software ist nicht selten günstiger, flexibler und langlebiger als proprietäre Produkte, weil sie nicht von einzelnen Personen oder Unternehmen, sondern in Communities entwickelt wird.

Durch die Entscheidung, unser Spiel als Open Source zu veröffentlichen, eröffnet sich eine Welt voller Möglichkeiten für Entwickler, Enthusiasten und die gesamte Community. Jeder, der ein Interesse an der Verbesserung des Spiels hat, kann dazu beitragen, sei es durch das Hinzufügen neuer Features, das Beheben von Fehlern oder das Anpassen des Spiels an individuelle Bedürfnisse.

Eine der Schlüsselkomponenten des Open-Source-Ansatzes ist die Wahl der Lizenz. Die Entscheidung ist im Fall von \ff auf die GPL 3 Lizenz gefallen. Die GPL (General Public License) ist eine der bekanntesten und am häufigsten verwendeten Open-Source-Lizenzen. Sie garantiert, dass der Quellcode des Spiels für jeden frei zugänglich ist und dass abgeleitete Werke ebenfalls unter derselben Lizenz veröffentlicht werden müssen. Dies bedeutet, dass die Community nicht nur von unserem Code profitieren kann, sondern auch dazu ermutigt wird, ihre eigenen Verbesserungen und Weiterentwicklungen zurückzugeben, sodass das gesamte Projekt davon profitiert.

Die Verwendung der GPL 3 Lizenz ist bietet starken Schutz für die Freiheiten der Nutzer und Entwickler bietet. Sie stellt sicher, dass die Software immer frei bleibt und nicht von einzelnen Akteuren monopolisiert oder eingeschränkt werden kann. Durch diese Lizenz zeigen wir unser Engagement für die Ideale der Open-Source-Bewegung und verfolge die Richtung für offene und transparente Softwareentwicklung.
%
% resets author
\renewcommand{\kapitelautor}{}