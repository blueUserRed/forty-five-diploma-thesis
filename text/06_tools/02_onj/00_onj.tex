
\section{Onj}\label{sec:onj}

Onj ist eine selbst entwickelte Markupsprache, die Ähnlichkeiten zu Sprachen wie Json, Toml oder Yaml aufweist,
diese allerdings um einige Features erweitert.
Onj-Dateien sind dafür ausgelegt, von Hand geschrieben zu werden und haben daher einige Quality of Life Features, die
ähnliche Sprachen nicht aufweisen.
Außerdem soll Onj auch für größere Projekte geeignet sein, ohne das der Wartungsaufwand zu groß wird.
Daher hat Onj Mechanismen um repetitiven Code zu vermeiden, wie imports oder Variablen, die Möglichkeit Strukturen
mit Schemas zu validieren und die Fähigkeit mittels Namespaces mit Kotlin-Code zu interagieren.

Onj-Dateien können nicht nur von Programmen eingelesen werden, Onj-Strukturen können auch mittels der
\inlineKotlin{buildOnjObject}-Funktion im Programm gebaut und in Dateien geschrieben werden.
Dabei können die Strukturen nicht nur zu gültigen Onj, sondern auch zu Json serialisiert werden.


\subsection{Warum Onj?}\label{subsec:warum-onj}

\renewcommand{\kapitelautor}{Autor: Irgendwer} % todo: replace

%
% text goes here
%

% resets author
\renewcommand{\kapitelautor}{}


\subsection{Überblick Syntax}\label{subsec:ueberblick-syntax}

\renewcommand{\kapitelautor}{Autor: Irgendwer} % todo: replace

%
% text goes here
%

% resets author
\renewcommand{\kapitelautor}{}


\subsection{Variablen}\label{subsec:variablen}

\renewcommand{\kapitelautor}{Autor: Marvin Kurka}

Um das Wiederholen von Code so gering wie möglich zu halten, werden in Onj Variablen verwendet, um häufige Strukturen
zu extrahieren.
Mittels des Variablennamen kann dan auf diese zugegriffen werden, auch Verknüpfungen mit Punkten sind erlaubt.

%! language = Onj
\begin{codeBlock}{onj}{Beispiel: Variablen in Onj}
var number = 5;
favoriteNumber: number,
otherNumber: (number + 1) * number,

var colors = {
    red: "#ff0000",
    green: "#00ff00",
    blue: "#0000ff",
};
favoriteColor: colors.blue,

var fruits = ["apple", "banana", "grape"];
// auf arrays wird mittels des Indexes zugegriffen
favoriteFruit: fruits.0,
\end{codeBlock}

Es können auch drei Punkte verwendet werden, um den Inhalt einer Variable in die aktuelle Struktur zu übernehmen.

%! language = Onj
\begin{codeBlock}{onj}{Beispiel: Triple-Dot Syntax in Onj}
var fruits = ["apple", "pear"];
fruitSalad: [
    "banana",
    "grape",
    ...fruits
],
\end{codeBlock}

Statt bei einem Zugriff einfach den Key-Namen oder den Index anzugeben, kann auch in Klammern eine Expression angegeben
werden.
Diese wird evaluiert und das Resultat wird für den Zugriff verwendet.

%! language = Onj
\begin{codeBlock}{onj}{Demo: dynamische Zugriffe in Onj}
var index = 1;
var arr = [0, 1, 2, 3, 4, 5];
myValue: arr.(index + 2) // => 3
\end{codeBlock}

% resets author
\renewcommand{\kapitelautor}{}


\subsection{Imports}\label{subsec:imports}

\renewcommand{\kapitelautor}{Autor: Irgendwer} % todo: replace

%
% text goes here
%

% resets author
\renewcommand{\kapitelautor}{}


\subsection{Onj Schemas}\label{subsec:onj-schemas}

\renewcommand{\kapitelautor}{Autor: Marvin Kurka}

Wenn Configdateien vom Programm eingelesen werden, muss überprüft werden, dass diese im richtigen Format sind.
Das kann vom Programmierer manuell gemacht werden, dies ist aber sehr umständlich, vor allem großen/komplexen Strukturen.
Stattdessen werden Schemas verwendet, die definieren wie eine Onj-Datei strukturiert sein darf.
Diese Schema-Dateien können vom Programm eingelesen und mit den Onj-Strukturen verglichen werden.

Schemas werden in \inlineCode{.onjschema} Dateien definiert.
Diese haben prinzipiell einen ähnlichen Syntax zu normalen \inlineCode{.onj} Dateien, statt Werten werden jedoch
Typen angegeben und einige Features sind nicht supported.

Die konkreten Unterschiede sind:
\begin{liste}
    \item Werte (\zB Zahlen, Strings, Booleans, \ldots) sind nicht erlaubt, stattdessen werden Typen angegeben.
    \item Funktionen sind nicht erlaubt.
    \item Variablenzugriffe (\inlineOnj{color.green}) sind nicht erlaubt.
\end{liste}

Anders als normale Onj-Dateien stellen Onjschema-Dateien allerdings Syntax um Typen zu definieren zur Verfügung.
Neben simplen literals (\inlineOnj{int}, \inlineOnj{float}, \inlineOnj{boolean}, \inlineOnj{string}) können auch
Objekte definiert werden.
Der Syntax für Objekte ist ident zu normalen Onj-Dateien.

\begin{codeBlock}{onj}{Demo: Objekt in OnjSchema}
myObject: {
    myNumber: int,
    myString: string
}
\end{codeBlock}

Arrays können auf zwei Arten definiert werden.
Entweder als Literal, die den Typen an jeder Stelle definiert, oder als Typ gefolgt von eckigen Klammern.
Bei der zweiten Variante kann in den Klammern optional die Länge des Arrays definiert werden.

\begin{codeBlock}{onj}{Demo: Arrays in OnjSchema}
myArray: [string, boolean, float],
myNumbers: int[],
threeNumbers: int[3]
\end{codeBlock}

Um anzugeben, dass ein Typ null sein darf, wird ein Fragezeichen verwendet (\inlineOnj{int?}).
Um anzugeben, dass ein beliebiger Typ verwendet werden darf, wird ein Stern (\inlineOnj{*}) verwendet.

Wenn in einem Schema ein Objekt definiert wird, darf dieses nur die Keys haben, die tatsächlich im Schema angegeben
wurden.
Hat ein Objekt einen Key, der im Schema nicht zu finden ist, führt das zu einem Fehler.
Um anzugeben, dass ein Objekt auch andere Keys erlauben soll wird der \inlineOnj{...*}-Syntax verwendet.

% resets author
\renewcommand{\kapitelautor}{}


\subsection{Named Objects}\label{subsec:named-objects}

\renewcommand{\kapitelautor}{Autor: Marvin Kurka}

Bei der Validation von komplexen Onj-Strukturen kann es unter gewissen Umständen zu Problemen können, vor allem wenn
an einer Stelle mehrere verschiedene Objekte möglich sind.
Um das Problem besser erkenntlich zu machen, hier ein Beispiel:

%! language = Onj
\begin{codeBlock}{onj}{Beispiel: Schlechte Umsetzung einer UI-Struktur 1}
// onj struktur
uiElements: [
    {
        type: "label",
        text: "Hello World",
        font: "red wing",
    },
    {
        type: "image",
        path: "./some/image.png"
    }
]
\end{codeBlock}

\begin{codeBlock}{onj}{Beispiel: Schlechte Umsetzung einer UI-Struktur 2}
// onj schema
var uiElement = {
    type: string,
    ...*
};
uiElements: uiElement[]
\end{codeBlock}

Im Schema kann nur garantiert werden, das ein UI-Element einen \inlineOnj{type}-Key hat, alles andere wird offen
gelassen.
Das macht die Schema-Validierung in diesem Fall praktisch nutzlos.
Die Lösung, die Onj für dieses Problem bietet, sind Named Objects.
Im Schema kann eine Named Object Group definiert werden, die mehrere Named Objects enthält.
Jedes Named Object hat einen Namen, über den es identifiziert wird und kann beliebige Keys mit beliebigen Typen
definieren.
Im Schema kann dann der Name der Named Object Group als Typ verwendet werden, der alle Named Objects, die Teil der Group
sind, erlaubt.

Hier noch einmal das selbe Beispiel, implementiert mit Named Objects:

%! language = Onj
\begin{codeBlock}{onj}{Beispiel: Gute Umsetzung einer UI-Struktur 1}
// onj struktur
uiElements: [
    $Label {
        text: "Hello World",
        font: "red wing",
    },
    $Image {
        path: "./some/image.png"
    }
]
\end{codeBlock}

\begin{codeBlock}{onj}{Beispiel: Gute Umsetzung einer UI-Struktur 2}
// onj schema
$UiElement {
    $Label {
        text: string,
        font: string
    }
    $Image {
        path: string
    }
}
uiElements: $UiElement[]
\end{codeBlock}

% resets author
\renewcommand{\kapitelautor}{}


\subsection{Namespaces}\label{subsec:namespaces}

\renewcommand{\kapitelautor}{Autor: Marvin Kurka}

Namespaces erlauben es, Onj zu erweitern, indem eigene Funktionen, zusätzliche globale Variablen oder
eigene Datentypen definiert werden.
Solche Namespaces werden in Kotlin definiert und mit Annotations markiert.

Funktionen werden im Namespace definiert und mit der \inlineKotlin{@RegisterOnjFunction} markiert.
Diese Annotation nimmt ein String mit einem OnjSchema als Parameter, der die Signatur der Funktion beschreibt.
Das ist notwendig, da Onj das Überladen von Funktionen erlaubt, \dah mehrere Funktionen dürfen denselben Namen haben.
Um herauszufinden, welche Funktion tatsächlich aufgerufen werden soll, vergleicht der Onj-Parser die mitgegebenen
Parameter mit dem Schema und ruft die erste Funktion auf, bei der diese übereinstimmen.

%! language = Kotlin
\begin{codeBlock}{kotlin}{Demo: Deklaration einer Onj-Funktion in Kotlin}
@OnjNamespace
object MyNamespace {

    @RegisterOnjFunction(schema = "params: [string, int]")
    fun repeatString(s: OnjString, times: OnjInt) = OnjString(s.value.repeat(times.value))
}
\end{codeBlock}

Neben normalen Funktionen stellt Onj noch drei spezielle Arten von Funktionen zur Verfügung:

\begin{itemize}
    \item Conversions: Diese werden mit folgendem Syntax aufgerufen: \inlineOnj{value#function}.
        Solche Funktion werden zum Beispiel verwendet, um einen Wert von einem Datentypen zu einem anderen zu
        konvertieren.
    \item Infix Funktionen: Diese werden mit folgendem Syntax aufgerufen: \inlineOnj{value1 function value2}.
        Beispiele sind die pow-Funktion (\inlineOnj{10 pow 5}) oder die in-Funktion (\inlineOnj{3 in [1, 2, 3, 4]}).
    \item Operator Overloading: Solche Funktionen erlauben es zu definieren, wie sich Operatoren wie
        \inlineOnj{+} oder \inlineOnj{*} für eigene Datentypen verhalten.
\end{itemize}

Wird so eine spezielle Funktion verwendet, wird das in der \inlineKotlin{@RegisterOnjFunction} Annotation angegeben.

%! language = Kotlin
\begin{codeBlock}{kotlin}{Demo: Deklaration einer Onj-Conversion in Kotlin}
@OnjNamespace
object MyNamespace {

    @RegisterOnjFunction(schema = "params: [string]", type = OnjFunctionType.CONVERSION)
    fun greeting(name: OnjString) = OnjString("Hello, ${name.value}")
}
\end{codeBlock}

Weiters können Namespaces globale Variablen definieren.

%! language = Kotlin
\begin{codeBlock}{kotlin}{Demo: Deklaration von globalen Onj-Variablen in Kotlin}
@OnjNamespace
object MyNamespace {

    @OnjNamespaceVariables
    val variables: Map<String, OnjObject> = mapOf(
        "myGlobal" to OnjInt(5)
    )
}
\end{codeBlock}

In Onj kann ein use-Statement verwendet werden, um einen Namespace zu inkludieren.

%! language = Onj
\begin{codeBlock}{onj}{Demo: Verwendung eines Onj-Namespaces}
use MyNamespace;

global: myGlobal,
greeting: "Reader"#greeting,
string: repeatString("a", 5)
\end{codeBlock}

% resets author
\renewcommand{\kapitelautor}{}

