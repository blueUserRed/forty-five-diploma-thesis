
\section{Onj}\label{sec:onj}

Onj ist eine selbst entwickelte Markupsprache, die Ähnlichkeiten zu Sprachen wie Json, Toml oder Yaml aufweist,
diese allerdings um einige Features erweitert.
Onj-Dateien sind dafür ausgelegt, von Hand geschrieben zu werden und haben daher einige Quality of Life Features, die
ähnliche Sprachen nicht aufweisen.
Außerdem soll Onj auch für größere Projekte geeignet sein, ohne das der Wartungsaufwand zu groß wird.
Daher hat Onj Mechanismen um repetitiven Code zu vermeiden, wie imports oder Variablen, die Möglichkeit Strukturen
mit Schemas zu validieren und die Fähigkeit mittels Namespaces mit Kotlin-Code zu interagieren.

Onj-Dateien können nicht nur von Programmen eingelesen werden, Onj-Strukturen können auch mittels der
\inlineKotlin{buildOnjObject}-Funktion im Programm gebaut und in Dateien geschrieben werden.
Dabei können die Strukturen nicht nur zu gültigen Onj, sondern auch zu Json serialisiert werden.


\subsection{Warum Onj?}\label{subsec:warum-onj}

\renewcommand{\kapitelautor}{Autor: Marvin Kurka}

Erfahrung, die wir in früheren Projekten \bzw Vorarbeit an \FF gesammelt haben zeigt, dass
es oft kurzfristig zu Änderungen und Anpassen bezüglich des Gameplays kommt.
Oft handelt es sich dabei um komplexere Funktionen, die Mechaniken des Gameplays ändern, oft sind es aber auch nur
simple Balancing-Änderungen, wo \zB nur der Schadenswert einer Karte geändert wird.
Daher hatten wir bei \FF den Anspruch, dass solche Balancing-Änderungen schnell und von Leuten mit keiner Kenntnis
des Programmaufbaus vorgenommen werden können.

Bei der Wahl der Markupsprache hatten wir die Wahl zwischen einen bereits etablierten Format wie XML, JSON oder YAML,
oder den selbst entwickelten Format Onj.
Unsere Wahl ist aus mehreren Gründen auf Onj gefallen.

\begin{itemize}
    \item Variablen und Imports erhöhen die Maintainability in größeren Projekten
    \item OnjSchemas helfen sicherzustellen, das die Dateien im korrekten Format sind
    \item Onjs Syntax ist leichter zu lesen als \zB der von Json
    \item Namespaces ermöglichen eine gute Integration mit dem Programm
    \item Onjs API ist für Kotlin ausgelegt und verwendet Features der Sprache, wie reified Generics oder
        Lambdas mit Receiver
    \item Dadurch, dass Onj selbst entwickelt wurde, können, falls notwendig, Features hinzugefügt werden (So wurde die
        Sprache \zB während des Projektes um einen Minifier erweitert, um Maps effizienter zu speichern.)
\end{itemize}

% resets author
\renewcommand{\kapitelautor}{}

\input{text/06_tools/02_onj/02_überblick_syntax.tex}

\subsection{Variablen}\label{subsec:variablen}

\renewcommand{\kapitelautor}{Autor: Marvin Kurka}

Um das Wiederholen von Code so gering wie möglich zu halten, werden in Onj Variablen verwendet, um häufige Strukturen
zu extrahieren.
Mittels des Variablennamen kann dann auf diese zugegriffen werden, auch Verknüpfungen mit Punkten sind erlaubt.

%! language = Onj
\begin{codeBlock}{onj}{Beispiel: Variablen in Onj}
var number = 5;
favoriteNumber: number,
otherNumber: (number + 1) * number,

var colors = {
    red: "#ff0000",
    green: "#00ff00",
    blue: "#0000ff",
};
favoriteColor: colors.blue,

var fruits = ["apple", "banana", "grape"];
// auf arrays wird mittels des Indexes zugegriffen
favoriteFruit: fruits.0,
\end{codeBlock}

Es können auch drei Punkte verwendet werden, um den Inhalt einer Variable in die aktuelle Struktur zu übernehmen.

%! language = Onj
\begin{codeBlock}{onj}{Beispiel: Triple-Dot Syntax in Onj}
var fruits = ["apple", "pear"];
fruitSalad: [
    "banana",
    "grape",
    ...fruits
],
\end{codeBlock}

Statt bei einem Zugriff einfach den Key-Namen oder den Index anzugeben, kann auch in Klammern eine Expression angegeben
werden.
Diese wird evaluiert und das Resultat wird für den Zugriff verwendet.

%! language = Onj
\begin{codeBlock}{onj}{Demo: dynamische Zugriffe in Onj}
var index = 1;
var arr = [0, 1, 2, 3, 4, 5];
myValue: arr.(index + 2) // => 3
\end{codeBlock}

% resets author
\renewcommand{\kapitelautor}{}


\subsection{Imports}\label{subsec:imports}

\renewcommand{\kapitelautor}{Autor: Marvin Kurka}

Oft ist es so, dass gewisse Strukturen öfter verwendet werden müssen.
Das kann Probleme mit der maintainability verursachen, da, falls eine Änderung vorgenommen wird, diese dann an vielen
Stellen repliziert werden muss.
Um solche Redundanzen innerhalb einer Datei zu vermeiden können Variablen verwendet werden, oft treten jedoch auch
dateiübergreifende Redundanzen auf.
Import-Statements erlauben häufige Strukturen in eine eigene Datei auszulagern und diese dann in anderen Dateien zu
importieren.
Bei einem Import-Statement wird neben dem Pfad der zu importierenden Datei auch ein Variablenname angegeben, unter dem
dann eine Variable mit dem Inhalt der importierten Datei erschaffen wird.

%! language = Onj
\begin{minted}{onj}
// toImport.onj
myValue: 5
\end{minted}

%! language = Onj
\begin{minted}{onj}
// importer.onj
import "toImport.onj" as imported;
importedValue: imported.myValue,
\end{minted}

Der Pfad kann, ähnlich wie ein Variablenzugriff, dynamisch gemacht werden.
Dazu werden Klammern verwendet.

%! language = Onj
\begin{minted}{onj}
var colorSchemes = ["lightMode.onj", "darkMode.onj"];
import "userPrefs.onj" as userPrefs;
import (colorSchemes.(userPrefs.preferredColorScheme)) as colorScheme;
\end{minted}

% resets author
\renewcommand{\kapitelautor}{}


\subsection{Onj Schemas}\label{subsec:onj-schemas}

\renewcommand{\kapitelautor}{Autor: Irgendwer} % todo: replace

%
% text goes here
%

% resets author
\renewcommand{\kapitelautor}{}


\subsection{Named Objects}\label{subsec:named-objects}

\renewcommand{\kapitelautor}{Autor: Irgendwer} % todo: replace

%
% text goes here
%

% resets author
\renewcommand{\kapitelautor}{}


\subsection{Namespaces}\label{subsec:namespaces}

\renewcommand{\kapitelautor}{Autor: Marvin Kurka}

Namespaces erlauben es, Onj zu erweitern, indem eigene Funktionen, zusätzliche globale Variablen oder
eigene Datentypen definiert werden.
Solche Namespaces werden in Kotlin definiert und mit Annotations markiert.

Funktionen werden im Namespace definiert und mit der \inlineKotlin{@RegisterOnjFunction} markiert.
Diese Annotation nimmt ein String mit einem OnjSchema als Parameter, der die Signatur der Funktion beschreibt.
Das ist notwendig, da Onj das Überladen von Funktionen erlaubt, \dah mehrere Funktionen dürfen denselben Namen haben.
Um herauszufinden, welche Funktion tatsächlich aufgerufen werden soll, vergleicht der Onj-Parser die mitgegebenen
Parameter mit dem Schema und ruft die erste Funktion auf, bei der diese übereinstimmen.

%! language = Kotlin
\begin{codeBlock}{kotlin}{Demo: Deklaration einer Onj-Funktion in Kotlin}
@OnjNamespace
object MyNamespace {

    @RegisterOnjFunction(schema = "params: [string, int]")
    fun repeatString(s: OnjString, times: OnjInt) = OnjString(s.value.repeat(times.value))
}
\end{codeBlock}

Neben normalen Funktionen stellt Onj noch drei spezielle Arten von Funktionen zur Verfügung:

\begin{liste}
    \item Conversions: Diese werden mit folgendem Syntax aufgerufen: \inlineOnj{value#function}.
        Solche Funktion werden zum Beispiel verwendet, um einen Wert von einem Datentypen zu einem anderen zu
        konvertieren.
    \item Infix Funktionen: Diese werden mit folgendem Syntax aufgerufen: \inlineOnj{value1 function value2}.
        Beispiele sind die pow-Funktion (\inlineOnj{10 pow 5}) oder die in-Funktion (\inlineOnj{3 in [1, 2, 3, 4]}).
    \item Operator Overloading: Solche Funktionen erlauben es zu definieren, wie sich Operatoren wie
        \inlineOnj{+} oder \inlineOnj{*} für eigene Datentypen verhalten.
\end{liste}

Wird so eine spezielle Funktion verwendet, wird das in der \inlineKotlin{@RegisterOnjFunction} Annotation angegeben.

%! language = Kotlin
\begin{codeBlock}{kotlin}{Demo: Deklaration einer Onj-Conversion in Kotlin}
@OnjNamespace
object MyNamespace {

    @RegisterOnjFunction(schema = "params: [string]", type = OnjFunctionType.CONVERSION)
    fun greeting(name: OnjString) = OnjString("Hello, ${name.value}")
}
\end{codeBlock}

Weiters können Namespaces globale Variablen definieren.

%! language = Kotlin
\begin{codeBlock}{kotlin}{Demo: Deklaration von globalen Onj-Variablen in Kotlin}
@OnjNamespace
object MyNamespace {

    @OnjNamespaceVariables
    val variables: Map<String, OnjValue> = mapOf(
        "myGlobal" to OnjInt(5)
    )
}
\end{codeBlock}

In Onj kann ein use-Statement verwendet werden, um einen Namespace zu inkludieren.

%! language = Onj
\begin{codeBlock}{onj}{Demo: Verwendung eines Onj-Namespaces}
use MyNamespace;

global: myGlobal,
greeting: "Reader"#greeting,
string: repeatString("a", 5)
\end{codeBlock}

% resets author
\renewcommand{\kapitelautor}{}


\subsection{Nachteile}\label{subsec:nachteile}

\renewcommand{\kapitelautor}{Autor: Irgendwer} % todo: replace

%
% text goes here
%

% resets author
\renewcommand{\kapitelautor}{}

