
\subsection{Überblick Syntax}\label{subsec:ueberblick-syntax}

\renewcommand{\kapitelautor}{Autor: Marvin Kurka}

Onjs Syntax ähnelt stark den von Json, jedoch gibt es einige Unterschiede:
\begin{liste}
    \item Bei Keys ohne Sonderzeichen sind Anführungszeichen optional
    \item Das Top-Level Objekt ist implizit
    \item Top-Level Arrays sind nicht erlaubt
    \item Trailing Kommas sind erlaubt
    \item Funktionen in verschiedenen Notationen sind unterstützt.
    \item Besondere Strukturen wie Variablen-Deklarationen oder import-statements
    \item Der Triple-Dot (include) Syntax
    \item Mathematische expressions sind erlaubt
\end{liste}

%! language = Onj
\begin{codeBlock}{onj}{Demo: Zusäztliche Features in Onj}
import "color.onj" as color;

use SomeNamespace;

var favoriteColor = color.red;
var boringColors = [color.black, color.white, color.gray];

string: "a\nstring",
colors: [color.blue, favoriteColor, ...boringColors,],
number: 2 + 2 * 2,
otherNumber: 4 * (10 pow 5),
functionCall: someFunction(10, 5),
conversion: 5#string,
isWhiteBoring: color.white in boringColors,
\end{codeBlock}

\begin{infoBox}
Die vollständige Grammatik von Onj kann im Anhang~\ref{ch:anhang-2} gefunden werden.
\end{infoBox}

% resets author
\renewcommand{\kapitelautor}{}
