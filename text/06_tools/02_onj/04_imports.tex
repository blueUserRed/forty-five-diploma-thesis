
\subsection{Imports}\label{subsec:imports}

\renewcommand{\kapitelautor}{Autor: Marvin Kurka}

Oft ist es so, dass gewisse Strukturen öfter verwendet werden müssen.
Das kann Probleme mit der Wartbarkeit verursachen, da, falls eine Änderung vorgenommen wird, diese dann an vielen
Stellen repliziert werden muss.
Um solche Redundanzen innerhalb einer Datei zu vermeiden können Variablen verwendet werden, oft treten jedoch auch
dateiübergreifende Redundanzen auf.
Import-Statements erlauben häufige Strukturen in eine eigene Datei auszulagern und diese dann in anderen Dateien zu
importieren.
Bei einem Import-Statement wird neben dem Pfad der zu importierenden Datei auch ein Variablenname angegeben, unter dem
dann eine Variable mit dem Inhalt der importierten Datei erschaffen wird.

%! language = Onj
\begin{codeBlock}{onj}{Demo: Imports in Onj 1}
// toImport.onj
myValue: 5
\end{codeBlock}

%! language = Onj
\begin{codeBlock}{onj}{Demo: Imports in Onj 2}
// importer.onj
import "toImport.onj" as imported;
importedValue: imported.myValue,
\end{codeBlock}

Der Pfad kann, ähnlich wie ein Variablenzugriff, dynamisch gemacht werden.
Dazu werden Klammern verwendet.

%! language = Onj
\begin{codeBlock}{onj}{Beispiel: Dynamischer Import in Onj}
var colorSchemes = ["lightMode.onj", "darkMode.onj"];
import "userPrefs.onj" as userPrefs;
import (colorSchemes.(userPrefs.preferredColorScheme)) as colorScheme;
\end{codeBlock}

% resets author
\renewcommand{\kapitelautor}{}
