
\subsection{Onj Schemas}\label{subsec:onj-schemas}

\renewcommand{\kapitelautor}{Autor: Marvin Kurka}

Wenn Configdateien vom Programm eingelesen werden, muss überprüft werden, dass diese im richtigen Format sind.
Das kann vom Programmierer manuell gemacht werden, dies ist aber sehr umständlich, vor allem großen/komplexen Strukturen.
Stattdessen werden Schemas verwendet, die definieren wie eine Onj-Datei strukturiert sein darf.
Diese Schema-Dateien können vom Programm eingelesen und mit den Onj-Strukturen verglichen werden.

Schemas werden in \inlineCode{.onjschema} Dateien definiert.
Diese haben prinzipiell einen ähnlichen Syntax zu normalen \inlineCode{.onj} Dateien, statt Werten werden jedoch
Typen angegeben und einige Features sind nicht supported.

Die konkreten Unterschiede sind:
\begin{itemize}
    \item Werte (\zB Zahlen, Strings, Booleans, \ldots) sind nicht erlaubt, stattdessen werden Typen angegeben.
    \item Funktionen sind nicht erlaubt.
    \item Variablenzugriffe (\inlineOnj{color.green}) sind nicht erlaubt.
\end{itemize}

Anders als normale Onj-Dateien stellen Onjschema-Dateien allerdings Syntax um Typen zu definieren zur Verfügung.
Neben simplen literals (\inlineOnj{int}, \inlineOnj{float}, \inlineOnj{boolean}, \inlineOnj{string}) können auch
Objekte definiert werden.
Der Syntax für Objekte ist ident zu normalen Onj-Dateien.

\begin{codeBlock}{onj}{Demo: Objekt in OnjSchema}
myObject: {
    myNumber: int,
    myString: string
}
\end{codeBlock}

Arrays können auf zwei Arten definiert werden.
Entweder als Literal, die den Typen an jeder Stelle definiert, oder als Typ gefolgt von eckigen Klammern.
Bei der zweiten Variante kann in den Klammern optional die Länge des Arrays definiert werden.

\begin{codeBlock}{onj}{Demo: Arrays in OnjSchema}
myArray: [string, boolean, float],
myNumbers: int[],
threeNumbers: int[3]
\end{codeBlock}

Um anzugeben, dass ein Typ null sein darf, wird ein Fragezeichen verwendet (\inlineOnj{int?}).
Um anzugeben, dass ein beliebiger Typ verwendet werden darf, wird ein Stern (\inlineOnj{*}) verwendet.

Wenn in einem Schema ein Objekt definiert wird, darf dieses nur die Keys haben, die tatsächlich im Schema angegeben
wurden.
Hat ein Objekt einen Key, der im Schema nicht zu finden ist, führt das zu einem Fehler.
Um anzugeben, dass ein Objekt auch andere Keys erlauben soll wird der \inlineOnj{...*}-Syntax verwendet.

% resets author
\renewcommand{\kapitelautor}{}
