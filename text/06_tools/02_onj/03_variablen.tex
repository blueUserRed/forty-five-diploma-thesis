
\subsection{Variablen}\label{subsec:variablen}

\renewcommand{\kapitelautor}{Autor: Marvin Kurka}

Um das Wiederholen von Code so gering wie möglich zu halten, werden in Onj Variablen verwendet, um häufige Strukturen
zu extrahieren.
Mittels des Variablennamen kann dann auf diese zugegriffen werden, auch Verknüpfungen mit Punkten sind erlaubt.

%! language = Onj
\begin{codeBlock}{onj}{Beispiel: Variablen in Onj}
var number = 5;
favoriteNumber: number,
otherNumber: (number + 1) * number,

var colors = {
    red: "#ff0000",
    green: "#00ff00",
    blue: "#0000ff",
};
favoriteColor: colors.blue,

var fruits = ["apple", "banana", "grape"];
// auf arrays wird mittels des Indexes zugegriffen
favoriteFruit: fruits.0,
\end{codeBlock}

Es können auch drei Punkte verwendet werden, um den Inhalt einer Variable in die aktuelle Struktur zu übernehmen.

%! language = Onj
\begin{codeBlock}{onj}{Beispiel: Triple-Dot Syntax in Onj}
var fruits = ["apple", "pear"];
fruitSalad: [
    "banana",
    "grape",
    ...fruits
],
\end{codeBlock}

Statt bei einem Zugriff einfach den Key-Namen oder den Index anzugeben, kann auch in Klammern eine Expression angegeben
werden.
Diese wird evaluiert und das Resultat wird für den Zugriff verwendet.

%! language = Onj
\begin{codeBlock}{onj}{Demo: dynamische Zugriffe in Onj}
var index = 1;
var arr = [0, 1, 2, 3, 4, 5];
myValue: arr.(index + 2) // => 3
\end{codeBlock}

% resets author
\renewcommand{\kapitelautor}{}
