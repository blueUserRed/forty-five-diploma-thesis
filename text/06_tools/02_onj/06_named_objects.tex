
\subsection{Named Objects}\label{subsec:named-objects}

\renewcommand{\kapitelautor}{Autor: Marvin Kurka}

Bei der Validation von komplexen Onj-Strukturen kann es unter gewissen Umständen zu Problemen kommen, vor allem wenn
an einer Stelle mehrere verschiedene Objekte möglich sind.
Um das Problem besser erkenntlich zu machen, hier ein Beispiel:

%! language = Onj
\begin{codeBlock}{onj}{Beispiel: Schlechte Umsetzung einer UI-Struktur 1}
// onj struktur
uiElements: [
    {
        type: "label",
        text: "Hello World",
        font: "red wing",
    },
    {
        type: "image",
        path: "./some/image.png"
    }
]
\end{codeBlock}

\begin{codeBlock}{onj}{Beispiel: Schlechte Umsetzung einer UI-Struktur 2}
// onj schema
var uiElement = {
    type: string,
    ...*
};
uiElements: uiElement[]
\end{codeBlock}

Im Schema kann nur garantiert werden, das ein UI-Element einen \inlineOnj{type}-Key hat, alles andere wird offen
gelassen.
Das macht die Schema-Validierung in diesem Fall praktisch nutzlos.
Die Lösung, die Onj für dieses Problem bietet, sind Named Objects.
Im Schema kann eine Named Object Group definiert werden, die mehrere Named Objects enthält.
Jedes Named Object hat einen Namen, über den es identifiziert wird und kann beliebige Keys mit beliebigen Typen
definieren.
Im Schema kann dann der Name der Named Object Group als Typ verwendet werden, der alle Named Objects, die Teil der Group
sind, erlaubt.

Hier noch einmal das selbe Beispiel, implementiert mit Named Objects:

%! language = Onj
\begin{codeBlock}{onj}{Beispiel: Gute Umsetzung einer UI-Struktur 1}
// onj struktur
uiElements: [
    $Label {
        text: "Hello World",
        font: "red wing",
    },
    $Image {
        path: "./some/image.png"
    }
]
\end{codeBlock}

\begin{codeBlock}{onj}{Beispiel: Gute Umsetzung einer UI-Struktur 2}
// onj schema
$UiElement {
    $Label {
        text: string,
        font: string
    }
    $Image {
        path: string
    }
}
uiElements: $UiElement[]
\end{codeBlock}

% resets author
\renewcommand{\kapitelautor}{}
