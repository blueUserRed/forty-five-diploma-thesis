
\subsection{Namespaces}\label{subsec:namespaces}

\renewcommand{\kapitelautor}{Autor: Marvin Kurka}

Namespaces erlauben es, Onj zu erweitern, indem eigene Funktionen, zusätzliche globale Variablen oder
eigene Datentypen definiert werden.
Solche Namespaces werden in Kotlin definiert und mit Annotations markiert.

Funktionen werden im Namespace definiert und mit der \inlineKotlin{@RegisterOnjFunction} markiert.
Diese Annotation nimmt ein String mit einem OnjSchema als Parameter, der die Signatur der Funktion beschreibt.
Das ist notwendig, da Onj das Überladen von Funktionen erlaubt, \dah mehrere Funktionen dürfen denselben Namen haben.
Um herauszufinden, welche Funktion tatsächlich aufgerufen werden soll, vergleicht der Onj-Parser die mitgegebenen
Parameter mit dem Schema und ruft die erste Funktion auf, bei der diese übereinstimmen.

%! language = Kotlin
\begin{codeBlock}{kotlin}{Demo: Deklaration einer Onj-Funktion in Kotlin}
@OnjNamespace
object MyNamespace {

    @RegisterOnjFunction(schema = "params: [string, int]")
    fun repeatString(s: OnjString, times: OnjInt) = OnjString(s.value.repeat(times.value))
}
\end{codeBlock}

Neben normalen Funktionen stellt Onj noch drei spezielle Arten von Funktionen zur Verfügung:

\begin{itemize}
    \item Conversions: Diese werden mit folgendem Syntax aufgerufen: \inlineOnj{value#function}.
        Solche Funktion werden zum Beispiel verwendet, um einen Wert von einem Datentypen zu einem anderen zu
        konvertieren.
    \item Infix Funktionen: Diese werden mit folgendem Syntax aufgerufen: \inlineOnj{value1 function value2}.
        Beispiele sind die pow-Funktion (\inlineOnj{10 pow 5}) oder die in-Funktion (\inlineOnj{3 in [1, 2, 3, 4]}).
    \item Operator Overloading: Solche Funktionen erlauben es zu definieren, wie sich Operatoren wie
        \inlineOnj{+} oder \inlineOnj{*} für eigene Datentypen verhalten.
\end{itemize}

Wird so eine spezielle Funktion verwendet, wird das in der \inlineKotlin{@RegisterOnjFunction} Annotation angegeben.

%! language = Kotlin
\begin{codeBlock}{kotlin}{Demo: Deklaration einer Onj-Conversion in Kotlin}
@OnjNamespace
object MyNamespace {

    @RegisterOnjFunction(schema = "params: [string]", type = OnjFunctionType.CONVERSION)
    fun greeting(name: OnjString) = OnjString("Hello, ${name.value}")
}
\end{codeBlock}

Weiters können Namespaces globale Variablen definieren.

%! language = Kotlin
\begin{codeBlock}{kotlin}{Demo: Deklaration von globalen Onj-Variablen in Kotlin}
@OnjNamespace
object MyNamespace {

    @OnjNamespaceVariables
    val variables: Map<String, OnjValue> = mapOf(
        "myGlobal" to OnjInt(5)
    )
}
\end{codeBlock}

In Onj kann ein use-Statement verwendet werden, um einen Namespace zu inkludieren.

%! language = Onj
\begin{codeBlock}{onj}{Demo: Verwendung eines Onj-Namespaces}
use MyNamespace;

global: myGlobal,
greeting: "Reader"#greeting,
string: repeatString("a", 5)
\end{codeBlock}

% resets author
\renewcommand{\kapitelautor}{}
