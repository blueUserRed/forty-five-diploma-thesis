
\subsection{Scene2D}\label{subsec:scene2D}

\renewcommand{\kapitelautor}{Autor: Marvin Kurka}

LibGdx stellt das Scene2D-Package zur Verfügung, dass das Bauen von UIs vereinfacht.
Es beinhaltet Features die oft für UIs benötigt werden, ein Event-System, Hit-Detection, Viewports und Draw-Logik.
Außerdem enthält es häufig gebrauchte Widgets, wie H- oder V-Boxen, Progress-Bars oder Listen.\zit{libGdxScene2DUi}

Die Basisklasse, für ein Element, dass auf dem Screen sichtbar ist, ist die Actor-Klasse, die die oben genannten
Funktionen wie Event-Handling implementiert.
Weiters hat LibGdx die Widget-Klasse, die von Actor erbt, die ausserdem Layout-Logik implementiert.
Alle UI-Elemente in \FF erben von Widget.\zit{libGdxScene2DUi}

Alle UI-Elemente sind Teil einer Scene.
Diese hat eine Kamera, verwaltet die Transformationen der UI und kann als InputProcessor verwendet werden, wodurch
auch Code außerhalb der Widgets auf Events der Scene reagieren kann.\zit{libGdxScene2D}

Viewports werden verwendet, um zu definieren, wie sich eine Scene verhält, wenn die Fenstergröße geändert wird.
Der ExtendViewport zum Beispiel, verlängert, falls notwendig, eine Achse des Koordinatensystems, um das Fenster zu füllen.
Der FitViewport allerdings, der auch von \FF verwendet wird, lässt die Größe der Stage gleich, und fügt schwarze Balken
hinzu um das Fenster zu füllen.\zit{libGdxViewports}


% resets author
\renewcommand{\kapitelautor}{}
