\renewcommand{\kapitelautor}{Autor: Felix Zwickelstorfer}

\section{Git}\label{sec:git}

Es gibt diverse VCS (Version Control System) wie zum Beispiel Git oder Mercurial.\cite{}
Dieses ermöglicht es, Änderungen an Dateien und Ordnern zu speichern.
Dadurch kann man wieder zu einem vorherigen "commit" zurückgehen, falls man etwas ungewollt geändert hat.

In diesem Projekt wurde Git verwendet, da es einerseits das meistverbreitete VCS ist und andererseits alle Projektmitglieder bereits Erfahrung damit hatten.

Bei Git sind alle Daten in einem Repository, welcher der Überordner aller Dateien ist.
In diesem befindet sich auch der ".git" Ordner, in dem alle relevanten Daten über das Repository gespeichert sind.
Dieses beinhaltet \zB lokale Commits und Branches, aber auch die Adresse des Online-Servers.
Ein Commit speichert die Änderungen seit dem letzten Commit.
Dadurch verbraucht das System weniger Speicher, als wenn es immer alles speichern würde.
Beim commit gibt man auch eine message an, welche beschreibt, welche Änderungen durchgeführt wurden.
Eine Branch ist eine neue separate Entwicklungslinie, das heißt, dass alle Commits nur in dieser angezeigt werden.
Dies wird vor allem verwendet, um neue Features hinzuzufügen, da dies meistens länger dauert, und eventuell unvollständiger Code "gepushed" wird,
der anderen Entwicklern Probleme verursachen könnte.
Wenn ein Feature funktioniert, kann man ein "pull request" erfordern, \dah, dass ein anderer Entwickler den Code durchgeht und sicherstellt,
dass der Code den Richtlinien entspricht.
Danach kann man Branches mergen, also zusammenführen.

Ein "push" beschreibt das Hochladen aller lokalen Commits auf den Server.
Das Gegenstück, also das Herunterladen der Daten, heißt "pull".
Da man nicht alle Daten auf den Server laden will, da diese eventuell von Gerät zu Gerät anders sein können, wie zum Beispiel die lokale Konfiguration,
gibt es die Möglichkeit, bestimmte Dateien zu ignorieren, indem man den relativen Pfad in die ".gitignore" Datei schreibt.
Bei einem pull, push, oder merge kann es zu einem "conflict" kommen, falls zwei verschieden Versionen eine Änderung in der gleichen Zeile haben.
Diese muss man manuell lösen.
Dabei werden beide Versionen des Codeabschnitts in die Datei geschrieben, und ein Entwickler muss die falsche Version löschen und einen neuen commit machen.\cite{gitHomePage}
\subsection{Workflow}\label{subsec:workflow}

\renewcommand{\kapitelautor}{Autor: Felix Zwickelstorfer}
Der git-workflow in dem Projekt wurde nach den folgenden Punkten umgesetzt:
\begin{enumerate}
    \item Beim Arbeiten an einer User-Story erstellt man eine branch namens "ff-xxx-name-of-user-story"
    \item Dabei sollte man nach folgenden Punkten commiten:
    \begin{itemize}
        \item Ein Teil eines Features ist fertig.
        \item Der Code ist ausführbar.
            Falls dies nicht der Fall ist, muss man es in der commit-message angeben.
        \item Falls man sich unsicher ist, ist es besser, öfter einen commit zu machen.
    \end{itemize}
    \item Nach der Fertigstellung erstellt man einen pull-request, beginnend mit "[ff-xxx]", welche sich jemand anders ansieht.
    \item Der zweite Entwickler schreibt dem Programmierer Kommentare zu Codestücken, welche nicht optimal sind.
    \item Der Programmier bessert den Code aus und eröffnet wieder eine pull-request.
        Dieser Vorgang wird so oft wiederholt, bis der zweite Entwickler die pull-request approved und diese anschließend auf die branch "main" gemerged wird.
\end{enumerate}


% resets author
\renewcommand{\kapitelautor}{}
\subsection{Github}\label{subsec:github}

%
% text goes here
%
