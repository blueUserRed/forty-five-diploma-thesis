\renewcommand{\kapitelautor}{Autor: Felix Zwickelstorfer}

\section{Git}\label{sec:git}

Es gibt diverse VCS (Version Control System) wie zum Beispiel Git oder Mercurial.
\begin{coolQuote}Versionsverwaltung ist ein System, welches die Änderungen an einer oder einer Reihe von Dateien über die Zeit hinweg protokolliert, sodass man später auf eine bestimmte Version zurückgreifen kann.
 ist ein System, welches die Änderungen an einer oder einer Reihe von Dateien über die Zeit hinweg protokolliert, sodass man später auf eine bestimmte Version zurückgreifen kann.
Die Dateien, die in den Beispielen in diesem Buch unter Versionsverwaltung gestellt werden, enthalten Quelltext von Software.
Tatsächlich kann in der Praxis nahezu jede Art von Datei per Versionsverwaltung nachverfolgt werden.
\end{coolQuote}\zid{gitVSC}

In diesem Projekt wurde Git verwendet, da es einerseits das meistverbreitete VCS ist und andererseits alle Projektmitglieder bereits Erfahrung damit hatten.

\renewcommand{\kapitelautor}{Autor: Felix Zwickelstorfer}
\subsection{Vokabular}\label{subsec:git-vokabular}


\renewcommand{\kapitelautor}{Autor: Felix Zwickelstorfer}

Bei Git gibt es mehrere Fachtermini, die in den folgenden Kapiteln verwendet werden.
Deshalb werden nun die wichtigsten Begriffe erläutert:

\begin{liste}
    \item repository: Der Bereich, in dem alle Dateien gespeichert sind, die versioniert werden sollen.
    \item commit: Erstellt eine Version des repository, wobei die Daten inkrementell gespeichert sind, \dah, nur die Änderungen.
    \item message: Beschreibung der Änderungen in einem commit.
    \item push/pull: Das Hoch- und Herunterladen aller commits auf \bzw von einem Server.
    \item main/main-branch: Die Hauptentwicklungslinie, die alle fertigen Features beinhaltet und auch für "builds" (=eine fertige Applikation, die man auch auf Steam hochladen kann) verwendet wird.
    \item branch: Eine separate Entwicklungslinie, um größere Änderungen meist über einen längeren Zeitraum getrennt von der main-branch zu implementieren.
    Diese werden vor allem verwendet, um Features hinzuzufügen.
    \item merge: Das Zusammenfügen von zwei Branches.
    \item pull/merge request: Wenn ein merge stattfinden soll, eröffnet man eine pull request.
    Danach prüft ein anderer Entwickler die Änderungen und kann entweder die request akzeptieren, wodurch der merge stattfindet, oder ablehnen.
    Dieses Feature wird verwendet, um die Codequalität sicherzustellen.
    \item merge conflict: Ein Konflikt tritt auf, wenn Git nicht automatisch entscheiden kann, welche Änderungen bei einem merge übernommen werden sollen.
    Dies kann geschehen, wenn in mehreren Branches der Code in derselben Datei geändert wird.\zit{gitHomePage}
\end{liste}

\renewcommand{\kapitelautor}{Autor: Felix Zwickelstorfer}
\subsection{Workflow}\label{subsec:workflow}

Der Git-workflow für Forty-Five wurde nach den folgenden Punkten umgesetzt:
\begin{enumerate}
    \item Beim Arbeiten an einer User-Story erstellt man eine branch namens ``ff-xxx-name-of-user-story''.
    \item Dabei sollte man nach folgenden Kriterien einen commit erstellen:
    \begin{liste}
        \item Ein Teil eines features ist fertig.
        \item Der Code ist ausführbar.
        Falls dies nicht der Fall ist, muss man dies in der commit-message angeben.
    \end{liste}
    \item Nach der Fertigstellung erstellt man einen pull-request, beginnend mit \quoted{[ff-xxx]}.
    Danach wird automatisch der andere Entwickler benachrichtigt.
    \item Der andere Entwickler schreibt Kommentare zu Codestücken, die nicht optimal sind.
    \item Der Programmierer verbessert den Code und eröffnet wieder eine pull-request.
    Dieser Vorgang wird so oft wiederholt, bis der andere Entwickler die pull-request akzeptiert und diese anschließend auf die branch main gemerged wird.
\end{enumerate}

Es gibt viele mögliche Erweiterungen für das System.
Ein Beispiel wäre das Hinzufügen von unit-tests, wodurch sichergestellt werden würde, dass Methoden genau so funktionieren, wie sie sollen.
Eine weitere Möglichkeit wären end-to-end Tests, bei denen sich ein Programm im GUI (graphical user interface) durchklickt und so das gesamte Programm aus Sicht des Endnutzers testet.
In \FF wurden diese Erweiterungen nicht eingesetzt, da der Arbeitsaufwand für das Schreiben der Tests zu groß wäre im Vergleich zu dem Nutzen, den sie bringen würden.
Weiterhin werden durch pull-requests einzelne Methoden vom jeweils anderen Entwickler überprüft, wodurch zumindest eine minimale Überprüfungsinstanz vorhanden ist.
Zusätzlich dazu wird jedes hinzugefügte feature vorher von dem Programmierer so gut wie möglich getestet, um die Gefahr zu verringern, dass bugs unbemerkt veröffentlicht werden.

\renewcommand{\kapitelautor}{Autor: Felix Zwickelstorfer}
\subsection{Github}\label{subsec:github}


Github ist ein öffentliches Webservice von Github Inc., welcher kostenlos git-server Dienste anbietet.
Es vereinfacht die Verwaltung von Projekten und Repositorys mit einem User Inferface und ermöglicht es auch Analysen über Projekte anzuzeigen.
Weiteres bietet es die Möglichkeit, Repositorys öffentlich zugänglich zu machen, wodurch jeder den Code herunterladen und auch erweitern kann.
Diese Erweiterungen müssen je nach Einstellung des Projektes von dem Besitzer des Repositorys genehmigt werden.

\renewcommand{\kapitelautor}{Autor: Felix Zwickelstorfer}
\textbf{Issues}

Issues sind ein Feature von Github, bei dem Bugs, Verbesserungsvorschläge oder Feedback für ein Repository auf Github veröffentlicht werden kann.
Dies ist hilfreich, da Entwickler alle Ideen gesammelt an einem Punkt finden können.
Man kann einem issue auch diverse Labels geben wie \zB "Bug", um schneller an die gewollten Informationen zu kommen.
Wenn ein Entwickler ein issue erledigt hat, kann er dieses "closen", um dem Verfasser zu zeigen, dass es bearbeitet wurde.
