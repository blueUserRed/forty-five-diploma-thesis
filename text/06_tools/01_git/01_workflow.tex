\subsection{Workflow}\label{subsec:workflow}

\renewcommand{\kapitelautor}{Autor: Felix Zwickelstorfer}
Der git-workflow in dem Projekt wurde nach den folgenden Punkten umgesetzt:
\begin{enumerate}
    \item Beim Arbeiten an einer User-Story erstellt man eine branch namens "ff-xxx-name-of-user-story"
    \item Dabei sollte man nach folgenden Punkten commiten:
    \begin{itemize}
        \item Ein Teil eines Features ist fertig.
        \item Der Code ist ausführbar.
            Falls dies nicht der Fall ist, muss man es in der commit-message angeben.
        \item Falls man sich unsicher ist, ist es besser, öfter einen commit zu machen.
    \end{itemize}
    \item Nach der Fertigstellung erstellt man einen pull-request, beginnend mit "[ff-xxx]", welche sich jemand anders ansieht.
    \item Der zweite Entwickler schreibt dem Programmierer Kommentare zu Codestücken, welche nicht optimal sind.
    \item Der Programmier bessert den Code aus und eröffnet wieder eine pull-request.
        Dieser Vorgang wird so oft wiederholt, bis der zweite Entwickler die pull-request approved und diese anschließend auf die branch "main" gemerged wird.
\end{enumerate}


% resets author
\renewcommand{\kapitelautor}{}