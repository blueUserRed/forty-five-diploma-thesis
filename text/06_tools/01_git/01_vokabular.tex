\renewcommand{\kapitelautor}{Autor: Felix Zwickelstorfer}
\subsection{Vokabular}\label{subsec:git-vokabular}


\renewcommand{\kapitelautor}{Autor: Felix Zwickelstorfer}

Bei git gibt es mehrere Fachtermini, welche in den folgenden Kapitel verwendet werden.
Deshalb werden nun die wichtigsten Begriffe erläutert:

\begin{itemize}
    \item Repository: Der Bereich, in dem alle Dateien gespeichert sind, welche versioniert werden sollen.
    \item Commit: Erstellt eine Version des repository, wobei die Daten inkrementell gespeichert sind, \dah nur die Änderungen.
    \item Message: Beschreibung der Änderungen in einem commit.
    \item Push/Pull: Das Hoch- und Herunterladen aller commit auf \bzw von einen Server.
    \item Main/Main-branch: Die Hauptentwicklungslinie, welche alle fertigen Features beinhaltet und auch für "builds" (fertige Applikation, die man auch auf Steam hochlädt) verwendet wird.
    \item Branch: Eine separate Entwicklungslinie um größere Änderungen meist über einen längeren Zeitraum getrennt von main zu machen.
    Diese werden vor allem verwendet um Features hinzuzufügen.
    \item Merge: Das Zusammenfügen von zwei Branches.
    \item Pull/Merge request: Wenn ein Merge stattfinden soll, eröffnet man eine Pull Request.
    Danach schaut sich diese ein anderer Entwickler an und kann entweder akzeptieren, wodurch der Merge stattfindet, oder ablehnen.
    Dieses Feature wird verwendet, um die Codequalität sicherzustellen.
    \item Merge conflict: Ein Conflict passiert, wenn git nicht von alleine weiß, welche Änderungen bei einem Merge übernommen werden soll.
    Dies kann passieren, wenn in mehreren Branches Code in der gleichen Datei geändert wird.\cite{gitHomePage}
\end{itemize}

%Bei Git sind alle Daten in einem Repository, welcher der Überordner aller Dateien ist.
%In diesem befindet sich auch der ".git" Ordner, in dem alle relevanten Daten über das Repository gespeichert sind.
%Dieses beinhaltet \zB lokale Commits und Branches, aber auch die Adresse des Online-Servers.
%Ein Commit speichert die Änderungen seit dem letzten Commit.
%Dadurch verbraucht das System weniger Speicher, als wenn es immer alles speichern würde.
%Beim commit gibt man auch eine message an, welche beschreibt, welche Änderungen durchgeführt wurden.
%Eine Branch ist eine neue separate Entwicklungslinie, das heißt, dass alle Commits nur in dieser angezeigt werden.
%Dies wird vor allem verwendet, um neue Features hinzuzufügen, da dies meistens länger dauert, und eventuell unvollständiger Code "gepushed" wird,
%der anderen Entwicklern Probleme verursachen könnte.
%Wenn ein Feature funktioniert, kann man ein "pull request" erfordern, \dah, dass ein anderer Entwickler den Code durchgeht und sicherstellt,
%dass der Code den Richtlinien entspricht.
%Danach kann man Branches mergen, also zusammenführen.

%Ein "push" beschreibt das Hochladen aller lokalen Commits auf den Server.
%Das Gegenstück, also das Herunterladen der Daten, heißt "pull".
%Da man nicht alle Daten auf den Server laden will, da diese eventuell von Gerät zu Gerät anders sein können, wie zum Beispiel die lokale Konfiguration,
%gibt es die Möglichkeit, bestimmte Dateien zu ignorieren, indem man den relativen Pfad in die ".gitignore" Datei schreibt.
%Bei einem pull, push, oder merge kann es zu einem "conflict" kommen, falls zwei verschieden Versionen eine Änderung in der gleichen Zeile haben.
%Diese muss man manuell lösen.
%Dabei werden beide Versionen des Codeabschnitts in die Datei geschrieben, und ein Entwickler muss die falsche Version löschen und einen neuen commit machen.\cite{gitHomePage}.