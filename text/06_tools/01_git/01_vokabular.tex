\renewcommand{\kapitelautor}{Autor: Felix Zwickelstorfer}
\subsection{Vokabular}\label{subsec:git-vokabular}


\renewcommand{\kapitelautor}{Autor: Felix Zwickelstorfer}

Bei Git gibt es mehrere Fachtermini, die in den folgenden Kapiteln verwendet werden.
Deshalb werden nun die wichtigsten Begriffe erläutert:

\begin{liste}
    \item repository: Der Bereich, in dem alle Dateien gespeichert sind, die versioniert werden sollen.
    \item commit: Erstellt eine Version des repository, wobei die Daten inkrementell gespeichert sind, \dah, nur die Änderungen.
    \item message: Beschreibung der Änderungen in einem commit.
    \item push/pull: Das Hoch- und Herunterladen aller commits auf \bzw von einem Server.
    \item main/main-branch: Die Hauptentwicklungslinie, die alle fertigen Features beinhaltet und auch für "builds" (=eine fertige Applikation, die man auch auf Steam hochladen kann) verwendet wird.
    \item branch: Eine separate Entwicklungslinie, um größere Änderungen meist über einen längeren Zeitraum getrennt von der main-branch zu implementieren.
    Diese werden vor allem verwendet, um Features hinzuzufügen.
    \item merge: Das Zusammenfügen von zwei Branches.
    \item pull/merge request: Wenn ein merge stattfinden soll, eröffnet man eine pull request.
    Danach prüft ein anderer Entwickler die Änderungen und kann entweder die request akzeptieren, wodurch der merge stattfindet, oder ablehnen.
    Dieses Feature wird verwendet, um die Codequalität sicherzustellen.
    \item merge conflict: Ein Konflikt tritt auf, wenn Git nicht automatisch entscheiden kann, welche Änderungen bei einem merge übernommen werden sollen.
    Dies kann geschehen, wenn in mehreren Branches der Code in derselben Datei geändert wird.\zit{gitHomePage}
\end{liste}
