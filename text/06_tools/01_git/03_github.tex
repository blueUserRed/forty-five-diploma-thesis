\renewcommand{\kapitelautor}{Autor: Felix Zwickelstorfer}
\subsection{Github}\label{subsec:github}

GitHub ist ein öffentliches Webservice von GitHub Inc., welcher kostenlos git-server Dienste anbietet.
Es vereinfacht die Verwaltung von Projekten und repositories mit einem user-interface und ermöglicht es auch, Analysen über Projekte anzuzeigen.
Weiterhin bietet es die Möglichkeit, repositories öffentlich zugänglich zu machen, wodurch jeder den Code herunterladen und auch erweitern kann.
Diese Erweiterungen müssen, je nach Einstellung des Projektes, von dem Besitzer des repositories genehmigt werden.

\renewcommand{\kapitelautor}{Autor: Felix Zwickelstorfer}
\subsubsection{Issues}\label{subsubsec:issues}

Issues sind ein feature von GitHub, bei dem bugs, Verbesserungsvorschläge oder Feedback für ein repository auf GitHub veröffentlicht werden können.
Dies ist hilfreich, da Entwickler alle Ideen gesammelt an einem Ort finden können.
Man kann einem issue auch diverse labels geben, wie \zB \quoted{Bug}, um schneller an die gewollten Informationen zu kommen.
Wenn ein Entwickler ein issue erledigt hat, kann er den Status auf \quoted{closed} setzten, um dem Verfasser zu zeigen, dass es bearbeitet wurde.