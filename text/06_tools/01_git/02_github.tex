\renewcommand{\kapitelautor}{Autor: Felix Zwickelstorfer}
\subsection{Github}\label{subsec:github}


Github ist ein öffentliches Webservice von Github Inc., welcher kostenlos git-server Dienste anbietet.
Es vereinfacht die Verwaltung von Projekten und Repositorys mit einem User Inferface und ermöglicht es auch Analysen über Projekte anzuzeigen.
Weiteres bietet es die Möglichkeit, Repositorys öffentlich zugänglich zu machen, wodurch jeder den Code herunterladen und auch erweitern kann.
Diese Erweiterungen müssen je nach Einstellung des Projektes von dem Besitzer des Repositorys genehmigt werden.

\renewcommand{\kapitelautor}{Autor: Felix Zwickelstorfer}
\textbf{Issues}

Issues sind ein Feature von Github, bei dem Bugs, Verbesserungsvorschläge oder Feedback für ein Repository auf Github veröffentlicht werden kann.
Dies ist hilfreich, da Entwickler alle Ideen gesammelt an einem Punkt finden können.
Man kann einem issue auch diverse Labels geben wie \zB "Bug", um schneller an die gewollten Informationen zu kommen.
Wenn ein Entwickler ein issue erledigt hat, kann er dieses "closen", um dem Verfasser zu zeigen, dass es bearbeitet wurde.