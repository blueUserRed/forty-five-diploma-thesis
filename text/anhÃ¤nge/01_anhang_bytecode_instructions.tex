
\chapter{Anhang 1 - Erklärung Bytecode Instruktionen}\label{ch:anhang-1}

\renewcommand{\kapitelautor}{Autor: Marvin Kurka}

Zuerst, hier nochmal der Code-Block:

%! language = kotlin
\begin{minted}{kotlin}
fun runLambda(lambda: () -> Unit) = lambda()
inline fun runLambdaInline(lambda: () -> Unit) = lambda()

fun test() {
    runLambda {
        println("runLambda")
    }
    runLambdaInline {
        println("runLambdaInline")
    }
}
\end{minted}

Und der Byte-Code:

\begin{minted}{text}
 0: getstatic     #34                 // Field TestKt$test$1.INSTANCE:LTestKt$test$1;
 3: checkcast     #18                 // class kotlin/jvm/functions/Function0
 6: invokestatic  #36                 // Method runLambda:(Lkotlin/jvm/functions/Function0;)V
 9: iconst_0
10: istore_0
11: iconst_0
12: istore_1
13: ldc           #37                 // String runLambdaInline
15: getstatic     #43                 // Field java/lang/System.out:Ljava/io/PrintStream;
18: swap
19: invokevirtual #49                 // Method java/io/PrintStream.println:(Ljava/lang/Object;)V
22: nop
23: nop
24: return
\end{minted}

Zwei Stellen im Bytecode ergeben auf den ersten Blick nicht wirklich Sinn, nämlich 9--12 und 22--23.
Diese sind hier erklärt.

\section{Stelle 1: 9-12}

\begin{minted}{text}
 9: iconst_0
10: istore_0
11: iconst_0
12: istore_1
\end{minted}

Hier schreibt der Compiler 0 in die Register 0 und 1, liest die Werte aber nie aus.
Ein Blick in den LVT der Funktion liefert einen Hinweis:

\begin{minted}{text}
LocalVariableTable:
    Start  Length  Slot  Name   Signature
    13      10     1 $i$a$-runLambdaInline-TestKt$test$2   I
    11      13     0 $i$f$runLambdaInline   I
\end{minted}

Der LVT oder Local Variable Table speichert Informationen zu den lokalen Variablen einer Funktion, die normalerweise
bei der Kompilation verloren gehen würden.
Er ist optional und nicht für die korrekte Ausführung des Codes notwendig, kann aber von
Debuggern verwendet werden.\zit{jvmspecLVT}

Da durch das inlinen der Funktion kein Frame am Callstack generiert wird, geht nützliche Information im Stacktrace
verloren.
Diese lokalen Variablen speichern Informationen zu aufgerufenen inline-Funktionen in ihren Namen, die dann von
Debugging-Tools ausgelesen werden können.\zit{youTrackFakeVariables}

\section{Stelle 2: 22-23}

Hier generiert der Compiler zwei nicht notwendige nop-Instruktionen.
Wie sich herausstellt, hat auch dieser Bytecode mit inline-Funktionen zu tun.
Unter gewissen Umständen, \zB wenn eine inline-Funktion nur eine andere inline-Funktion aufruft,
wird für die erste Funktion kein Bytecode generiert.
Das führt zu mehreren Problemen, zum Beispiel kann dadurch keine Assoziation zu einer Zeilennummer hergestellt werden,
da Informationen im LineNumberTable immer mit einem Bytecode-Offset verknüpft sind.
Außerdem führt das zu Problemen beim debuggen, da auch Breakpoints immer auf eine Instruktion im Bytecode verweisen.
Deswegen wird für eine inline-Funktion immer ein \inlineCode{nop} ausgegeben, um zu garantieren, dass
zumindest eine Instruktion für jede inline-Funktion ausgegeben wird.
Im oberen Code sind zwei \inlineCode{nop} Instruktionen vorhanden, da in Kotlin \inlineKotlin{println} auch eine
inline-Funktion ist.\zit{youTrackNops}

% resets author
\renewcommand{\kapitelautor}{}
