
\section{Effektsystem}\label{sec:effectsystem}

\renewcommand{\kapitelautor}{Autor: Marvin Kurka}

\FF zeichnet sich vor allem durch die Effekte der Karten aus, die es dem Spieler ermöglichen, komplexe Kombinationen
von Karten zu spielen, um verschiedene Effektketten auszulösen.
Um es möglich zu machen, solche Effekte schnell und einfach zu definieren, wurde ein System entwickelt, dass es
ermöglichen soll, neue Effekte zu definieren ohne viel in vorhandenen Code eingreifen zu müssen.

Effekte werden in zwei Kategorien eingeteilt: Trait Effekte und normale Effekte.
Trait Effekte werden nicht von dem Effektsystem verwaltet und werden für alle Situationen verwendet, in denen das
normale Effektsystem nicht ausreicht.
So gibt es zum Beispiel den ``reinforced`` Trait-Effekt, der dafür sorgt, dass eine Bullet immer den ganzen Damage
des Gegners parried, unabhängig von ihrem eigenen Damage-Wert.
Da diese Logik von dem Parry-System übernommen werden muss, ist dieser Effekt ein Trait-Effekt und wird somit nicht
über das normale Effektsystem definiert.

Normale Effekte bestehen aus zwei Teilen: dem Trigger und dem Effekt selber.
Der Trigger gibt an, wann der Effekt aktiv wird.
Beispiele sind: \inlineKotlin{ON_SHOT}, \inlineKotlin{ON_ENTER} oder \inlineKotlin{ON_REVOLVER_ROTATION}.
Jedes Mal, wenn der GameController eine Aktion ausführt, die Effekte auslösen könnte, werden alle relevanten Karten
im Spiel auf den zu der Aktion passenden Trigger überprüft.
Trigger können auch kaskadieren, das heißt, wenn \zB die \inlineKotlin{ON_BOUNCE} oder \inlineKotlin{ON_DESTROY}
Trigger ausgelöst werden, wird automatisch auch der \inlineKotlin{ON_LEAVE} Trigger ausgelöst.
Definiert eine Karte einen oder mehrere Effekte für diesen Trigger, werden diese aktiv.

Die Effekte machen starken Gebrauch von Timelines, die in Kapitel~\ref{subsec:timelines} erklärt sind.
Wird ein Effekt ausgelöst, wird die \inlineKotlin{Effect::onTrigger} Funktion aufgerufen, die eine Timeline zurückgibt.
Diese wird in zusammen mit anderen Timelines, die \zB von anderen Effekten stammen, in eine Timeline eingearbeitet,
die dann ausgeführt wird.
Um das Spiel zu beeinflussen, kann der Effekt einige Funktionen aus dem GameController verwenden, um Timelines
zu bekommen, die \zB den Revolver drehen, oder eine Karte zerstören.

Effekte, die sich auf andere Bullets im Revolver beziehen, verwenden die \inlineKotlin{BulletSelector} Klasse um
auszuwählen, auf welche Bullets der Effekt angewandt wird.
Je nach Definition des Effekts wird dann ein Popup geöffnet, das den Spieler nach einer Target-Bullet für den Effekt
fragt, oder ein anderes Kriterium wird verwendet, um festzustellen, welche Bullets betroffen sind.

Effekte, die Damage-Werte von anderen Karten im Revolver verändern, verwenden die \inlineKotlin{Card.CardModifier}
Klasse.
Diese erlaubt es eine temporäre Damage-Änderung zu definieren, entweder in der Form einer Addition oder der eines
multiplikativen Faktors.
Über den Konstruktor dieser Klasse können auch Lambdas mitgegeben werden, die bestimmen, wann der Modifier wieder
entfernt wird, oder in welchem Zeitraum er gültig ist.
Die Leaders Bullet zum Beispiel gibt allen anderen Karten im Revolver einen Modifier, der abläuft, wenn Leaders Bullet
den Revolver verlässt.
Die +P+ Bullet verwendet dieses System, um sich selbst einen Modifier zu geben, der nur aktiv ist, wenn der Gegner einen
Status Effekt hat.

Alle Karten im Spiel werden in \inlineCode{assets/config/cards.onj} definiert.
Diese Datei inkludiert den Cards-Namespace, der Funktionen zur Verfügung stellt um Effekte, Status-Effekte, und
Ähnliches zu definieren.
Eine Definition für eine Karte sieht zum Beispiel so aus:

%! language = Onj
\begin{codeBlock}{onj}{Beispiel: Definition einer Karte aus cards.onj}
    {
        name: "hollowpointBullet",
        title: "Hollowpoint Bullet",
        flavourText: "",
        description: "If the enemy has shield:\nThis Bullet has +6 dmg",
        baseDamage: 10,
        coverValue: 0,
        traitEffects: [],
        rotation: $Right { amount: 1 },
        highlightType: "standard",
        effects: [
            buffDmg("enter", bSelects.self, 6#val, $TargetedEnemyShieldIsAtLeast { value: 1 }#activeChecker)
        ],
        forceLoadCards: [],
        dark: false,
        cost: 1,
        type: $Bullet { },
        tags: ["pool1", "rarity2"],
        price: 0,
    },
\end{codeBlock}

Hier wird die \inlineOnj{buffDmg()} Funktion verwendet, um den Effekt der Karte zu definieren.
Der erste Parameter ist der Trigger, der zweite ist der \inlineKotlin{BulletSelector} und der dritte der Schadenswert.
Der Schadenswert ist vom Typ \inlineKotlin{EffectValue}, was ein typealias für
\inlineKotlin{(controller: GameController, card: Card?) -> Int} ist.
Den Wert in ein Lambda zu wrappen erlaubt es, den Wert von anderen Werten abhängig zu machen.
Die Glass Bullet verwendet dieses System zum Beispiel, um den Spieler so viele Karten ziehen zu lassen, wie die teuerste
Bullet im Revolver kostet.
Der letzte Parameter in der \inlineOnj{buffDmg()} Funktion ist ein ActiveChecker, der, wie schon vorher erklärt,
bestimmt, wann der Damage-Modifier aktiv ist.
In diesem Fall wird das \inlineKotlin{GamePredicate} \inlineKotlin{TargetedEnemyShieldIsAtLeast} verwendet, um
festzustellen ob der Enemy gerade einen Shield-Wert größer als null hat, und den Modifier nur dann zu aktivieren.

% resets author
\renewcommand{\kapitelautor}{}
