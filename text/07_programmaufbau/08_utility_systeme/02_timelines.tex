
\subsection{Timelines}\label{subsec:timelines}

\renewcommand{\kapitelautor}{Autor: Marvin Kurka}

Ein Encounter in \FF ist zu einem großen Teil aus linearen Abfolgen von Aktionen aufgebaut.

Ein Beispiel: Der Spieler drückt die "shoot"-Taste:
\begin{liste}
    \setlength\itemsep{0.2em}
    \item Die Karte im Slot 5 des Revolvers wird entfernt
    \item Dem Gegner wird Schaden hinzugefügt
    \item Die Schadensanimation des Gegners wird gestartet (und läuft ab dann parallel und selbstständig)
    \item Es wird überprüft, ob die geschossene Karte Effekte mit dem Trigger "ON\_SHOT" hat
    \item Wenn ja, werden diese ausgeführt
    \item Die Rotations-Animation des Revolvers wird abgespielt
    \item Es wird darauf gewartet, dass die Rotations-Animation fertig ist, bevor weitere Aktionen durchgeführt werden
\end{liste}

Dieses Beispiel ist etwas simplifiziert, in Wahrheit müssen auch Statuseffekte, Encounter Modifier und weitere Mechaniken
beachtet werden.

Um solche lineare Abfolgen einfach verwalten zu können, wurde die Timeline-Klasse entwickelt.
Diese Klasse speichert eine Serie an Aktionen und führt eine nach der anderen aus.
Eine Aktion kann dabei auch nachfolgende auch blockieren, zum Beispiel die Ausführung um eine Anzahl an
Millisekunden verzögern oder warten, bis eine Bedingung zutrifft.

Timelines können einfach mittels der \inlineKotlin{Timeline::timeline} Funktion erstellt werden.
Diese nimmt ein Builder-Lambda mit einer Instanz der \inlineKotlin{TimelineBuilderDSL} Klasse als Receiver, die
diverse Funktionen zur Verfügung stellt.

%! language = Kotlin
\begin{codeBlock}{kotlin}{Demo: Definition einer Timeline}
fun createTimeline() = Timeline.timeline {
    action {
        println("Hello")
    }
    delay(500)
    action {
        println("World")
    }
}
\end{codeBlock}

Mittels der \inlineKotlin{TimelineBuilderDSL::include} Funktion können auch die Aktionen anderer Timelines inkludiert
werden.
Hängt eine Sub-Timeline von Werten ab, die während des Bauens noch nicht bekannt sind, kann die
\inlineKotlin{TimelineBuilderDSL::includeLater} Funktion verwendet werden, die mit dem Erstellen der Sub-Timeline
wartet, bis die Aktion tatsächlich an der Reihe ist.
Mit der \inlineKotlin{TimelineBuilderDSL::delayUntil} Funktion kann die Ausführung der Timeline verzögert werden, bis
eine Bedingung zutrifft.

In der GameController Klasse, die große Teile des Encounters kontrolliert, sind viele Funktionen so aufgebaut, dass
sie statt eine gewisse Aufgabe zu erfüllen eine Timeline zurückgeben, die diese Aufgabe erfüllt.
Ein Beispiel ist die \inlineKotlin{GameController::rotateRevolver} Funktion, die, wenn sie aufgerufen wird, erst einmal
nichts tut.
Stattdessen gibt sie eine Timeline zurück, die erst wenn sie gestartet wird den Revolver rotiert.
Funktionen wie die \inlineKotlin{GameController::shoot} Funktion binden diese Timeline dann in ihre eigene ein.
Das erlaubt es dem GameController nach einem Userinput dynamisch eine Timeline aufzubauen, die Schritt für Schritt alle
notwendigen Aktionen durchführt.

% resets author
\renewcommand{\kapitelautor}{}
