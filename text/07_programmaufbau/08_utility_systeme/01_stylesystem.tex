
\subsection{Stylesystem}\label{subsec:stylesystem}

\renewcommand{\kapitelautor}{Autor: Marvin Kurka}

Wie im Kapitel~\ref{sec:screens} erklärt, sind Screens in Onj-Dateien definiert.
Dabei war es uns wichtig, Widgets schnell und einfach designen und auch simple Animationen hinzufügen zu können.
Deswegen wurde ein Style-System entwickelt, das es ermöglicht gewisse Attribute von Widgets über die Screen-Dateien
zu ändern, und dass auch abhängig von Bedingungen und animiert.

Jedes Widget, dass über dieses System gestyled werden kann, implementiert das StyledActor-Interface und
bekommt einen StyleManager vom ScreenBuilder zugewiesen.
Der StyleManager speichert die StyleProperties des Widgets.
StyleProperties sind Aspekte des Widgets, die das StyleSystem ändern kann, \zB den Hintergrund.

Jede StyleProperty hat verschiedene StyleInstructions, die versuchen der Property einen konkreten Wert zuzuweisen,
\zB \inlineOnj{background: "black_texture"} wäre eine StyleInstrunction für die background-StyleProperty.
Jede Instruction kann auch eine Bedingung haben, die angibt, ob sie gerade aktiv ist, \zB könnte eine Instruction
den Hintergrund nur dann setzen, wenn über das Widget gehovert wird.
Kommen mehrere Instructions für eine Property in Frage, wird eine Priority verwendet, um zu entscheiden, welche
Instruction gewinnt.

Um simple Animationen zu ermöglichen, wird eine AnimatedStyleInstruction, eine Subklasse von StyleInstruction, verwendet.
Diese speichert den Zeitpunkt, an dem sie die Kontrolle erlangt hat, und interpoliert basieren darauf ihren Wert.
Das funktioniert nur für numerische Werte.

Hier ein Beispiel aus \inlineCode{choose_card_screen.onj}, dass das System in Verwendung zeigt.

%! language = Onj
\begin{codeBlock}{onj}{Beispiel: Definition eines animierten Widgets aus chooseCardScreen.onj (gekürzt)}
styles: [
    // Die nächsten beiden Styles erzeugen eine animierte Bewegung, wenn ein anderes Widget über dieses gedragged wird
    {
        style_priority: 1,
        positionBottom: -5#percent,
        style_animation: {
            duration: 0.1,
            interpolation: interpolation.linear
        },
    },
    {
        style_priority: 2,
        style_condition: actorState("draggedHover"),
        positionBottom: 0#percent,
        style_animation: {
            duration: 0.1,
            interpolation: interpolation.linear
        },
    },
]
\end{codeBlock}

Die \inlineOnj{actorState}-Funktion fragt hier den Zustand des Widgets ab.
In welchen States ein Widget ist, wird vom Programm mittels den \inlineKotlin{StyledActor::enterActorState} und
\inlineKotlin{StyledActor::leaveActorState} Funktionen festgelegt.
Diese Bedingungen können auch verknüpft werden, \zB
\inlineOnj{(hover() and state("something")) or not(actorState("something else"))}.

Die hier verwendeten Funktionen und globalen Variablen sind in einem Namespace, nämlich dem
\inlineKotlin{StyleNamespace}, definiert.

% resets author
\renewcommand{\kapitelautor}{}
