%! Author = felix
%! Date = 17/03/2024
\renewcommand{\kapitelautor}{Autor: Felix Zwickelstorfer}
\subsection{Parameter}\label{subsec:parameter}
\renewcommand{\kapitelautor}{Autor: Felix Zwickelstorfer}

Jeder screen hat in Onj drei Bereiche, welcher die wichtigsten Informationen für diesen Screen beinhaltet.


\renewcommand{\kapitelautor}{Autor: Felix Zwickelstorfer}
\subsubsection{Assets}\label{subsubsec:assets}
\renewcommand{\kapitelautor}{Autor: Felix Zwickelstorfer}
Die Assets beschreiben, welche Ressourcen dieser Screen immer laden soll.
Bei Forty-Five werden normalerweise Ressourcen wie Bilder, Animationen als auch Schriftarten nur im Screen gespeichert.
Das heißt, dass man sie bei jedem Screen neu lädt.
Deshalb gibt man dem Screen eine Liste von assets mit, welche Ressourcen der Screen immer benötigt.
Diese beinhaltet \zB Schriftarten oder Hintergründe.
Manche Ressourcen, wie \zB Karten, werden erst vom Programm geladen, da es keinen Sinn ergeben würde, jederzeit alle Karten zu laden.
Darauf wird allerdings genauer bei~\ref{sec:resourcenmanagement} eingegangen.

\renewcommand{\kapitelautor}{Autor: Felix Zwickelstorfer}
\subsubsection{Viewport}\label{subsubsec:viewport}
\renewcommand{\kapitelautor}{Autor: Felix Zwickelstorfer}
Der Viewport beschreibt die logische Anzahl an "Points" pro Screen in der Breite und Höhe.
Man kann diese Points mit Pixeln vergleichen, es ist allerdings komplett von der eigentlichen Auflösung des Spiels getrennt.
Er wird verwendet, damit man als Entwickler Elementen bestimmte Größen im Verhältnis zum Screen geben kann.
Weiteres wird er verwendet, um bei Boxen die Kind-Elemente zu platzieren.
Diese werden immer nur auf ganzzahlige Points gesetzt, wodurch es bei zu niedriger logischer Auflösung zu nicht korrekt platzierten Elementen kommen kann.


\renewcommand{\kapitelautor}{Autor: Felix Zwickelstorfer}
\subsubsection{Optionen}\label{subsubsec:optionen}
\renewcommand{\kapitelautor}{Autor: Felix Zwickelstorfer}
Die Optionen sind der vermutlich wichtigste Parameter eines Screens.
Sie beinhalten einerseits Elemente wie den Hintergrund, welcher allerdings meistens vom root-element überschrieben wird als auch Parameter für die Musik.
Andererseits werden auch die InputMaps mitgegeben, welche für einfache User-Interaktionen zuständig sind.
Diese sind genauer in~\ref{sec:key-input-system} beschrieben.
Das wichtigste Element ist allerdings der Screen-Controller.
Dieser ist das zentrale Steuerelement eines Screens und wird bei~\ref{subsec:screen-controller} beschrieben.
Wie das folgende Beispiel zeigt, gibt man dem Screen-Controller die Namen bestimmter Elemente mit, welche dieser verwaltet.
Das erlaubt es, dass man die Elemente-Struktur des Screens ändern kann, aber trotzdem noch auf die richtigen Elemente im Code referenziert.

\begin{codeBlock}{onj}{Beispiel: Optionen des HealOrMapHP\-Screens}
options: {
    background: "hover_detail_background",
    transitionAwayTime: 1.5,
    music: "map_theme",
    playAmbientSounds: true,
    inputMap: [
        ...(inputMaps.defaultInputMap),
        ...(inputMaps.healOrMaxInputMap),
        ...(inputMaps.addMaxHPInputMap),
    ],
    screenController: $HealOrMaxHPScreenController {
        addLifeActorName: "add_lives_option",
    }
}
\end{codeBlock}