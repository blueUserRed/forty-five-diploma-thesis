%! Author = felix
%! Date = 17/03/2024
\renewcommand{\kapitelautor}{Autor: Felix Zwickelstorfer}
\subsection{Parameter}\label{subsec:parameter}
\renewcommand{\kapitelautor}{Autor: Felix Zwickelstorfer}

Jeder screen hat in Onj drei Bereiche, die die wichtigsten Informationen für diesen screen beinhalten. 
Diese umfassen die logische Verwaltung und Gliederung, aber auch die benötigten Ressourcen.

\renewcommand{\kapitelautor}{Autor: Felix Zwickelstorfer}
\subsubsection{Assets}\label{subsubsec:assets}
\renewcommand{\kapitelautor}{Autor: Felix Zwickelstorfer}
Die assets beschreiben, welche Ressourcen dieser screen immer laden soll. 
Bei Forty-Five werden normalerweise Ressourcen wie Bilder, Animationen sowie Schriftarten nur im screen gespeichert. 
Das heißt, dass sie bei jedem screen neu geladen werden. 
Deshalb gibt man dem screen eine Liste von assets mit, welche Ressourcen der screen immer benötigt. 
Diese beinhaltet \zB Schriftarten oder Hintergründe. 
Manche Ressourcen, wie \zB Karten, werden erst vom Programm geladen, da es keinen Sinn ergeben würde, jederzeit alle Karten zu laden. 
Darauf wird allerdings genauer bei~\ref{sec:resourcenmanagement} eingegangen.

\renewcommand{\kapitelautor}{Autor: Felix Zwickelstorfer}
\subsubsection{Viewport}\label{subsubsec:viewport}
\renewcommand{\kapitelautor}{Autor: Felix Zwickelstorfer}
Der viewport beschreibt die logische Anzahl an "points" pro screen in der Breite und Höhe. 
Man kann diese points mit Pixeln vergleichen, allerdings ist es komplett von der eigentlichen Auflösung des Spiels getrennt. 
Er wird verwendet, damit man als Entwickler Elementen bestimmte Größen im Verhältnis zum screen geben kann. 
Zusätzlich wird er verwendet, um bei box-Elementen die child-Elemente zu platzieren.
Diese werden immer nur auf ganzzahlige points gesetzt, wodurch es bei zu niedriger logischer Auflösung zu nicht korrekt platzierten Elementen kommen kann.

\renewcommand{\kapitelautor}{Autor: Felix Zwickelstorfer}
\subsubsection{Optionen}\label{subsubsec:optionen}
\renewcommand{\kapitelautor}{Autor: Felix Zwickelstorfer}
Die Optionen sind der vermutlich wichtigste Parameter eines screens. 
Sie beinhalten einerseits Elemente wie den Hintergrund, der allerdings meistens vom root-element überschrieben wird, sowie Parameter für die Musik.
Andererseits werden auch die input maps mitgegeben, welche für einfache user-Interaktionen zuständig sind. 
Diese sind genauer in~\ref{sec:key-input-system} beschrieben. 
Das wichtigste Element ist allerdings der screen-controller. 
Dieser ist das zentrale Steuerelement eines screens und wird bei~\ref{subsec:screen-controller} beschrieben. 
Wie das folgende Beispiel zeigt, gibt man dem screen-controller die Namen bestimmter Elemente mit, welche dieser verwaltet. 
Das erlaubt es, dass man die Elemente-Struktur des screens ändern kann, aber trotzdem noch auf die richtigen Elemente im Code referenziert.

\begin{codeBlock}{onj}{Beispiel: Optionen des HealOrMapHP\-Screens}
options: {
    background: "hover_detail_background",
    transitionAwayTime: 1.5,
    music: "map_theme",
    playAmbientSounds: true,
    inputMap: [
        ...(inputMaps.defaultInputMap),
        ...(inputMaps.healOrMaxInputMap),
        ...(inputMaps.addMaxHPInputMap),
    ],
    screenController: $HealOrMaxHPScreenController {
        addLifeActorName: "add_lives_option",
    }
}
\end{codeBlock}