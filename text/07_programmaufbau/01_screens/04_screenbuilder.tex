%! Author = felix
%! Date = 17/03/2024

\renewcommand{\kapitelautor}{Autor: Felix Zwickelstorfer}
\subsection{Screen-Builder}\label{sec:screen-builder}
\renewcommand{\kapitelautor}{Autor: Felix Zwickelstorfer}
Der screen-builder baut, wie der Name bereits sagt, den screen aus der onj-Datei auf.
Er liest zuerst die Parameter und templates und erstellt anschließend die Elemente in einem rekursiven Prozess.
Dabei wird zuerst das root-Element erstellt, und anschließend die child-Elemente davon, und so weiter.
Für das Erstellen eines Elements gibt es die Methode \inlineCode{getWidget()}, welche abhängig vom in onj angegebenen \inlineCode{OnjNamedObject} daraus ein widget generiert.
Anschließend wird unabhängig vom erstellten Element noch \inlineCode{applyWidgetKeysFromOnj()} ausgeführt, welche allgemeine Daten wie \zB den Namen oder die styles setzt.

Der screen-builder wird nach dem erstmaligen Aufbau nur für das Hinzufügen von widgets mit templates verwendet.
Dabei kann man entweder Daten mittels dem \inlineCode{$None} OnjObject zu einem bereits bestehenden Element hinzufügen, wie \zB Karten, oder komplett neue Elemente erzeugen, wie die Warnungen im backpack.
