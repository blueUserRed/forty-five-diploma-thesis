%! Author = felix
%! Date = 17/03/2024

\renewcommand{\kapitelautor}{Autor: Felix Zwickelstorfer}
\subsection{Widgets}\label{sec:widgets}
\renewcommand{\kapitelautor}{Autor: Felix Zwickelstorfer}
Ein widget beschreibt jedes Element, welches auf einem screen sichtbar ist. 
Einige werden von jedem screen gebraucht, wie \zB das box-widget, welches eine flexbox darstellt, oder auch ein Bild oder ein Text.
Diese werden im Folgenden als gewöhnliche widgets bezeichnet und können auch verschachtelt werden.
Es gibt allerdings Elemente, die durch ihre Komplexität oder Dynamik nicht als eine Schachtelung von anderen Elementen dargestellt werden können. 
Diese werden zu ihrem eigenen widget, welches nicht vom controller, sondern vom sich selbst verwaltet wird, wie \zB die Map.
\renewcommand{\kapitelautor}{Autor: Felix Zwickelstorfer}
\subsubsection{Widgets in Onj}\label{subsubsec:widgetsinonj}
\renewcommand{\kapitelautor}{Autor: Felix Zwickelstorfer}

Alle widgets sind definiert in \inlineCode{assets/onjschemas/screen.onjschema}.
Als Beispiel folgt nun die Konfiguration für das grüne Icon in der Mitte:
\begin{codeBlock}{onj}{Beispiel: Definition eines Images aus heal\_or\_max\_screen.onj}
    $Image {
        name: "heal_icon",
        textureName: "map_node_heal",
        zIndex: 160,
        scaleX: 1.0,
        scaleY: 1.0,
        styles: [
            {
                positionType: positionType.absolute,
                positionTop: 19#percent,
                positionLeft: 47.15#percent,
            }
        ],
    }
\end{codeBlock}
Es gibt diverse Parameter, die nur für images existieren wie \zB \quoted{textureName} oder \quoted{scaleX}, während andere "widget shared keys" sind, \dah, dass sie für alle gewöhnlichen widgets verfügbar sind.
Davon beinhaltet das oben angeführten Beispiel "name", "zIndex" und "styles".
Styles werden bei~\ref{subsec:stylesystem} genauer beschrieben.
Es gibt auch die keys "behaviours" und "dragAndDrop", welche das Verhalten beim Interagieren des Benutzers definieren.

Ein Beispiel für ein spezielles widget ist der backpack:
\begin{codeBlock}{onj}{Beispiel: backpack-Widget aus screens.onjschema}
    $Backpack {
        cardsFile: string,
        backpackFile: string,
        deckNameWidgetName: string,
        deckSelectionParentWidgetName: string,
        deckCardsWidgetName: string,
        backPackCardsWidgetName: string,
        backpackEditIndicationWidgetName: string,
        sortWidgetName: string,
        sortReverseWidgetName: string,
        ...widgetSharedKeys,
         children?: $Widget[],
         partOfSelectionHierarchy?: boolean,
    }
\end{codeBlock}
Der backpack ist eine flexbox mit erweiterten Funktionen.
Man kann Kartendecks bauen, benennen und die Karten außerhalb des Decks, in der sogenannten Kartenablage, nach gewissen Kriterien sortieren.
Es ist ein eigenes widget wodurch es auf mehreren screens verwendet werden kann.
Es beinhaltet alle keys, die auch ein box-widget hat (alles ab den "widget shared keys"), und zusätzlich die Namen zu bestimmten Konfigurationsdateien (cards und backpack), um die Daten richtig darzustellen.
Diese werden mitgegeben, da es eine bessere Lösung ist, als die Pfade hartkodiert im Code stehen zu haben, da es in Zukunft leichter anpassbar ist.
Weiters bekommt es die Namen bestimmter child-Elemente.
Diese Elemente werden statt von dem screen-controller, von dem backpack-widget verwaltet \bzw verwendet.

\renewcommand{\kapitelautor}{Autor: Felix Zwickelstorfer}
\subsubsection{Widgets in Kotlin}\label{subsubsec:widgetsinkotlin}
\renewcommand{\kapitelautor}{Autor: Felix Zwickelstorfer}
Bei LibGdx ist ein widget immer ein \inlineCode{com.badlogic.gdx.scenes.scene2d.Actor} oder eine Unterklasse davon.
In \FF werden keine der Klassen von LibGdx direkt verwendet, sondern immer eigene Erweiterungen \zB wegen des style-systems.
Weiters werden sie benötigt, um eigene events verarbeiten zu können, wie das hover-event.
Programmtechnisch gibt es vier Hauptänderungen, die überschrieben werden:
\begin{enumerate}
    \item Die \quoted{layout} Funktion, welche beschreibt, wo Elemente positioniert sind und bei Boxen auch deren Kinderelemente.
    \item Die \quoted{draw} Funktion, welche für die Darstellung verantwortlich ist. Das ermöglicht es, shader (siehe \ref{subsubsec:shader}) hinzuzufügen oder das style-system einzubauen.
    Sie wird in jedem frame einmal ausgeführt und wird daher ebenfalls für updates von timelines (siehe~\ref{subsec:timelines}) verwendet.
    \item Interfaces werden hinzugefügt, welche einerseits für das style-system verwendet werden, als auch generell das Arbeiten mit widgets vereinfachen.
\end{enumerate}
\renewcommand{\kapitelautor}{Autor: Felix Zwickelstorfer}
\subsubsection{Templates}\label{sec:templates}
\renewcommand{\kapitelautor}{Autor: Felix Zwickelstorfer}
Templates sind ein weiterer wichtiger Part von \FF, da sie das dynamische Erstellen von Elementen im Code erleichtern.
Außerdem verbessern sie die Lesbarkeit des Codes, da für ähnliche widgets ein template mit mehreren Parametern definiert werden kann.
Alle gewöhnlichen widgets können aus einem template erstellt werden.
Dies wird vor allem bei den Karten verwendet, da das Design und das Verhalten innerhalb eines screens für die Karten großteils gleich ist.
Deshalb erstellt man ein widget aus einem template und es wird nur das Bild und der hover-text auf die entsprechende Karte angepasst.
Templates werden allerdings auch an vielen anderen Stellen verwendet, \zB bei den Warnungen und Hinweisen im backpack, als auch bei dem HealOrMaxHP-screen für die beiden auswählbaren widgets.
Ein template zeichnet sich dadurch aus, dass es zusätzlich zu den normalen keys noch einen "template name" und die "template keys" hat.
Diese sehen bei dem Beispiel screen folgendermaßen aus:
\begin{codeBlock}{onj}{Beispiel: Ausschnitt eines Templates aus heal\_or\_max\_screen.onj}
templates: [
    $Box {
        template_name: "healOptionTemplate",
        template_keys: {
            "name": "name",
            "children.0.textureName": "textureName",
            "children.1.template": "templateMainText",
        },
        name: "Die name-value wird überschrieben, weil der zugehörige key in den templates angegeben ist",
        children: [
            $Image {
                textureName: "heal_or_max_add_max_health",
                scaleX: 0.8,
                scaleY: 0.8,
            },
            $TemplateLabel {
                template: "+5 Max HP",
                font: "red_wing_cm",
                color: color.dark_brown,
                fontScale: 1.1,
                align: "center",
            },
        ],
    }
]
\end{codeBlock}
Im Programmcode gibt man, wenn man ein template verwenden will, den \quoted{template name} an, und man gibt ein \inlineCode{OnjObject} mit, das die zu überschreibenden Daten beinhaltet.
In diesem Beispiel sind es der angezeigte Text und das Bild, da diese sich von der linken und rechten Seite unterscheiden.
Zusätzlich wird über das template auch der interne Name des widgets gesetzt.