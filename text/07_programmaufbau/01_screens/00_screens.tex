\renewcommand{\kapitelautor}{Autor: Felix Zwickelstorfer}
\section{Screens}\label{sec:screens}
\renewcommand{\kapitelautor}{Autor: Felix Zwickelstorfer}
In \FF werden grafische Elemente in onj als Teil eines screens definiert.
Dabei sind sich stark voneinander unterscheidende Elemente ein eigener screen, wie \zB der Kampf, die Karte oder der Shop.
Ein screen besteht dabei aus einem root-Element mit untergeordneten child-Elementen.
Zusätzlich hat kann ein screen einen screen-controller haben, welcher für die Verwaltung des screens zuständig ist und komplexere Aufgaben in diesem durchführt.

Im Folgenden werden die einzelnen Bestandteile genauer erklärt und als Beispiel ein Teil des  \inlineCode{HealOrMaxHPScreen} genommen.
Bei diesem kann sich der Spieler entweder heilen oder seine maximale Lebensanzahl erhöhen.
\begin{figure}[H]
    \centering
    \includegraphics[width=1.0\textwidth]{healormaxhpevent.png}
    \caption{Screenshot des HealOrMaxHP-Screens}
\end{figure}

%! Author = felix
%! Date = 17/03/2024

\renewcommand{\kapitelautor}{Autor: Felix Zwickelstorfer}
\subsection{Widgets}\label{sec:widgets}
\renewcommand{\kapitelautor}{Autor: Felix Zwickelstorfer}
Ein widget beschreibt jedes Element, welches auf einem screen sichtbar ist. 
Einige werden von jedem screen gebraucht, wie \zB das box-widget, welches eine flexbox darstellt, oder auch ein Bild oder ein Text.
Diese werden im Folgenden als gewöhnliche widgets bezeichnet und können auch verschachtelt werden.
Es gibt allerdings Elemente, die durch ihre Komplexität oder Dynamik nicht als eine Schachtelung von anderen Elementen dargestellt werden können. 
Diese werden zu ihrem eigenen widget, welches nicht vom controller, sondern vom sich selbst verwaltet wird, wie \zB die Map.
\renewcommand{\kapitelautor}{Autor: Felix Zwickelstorfer}
\subsubsection{Widgets in Onj}\label{subsubsec:widgetsinonj}
\renewcommand{\kapitelautor}{Autor: Felix Zwickelstorfer}

Alle widgets sind definiert in \inlineCode{assets/onjschemas/screen.onjschema}.
Als Beispiel folgt nun die Konfiguration für das grüne Icon in der Mitte:
\begin{codeBlock}{onj}{Beispiel: Definition eines Images aus heal\_or\_max\_screen.onj}
    $Image {
        name: "heal_icon",
        textureName: "map_node_heal",
        zIndex: 160,
        scaleX: 1.0,
        scaleY: 1.0,
        styles: [
            {
                positionType: positionType.absolute,
                positionTop: 19#percent,
                positionLeft: 47.15#percent,
            }
        ],
    }
\end{codeBlock}
Es gibt diverse Parameter, die nur für images existieren wie \zB \quoted{textureName'' oder ``scaleX}, während andere "widget shared keys" sind, \dah, dass sie für alle gewöhnlichen widgets verfügbar sind.
Davon beinhaltet das oben angeführten Beispiel "name", "zIndex" und "styles".
Styles werden bei~\ref{subsec:stylesystem} genauer beschrieben.
Es gibt auch die keys "behaviours" und "dragAndDrop", welche das Verhalten beim Interagieren des Benutzers definieren.

Ein Beispiel für ein spezielles widget ist der backpack:
\begin{codeBlock}{onj}{Beispiel: backpack-Widget aus screens.onjschema}
    $Backpack {
        cardsFile: string,
        backpackFile: string,
        deckNameWidgetName: string,
        deckSelectionParentWidgetName: string,
        deckCardsWidgetName: string,
        backPackCardsWidgetName: string,
        backpackEditIndicationWidgetName: string,
        sortWidgetName: string,
        sortReverseWidgetName: string,
        ...widgetSharedKeys,
         children?: $Widget[],
         partOfSelectionHierarchy?: boolean,
    }
\end{codeBlock}
Der backpack ist eine flexbox mit erweiterten Funktionen.
Man kann Kartendecks bauen, benennen und die Karten außerhalb des Decks, in der sogenannten Kartenablage, nach gewissen Kriterien sortieren.
Es ist ein eigenes widget wodurch es auf mehreren screens verwendet werden kann.
Es beinhaltet alle keys, die auch ein box-widget hat (alles ab den "widget shared keys"), und zusätzlich die Namen zu bestimmten Konfigurationsdateien (cards und backpack), um die Daten richtig darzustellen.
Diese werden mitgegeben, da es eine schönere Lösung ist, als die Pfade hartkodiert im Code stehen zu haben.
Weiters bekommt es die Namen bestimmter child-Elemente.
Diese Elemente werden statt von dem screen-controller, von dem backpack-widget verwaltet \bzw verwendet.

\renewcommand{\kapitelautor}{Autor: Felix Zwickelstorfer}
\subsubsection{Widgets in Kotlin}\label{subsubsec:widgetsinkotlin}
\renewcommand{\kapitelautor}{Autor: Felix Zwickelstorfer}
Bei LibGdx ist ein widget immer ein \inlineCode{com.badlogic.gdx.scenes.scene2d.Actor} oder eine Unterklasse davon.
In \FF werden keine der Klassen von LibGdx direkt verwendet, sondern immer eigene Erweiterungen \zB wegen des style-systems.
Weiteres werden sie benötigt, um eigene events verarbeiten zu können, wie das hover-event.
Programmtechnisch gibt es vier Hauptänderungen, die überschrieben werden:
\begin{enumerate}
    \item Die \quoted{layout} Funktion, welche beschreibt, wo Elemente positioniert sind und bei Boxen auch deren Kinderelemente.
    \item Die \quoted{draw} Funktion, welche für die Darstellung verantwortlich ist. Das ermöglicht es, shader (siehe \ref{subsubsec:shader}) hinzuzufügen oder das style-system einzubauen.
    Sie wird in jedem frame einmal ausgeführt und wird daher ebenfalls für updates von timelines (siehe~\ref{subsec:timelines}) verwendet.
    \item Interfaces werden hinzugefügt, welche einerseits für das style-system verwendet werden, als auch generell das Arbeiten mit widgets vereinfacht.
\end{enumerate}
\renewcommand{\kapitelautor}{Autor: Felix Zwickelstorfer}
\subsubsection{Templates}\label{sec:templates}
\renewcommand{\kapitelautor}{Autor: Felix Zwickelstorfer}
Templates sind ein weiterer wichtiger Part von \FF, da sie das dynamische Erstellen von Elementen im Code erleichtern.
Außerdem verbessern sie die Lesbarkeit des Codes, da für ähnliche widgets ein template mit mehreren Parametern definiert werden kann.
Alle gewöhnlichen widgets können aus einem template erstellt werden.
Dies wird vor allem bei den Karten verwendet, da das Design und das Verhalten innerhalb eines screens für die Karten großteils gleich ist.
Deshalb erstellt man ein widget aus einem template und es wird nur das Bild und der hover-text auf die entsprechende Karte angepasst.
Templates werden allerdings auch an vielen anderen Stellen verwendet, \zB bei den Warnungen und Hinweisen im backpack, als auch bei dem HealOrMaxHP-screen für die beiden auswählbaren widgets.
Ein template zeichnet sich dadurch aus, dass es zusätzlich zu den normalen keys noch einen "template name" und die "template keys" hat.
Diese sehen bei dem Beispiel screen folgendermaßen aus:
\begin{codeBlock}{onj}{Beispiel: Ausschnitt eines Templates aus heal\_or\_max\_screen.onj}
templates: [
    $Box {
        template_name: "healOptionTemplate",
        template_keys: {
            "name": "name",
            "children.0.textureName": "textureName",
            "children.1.template": "templateMainText",
        },
        name: "Die name-value wird überschrieben, weil der zugehörige key in den templates angegeben ist",
        children: [
            $Image {
                textureName: "heal_or_max_add_max_health",
                scaleX: 0.8,
                scaleY: 0.8,
            },
            $TemplateLabel {
                template: "+5 Max HP",
                font: "red_wing_cm",
                color: color.dark_brown,
                fontScale: 1.1,
                align: "center",
            },
        ],
    }
]
\end{codeBlock}
Im Programmcode gibt man, wenn man ein template verwenden will, den \quoted{template name} an, und man gibt ein \inlineCode{OnjObject} mit, das die zu überschreibenden Daten beinhaltet.
In diesem Beispiel sind es der angezeigte Text und das Bild, da diese sich von der linken und rechten Seite unterscheiden.
Zusätzlich wird über das template auch der interne Name des widgets gesetzt.
%! Author = felix
%! Date = 17/03/2024
\renewcommand{\kapitelautor}{Autor: Felix Zwickelstorfer}
\subsection{Parameter}\label{subsec:parameter}
\renewcommand{\kapitelautor}{Autor: Felix Zwickelstorfer}

Jeder screen hat in Onj drei Bereiche, die die wichtigsten Informationen für diesen screen beinhalten. 
Diese umfassen die logische Verwaltung und Gliederung, aber auch die benötigten Ressourcen.

\renewcommand{\kapitelautor}{Autor: Felix Zwickelstorfer}
\subsubsection{Assets}\label{subsubsec:assets}
\renewcommand{\kapitelautor}{Autor: Felix Zwickelstorfer}
Die assets beschreiben, welche Ressourcen dieser screen immer laden soll. 
Bei Forty-Five werden normalerweise Ressourcen wie Bilder, Animationen sowie Schriftarten nur im screen gespeichert. 
Das heißt, dass sie bei jedem screen neu geladen werden. 
Deshalb gibt man dem screen eine Liste von assets mit, welche Ressourcen der screen immer benötigt. 
Diese beinhaltet \zB Schriftarten oder Hintergründe. 
Manche Ressourcen, wie \zB Karten, werden erst vom Programm geladen, da es keinen Sinn ergeben würde, jederzeit alle Karten zu laden. 
Darauf wird allerdings genauer bei~\ref{sec:resourcenmanagement} eingegangen.

\renewcommand{\kapitelautor}{Autor: Felix Zwickelstorfer}
\subsubsection{Viewport}\label{subsubsec:viewport}
\renewcommand{\kapitelautor}{Autor: Felix Zwickelstorfer}
Der viewport beschreibt die logische Anzahl an "points" pro screen in der Breite und Höhe. 
Man kann diese points mit Pixeln vergleichen, allerdings ist es komplett von der eigentlichen Auflösung des Spiels getrennt. 
Er wird verwendet, damit man als Entwickler Elementen bestimmte Größen im Verhältnis zum screen geben kann. 
Zusätzlich wird er verwendet, um bei box-Elementen die child-Elemente zu platzieren.
Diese werden immer nur auf ganzzahlige points gesetzt, wodurch es bei zu niedriger logischer Auflösung zu nicht korrekt platzierten Elementen kommen kann.

\renewcommand{\kapitelautor}{Autor: Felix Zwickelstorfer}
\subsubsection{Optionen}\label{subsubsec:optionen}
\renewcommand{\kapitelautor}{Autor: Felix Zwickelstorfer}
Die Optionen sind der vermutlich wichtigste Parameter eines screens. 
Sie beinhalten einerseits Elemente wie den Hintergrund, der allerdings meistens vom root-element überschrieben wird, sowie Parameter für die Musik.
Andererseits werden auch die input maps mitgegeben, welche für einfache user-Interaktionen zuständig sind. 
Diese sind genauer in~\ref{sec:key-input-system} beschrieben. 
Das wichtigste Element ist allerdings der screen-controller. 
Dieser ist das zentrale Steuerelement eines screens und wird bei~\ref{subsec:screen-controller} beschrieben. 
Wie das folgende Beispiel zeigt, gibt man dem screen-controller die Namen bestimmter Elemente mit, welche dieser verwaltet. 
Das erlaubt es, dass man die Elemente-Struktur des screens ändern kann, aber trotzdem noch auf die richtigen Elemente im Code referenziert.

\begin{codeBlock}{onj}{Beispiel: Optionen des HealOrMapHP\-Screens}
options: {
    background: "hover_detail_background",
    transitionAwayTime: 1.5,
    music: "map_theme",
    playAmbientSounds: true,
    inputMap: [
        ...(inputMaps.defaultInputMap),
        ...(inputMaps.healOrMaxInputMap),
        ...(inputMaps.addMaxHPInputMap),
    ],
    screenController: $HealOrMaxHPScreenController {
        addLifeActorName: "add_lives_option",
    }
}
\end{codeBlock}
%! Author = felix
%! Date = 17/03/2024
\renewcommand{\kapitelautor}{Autor: Felix Zwickelstorfer}
\subsection{Screen-Controller}\label{subsec:screen-controller}
\renewcommand{\kapitelautor}{Autor: Felix Zwickelstorfer}

Der screen-controller ist die zentrale Steuereinheit eines screens und verwaltet diesen.
Er ist für die gesamte Logik zuständig sowie für die meisten User-interactions.
Die einzigen Ausnahmen sind komplexe widgets (siehe~\ref{sec:widgets}) und input maps (siehe~\ref{sec:key-input-system}).
Als Beispiel folgen nun Ausschnitte des \inlineCode{HealOrMaxHPScreen-Controllers} erklärt:

\begin{codeBlock}{kotlin}{Beispiel: Screen-Controller des HealOrMapHP-Screens}
    override fun init(onjScreen: OnjScreen, context: Any?) {
        val rnd = Random(context.seed)
        if (context !is HealOrMaxHPMapEvent) throw RuntimeException("context for ${this.javaClass.simpleName} must be a ChooseCardMapEvent")
        amount = context.healthRange.random(rnd) to context.maxHPRange.random(rnd)
        TemplateString.updateGlobalParam(
            "map.cur_event.heal.lives_new",
            min(SaveState.playerLives + amount.first, SaveState.maxPlayerLives)
        )
        // ...
    }
\end{codeBlock}


Die \inlineCode{init} Methode wird aufgerufen, nachdem alle Elemente vom screen geladen worden sind.
Der Parameter \quoted{context} ist ein MapEvent, da dieser screen nur als Event aufgerufen werden kann.
Das Event beinhaltet Daten wie den Seed des Events und die Heil-Bereiche.
Anschließend berechnet der screen-controller eine zufällige Zahl aus diesen Bereichen und bestimmt dadurch, wie viel man jeweils geheilt werden kann.
Der Seed ist notwendig, damit diese Zahl immer gleich ist.
Wäre es immer zufällig, könnte man als Spieler das Event verlassen und wieder betreten.
Das könnte man so oft wiederholen, bis man die maximale Anzahl an Leben erhält.
Zuletzt aktualisiert der screen-controller noch die Zahlen in den Texten, damit der Spieler auch sieht, wie viele Leben er hinzubekommt.

Anschließend kann dieser screen "completed", also abgeschlossen, werden.
Dafür hat er folgende Methode:
\begin{codeBlock}{kotlin}{Beispiel: Screen\-Controller des HealOrMapHP\-Screens}
    override fun completed() {
        SoundPlayer.situation("heal", screen)
        if ((screen.namedActorOrError(healChosenTarekGeorgWidgetName) as CustomFlexBox).inActorState("selected")) {
            val newLives = min(SaveState.playerLives + amount.first, SaveState.maxPlayerLives)
            FortyFiveLogger.debug(logTag, "Lives healed from ${SaveState.playerLives} to $newLives!")
            SaveState.playerLives = newLives
        }
        // ...
        context?.completed()
    }
\end{codeBlock}

Wie man im Code sieht, wird beim Abschließen des screens ein Sound abgespielt.
Anschließend wird die Wahl des Spielers überprüft, und falls er sich heilen wollte, wird seine Lebensanzahl entsprechend angepasst.
Zum Schluss wird der "context" ebenfalls abgeschlossen, was in diesem Fall heißt, dass man zurück zur Map gelangt und dass man dieses Event nicht mehr ausführen kann.

Dieser screen hat nicht viele Operationen, wodurch auch der screen-controller sehr wenig zu verwalten hat, weshalb er als Beispiel optimal ist.
Üblicherweise beinhalten screens komplexere Programmlogiken.
%! Author = felix
%! Date = 17/03/2024

\renewcommand{\kapitelautor}{Autor: Felix Zwickelstorfer}
\subsection{Screen-Builder}\label{sec:screen-builder}
\renewcommand{\kapitelautor}{Autor: Felix Zwickelstorfer}
Der screen-builder baut, wie der Name bereits sagt, den screen aus der onj-Datei auf.
Er liest zuerst die Parameter und templates und erstellt anschließend die Elemente in einem rekursiven Prozess.
Dabei wird zuerst das root-Element erstellt, und anschließend die sub-Elemente davon, und so weiter.
Für das Erstellen eines Elements gibt es die Methode \inlineCode{getWidget()}, welche abhängig vom in onj angegebenen \inlineCode{OnjNamedObject} daraus ein widget generiert.
Anschließend wird unabhängig vom erstellten Element noch \inlineCode{applyWidgetKeysFromOnj()} ausgeführt, welche allgemeine Daten wie \zB den Namen oder die styles setzt.

Der screen-builder wird nach dem erstmaligen Erstellen nur für das Hinzufügen von widgets mit templates verwendet.
Dabei kann man entweder Daten mittels dem \inlineCode{$None} OnjObject zu einem bereits bestehenden Element hinzufügen, wie \zB bei Karten, oder komplett neue Elemente erzeugen wie die Warnungen im Backpack.