
\section{Biome}\label{sec:biome}

\renewcommand{\kapitelautor}{Autor: Marvin Kurka}

Um den Spieler mehr Abwechslung zu bieten wurden Biome implementiert.
Das Biom beeinflusst mehrere Aspekte des Spiels:

\begin{liste}
    \item Den Hintergrund der Map
    \item Die Dekorationen auf der Map
    \item Die Hintergrundgeräusche, die abgespielt werden
    \item Der Hintergrund im Encounter
    \item Die Encounter die auf der Road auftauchen
    \item Die Karten, die gekauft/gefunden werden können
\end{liste}

Sämtliche Änderungen auf der Map selber können direkt in der Konfiguration des Map-Generators implementiert werden.
Selbiges gilt für die Screen-Hintergründe, diese können direkt in der Screen-Definition konfiguriert werden, indem
der Zustand des Programmes mittels der \inlineOnj{state()} Funktion abgefragt wird.

Karten sind in Pools aufgeteilt.
Das hilft sicherzustellen, dass der Spieler nicht alle Karten sofort bekommen kann, sondern, besonders Karten mit
komplizierten Effekte, erst später im Spiel erhält.
Das ist mit den Biomen verknüpft und verwendet das RandomCardSelection Objekt.
Auch dieses Objekt kann über eine onj-Datei konfiguriert werden und bestimmt welche Karten wann und mit welcher
Wahrscheinlichkeit im Spiel auftauchen.
Hier werden bestimmte Karten anhand der Road/des Biomes geblacklistet und sind somit nicht erhältlich.

% resets author
\renewcommand{\kapitelautor}{}
