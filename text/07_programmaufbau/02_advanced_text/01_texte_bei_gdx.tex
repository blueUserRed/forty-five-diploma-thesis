%! Author = felix
%! Date = 17/03/2024

\renewcommand{\kapitelautor}{Autor: Felix Zwickelstorfer}
\subsection{Labels in LibGdx}\label{subsec:labels-in-gdx}
\renewcommand{\kapitelautor}{Autor: Felix Zwickelstorfer}

Die Hauptbibliothek, die in Forty-Five verwendet wird, hat zur Darstellung von Texten Elemente, welche Labels genannt werden.
Diese können eine bestimmte Farbe, Schriftart oder Schriftgröße haben.
Man kann auch manuelle Zeilenumbrüche innerhalb eines Elements machen.
Diese Funktionen sind gut und funktionieren auch in den meisten Fällen auch, sind allerdings nicht für alles ausreichend.
Dadurch, dass in Forty-Five viele dynamische Texte vorkommen, ist es zu viel Aufwand, jedes Mal per Hand es zu ändern, wann welcher Zeilenumbruch gesetzt werden soll.
Weites ist es komplex, die Positionen und Breiten für jedes Element neu zu berechnen, weshalb ein allgemeiner Parser geschrieben wurde, der es etwas simpler löst.
% todo, nacheinander rendern von zeichen nicht möglich, außerdem farbe nicht möglich, weil wenn input <red>Text</red> fe, würde beim einzeln anzeigen noch das <red> anzeigen, bis es dann programm checkt