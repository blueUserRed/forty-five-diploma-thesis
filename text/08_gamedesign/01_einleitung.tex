
\section{Einleitung}\label{sec:einleitung}

\renewcommand{\kapitelautor}{Autor: Philip Jankovic}


\FF ist ein digitales Kartenspiel in einem Wild West Setting. Der spieler reist durch den Wilden Westen auf einer Map,
sammelt Karten und bekämpft damit Gegner um auf der Map fortzuschreiten. Karten sind Bullets und werden in das Spielfeld, also den Revolver geladen.
Jede Bullet hat einen eigenen Effekt, der von dem Spieler genützt werden kann um immer bessere Combos aufzubauen und damit immer stärkere Gegner zu bekämpfen.
Der Revolver wird dazu verwendet, Bullets auf den Gegner abzufeuern. Nachdem eine Bullet geschossen wurde, verlässt die
Bullet den Revolver dreht sich eines nach rechts. Der Gegner kann verschiedene Gegneraktionen ausführen und damit dem Spieler schaden.
Durch die  immer besser werdenden Bullets des Spielers, ist es ihm möglich, die immer schwerern Gegner zu bezwingen und damit den Wilden Westen zu durchqueren.
Die Map enthält verschiedene Biome, also Gebiete, die sich durch Aussehen, Vegetation oder andere Eigenschaften unterscheiden.
In diesen Biomen gibt es Events, durch die der Spieler neue Karten erwirbt oder er sich heilen kann.



\subsection{Begriffserklärung}\label{begriffserklärung}
Mechanik: In \FF sind Mechaniken einzelne Komponente des Spieles, welche zusammgesetzt das Spiel ergeben.\zit{mechanik}

Combo: In \FF wird eine Combo als eine Kombination von Karten mit guter Synergie zueinander bezeichnet.


Map: Die Karte von \FF, auf der sich der Spieler bewegt und diverse Events auswählt, wie \zB Kämpfe oder Shops.

Road: Eine Road in \FF ist eine zufällig generierter Abschnitt des Spieles. Eine Road befinndet sich immer zwischen zwei Areas.

Area: Nicht zufällig gnerierte Abschnitte der \FF Map.

Rogue-Like: Ein Genre von Videospiel, bei welchem der Fortschritt des Spielers verloren geht, sollte er streben/verlieren. \zit{zitatdeckbuilding}

Run: Ein Run ist ein Durchlauf in einem Rogue-like oder Rogue-lite Spiel. \zit{zitatdeckbuilding, liteandlike}

Rogue-Lite: Ein Subgenre des Rogue-Likes, bei welchem Teile des Fortschrittes in den nächsten Run mitgenommen werden.
Es ist einfacher und weniger frustrierend als ein Rogue-Like \zit{zitatdeckbuilding, liteandlike}

% text goes here
%

% resets author
\renewcommand{\kapitelautor}{}
