
\section{Flavour}\label{sec:flavour}

\renewcommand{\kapitelautor}{Autor: Irgendwer} % todo: replace

\subsection{Die Wichtigkeit des Settings und des Flavors für ein Kartenspiel}\label{subsec:wichtigkeit-des-flavours}

%
Mit Flavor, im bezug auf Kartenspiele, ist gemeint, welche Geschichte und Hintergründe hinter einer Karte steckt bzw wie gut die Art, der Effekt,
die Geschichte hinter jeder KArte zu dem Konzept/Namen der Karte passt. Flavor gibt jeder Karte etwas eigenes, was die karte von anderen unterschiedet.

"Flavor is the Soul of the game"\cite{soulOfTheGame}


Flavor kann durch verschiedene Art auftretten. Es kann Flavor in Karten und generell in spielkonzepten/Regeln exestieren.
Es verknüpft die Karten, das Setting, die Story und die Spielregeln. Grob kann ein Kartenspiel in Funktion und Flavor aufgeteitlt werden.
Beide exestieren zusammen und keines ist wichtiger als das andere. Unter Funktion versteht wie das Spiel gespielt wird was regeln und Mechaniken einschließt. \cite{flavorAndFunction}

% text goes here
%
\subsection{Flavor durch Spielmechaniken und Spielmechaniken durch Flavor}\label{subsec:flavour-durch-mechaniken}

Flavor und Funktion, also Spielnechaniken ist das erste über das sich Gednken gemacht werden sollte.
Flavor in den Spielregeln, erlaubt spielern auch Wissen, das sie Bereits besitzen, in dem Spieel anzuwenden und das Spiel dadurch besser zu verstehen. \cite{flavorAndFunction}

In \FF zum Beispiel, ist das spielfeld der Revolver. Vielen Spielern wird bereits bewusst sein, dass man Bullets in einen Revolver reinlädt und da alle Karten Bullets sind kann sich der Spieler
denken was er zu tun hat. Die Mechanik des Kartens in das Runde Spielfeld zu laden, greift in den Flavor, der besagt diese Karten sind Bullets und das runde Spielfeld ist ein Revolver.
Wären die Karten keinen Bullets sondern kreaturen und der Revolver kein Runder Kreis der rotieren kann, wäre \FF nicht \FF sondern ein anderes Spiel mit dementsprechend auch anderem Setting.

Die Roattion und eigenschaften des PSielfeldes in \FF machen nur Sinn, da es ein Revolver ist.
Das Karten Bullets sind, macht nur Sinn, weil das Setting im Wilden Westen spielt und das Spielfeld ein Revolver ist.
Das das Spielfeld ein revolver ist, macht nur sinn, da er nach rechts rotiert und die vorderste Karte rausschießt wenn er geschossen wird.
Ohne den Flavor sind die SPielregln von \FF unnachfolziebar und können ohne das andere nicht exestieren.

Auch in zum Beispiel Magic the Gathering, Inscription oder Balatro ist dieses Konzept sichtbar. In Magic ist das Spielkonzept breit und umfangreich. Das Setting ist nicht so streng definiert in \FF, was mehr verschiedene Spielmechaniken und Karten erlaubt. Dafür kann das Spiel sich manchmal nicht wirklich mehr wie Magic anfühlen,
was bei \FF durch die strikte einhaltung der Machaniken und der Karten, also  Bullets und der revolver nicht passieren kann. #
Der Flavor von \FF ist stark und das durchziehende Konzept der Bullets so konstant, das das SPiel und seine Regeln immer
wiederzuerkennen sind. Ähnlich ist es bei Balatro, ein pokerinspiriertes roguelite deckbuilding game.
So wie die sammelbaren Karten in \FF Bullets sind, sammelt der Spieler in balatro verschiedene Joker KArten.
Obwohl Balatro erst veröffentlicht wurde als die erste Release version von \FF die entwicklung bereits abgeschlossen hatte,
ist das Konzept von den wirklich verschiedenen und komischen Jokern von Balatro sehr ähnlich zu den verschiedenen und
komischen Bullets von \FF.
Inscription jedoch verfügt über weniger konstantes Karten design, auch wenn jede Karte ein anderes Tier darstellt,
legt dafür aber wichtigkeit auf das setting und Stimmung die das SPiel rüberbringt. Durch opfern von Karten können stärkere Karten
gespielt werden, was gut zu dem Horrorsetting passt. Auch Lebenspunkte werden abgerechnet durch herausgerissene Zähne und Augen.
All diese Beispiele zeigen wie Spiele aus dem Selben oder ser ähnlichen genre sich durch ihr setting, Falvor und Spielmecganiken unterscheiden.
%TODO Quellen von den Spielen!!


Die erschaffung der Spielregeln und damit auch des Flavors von \FF, gingen Hand in Hand zu einander.
Zuerst kam die Idee des Setting des Wilden Westens, dann das Spielkonzept.
Mehr zu Methoden der Ideenfindung kann später nachgelesen werden. Wichtiig für dieses Kapitel ist,
das selbst nach zwei Jahren Arbeit und vielen Änderungen an Spielreglen, das Grundkonzzept noch immer das selbe ist.
Viele Karten, die damals im Notizuch aufgeezichent wurden, befinden ssich fast ident noch immer im Spiel.
%TODO bilder von Karten damals vs heute usw



Spielmechaniken ist jedoch nicht das einzige, in dem Flavor eine wichtige Rolle spielt

%
% text goes here
%

\subsection{Cardflavor}\label{subsec:flavour-durch-mechaniken}

Cardflavor ist der Flavor einzelner Karten.
Dazu gehören ihr Name, ihre Art, ihr Effekt und wie der Effekt mit der Name der karte zusammenhängt,
ihr Platz in der Welt und die Hintergrundgeschichte dieser Karte. Je nach Karte sind mehr oder weniger der gerade ganannten
eigenschaften vertreten.

Flavortext:
Zusätzlich zu dem Effekt der Karte, welcher sichtbar wird wenn der Spieler über die Karte hovert, haben einige Karten zusättzlich einen sogenannten Flavortext.

Flavortexte sind zum Beispiele auch in Magic the Gathering vertreten und beschreiben die Welt und zeigen Zitate von verschiedenen Charackteren in der Magic-Welt.\cite{magicarena} \cite{soulOfTheGame}%TODO BSP und quelle

In \FF kann ein Falvortext entweder Worldbuilding sein oder auch ein SPruch, Witz oder Anspielung die zur Karte passt.








%für später
In Magic: The Gathering zum Beispiel
gibt es sogenannte Flavortexte, kleine Texte die einen Spuruch oder ähnliches aus der Geschichte von Magic zeigen, sei es jetzt über einen charakter, oder die Welt.
Forty-five verfügt auch über Flavortexte, jedoch werden sie ein wenig anders benützt. Nicht jede Karte hat einen Falvortext,  die Bullets die jedoch einen haben, haben entweder Worldbuilding,
Context für die Story, woraf zu einem späteren Zeitounkt nochh zu sprechen gekommen wird, Einen Anspielung auf ein anderes Medium, oder einen Witz/Spruch der nochmal
die Bullet unterstreicht. Flavortexte sind jedoch nicht der einzige Weg Flavor in Karten zu bringen.


\subsection{Kein teil des SPieles ohne Flavor}\label{subsec:flavour-durch-mechaniken}

\subsection{Story und Worldbuilding}\label{subsec:flavour-durch-mechaniken}


% resets author
\renewcommand{\kapitelautor}{}
