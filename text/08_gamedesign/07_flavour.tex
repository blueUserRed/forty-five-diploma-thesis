
\section{Flavour}\label{sec:flavour}

\renewcommand{\kapitelautor}{Autor: Philip Jankovic}

\subsection{Die Wichtigkeit des Settings und des Flavors für ein Kartenspiel}\label{subsec:wichtigkeit-des-flavours}

%
Mit Flavor, im Bezug auf Kartenspiele, ist gemeint, welche Geschichte und Hintergründe hinter einer Karte stecken bzw.
wie gut die Art, der Effekt oder die Geschichte hinter jeder Karte zu dem Konzept/Namen der davon passt.
Flavor gibt jeder Karte etwas Eigenes, was sie von anderen unterscheidet.

\quoted{Flavor is the Soul of the game}\zit{soulOfTheGame}


Flavor kann auf verschiedene Arten auftreten. Flavor kann in Karten und generell in Spielkonzepten/Regeln existieren.
Es verknüpft die Karten, das Setting, die Story und die Spielregeln. Grob kann ein Kartenspiel in Funktion und Flavor aufgeteilt werden.
Beide existieren zusammen und keines ist wichtiger als das andere.
Unter Funktion wird verstanden, wie das Spiel gespielt wird, was Regeln und Mechaniken einschließt.\zit{flavorAndFunction}


\subsection{Flavor durch Spielmechaniken und Spielmechaniken durch Flavor}\zit{subsec:flavour-durch-mechaniken}

Flavor und Funktion, zusammen also Spielmechaniken, sind das Erste, über das nachgedacht werden sollte.
Flavor in den Spielregeln erlaubt Spielern, bereits vorhandenes Wissen im Spiel anzuwenden und das Spiel dadurch besser zu verstehen. \zit{flavorAndFunction}


Im \FF zum Beispiel ist das Spielfeld der Revolver. Vielen Spielern wird bereits bewusst sein, dass Kugeln in einen
Revolver geladen wurd und da alle Karten Kugeln sind, kann sich der Spieler denken, was er zu tun hat. Die Mechanik des Ladens
der Karten in das runde Spielfeld greift in den Flavor, der besagt, dass diese Karten Kugeln sind und das runde Spielfeld
ein Revolver ist. Wären die Karten keine Kugeln, sondern zum Beispiel Kreaturen und der Revolver kein runder Kreis,
der rotieren kann, wäre \FF nicht \FF, sondern ein anderes Spiel mit dementsprechend anderem Setting.


Die Rotation und Eigenschaften des Spielfeldes in \FF machen nur Sinn, da es ein Revolver ist.
Das Karten Bullets sind, macht nur Sinn, weil das Setting im Wilden Westen spielt und das Spielfeld ein Revolver ist.
Das das Spielfeld ein revolver ist, macht nur sinn, da er nach rechts rotiert und die vorderste Karte rausschießt wenn er geschossen wird.
Ohne den Flavor sind die SPielregln von \FF unnachfolziebar und können ohne das andere nicht exestieren.


Anders gesagt, gibt der Flavor einem Kartenspiel, welches eigentlich nur ein abstraktes Konzept aus Nummern und Regeln ist,
an welche sich Menschen halten, damit ein Spiel zustande kommt, einen Grund für die Funktionalitäten und damit eine \quoted{Seele}.


Auch in zum Beispiel “Magic the Gathering”, “Inscription” oder “Balatro” ist dieses Konzept sichtbar. In “Magic” ist das Spielkonzept breit und umfangreich.
Das Setting ist nicht so streng definiert wie in \FF, was mehr verschiedene Spielmechaniken und Karten erlaubt.
Dafür kann das Spiel sich manchmal nicht wirklich mehr wie “Magic” anfühlen,
was bei \FF durch die strikte Einhaltung der Mechaniken und der Karten, also Bullets und dem Revolver, nicht passieren kann.
Der Flavor von \FF ist stark und das durchziehende Konzept der Bullets so konstant, dass das Spiel und seine Regeln immer
wiederzuerkennen sind. Ähnlich ist es bei “Balatro”, einem pokerinspirierten roguelite deckbuilding game.
So wie die sammelbaren Karten in \FF Bullets sind, sammelt der Spieler in “Balatro” verschiedene Joker Karten.
Obwohl “Balatro” erst veröffentlicht wurde, als die erste Release-Version von \FF die Entwicklung bereits abgeschlossen hatte,
ist das Konzept von den wirklich verschiedenen und komischen Jokern von “Balatro” sehr ähnlich zu den verschiedenen und
komischen Bullets von \FF.
”Inscription” wiederum verfügt über weniger konstantes Kartendesign, auch wenn jede Karte ein anderes Tier darstellt,
legt dafür aber Wichtigkeit auf das Setting und die Stimmung, die das Spiel rüberbringt. Durch Opfern von Karten können stärkere Karten
gespielt werden, was gut zu dem Horrorsetting passt. Auch Lebenspunkte werden abgerechnet durch herausgerissene Zähne und Augen.
All diese Beispiele zeigen, wie Spiele aus demselben oder sehr ähnlichen Genre sich durch ihr Setting, Flavor und Spielmechaniken unterscheiden.
%TODO Quellen von den Spielen!!


Die Erschaffung der Spielregeln und damit auch des Flavors von \FF gingen Hand in Hand miteinander.
Zuerst kam die Idee des Settings des Wilden Westens, dann das Spielkonzept. Mehr zu Methoden der Ideenfindung kann später
nachgelesen werden. Wichtig für dieses Kapitel ist, dass selbst nach zwei Jahren Arbeit und vielen Änderungen an Spielregeln,
das Grundkonzept noch immer das gleiche ist. Viele Karten, die damals im Notizbuch aufgezeichnet wurden, befinden sich
fast identisch noch immer im Spiel.
%TODO bilder von Karten damals vs heute usw



Spielmechaniken sind jedoch nicht das Einzige, in dem Flavor eine wichtige Rolle spielt.

\subsection{Cardflavor}\label{subsec:cardflavor}

Cardflavor ist der Flavor einzelner Karten.
Dazu gehören ihr Name, ihre Art, ihr Effekt und wie der Effekt mit dem Namen der Karte zusammenhängt,
ihr Platz in der Welt und die Hintergrundgeschichte dieser Karte. Je nach Karte sind mehr oder weniger der gerade genannten
Eigenschaften vertreten.


Flavortext:
Zusätzlich zu dem Effekt der Karte, der sichtbar wird, wenn der Spieler über die Karte hovert, haben einige Karten zusätzlich einen sogenannten Flavortext.


Flavortexte sind zum Beispiel auch in Magic the Gathering vertreten und beschreiben die Welt der Spieles oder geben
Zitate von verschiedenen Charakteren in der Magic-Welt wieder.\zit{magicarena}\zit{soulOfTheGame} %TODO Bild


In \FF kann ein Flavortext entweder Worldbuilding sein oder auch ein Spruch, Witz oder Anspielung, die zur Karte passt.

%TODO bilder mit beispielen

Karteneffekte:
Die zweite Art, einer Karte Flavor zu verleihen, ist es, den Karteneffekt an den Namen der Karte anzupassen.
Ein Beispiel dafür ist Golddiggers Bullet.


In den meisten Kartenspielen ist das Ziehen von Karten ein großer Vorteil für einen Spieler, da eine größere Kartenauswahl
auch mehr Kontrolle bedeutet. Dies wird in den Communities umgangssprachlich als \quoted{Value} bezeichnet. \zit{whatsvalue}


Auch in \FF spielt Value eine große Rolle.
Um den Karten Flavor zu geben, sind alle Karten, die Karten ziehen, einer Bullet zugeordnet, welche auf eine Art und Weise ein wertvolles Material darstellt.
Beispiele dafür sind zum Beispiel Silver Bullet oder Gold Bullet. Diese Verbindung zwischen wertvollen
Materialien und Karten ziehen ist für viele Spieler nicht sofort erkennbar, jedoch werden die meisten Spieler die Verbindung zu Golddiggers Bullet verstehen.


Golddiggers Bullet hat den Effekt, dass immer wenn der Spieler eine Karte zieht, sie gestärkt wird und dem Spieler Reserves gibt. Die “wertvollen” Karten wie z.B Gold Bullet ziehen also Karten, wovon Golddiggers Bullet profitiert und die
gezogenen Karten in \quoted{Geld}, also Reserven, umwandelt, mit dem wieder Karten gespielt werden können. Also wie ein Goldgräber, der Gold findet und es anschließend verkauft.

Auf einem ähnlichen Konzept basiert Gravediggers Bullet, die davon profitiert, wenn Karten \quoted{sterben}, also zerstört werden.


Diese Flavor Effekte ziehen sich durch das ganze Spiel und nutzen die Spielmechaniken auch um die Konzepte der Bullets umzusetzen.
Rotten Bullet zum Beispiel verliert mit jedem Mal, die sie rotiert, einen dmg ihrer dmg-value. Gold Bullet wird immer mehr
wert, mit der Zeit, da sie, wenn sie abgeschossen wird, also \quoted{verkauft wird}, x Karten zieht, wobei x die Nummer der Drehungen
von Gold Bullet ist. Es werden also Eigenschaften von Gold, nämlich dass es immer mehr wert wird, je länger es in Besitz ist,
in die Spielmechaniken von \FF integriert.


Kartenflavor ist unter anderem wichtig für das Kartenspiel, da damit Unglaubwürdigkeiten aus dem Weg geräumt werden.
Nicht alle Karten erhalten die Qualität des Cardflavors wie zum Beispiel Golddiggers Bullet, was jedoch auch nicht wirklich nötig ist.
Dennoch muss ein gewisser Grundzusammenhang zwischen dem Kartennamen und dem Effekt existieren.
Ein Beispiel dafür wäre, dass eine Feuer Bullet irgendwas mit Feuer zu tun hat und nicht den Himmel Frösche regnen lässt,
übertrieben ausgedrückt. Jeder Teil des Spiels sollte im besten Fall mit Flavor strotzen, nicht nur Karten und Spielregeln.



\subsection{Kein Teil des Spieles ohne Flavor}\label{subsec:keinTeildesSpielesOhneFlavor}

Während der Entwicklung von \FF wurde viel Wert darauf gelegt, dass das Spiel so viel Persönlichkeit wie möglich hat.
Vor allem durch die aufkommenden AI-Modelle war es dem Entwicklungsteam wichtig, dass es merkbar ist,
dass das Produkt hundertprozentig Werk einer Person ist und nicht einer KI.
Der Flavor und die kleinen Eigenheiten von \FF tragen dazu bei.
Das ganze Spiel ist voll mit Anspielungen, Inspirationen von anderen Medien und kleinen Sprüchen oder Witzen. Mehr zu Inspirationen und Kreativfindungsmethoden
kann in einem späteren Kapitel gelesen werden, wichtig ist jedoch, dass sich der beschriebene Flavor konstant im ganzen Spiel zu finden ist.
Zwei Beispiele dafür sind zum Beispiel die Encounter Modifier, welche ähnlich wie Cardeffect Flavor auf die Spielmechaniken
angepasst wurden oder die eigens für das Spiel erstellte Schrift, die immer wieder im Spiel zu finden ist. Auch Ortsnamen
haben oft ein Wortspiel mit \quoted{Bullet} wie zum Beispiel \quoted{Tabu Letter Outpost} oder \quoted{Aqua Balle}
Ein wichtiger Einfluss des Flavors ist außerdem die Welt von \FF und ihre Eigenheiten und Worldbuilding.



\subsection{Storytelling und Worldbuilding}\label{subsec:storytellingUndWorldbuilding}

\FF spielt in einer fiktionalen Version der USA, in der sogenannten Frontier. Gut 40 Jahre vor den Ereignissen von \FF
entdeckte ein Reisender die Frontier, die bewohnt wurde vom einheimischen Stamm der Onathahans. Heutzutage ist die Frontier
eine scheinbar gesetzlose Zone, gefüllt mit Outlaws und Personen, die ihr Glück in der Frontier finden wollen.
Aus unbekannten Gründen sind verhexte Bullets in der Frontier verteilt, welche von den Outlaws zu Geld gemacht werden wollen,
oder um sich diese starken Bullets zu eigenem Vorteil zu nutzen. Was genau vor 40 Jahren vorgefallen ist, was mit den Onathahans
passiert ist und mehr wird im Laufe des Spiels durch Flavortexte und Dialoge erklärt. Sollte der Spieler die Texte lesen,
was ihm jedoch offen gestellt ist ob er dies tut, kann er die einzelnen Puzzlestücke zusammensetzen und damit die Welt von \FF kennenlernen.
Die Flavortexte auf den Karten können aufgrund der zufälligen Reihenfolge in der der Spieler sie zu Gesicht bekommt in beliebiger Reihenfolge gelesen werden.


Der Ansatz nennt sich \quoted{nicht lineares Storytelling} und ist zum Beispiel auch in Magic the Gathering zu finden.\zit{nonlinearstorytelling}


Die Flavortexte der Karten bieten dem Spieler Informationen über die Ereignisse von vor 40 Jahren und die Mysterien,
die die Frontier und die Bullets umgeben. Dialoge mit Charakteren liefern jedoch Informationen über den Handlungsstrang,
der sich in der Gegenwart abspielt. Dazu gehört auch zum Beispiel der mysteriöse Governor, der großes Interesse am Sammeln der verhexten Bullets hat.


Jede Story-Interaktion in \FF ist freiwillig und kann daher ignoriert werden, falls der Spieler kein Interesse an der Story hat.
Das Worldbuilding ist jedoch über das ganze Spiel verteilt. Bullets, welche den Revolver nach links, statt nach rechts drehen,
werden mit den Onathahans in Verbindung gebracht, welche von Außenstehenden einfach nur Hexen genannt werden.
Auch Orte und Biome werden von der Story beeinflusst, wie zum Beispiel Salem, die damalige Heimat der Onathahans, oder Aqua Balle,
der Eingang zur Frontier, an dem sich alle Outlaws sammeln, welche die Frontier als Ziel haben.


Die bereits erwähnte Schrift stellt die Schrift der Onathahans dar, um dem Worldbuilding mehr Glaubwürdigkeit zu geben,
ähnlich wie J.R.R. Tolkien eine eigene Sprache für Elben in seinem Meisterwerk \quoted{Herr der Ringe} entwickelte. \zit{elbenSprache}

% resets author
\renewcommand{\kapitelautor}{}
