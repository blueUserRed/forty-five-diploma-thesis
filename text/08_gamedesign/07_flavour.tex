
\section{Flavour}\label{sec:flavour}

\renewcommand{\kapitelautor}{Autor: Irgendwer} % todo: replace

\subsection{Die Wichtigkeit des Settings und des Flavors für ein Kartenspiel}\label{subsec:wichtigkeit-des-flavours}

%
Mit Flavor, im bezug auf Kartenspiele, ist gemeint, welche Geschichte und Hintergründe hinter einer Karte steckt bzw wie gut die Art, der Effekt,
die Geschichte hinter jeder KArte zu dem Konzept/Namen der Karte passt. Flavor gibt jeder Karte etwas eigenes, was die karte von anderen unterschiedet.

"Flavor is the Soul of the game"\cite{soulOfTheGame}


Flavor kann durch verschiedene Art auftretten. Es kann Flavor in Karten und generell in spielkonzepten/Regeln exestieren.
Es verknüpft die Karten, das Setting, die Story und die Spielregeln. Grob kann ein Kartenspiel in Funktion und Flavor aufgeteitlt werden.
Beide exestieren zusammen und keines ist wichtiger als das andere. Unter Funktion versteht wie das Spiel gespielt wird was regeln und Mechaniken einschließt. \cite{flavorAndFunction}

% text goes here
%
\subsection{Flavor durch Spielmechaniken und Spielmechaniken durch Flavor}\label{subsec:flavour-durch-mechaniken}

Flavor und Funktion, also Spielnechaniken ist das erste über das sich Gednken gemacht werden sollte.
Flavor in den Spielregeln, erlaubt spielern auch Wissen, das sie Bereits besitzen, in dem Spieel anzuwenden und das Spiel dadurch besser zu verstehen. \cite{flavorAndFunction}


In \FF zum Beispiel, ist das spielfeld der Revolver. Vielen Spielern wird bereits bewusst sein, dass man Bullets in einen Revolver reinlädt und da alle Karten Bullets sind kann sich der Spieler
denken was er zu tun hat. Die Mechanik des Kartens in das Runde Spielfeld zu laden, greift in den Flavor, der besagt diese Karten sind Bullets und das runde Spielfeld ist ein Revolver.
Wären die Karten keinen Bullets sondern kreaturen und der Revolver kein runder Kreis der rotieren kann, wäre \FF nicht \FF sondern ein anderes Spiel mit dementsprechend auch anderem Setting.


Die Rotation und Eigenschaften des Spielfeldes in \FF machen nur Sinn, da es ein Revolver ist.
Das Karten Bullets sind, macht nur Sinn, weil das Setting im Wilden Westen spielt und das Spielfeld ein Revolver ist.
Das das Spielfeld ein revolver ist, macht nur sinn, da er nach rechts rotiert und die vorderste Karte rausschießt wenn er geschossen wird.
Ohne den Flavor sind die SPielregln von \FF unnachfolziebar und können ohne das andere nicht exestieren.


Anders gesagt gibt der Flavor einem Kartenspiel, welches eigenltich nur ein abstraktes Konzept aus Nummern und Regeln ist,
an welche sich Menschen halten, damit ein Spiel zustandekommt, einen Grund für die Funktionalitäten.


Auch in zum Beispiel Magic the Gathering, Inscription oder Balatro ist dieses Konzept sichtbar. In Magic ist das Spielkonzept breit und umfangreich.
Das Setting ist nicht so streng definiert wie in \FF, was mehr verschiedene Spielmechaniken und Karten erlaubt.
Dafür kann das Spiel sich manchmal nicht wirklich mehr wie Magic anfühlen,
was bei \FF durch die strikte einhaltung der Machaniken und der Karten, also  Bullets und der revolver nicht passieren kann.
Der Flavor von \FF ist stark und das durchziehende Konzept der Bullets so konstant, das das SPiel und seine Regeln immer
wiederzuerkennen sind. Ähnlich ist es bei Balatro, ein pokerinspiriertes roguelite deckbuilding game.
So wie die sammelbaren Karten in \FF Bullets sind, sammelt der Spieler in Balatro verschiedene Joker KArten.
Obwohl Balatro erst veröffentlicht wurde als die erste Release version von \FF die entwicklung bereits abgeschlossen hatte,
ist das Konzept von den wirklich verschiedenen und komischen Jokern von Balatro sehr ähnlich zu den verschiedenen und
komischen Bullets von \FF.
Inscription jedoch verfügt über weniger konstantes Karten design, auch wenn jede Karte ein anderes Tier darstellt,
legt dafür aber wichtigkeit auf das setting und Stimmung die das SPiel rüberbringt. Durch opfern von Karten können stärkere Karten
gespielt werden, was gut zu dem Horrorsetting passt. Auch Lebenspunkte werden abgerechnet durch herausgerissene Zähne und Augen.
All diese Beispiele zeigen wie Spiele aus dem Selben oder ser ähnlichen genre sich durch ihr setting, Falvor und Spielmecganiken unterscheiden.
%TODO Quellen von den Spielen!!


Die erschaffung der Spielregeln und damit auch des Flavors von \FF, gingen Hand in Hand zu einander.
Zuerst kam die Idee des Setting des Wilden Westens, dann das Spielkonzept.
Mehr zu Methoden der Ideenfindung kann später nachgelesen werden. Wichtiig für dieses Kapitel ist,
das selbst nach zwei Jahren Arbeit und vielen Änderungen an Spielreglen, das Grundkonzzept noch immer das selbe ist.
Viele Karten, die damals im Notizuch aufgeezichent wurden, befinden ssich fast ident noch immer im Spiel.
%TODO bilder von Karten damals vs heute usw



Spielmechaniken ist jedoch nicht das einzige, in dem Flavor eine wichtige Rolle spielt

%
% text goes here
%

\subsection{Cardflavor}\label{subsec:cardflavor}

Cardflavor ist der Flavor einzelner Karten.
Dazu gehören ihr Name, ihre Art, ihr Effekt und wie der Effekt mit der Name der karte zusammenhängt,
ihr Platz in der Welt und die Hintergrundgeschichte dieser Karte. Je nach Karte sind mehr oder weniger der gerade ganannten
eigenschaften vertreten.


Flavortext:
Zusätzlich zu dem Effekt der Karte, welcher sichtbar wird wenn der Spieler über die Karte hovert, haben einige Karten zusättzlich einen sogenannten Flavortext.


Flavortexte sind zum Beispiele auch in Magic the Gathering vertreten und beschreiben die Welt und zeigen Zitate von verschiedenen Charackteren in der Magic-Welt.\cite{magicarena,soulOfTheGame}%TODO BSP und quelle


In \FF kann ein Flavortext entweder Worldbuilding sein oder auch ein Spruch, Witz oder Anspielung die zur Karte passt.

%TODO bilder mit beispielen

Karteneffekte:
Die zweite Art einer Karte Falvor zu verleihen ist es, den Karteneffekt auf den Namen der KArte anzupassen.
Ein Beispiel dafür ist Golddiggers Bullet.


In den meisten Kartenspielen ist das Ziehen von karten ein irseniger Vorteil
für einen Spieler, da mehr Karten auswahl auch eine größere Kontrolle bedeutet.
Das wird in den Communities umgangsprachlich Value genannt. \cite{whatsvalue}


Auch in \FF spielt Value eine große Rolle.
Um den Karten Flavor zu geben, wurden alle Karten, die Karten ziehen einer Bullet zugeordnet, weleche auf eine Art und Weiße ein wertvolles Material darstellt.
Beispiele dafür sind zum Beispiel Silver Bullet oder Gold Bullet. Diese verbindung zwischen wertvollen
Materiallien und Karten ziehen ist für viele Spieler nicht sofort erkennbar, jedoch werden die meisten Spieler die verbindung zu Golddiggers Bullet verstehen.
Golddiggers Bullet hat den effekt, dass immer wenn der Spieler eine Karte zieht, sie gestärkt wird und dem Spieler Reserves gibt.
Goldiggers Bullet die wertvollen Karten wie gold Bullet ziehen also Karten, wovon Golddiggers Bullet profetiert und die
gezogenen Karten in "Geld" also reserves umwandelt, mit dem wieder Karten gespielt werden können. Auf einem ähnlichen Konzept basiert Gravediggers Bullet, die davon profetiert wenn Karten "sterben", also zerstört werden.
Diese Flavor Effekte ziehen sich durch das ganze Spiel und nutzen die Spielmechaniken auch um die Konzepte der Bullets umzusetzten.
Rotten Bullet zum Beispiel verliert mit jedem Mal die sie rotiert einen dmg ihrer dmg-value. Gold Bullet, wird immer mehr
wert, mit der zeit, da sie wenn sie abgeschossen wird, also "verkauft wird" x Karten, wobei x die Nummer der Drehungen
von Gold Bullet ist. Es werden also Eigenschaften von Gold, nähmlich dass es immer mehr wert wird, je länger man es Besitzt,
in die Spielmechaniken von \FF integriert.


Kartenflavor ist unter anderem wichtig für das Kartenspiel, da damit Unglaubwürdigkeiten aus dem Weg geschafft werden.
Nicht alle Karten erhalten die Qualität des Cardflavors wie zum Beispiel Golddiggers Bullet, was jdoch auch nicht wirklich nötig ist, jedoch muss ein gewisser Grundzusammenhand zwischen dem Kartennamen und dem Effekt exesieren.
Ein Beispiel dafür wäre, das eine Feuer Bullet irgendwas mit Feuer zu tun hat und nicht den Himmel Frösche regnen lässt,
übertrieben ausgedrückt. Jeder Teil des Spieles sollte im besten Fall mit Flavor strotzen, nicht nur Karten und Spielregeln.



\subsection{Kein Teil des Spieles ohne Flavor}\label{subsec:keinTeildesSpielesOhneFlavor}

Wärend der Entwicklung von \FF wurde viel Wert darauf gelegt, das das spiel sich so viel persöhnlichkeit wie möglich hat.
Vorallem durch die aufkommenden AI-Modele war es dem Entwicklungsteam wichtig, das es merkbar ist,
das das Produkt hundert prozentik werk einer Person ist und nicht einer KI.
Der Flavor und die kleinen Eigenheiten von \FF tragen dazu bei.
Das ganze Spiel ist voll mit Anspielungen, Inspirationen von anderen Medien und kleinen Sprüchen oder Witzen. Mehr zu Inspirationen und Kreativfindungsmethoden
kann in einem späteren Kapitel gelsen werden, wichtig ist jedoch, das sich der Beschriebene Flavor konstant im ganzen Spiel zu finden ist.
Zwei Beispiele dafür sind zum Beispiel die Encounter Modifier, welche ähnlich wie Cardeffect Flavor auf die Spielmechaniken
angepasst wurden oder die eigens für das Spiel erstellte Schrift, die immer wieder im Spiel zu finden ist.
Ein wichtiger einfluss des Falvors ist außerdem die Welt von \FF und ihre eigenheiten und Worldbuilding



\subsection{Storytelling und Worldbuilding}\label{subsec:storytellingUndWorldbuilding}





% resets author
\renewcommand{\kapitelautor}{}
