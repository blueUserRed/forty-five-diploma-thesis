
\section{Flavour}\label{sec:flavour}

\renewcommand{\kapitelautor}{Autor: Irgendwer} % todo: replace

\subsection{Die Wichtigkeit des Settings und des Flavors für ein Kartenspiel}\label{subsec:wichtigkeit-des-flavours}

%
Mit Flavor, im bezug auf Kartenspiele, ist gemeint, welche Geschichte und Hintergründe hinter einer Karte steckt bzw wie gut die Art, der Effekt,
die Geschichte hinter jeder KArte zu dem Konzept/Namen der Karte passt. Flavor gibt jeder Karte etwas eigenes, was die karte von anderen unterschiedet.

Flavor kann durch verschiedene Art auftretten. Es kann Flavor in Karten und generell in spielkonzepten/Regeln exestieren.
Es verknüpft die Karten, das Setting, die Story und die Spielregeln. Grob kann ein Kartenspiel in Funktion und Flavor aufgeteitlt werden.
Beide exestieren zusammen und keines ist wichtiger als das andere. Unter Funktion versteht wie das Spiel gespielt wird was regeln und Mechaniken einschließt. \cite{flavorAndFunction}

% text goes here
%
\subsection{Flavor durch Spielmechaniken und Spielmechaniken durch Flavor}\label{subsec:flavour-durch-mechaniken}

Falvor und Funktion, also Spielnechaniken ist das erste über das sich Gednken gemacht werden sollte.
Flavor in den Spielregeln, erlaubt spielern auch Wissen, das sie Bereits besitzen, in dem Spieel anzuwenden und das Spiel dadurch besser zu verstehen. \cite{flavorAndFunction}

In \FF zum Beispiel, ist das spielfeld der Revolver. Vielen Spielern wird bereits bewusst sein, dass man Bullets in einen Revolver reinlädt und da alle Karten Bullets sind kann sich der Spieler
denken was er zu tun hat. Die Mechanik des Kartens in das Runde Spielfeld zu laden, greift in den Flavor, der besagt diese Karten sind Bullets und das runde Spielfeld ist ein Revolver.
Wären die Karten keinen Bullets sondern kreaturen und der Revolver kein Runder Kreis der rotieren kann, wäre \FF nicht \FF sondern ein anderes Spiel mit dementsprechend auch anderem Setting.

Die Roattion und eigenschaften des PSielfeldes in \FF machen nur Sinn, da es ein Revolver ist.
Das Karten Bullets sind, macht nur Sinn, weil das Setting im Wilden Westen spielt und das Spielfeld ein Revolver ist.
Das das Spielfeld ein revolver ist, macht nur sinn, da er nach rechts rotiert und die vorderste Karte rausschießt wenn er geschossen wird.
Ohne den Flavor sind die SPielregln von \FF unnachfolziebar und können ohne das andere nicht exestieren.

Die erschaffung der Spielregeln/des Spielkonzeptes und damit auch des Flavors,


%
% text goes here
%



\subsection{Cardflavor}\label{subsec:flavour-durch-mechaniken}


%für später
In Magic: The Gathering zum Beispiel
gibt es sogenannte Flavortexte, kleine Texte die einen Spuruch oder ähnliches aus der Geschichte von Magic zeigen, sei es jetzt über einen charakter, oder die Welt.
Forty-five verfügt auch über Flavortexte, jedoch werden sie ein wenig anders benützt. Nicht jede Karte hat einen Falvortext,  die Bullets die jedoch einen haben, haben entweder Worldbuilding,
Context für die Story, woraf zu einem späteren Zeitounkt nochh zu sprechen gekommen wird, Einen Anspielung auf ein anderes Medium, oder einen Witz/Spruch der nochmal
die Bullet unterstreicht. Flavortexte sind jedoch nicht der einzige Weg Flavor in Karten zu bringen.


\subsection{Kein teil des SPieles ohne Flavor}\label{subsec:flavour-durch-mechaniken}

\subsection{Story und Worldbuilding}\label{subsec:flavour-durch-mechaniken}


% resets author
\renewcommand{\kapitelautor}{}
