
\section{Bullets und Effekte}\label{sec:bullets-und-effekte}

\renewcommand{\kapitelautor}{Autor: Philip Jankovic}


\FF in game visuals sind Handgezeichnet und oft statisch. Beim zeichnen wurde zuerst eine reference rausgesucht.


\subsection{Referances}\label{subsec:references}
Referances beimZecchnen sind Bilder oder andere Artworks an denen sich oritentiert wird. Vorallem am Anfang ist das
Arebiten mit References fast ein muss, da das einfache vorstellen von der zeichnung im kopf und das anschließende zeichnen davon eine Meisterklasse ist.
Perspektive, die art und weiße sie sich kleidung verhält und Körperposen können von der Referance inspierite werden.
Jedoch wird versucht die referenace nicht 100\% abzuzeichnen, sondern auch seinen eigenen stil und touch hinzufügen. \zit{referance}


Für \FF wurden referenaces von Pinterest gesucht, jedoch auch Inspiration aus Filmen und anderen Medien bezogen.


Ein Beipsiel für die nutzung für die verschiedenesten References ist der Manga "JOJO's Bizzzar Adventure" von Hirohiko Araki\zit{jojo},
welcher acuh eine inspiration für einige Teile von \FF ist, nicht nur Art-Wise sondern auch von den Krativfindungsmethoden, welche von dem Author benützt werden, mehr dazu in dem
Kapitel zu Kreativfindungsmethoden. %TODO verweis
Araki verwendet eine große auswahl als referances für seine Art, vorallem aus der Welt
der Fashion, was auch in seinem Artstyle zu sehen ist.
%TODO bilder von ARAKI und Referances!!! https://jojowiki.com/Reference_Gallery#2-115


\FF bezieht ein zwei references von "JOJO's Bizzzar Adventure", jedoch ist die Auswahl der benützen references groß und abweichslungsreich.
Auch eigene refernece Bilder wurden verwnedet für den Main charackter des Spieles.
%TODO selbst gemachtes refernce bild und jojo pointing finger in the air vs im spiel dann


Bei dem nutzen einer refernce muss jedoch darauf geachtet werden, das Bild nicht nur abzuzeichnen. Je nach zeichenstil
oder wie stilisiert ein Zeichenstil ist ist es eine Gafahr, ein  Bild einfachjh nur abzuzeichnen, anstatt es als refernace zu vernweden.
Das bewahrens eines eigenen Zeichenstiles ist dabei besonders wichtig.


\subsection{Artsytle von \FF}\label{subsec:references}

Der Artstyle von \FF ist in einem Comic stil gehalten, mit groben, beleistift outlines und coloring auf der Ebene darunter.
Coloring wird mit einem farb pinsel auf 100\% fluss gemacht, damit die Kanten schön hart sind und sich besser von dem hintergrund abtrennen.
Auch wenn die Zeichnungen selber nicht relasitsich gezeichnet sind, halten sie sich an realistische posen und propertionen, sind also nicht wirklich stilisiert.
Das Spiel nimmt sich dadurch nicht zu ernst, passt gut zu dem rauen wild west setting und zeigt trotz all dieser komischen Bullet
trotzdem einen Anker in der realität. Wichtig ist es, dass das spiel sich selber nicht alzu ernst nimmt, da die Karten und sprüche eher
spielerisch und unernst sind.
%TODO Bilder!!!

Um den zeichnungen mehr tiefe zu geben wird über der color ebene eine Muliplate-Ebene auf 50\% oppacity verwendet um schatten
darzustellen. Geshaded wird mit dem selben Pinsel mit welchem gecolored wird. Das shading bleibt bei diesem einen grauen ton,
härtere shadows werden durch crosshatching mit dem outline pinsel gemacht.\zit{crosshatching}
%TODO BILDER

%
% evolution der artstyles


%artstyle
%

% resets author
\renewcommand{\kapitelautor}{}
