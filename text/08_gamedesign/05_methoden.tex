
\section{Methoden}\label{sec:methoden}

\renewcommand{\kapitelautor}{Autor: Philip Jankovic}

%

\subsection{Kreativfindungsmethoden}\label{subsec:kreativfindungsmethoden}

Während der Entwicklung von \FF werden verschiedenste eigene Kreativfindungsmethoden verwendet. Diese werden nicht wirklich
geplant oder eingeleitet, sondern kommen eher an verschiedenen Zeitpunkten. Was sie jedoch verbinden, ist eine zusätzliche
tätigkeit, welche vertraut ist und die schon so gut gekonnt wird, das sie auch \quoted{automatisch} gemacht werden kann.
Dazu gehören \zB Autofahrten, duschen oder Skifahren. Außerdem wird sich viel an der Welt rundherum inspiriert.
Konzepte, welche in der echten Welt zu finden sind, sind der Grundstein von \FF's Kreativfindung.

\begin{coolQuote}
Good ideas come from one's life experience, losing interest in the world means losing the ability to come up with ideas.
If you're always able to maintain interest in something, and keep your antennea poised to pick up on and react with openess
to the occurences around you, you won't run out of ideas.
\end{coolQuote}
\zidbuch{mangabook}

Dieses Zitat ist eine große Inspiration für Philip Jankovic. Es beschreibt gut, weshalb er gerne reist und sich die Welt
anschaut und dadurch auf seine Ideen kommt. Fast alle Bullets in \FF hatten zuerst einen Namen und der Effekt wurde
im nachhinein hinzugefügt, falls ein passender gefunden wurde.


\subsection{Zahlen verändern und Usability Tests}\label{subsec:usability}

Bei der Enticklung eines Kartenspieles müssen immens viele Änderungen über den Lauf der Entwicklung gemacht werden.
Bei \FF wurde eine Kreislauf von Definieren und Testen festgelegt. Oft wird die ursprüngliche Zahl, seien es Kosten,
Schaden oder Gegnerangriffe grob festgelegt. Wenn der neue Build fertig ist, wird geschaut ob die Zahlen passen, aufgeschrieben und danach
je nachdem angepasst. Es werden sich jedoch bei der urspünglichen Definition neue Effekte und Zahlen gedanken gemacht.
\zB Werden Bullets, welche indirekten Schaden durch \zB \quoted{On-Turn-Beginn} machen, wie \quoted{Unpleasent Gradient Bullet}, einen niedrigen Schaden als andere Bullets.
Je nachdem welchen Effekt die Karte haben werden Kosten-Wert und Damage-Wert angepasst.

% kreativfindung und brainstorm methoden


% araki zitat über kreativität

% Inspirationen
%

% resets author
\renewcommand{\kapitelautor}{}
