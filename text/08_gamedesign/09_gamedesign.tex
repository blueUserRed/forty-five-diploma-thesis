
\section{Gamedesign}\label{sec:differenzierung}

\renewcommand{\kapitelautor}{Autor: Philip Jankovic} % todo: replace


\subsection{Gamedesign behind the Bullets}\label{subsec:gamedesignBehindBullets}

Die Vision von \FF's Regeln war es, den Spieler dazu zu bringen, den Revolver vollständig auszunutzen. Damit ist gemeint,
den Spieler dazu zu bringen, jegliche Mechaniken des Revolvers zu nutzen und kreative Möglichkeiten zu finden, ihn auszunutzen.
Das folgende Problem ist ein Beispiel für das Problem der \FF Regeln, welches das Kartendesign beheben musste.
Der Spieler zieht ganz normal seine Karten und platziert eine Kugel im vordersten Slot des Revolvers.
Anstatt sich einen Revolver mit mehreren Kugeln aufzubauen, platziert er die Kugel und schießt sie sofort ab.
Das Gleiche passiert mit der nächsten Kugel. Diese Art von Gameplay ist genau das, was vermieden werden wollte.
Es ist langweilig, unkreativ und fordert keinerlei Denkanstrengungen oder Gedanken des Spielers.
Dieses Problem war auch das Problem einer früheren Version von \FF. Um dieses Problem zu beheben, wurde ein Designprinzip
ins Leben gerufen, welches sich "Placement Matters" nennt.


\subsection{Placement matters and you do too :)}\label{subsec:placementMatters}

"Placement Matters" ist ein Begriff, der das Gamedesign von \FF gut beschreibt. Durch den Revolver, seine limitierenden
Eigenschaften und den begrenzten Platz entsteht durch diese Designphilosophie die erwünschte Spielweise der Spieler.
"Placement Matters" beschreibt das Konzept, dass beim Platzieren einer Kugel der Ort, die Reihenfolge und der Zeitpunkt
eine wichtige Rolle spielen. Durch Trigger wie zum Beispiel "On-Rotate" oder "On-Turn-Begin" profitieren manche Kugeln davon,
am Ende des Revolvers platziert zu werden, damit sie trotz der Rotation so lange wie möglich im Revolver bleiben können.
Der Statuseffekt "Burning" zum Beispiel, erhöht den Schaden der nächsten paar Kugeln um 50\%.
Durch diesen Effekt ist es also ratsam, Kugeln mit diesem Effekt vor anderen Kugeln zu feuern, um mehr Schaden zu erzielen.
Durch "Leaders Bullet" werden alle Kugeln gestärkt, solange sie sich selbst im Revolver befinden. Platziert man sie also
weiter vorne im Revolver, verlieren alle Kugeln den Buff, da "Leaders Bullet" vor ihnen den Revolver verlässt.
Der Effekt von "Guarding Angles Bullet" wiederum bezieht sich zum Beispiel auf einen bestimmten Slot, nämlich Slot 3,
was den Spieler dazu bringt, die Platzierung seiner Karten genau zu überdenken. Diese Einschränkungen und Herausforderungen
sind der Grundstein für das Revolver-Gameplay von \FF.


\subsection{Archetype design und effekt design}\label{subsec:placementMatters}

FF's Karten können grob in Archetypen aufgeteilt werden. Ein Archetyp in Kartenspielen beschreibt eine bestimmte Strategie, welche Karten verfolgen.
Spieler können dann basierend auf diesen Archetypen Decks bauen und damit verschiedene Strategien verfolgen.\cite{whatIsAnArchetype}


In \FF wurde versucht, die Archetypen so breit wie möglich zu gestalten. Das bedeutet, dass es bereits dezidierte Archetypen gibt,
jedoch viele Karten in mehreren Archetypen spielbar sind und die Archetypen sich auch überschneiden.
Während der Entwicklung wurde dieses Konzept als "Broad Archetypes" bezeichnet. Gründe für diesen Ansatz sind unter anderem
die bereits sehr dezidierten und speziellen Spielregeln von \FF und die zufällige Reihenfolge, in der der Spieler die Karten zieht.


Im Vergleich zu Magic, wo die Archetypen strenger sind und sogar durch Farben voneinander getrennt werden,
kann der Spieler sich nicht direkt aussuchen, welche Karten er gerne hätte, sondern muss eine von den drei Karten nehmen,
die ihm vorgelegt werden. Magics Regeln sind außerdem viel breiter gefächert und basieren nicht auf einem limitierten
Konzept wie dem Revolver. Wenn ein Effekt also "On-Rotate" auslöst, ist er in vielen Decks spielbar und nicht nur in Decks
eines Rotation-Archetyps, da die Drehung des Revolvers oft passiert. Diese Methode erlaubt jedoch auch unzählige
Kombinationen und da viele Karten, nicht alle, miteinander funktionieren, hat der Spieler fast immer ein gutes Erlebnis
beim Entdecken einer neuen Kombination.\cite{magicarena}


Ein gutes Beispiel für diese breiten Archetypen und die Interkonnektivität der Archetypen ist zum Beispiel “High Velocity Bullet”.
”High Velocity” ist eine 1 Kosten, 0 Schaden Karte, die über den Effekt verfügt, dass sie dem Gegner Schaden zufügt, wenn sie "On-Leave" ist.
Für viele Spieler scheint diese Bullet jedoch bizarr. Der "On-Leave" Trigger ist bei “High Velocity” dafür verantwortlich,
dass wenn die Karte geschossen wird, der Gegner Schaden bekommt, obwohl “High Velocity Bullet” eine 0 bei ihrer Schaden-Wert hat.
Die Formulierung des Triggers erlaubt es, dass der Effekt von “High Velocity Bullet” durch jegliche Art und Weise, durch
die die Bullet den Revolver verlässt, aktiviert wird. Wenn der Spieler “High Velocity Bullet” das erste Mal verwendet,
ist “High Velocity Bullet” nicht viel anders als eine normale Bullet, jedoch stößt er mit der Zeit auf “Deputies Bullet”.


"Deputies Bullet" hat den Effekt, dass wenn sie im Revolver platziert wird, die Karte im Slot 3 wieder in die Hand
zurückgegeben wird. Auch hier kommt das bereits erwähnte "Placement matters" ins Spiel, da der Spieler sich bewusst sein
muss, welche Karte auf seine Hand zurückgegeben wird, bevor er "Deputies Bullet" spielt. Zuerst wird der Spieler den
Effekt von "Deputies Bullet" als etwas Negatives sehen, um den erhöhten Schaden der Bullet auszugleichen,
eine bewusste Entscheidung, die später noch einmal ins Spiel kommt.


Kombiniert mit “High Velocity Bullet”, kann der eher negative Effekt von “Deputies Bullet” verwendet werden, um
“High Velocity Bullet” den Revolver verlassen zu lassen und damit vorzeitig dem Gegner Schaden zuzufügen.
Zusätzlich dazu, dass der Effekt von “Deputies Bullet”, umgangssprachlich "Bounce" genannt, "On-Leave" Effekte triggert,
ist es eine Möglichkeit Karten mit einem "On-Placedown" Trigger auf die Hand zu geben. Das erlaubt es, den Effekt damit noch einmal zu aktivieren.
Eine andere Möglichkeit ist, dass der Spieler auf eine "Purge Bullet" stößt. “Purge Bullet” hat ähnlich wie “Deputies Bullet”
erhöhten Schaden, dafür ist ihr Effekt, die Zerstörung von zwei Bullets neben ihr, wenn sie platziert wird.
Dies kann jedoch vom Spieler vermieden werden indem er - wieder durch Placement Matters- “Purge Bullet” auf eine Art und
Weise in den Revolver platziert, sodass keine Karte neben ihr liegt und damit auch nicht zerstört wird.
Ähnlich wie bei “Deputies Bullet” jedoch, kann “High Velocity Bullet” mit “Purge Bullet” zerstört werden, was wiederum als
Verlassen des Revolvers zählt, also den Effekt von “High Velocities” Bullet aktiviert. “Purge Bullet” ist außerdem Teil
des Destroy Archetypes, welcher darauf abzielt, dass Karten zerstört werden im Revolver.
Dieser relativ abgesonderte Archetype, kann trotzdem mit “On-Leave” Effekten kombiniert werden, was wiederum das breite
Kartendesign von FF veranschaulicht. %TODO bilder der Bullets
%TODO bild von archetype chart in draw Io umgesetzt


Der breite Ansatz ermöglicht es dem Spieler, mit fast jeder neuen Karte, die er erhält, neue Kombinationen und Nutzungsmöglichkeiten zu entdecken.
Es stärkt außerdem das Verständnis der Spielmechaniken.
Jedoch funktionieren nicht alle \FF Archetypen mit jedem anderen, es würde sonst den Spaß aus dem Deckbau nehmen
Wäre das nämlich der Fall, wäre das Bauen eines Decks komplett obsolet und der Spieler müsste sich keine Gedanken darüber machen, wie er sein Deck zusammenstellt.



\subsection{Kartensammeln und Kartenpools}\label{subsec:placementMatters}

Durch die Vielzahl an Karten und den Unterschieden in der Komplexität zwischen diesen Karten werden Karten in \FF in Pools eingeteilt.
Diese Kartenpools dienen dazu, zu kontrollieren, wann welche Karten vom Spieler nutzbar sind.
Sie halten komplexe und zu starke Karten vom Spieler fern, bis dieser bereit ist, sie zu erhalten.
Zum jetzigen Zeitpunkt gibt es zwei Kartenpools mit jeweils rund 25 Karten. Combos in Pool 1 sind oft einfachere
Versionen der später komplizierteren und stärkeren Combos, die durch Pool 2 Karten möglich sind.
Pool 1 ist darauf ausgelegt, dass die Grundmechaniken von \FF dem Spieler bewusst sind.
%pool, draft usw


\subsection{Backback und Deck Gamedesign}\label{subsec:placementMatters}

Am Anfang des Spiels besitzt der Spieler 7-8 Bullets. Im Laufe des Spiels, sei es durch absolvierte Kämpfe
oder Map-Events, sammelt der Spieler Karten. Wenn eine Karte gesammelt wird, kann der Spieler sich entscheiden,
ob er die Karte in sein Deck gibt oder in den Rucksack. Hat er jedoch die Mindestanzahl an Karten in seinem Deck
noch nicht erreicht, steht ihm nur das Hinzufügen zum Deck offen. Erst wenn der Spieler genug Karten gesammelt hat,
um die Mindestanzahl des Decks einzuhalten, kann er Karten aus dem Deck verschieben. Es gibt einige Dinge
in \FF, die das Bauen bzw. die Wahl des Decks beeinflussen. Eines davon sind die bereits erwähnten Pools und Archetypen.
Je nachdem, wie viele Bullets von einem bestimmten Archetyp ein Spieler besitzt, ist sein Deck mehr ausgeprägt oder auch nicht.
Die fehlenden Karten des Archetyps können aufgefüllt werden durch andere Karten, die ins Deck passen oder generell immer nutzbar sind.
Deckdesign in \FF basiert auf dem sogenannten "Drei Säulen Modell".

%mindestanzahl deck
%deckbauen, und wie es beinflusst wird durch pools und encounter modifier

\subsection{Das drei Säulen Modell}\label{subsec:placementMatters}

Das Drei-Säulen-Modell bezieht sich auf die drei "Arten" von Bullets, die in einem idealen Deck vorhanden sein sollten.
Diese Säulen werden auch beim Designen der Archetypen beachtet, um sicherzustellen, dass jedes Deck über die nötigen
Möglichkeiten verfügt, alle drei Säulen zu erfüllen. Diese Säulen sind ein internes Konzept, das dafür verwendet wird,
sicherzustellen, dass viele verschiedene Decks möglich sind und nicht durch das Fehlen von bestimmten Karten ausgebremst
werden. Der Spieler baut Decks nach dem Säulenkonzept automatisch und durch Ausprobieren, da ihm mit der Zeit bewusst wird,
dass ihm zum Beispiel die Handkarten immer ausgehen und er dementsprechend Maßnahmen ergreift. Dieses richtige Deckbauen
ist auch Teil der Schwierigkeit von \FF, genauso wie das Meistern der Mechaniken. Jede dieser Säulen hat außerdem sogenannte "Staples".
Also Karten, die in jedem Deck gut sind und auch immer in Decks gegeben werden sollten, falls möglich.


Die erste Säule: Ressourcen:
Die Säule der Ressourcen bezieht sich auf Karten, welche dem Spieler einen Vorteil verschaffen. Darin ist der bereits
erwähnte "Value" inkludiert, also das Ziehen von Karten und die Erzeugung von Reserven. Dadurch verschafft sich der Spieler einen Vorteil,
da mehr Karten in der Hand mehr Möglichkeiten für Züge bedeuten und die extra Reserven auch das Spielen dieser Extrakarten ermöglicht.
Anders gesagt erlaubt die Ressourcen-Säule dem Spieler mehr in weniger Zügen zu spielen.
Staples inkludieren Silver Bullet, in späteren Pools auch Gold Bullet für das Ziehen von Karten und Workers Bullet für das Erzeugen von Reserven.
Je nach Archetyp gibt es auch Karten, welche die Rolle der ersten Säule in ihrem jeweiligen Archetyp übernehmen.
Siehe zum Beispiel "Gravediggers Bullet" für den Destroy-Archetyp.


Die zweite Säule: Schutz:
Die Säule des Schutzes beschäftigt sich damit, gegnerische Angriffe abzuwehren. Dazu gehören Bullets, die dem Spieler
Schutz geben oder besonders stark beim Parieren sind. Schutz ist die kleinste und unwichtigste Säule der drei,
da jede Bullet die Möglichkeit bietet, mit ihr zu parieren und damit Schaden abzuwehren. Jedoch sind auch einfach nur große,
starke Bullets wichtig, um hohe Angriffe von Gegnern abzublocken. Da jede Bullet parieren kann, gibt es nicht wirklich Staples,
"Turtle Bullet" ist jedoch die beste Wahl für eine einfache und starke Schutz-Bullet.


Die dritte Säule: Wincon:
Die dritte Säule ist der Kern eines Decks und beinhaltet alle Karten, die dafür verwendet werden, durch die Deckstrategie einen Kampf zu gewinnen.
Das Wort "Wincon" bezieht sich dabei auf die Wincondition, welche je nach Archetyp des Decks unterschiedlich ist.
Ein Rotation Deck hat andere Karten in der dritten Säule als ein Destroy Deck. Stables existieren nur in den jeweiligen Archetypen
und nicht in der Gesamtheit der Wincon, auch wenn es einzelne Karten gibt, die so stark sind, dass sie in mehreren Decks verwendet werden können.
Ein Beispiel dafür ist "Bullet" oder "Bewitched Bullet".


\subsection{Encounter modifier? Was steck dahinter?}\label{subsec:placementMatters}

Der Spieler hat die Möglichkeit, mehrere Decks zu bauen und zwischen bis zu 5 Decks hin und her zu wechseln.
Um den Spieler dazu zu bringen, auch mal ein anderes Deck zu spielen und nicht immer das selbe, und um mehr
Abwechslung ins Spiel zu bringen, wurden Encounter Modifier eingeführt.
Diese Modifier ändern die Spielregeln mal leicht, mal schwer und je nachdem haben große Auswirkungen darauf, wie der Spieler das Spiel spielt.
Ein Beispiel für einen Modifier, welcher das Ändern des Decks als Ziel hat, ist "Moist". Durch "Moist" verliert jede Bullet einen Schadenspunkt, wenn sie sich im Revolver dreht.
Das wirkt sich vor allem auf Decks aus, welche auf Rotationen basieren. Diese Decks sind noch immer spielbar, jedoch geschwächt.
Ein Beispiel für einen Encounter Modifier, welcher mehr Abwechslung ins Spiel bringt, ist "Steel Nerves". "Steel Nerves" führt
einen Timer ein, welcher, falls er Null erreicht, den Revolver automatisch schießt. Das bringt einen gewissen Zeitdruck
ins Spiel und der Spieler muss mit der Stresssituation umgehen und dabei weiterhin versuchen, gute Züge zu spielen, um den Gegner zu besiegen. %TODO Bilder von Encounter Modifier

\subsection{Enemy Gamedesign and difficulty scaling}\label{subsec:placementMatters}

Der Gegner ist ein wichtiger teil des Gameplays von \FF. Nicht nur gilt er als Ziel für die Angriffe des Spielers, ohne ihn würde es kein Gameplay geben.
Der Gegner reagiert nicht auf den Spieler, sondern der Spieler auf den gegener. Er displayed seine nächte aktion über seinem kopf, noch währende der Spieler dran ist.
Das bewirk, dass der Spieler seinen Zug anpasst je nachdem was der Gegner macht. Jedoch kann er es auch einfach igrnoiren,
falls er gerade eine Combo hat oder glaubt die Aktion des Gegner ihm keine Probleme macht.


Ein gegner kann verschiedene Aktionen ausführen, dazu zählt SChaden zu verurscahen, welccher von dem SPieler geparried werden kann,
sich slebst schild zu geben oder eine Gegner spezifische Aktion auszuführen. Der GEgnertyp bestimmt, wie viel schaden der
GEgner macht, wie viel Schild er sich selbst geben kann und welche Aktionen er ausführen kann. Die Witch zum Beisppiel
kann den revolver des Spielers nach links drehen. gegnertypen sind in verschiedene Klassen eingeteilt, was durch ihre
effektivness in verschiedenen Kampfzentarien bestimmt wird. Der Pyro ist ein Support gegner, da er vorallem zusammen mit
anderen Gegnern glänzt, da er den Spieler in Brand setzt, damit die anderen gegner mit ihren Angriffen mehr schaden machen.
Außerdem sorgt die Aktion "Hot Potato" dafür, das dem Spieler die Reserves kanpp werden. "Hot Potato" gibt dem Spieler
Scorching Bullet in die Hand, eine feindliche Bullet, welche dem spieler 10 Schaden verursachtet immer am ende eines zuges
falls sich scortching Bullet in der Hand des Spielers oder im Revolver befindet. Sie kostet 3 reserves, bekommt der Spieler
sie also in die Hand, muss er sich entscheiden ob er die Bullet behält und seine Reserves für etwas anderes verwendet oder
er die drei reserves bezahlt und die Bullet damit in den revolver lädt und wegschießt.





%special attacks
%3phasen prinzip
%difficulty im kampf und auf einer road (auch mit encounter file)
%witch und die änderungen mit der witch


%wichtig waren von marvin die config files
%challanging, da es nicht wirklich etwas zum Nachlesen gab und das für jedes Spiel anders ist

%noch viel zu tun


\subsection{Carddescriptions}\label{subsec:placementMatters}

Wichtig in einem Kartenspiel wie \FF sind die Beschreibungen der Effekte der Karten. Die Terminologie der Beschreibungen
soll konsistent, verständlich und so kurz wie möglich sein, ohne dass Information dabei verloren geht. Bei \FF sind
Effekt Trigger durch mit Bindestrichen zusammenhängende Wörter geschrieben, verknüpft mit dem Wort "on" am Phrasen Anfang, um zu symbolisieren,
dass der Effekt "on" diesem Event getriggert und um den Trigger auch verständlich zu machen ohne ihn zu erklären.
"On-Placedown" zum Beispiel aktiviert den Effekt, wenn die Kugel in den Revolver gelegt wird, also "down geplaced" wird.
Diese Art von Trigger wird intern "Major Trigger" genannt.
Übergeordnet über den "Major Triggern" sind die "Trigger Conditions". Sie werden zuerst gecheckt, wie zum Beispiel die "While in Revolver:" Condition.
Der nachfolgende "On-Turn-Begin" Trigger kann nur getriggert werden, wenn die Condition erfüllt ist, also wenn die Kugel sich im Revolver befindet. %TODO Bild von workers Bullet Effekt hier
Sogenannte "spezifische Trigger" werden ausgeschrieben, da sie zu lang, zu spezifisch und zu selten für die Schreibweise der "Major Trigger" sind. Ein Beispiel dafür ist
"Whenever this rotates into the slot that it was originally placed into:".


Zusätzlich zu abgekürzten Trigger-Bezeichnungen gibt es Keywords. Keywords in \FF sind Effekte, die immer die gleiche
Erklärung haben und deswegen wird die Erklärung in eine extra Info-Box ausgelagert. %TODO BILD
Keywords sind ein guter Weg, erfahrenen Spielern die Infos auf einem Blick zu geben, da sie bereits grob wissen, was das
Keyword bedeutet, trotzdem aber neuen Spielern die Möglichkeit bietet, sich die Erklärung noch einmal durchzulesen.
Keywords werden angewendet bei Trait-Effekten und Status-Effekten. Trait-Effekte sind extra Eigenschaften für Bullets
wie zum Beispiel "spray", durch welchen alle Gegner von der Bullet getroffen werden anstatt nur einer.
Um Status-Effekte und Trait-Effekte variabel zu halten, werden Parameter verwendet. %TODO Bild mit Erklärung zu Parametern noch



%keywords -> für flvor und damit der psiler der schon läänger spielt nicht immer alles lesen muss. und statuseffekte---------------------------------------------------------------------------------------------
%Slots und sloticons


\subsection{Tutoriel}\label{subsec:placementMatters}


%Broad Bullet design für draft

%

%gegner gamedesign :((((

\subsection{Wie eine Entscheidung das Spiel verändert am Beispiel des Kartennachziehens}\label{subsec:placementMatters}

Während der Entwicklung einen Kartenspieles müssen viele Entscheidungen getroffen werden, wie zum Beispiel die Enstcheidung,
Karten, welche gschossen wurden nun unter das deck wieder zu legen. In früheren \FF versionen wurden geschossene Karten einfach aus dem Kampf entfernt.
Jedoch waren die Kämpfe dadurch relativ schnell vorbei für den Spieler, da nur mehr normale Bullets gezogen wurden nachdem das Deck leer gezogen wurde.
Dadurch das das Deck nach der Änderung nie leer wird, bleibt der kampf interessant, anders als mit immer nur der selben normal Bullet.
Das Spiel wurde auf einen Schlag viel dynamischer und es konnten viel mehr verschiedene und stärkere combos ausgeführt werden. Das "Cylen"
von karten ist wichtg, da Karten dadurch in einem Kapmf öfters verwendet werden könnten, ohne drei verschiedene AKrten,
die dasselbe nur ein wenig anders tun in seinem Deck zu haben. Decks sind abweschlungsreicher und das Potenzial der Bullets
kann besser von dem Spieler benutzt werden.


Diese simple Änderung zeigt, wie auch nur die kleinste Entscheidung das Spiel komplett verändern kann, weshalb playtesting
wichtig für ein Spiel wie \FF ist. Dies wurde auch bei der technischen Umsetzung bedacht, weshalb \FF durch so viele config Files verfügt.


\subsection{Gameplay rework ...v23}\label{subsec:placementMatters}
Über die Entwicklung von \FF gab es viele verschiedene Version, manche mit kleinen Änderungn, manche jedoch mit komplett veränderten Mecahniken und regeln.
Die Entwicklung der jetztigen version zog sich über 2 Jahre und unzählige änderungen wurden vorgenommen.


Unter anderem wurde das damilge schutz system der cover Karten mit dem jetztigen Parry system ausgetauscht. Cover Karten
war ein zusätzlicher KArtentyp zu den Bullets, hinter welchen sich der Spieler "verstecken" konnte. Bei tests gab es
jedoch einige Probleme mit den Covern, da sich die Spieler immer nur hinter den Covern verstcekten und sonst nicht wirklich getan haben.
Um diesem Problem entgegbenzuwirken, wurde die parry mechanik eingeführt, welche auch Bullets zum Schutz des Spielers verwendet,
um sich mehr auf den Kern von \FF - die Bullets- zu konzentrieren.Das Einführen von Parry makiert den start der jetzigen \FF version.



\subsection{Angewanntes Gamedeign: The pinacle of forty-five: Bewitched Bullet}\label{subsec:placementMatters}

%angewantes gamedeign: die pinacle of forty-five: Bewtisched Bullet

% resets author
\renewcommand{\kapitelautor}{}
