
\section{Gamedesign}\label{sec:differenzierung}

\renewcommand{\kapitelautor}{Autor: Irgendwer} % todo: replace


\subsection{Gamedesign behind the Bullets}\label{subsec:gamedesignBehindBullets}

Die Vision von \FF 's Regeln war es, den Spieler dazu zu bringen, den Revolver voll und ganz auszunutzen.
Damit ist gemeint, den Spieler dazu zu bringen, jegliche Mechaniken des Revolver zu nutzen und kreative Möglichkeiten zu finden ihn auszunutzen.
Das folgende Problem ist ein Beispiel für das Problem der \FF regeln, welches das Kartdesign beheben musste.
Der Spieler zieht ganz normal seine Karten, und platziert eine Bullet in dem vordersten Slot des Revolvers.
Anstatt sich einen Revolver mit mehreren Bullets aufzubauen, platziert er die Bullet und schießt sie sofort weg. Das selbe passiert mit der nächsten Bullet.
Diese Art von Gameplay ist genau jenes, welcher vermiden werden wollte. Es ist langweilig, uncreativ und fordert keinerei denkanstrengungen oder Gedanken der Spielers.
Dieses Problem war auch das Problem einer früheren Version von \FF. Um diese Lösung zu beheben wurde ein Designprinzip ins Leben gerufen, welches sich "Placement Matters" nennt.


\subsection{Placement matters and you do too :)}\label{subsec:placementMatters}

Placement Matters ist ein Bergiff, welches das Gamedeign von \FF gut beschreibt. Durch den Revolver, seine limitierenden
Eigenschaften und den limitierten PLatz ensteht durch diese Deisgn Philosphooie wurde die erwünschte Spielweiße der Psieler erreicht.
Placement Matters, beschreibt das Konzept, das bei dem platzieren einer Bullet der Ort, die Reihenfolge und der Zeitpunkt eine wichtige rolle spielt.
Durch Trigger wie zum Beispiel dem "On-Rotate" oder "On-Turn-Begin" profetieren manche Bullet davon, am Ende des Revolver platziert zu werden, damit sie trotz der rotation so lange wie möglich im revolver bleiben können.
Der Statuseffekt "burning" zum Beispiel, erhöht den Schaden der nächsten paar Bullets um 50\%. Durch diesen Effekt, ist es also ratsam, Bullets mit diesem Effekt vor anderen Bullets zu feuern um mehr Schaden rauszuholen.
Durch Leaders Bullet werden Alle Bullets gestärkt solange sie sich slebst im Revolver befindet. Platziert man sie also
weiter vorne im revolver, verlieren alle Bullets den Buff, da Leaders Bullet ja vor ihnen den Revolver verlässt.
Der Effekt von Guarding Angles Bullet bezieht sich zum beispiel auf einen gewissen Slot, nähmlich slot 3, was den Spieler auch dazu bring,die Platzierung seiner Karten genau zu überdenken.
Diese einschränkungen und Challanges sind der Grundstein für das revolver Gameplay von \FF.




\subsection{Archetype design und effekt design}\label{subsec:placementMatters}

\FF's Karten können grob in archetypes aufgeteilt werden. Ein rachetype in Kartenspielen beschreibt eine bestimmte Strategie, welche Karten verfolgen.
Spieler können dann basierend auf diesen Archetypes Decks bauen und damit verschiedene Strategien verfolgen.\cite{whatIsAnArchetype}


In \FF wurde versucht, die Archetypes so breit wie möglich zu gestallten. Das bedeutet, dass es schon dezentierte Archetypes
gibt, jedoch viele Karten in mehreren Archetypes spielbar sind und die Archetypes sich auch überscheiden.
Während der Entwicklung wurde dieses Konzept als "Broad Archetypes bezeichnet". Gründe für diesen Ansatz sind unter anderem,
die bereist sehr dezidierten und speziellen spielregeln von \FF  und die zufällige Reihenfolge in der der Spieler die Karten zieht.


Verglichen mit Magic, wo die Archetypes strikter sind und sogar durch Farben von einander getrennt werden, kann der
Spieler sich nicht direkt aussichen welcher Karten er gerne hätte, sondern muss eine von den drei Karten nehmen, die ihm vorgelegt werden.
Magics regeln sind auserdem viel breiter gefächter und basieren nicht auf einem lmitierten Konzept wie der Revolver.
Wenn eine Effekt also "On-Rotate" triggert, ist sie in vielen Decks spielbar und nicht nur in Decks eines Turn Archetypes,
da die Drehung des Revolvers etwas ist, was oft passiert. Diese Methode erlaubt jedoch auch unmengen an Kombos und da
viele Karte, nicht alle, miteinander funktionieren, hat der Spieler fast immer eine gutes Erlebniss beim Entdecken einer neuen Combo.


Ein gutes Beispiel für diesen broad arcetypes und die interkonektivität der Archetypes ist zu Biepiel High velocity Bullet.
High Velococity ist eine 1 Kosten, 0 dmg Karte, welche über den Effekt verfügt, dass sie sie "on-Leave" them gegner Scahden zufügt.
Für viele Spieler scheint diese Bullet jedoch bizzar. Der On-Leave Trigger ist bei High Velocity dafür verantwortlich,
dass wenn die Karte geschossen wird, trotzdem der Gegner dmg bekommt, obwohl high veolcity eine 0 bei ihrer dmg value stehen hat.
Die formolierung des Triiggers erlaubt es, das der Effekt von high veolcity Bullet durch jeglich Art und Weise durch die die bullet den revolver verlässt aktiviert wird.
Wenn der Spieler High velocity Bullet das erste Mal verwednet ist High velocity bUllet nicht viel anders als eine normale Bullet,
jedoch stößt er mit der Zeit auf Deputies Bullet.


Deputies Bullet hat den effekt, dass wen sie im Revolver platziert wird
die Karte in Slot 3 wieder in die Hand zurückgegeben wird. Auch hier komm das bereits erwähnte Placement matters in spiel,
da der spieler sich beusst sein muss, welche Karte auf seine Hand zurüückgegeben wird bevor er deputies Bullet spielt.
Zuerst wird der Spieler den Effekt von Deputies Bullet als etwas negatives sehen, um den erhörten dmg der Bullet auszugleichen,
eine bewisste Entscheidung die später noch einmal ins Spiel kommt.

Kombiniert mit High Velocity Bullet, kann der eher
negative effekt von Deputies Bullet verwendet werden, um High velocity den Revolver verlassen zu lassen und damit vorzeitig dem Gegner Schaden zuzufügen.
Zusätzlich dazu, das der Effekt von Deputies Bullet, umgangssprachlich "bounce" genannt "On-Leave" Effekte triggert,
ist es eine Möglichkeit Karten mit einem "On-Placedown" Trigger auf die Hand zu geben. Das erlaubt es, den Effekt damit nocheinmal zu aktivieren.
Eine andere Möglichkeit ist, dass der spieler auf eine "Purge Bullet" stößt. Purge Bullet hat ähnlich wie deputies Bullet
erhöhten Schaden, dafür ist ihr Effekt, die szerstörung von zwei Bullets neben ihr, wenn sie platziert wird. Dies kann
jedoch vom Spieler vermieden werden indem er - wieder durch Placement matters- Purge Bullet auf eine Art und weise in
den revolver platziert, sodas keine Karte neben ihr liergt und damit auch nicht zerstört wird. Ähnlich wie bei Deputies
Bullet jedoch, kann High veolcity Bullet mit Purge Bullet zerstört wrde, was wiederum als verlassen des Revolvers zählt,
also den Effekt von High Velocities Bullet aktiviert. Purge Bullet ost außerdem Teil des destroys Archetype, welcher
darauf abzeihlt, dass Karten zerstört werden im revolver.
Dieser relativ abgesonderte Archetype, kann trptzdem mit on-Leave Effekten kombiniert werden, was wiederum das breite Karten design von \FF veranschaulicht. %TODO bilder der Bullets
%TODO bild von archetype chart in draw Io umgesetzt


Der broad approach ermöglicht den Spieler mit fast jeder neuen Karte die er erhält neue Combos und nutzungsmöglcihkeiten zu endecken.
Es stärkt außerdem das VErständniss der Spielmechaniken.
Jedoch funktionieren nicht alle \FF Archetypes mit jedem anderen. %würde sonst den spaß aus dem deckbauen nehmen
Würde das nähmlcih der Fall sein, wär das Bauen eines Deckes komplett obsolet und der Spieler müsste sich keine
Gedanken darüber machen wie er sein Deck zusammenstellt.

%halt mit placement matters im hinterkopf
%broad apporach
% Archetypes und ihre verbindung zueinander


\subsection{Kartensammeln und Kartenpools}\label{subsec:placementMatters}

Durch die Menge an Karten und den Unterschieden in Komplexität zwischen jenen Karten werden Karten in \FF in Pools eingeteilt.
Diese Kartenpools sind da, um zu kontrollieren, wann welxhe karten von dem Spieler nutzbar sind.
Es hält komplexe und zu Starke Karten vom Spieler fern, bis der nicht dazu bereit ist, sie zu bekommen.
Zu dem jetztigen zeitpunkt gibt es Zwei Kartenpools mit jeweils rund 25 Karten. Combos in pool 1 sind oft simplere
Versionen von den später komplizierteren und stärkeren Combos, welche durch pool 2 Karten möglich sind.
Pool 1 ist darauf ausgelegt, dass die Grundmechaniken von \FF dem Spieler bewusst sind.
%pool, draft usw


\subsection{Backback und Deck Gamedesign}\label{subsec:placementMatters}
%mindestanzahl deck
%deckbauen, und wie es beinflusst wird durch pools und encounter modifier

\subsection{Das drei Säulen Modell}\label{subsec:placementMatters}

\subsection{Encpunter modifier? Was steck dahinter?}\label{subsec:placementMatters}


\subsection{Enemy Gamedesign and difficulty scaling}\label{subsec:placementMatters}


%special attacks
%3phasen prinzip
%difficulty im kampf und auf einer road (auch mit encounter file)
%witch und die änderungen mit der witch


%wichtig waren von marvin die config files
%challanging, da es nicht wirklich etwas zum Nachlesen gab und das für jedes Spiel anders ist

%noch viel zu tun


\subsection{Carddescriptions}\label{subsec:placementMatters}

\subsection{Tutoriel}\label{subsec:placementMatters}


%Broad Bullet design für draft

%

%gegner gamedesign :((((

\subsection{Wie eine entscheidung das Spiel verändert am Beispiel des Kartennachziehens}\label{subsec:placementMatters}

%angewantes gamedeign: die pinacle of forty-five: Bewtisched Bullet

% resets author
\renewcommand{\kapitelautor}{}
