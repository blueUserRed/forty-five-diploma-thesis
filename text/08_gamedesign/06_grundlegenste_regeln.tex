
\section{.Forty-Fives grundlegendste Regeln}\label{sec:grundlegenste-regeln}

\renewcommand{\kapitelautor}{Autor: Irgendwer} % todo: replace

%
.Forty-Five ist ein digitales Kartenspiel des Subgenres des Rogue-like Deckbuilders. %TODO: Quelle für Erklärung
Der Spieler bewegt sich über eine Map, die prozentual generiert ist, sammelt Karten und benutzt diese Karten,
um Gegner zu bekämpfen. Sollte der Spieler einen Kampf verlieren, stirbt er, seine gesammelten Karten
gehen verloren und er startet wieder am Anfang. Da es sich jedoch um ein "Rogue-Lite" handelt und nicht um ein "Rogue-Like",
gibt es eine Art von speicherbarem Fortschritt, den der Spieler in sein nächstes Leben mitnehmen kann.
Ein Durchlauf bzw. ein Leben des Spielers wird als "Run" bezeichnet. Ein Run endet mit dem Tod des Spielers.
Während der Entwicklung wurden jene Elemente, die in das nächste Leben übergehen, als "Rogue-Lite-Elemente" bezeichnet.


%zuerst erklären was roads und so sind????
\subsection{.Forty-Fives Rogue-Lite-Elemente}\label{rogue_lite_elemente:}

Um das Spielerlebnis des Spielers nicht allzu frustrierend zu gestalten, wurden zwei verschiedene Mechaniken eingeführt.
Bei der ersten Mechanik handelt es sich um die Erhöhung der maximalen Lebenspunkte des Spielers bei Heilpunkten im Spiel. Der Spieler hat die Möglichkeit
sich zu entscheiden, ob er sich lieber nur für diesen Run heilt und dafür auch eine größere Menge an Lebenspunkten erhält, oder ob er
lieber in das Erhöhen seiner Lebenspunktkapazität investiert. Bei letzterem handelt es sich natürlich um eine kleinere Zahl als bei der anderen Wahl.
Dies dient dazu, dem Spieler die Entscheidung offen zu halten, entweder etwas Langwieriges aber Bleibendes zu investieren oder lieber noch einmal ordentlich
Lebenspunkte aufzutanken, was bei der zweiten Mechanik ins Spiel kommt.

Schafft der Spieler einen Abschnitt des Spiels, also eine sogenannte Road, werden seine bis zu dem Zeitpunkt gesammelten Karten
gespeichert und sind ab dann selbst nach einem Tod jederzeit zugänglich und spielbar. Verknüpft mit der ersten Mechanik
hat der Spieler die Möglichkeit, sich dazu zu entscheiden, seine Leben wieder aufzuheilen und das Ende der Road anzustreben.
Ist er der Meinung, es sowieso nicht mehr zu dem Abschnittsende zu schaffen, wählt er die Erhöhung der maximalen
Lebenspunkte, um so zumindest einen Vorteil in den nächsten Runs zu haben. Der Spieler muss also das Risiko und die
Belohnung abwägen und daran entscheiden, welche Wahl er trifft.


%

% resets author
\renewcommand{\kapitelautor}{}
