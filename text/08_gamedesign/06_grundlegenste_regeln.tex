
\section{.Forty-Fives grundlegendste Regeln}\label{sec:grundlegenste-regeln}

\renewcommand{\kapitelautor}{Autor: Irgendwer} % todo: replace

%
.Forty-Five ist ein digitales Kartenspiel des Subgenres des Rogue-like Deckbuilders. %TODO: Quelle für Erklärung
Der Spieler bewegt sich über eine Map, die prozentual generiert ist, sammelt Karten und benutzt diese Karten,
um Gegner zu bekämpfen. Sollte der Spieler einen Kampf verlieren, stirbt er, seine gesammelten Karten
gehen verloren und er startet wieder am Anfang. Da es sich jedoch um ein "Rogue-Lite" handelt und nicht um ein "Rogue-Like",
gibt es eine Art von speicherbarem Fortschritt, den der Spieler in sein nächstes Leben mitnehmen kann.
Ein Durchlauf bzw. ein Leben des Spielers wird als "Run" bezeichnet. Ein Run endet mit dem Tod des Spielers.
Während der Entwicklung wurden jene Elemente, die in das nächste Leben übergehen, als "Rogue-Lite-Elemente" bezeichnet.


%zuerst erklären was roads und so sind????
\subsection{.Forty-Fives Rogue-Lite-Elemente}\label{rogue_lite_elemente}

Um das Spielerlebnis des Spielers nicht allzu frustrierend zu gestalten, wurden zwei verschiedene Mechaniken eingeführt.
Bei der ersten Mechanik handelt es sich um die Erhöhung der maximalen Lebenspunkte des Spielers bei Heilpunkten im Spiel. Der Spieler hat die Möglichkeit
sich zu entscheiden, ob er sich lieber nur für diesen Run heilt und dafür auch eine größere Menge an Lebenspunkten erhält, oder ob er
lieber in das Erhöhen seiner Lebenspunktkapazität investiert. Bei letzterem handelt es sich natürlich um eine kleinere Zahl als bei der anderen Wahl.
Dies dient dazu, dem Spieler die Entscheidung offen zu halten, entweder etwas Langwieriges aber Bleibendes zu investieren oder lieber noch einmal ordentlich
Lebenspunkte aufzutanken, was bei der zweiten Mechanik ins Spiel kommt. %todo quelle füg max hp und hp generell

Schafft der Spieler einen Abschnitt des Spiels, also eine sogenannte Road, werden seine bis zu dem Zeitpunkt gesammelten Karten
gespeichert und sind ab dann selbst nach einem Tod jederzeit zugänglich und spielbar. Verknüpft mit der ersten Mechanik
hat der Spieler die Möglichkeit, sich dazu zu entscheiden, seine Leben wieder aufzuheilen und das Ende der Road anzustreben.
Ist er der Meinung, es sowieso nicht mehr zu dem Abschnittsende zu schaffen, wählt er die Erhöhung der maximalen
Lebenspunkte, um so zumindest einen Vorteil in den nächsten Runs zu haben. Der Spieler muss also das Risiko und die
Belohnung abwägen und daran entscheiden, welche Wahl er trifft.

\subsection{Das Sammeln der Karten}\label{sammeln_der_Karten}

Nach jedem überlebten Kampf bekommt der Spieler eine neue Karte. Er bekommt drei verschiedene Karten zur Auswahl gezeigt und darf sich eine davon aussuchen.
Außerdem gibt es auf der Map verteilt noch zusätzliche Events, die der Spieler aufsuchen kann, um seine Sammlung zu erweitern.
Das bringt nicht nur wieder eine Entscheidung des Spielers mit sich, die den Verlauf des Runs ändert, sondern bringt auch mehr Abwechslung,
da der Spieler nicht immer nur die Karten nehmen kann, die er gerne hätte. Es bringt auch eine Art Entdeckerlust mit sich, da der Spieler
auf diese Weise natürlich die Karten erst nach und nach sieht und nicht alle gleich auf einmal (siehe z.B. Magic Arena, wo alle Karten sofort in einer eigenen Liste zugänglich sind). %todo: Quelle
Diese Art des Erwerbs von Karten wird "Draft" genannt %TODO Quelle
und zwingt den Spieler mit begrenzter Wahl und Ressourcen das Beste daraus zu machen. Diese Designpolitik wird beim Design der Karten beachtet und wird zu einem späteren Zeitpunkt genauer erläutert %TODO macht man das so?
Viele Rogue-like Deckbuilder haben ein ähnliches System. Inspiriert wurde das Sammeln der Karten in ".Forty-Five" von Spielen wie Inscription und Slay the Spire, %Todo Quelle
hat jedoch einen großen Unterschied. Anders als in z.B. Slay the Spire können Karten ganz einfach aus dem Kartendeck entfernt werden, nachdem sie einmal ausgewählt wurden.
Karten können aus dem Deck in den Backpack verschoben werden und ermöglichen dadurch ein flexibleres Spielerlebnis. %todo Quelle. Es gibt auch einen Deckbuilder, der das so macht, keine Ahnung wie der heißt.


\subsection{Der Backpag und das Deck}\label{backpack_and_deck}
\begin{infoBox}
\end{infoBox}
Note: Zu diesem Zeitpunkt geht es nicht um die Designprinzipien hinter dem Deck und den Karten,
sondern nur um die Erklärung der grundlegenden Mechaniken. Das Gamedesign wird zu einem späteren Zeitpunkt beschrieben.

Der Backbag und das Deck dienen beide als Speicherort für Karten, mit dem Unterschied, dass die Karten im Deck aktiv im
Kampf eingesetzt werden und die Karten im Backbag mehr als Reserve gelten.
Karten können frei zwischen den beiden verschoben werden, solange die festgelegte Mindestanzahl an Karten im Deck eingehalten wird.
Da Forty-Five viele verschiedene Strategien bietet, gibt es mehrere Decks, zwischen denen man einfach wechseln kann.
Das hat zur Folge, dass der Spieler nicht immer ein Deck zerlegen muss, um eine andere Strategie auszutesten.
Außerdem können Decks umbenannt werden, um besser wiedergefunden und erkannt zu werden.
Karten können im Backbag nach Kriterien wie Kosten oder Namen auf- und absteigend sortiert werden.
Sammelt der Spieler eine neue Karte, kann er sich entscheiden, ob er die Karte gleich seinem Deck oder doch eher seinem
Backbag hinzufügen möchte.
Der Backbag und das Deck ermöglichen es dem Spieler, Karten, die er gesammelt hat, aus dem Deck zu nehmen und damit
nicht zu verwenden, sowie Decks zu bauen, die zu einer Strategie passen.
Startet der Spieler nun einen Kampf, wird das zuletzt ausgewählte deck verwendet.


\subsection{Der Kampf}\label{backpack_and_deck}
%grundsätzliche sachen wie reserves, gegner und spieler, gewonnen wenn gegner tod usw karten am anfang ziehe zwei karten am anfang vom turn

Während der Spieler über die Map reist, sind Kämpfe unausweichlich.
Gekämpft wird mit den gesammelten Karten, mit dem Ziel, den oder die Gegner zu besiegen und dabei so wenig Lebenspunkte
wie möglich zu verlieren bzw. nicht zu sterben. Gewonnen hat der Spieler, wenn alle Gegner besiegt wurden.
Ein Kampf ist aufgeteilt in Züge. Immer abwechselnd ist entweder der Spieler oder der Gegner dran. Am Anfang des Kampfes
werden eine vordefinierte Anzahl an Karten vom Deck des Spielers gezogen. %todo die genaue Anzahl fehlt noch (5 oder 6?)
Anschließend hat der Spieler die Möglichkeit, nach Belieben seinen Zug auszuführen. Ein Spielerzug wird mit dem "End Turn"
Button beendet, und das Betätigen des Knopfes startet den automatischen Ablauf des Gegnerzuges. Der Gegner führt eine mehr oder weniger zufällige Aktion aus, und danach ist wieder der Zug des Spielers.
Am Start des Zuges des Spielers werden zwei Karten vom Deck gezogen. Pro Zug des Spielers stehen dem Spieler 4 "Reserves"
zur Verfügung. Diese werden jeweils auch am Anfang des Zuges wieder auf die maximale Anzahl aufgefüllt.
Reserves können dazu verwendet werden, Karten zu bezahlen und damit auch zu spielen.


\subsection{Karten}\label{Karten}
Cards in \FF are Bullets. Each Bullet has a cost that must be paid in order to play the card.
The reserves needed for the Bullet are paid automatically IF the player is able to afford the Bullet, the moment the Bullet is played.
Das Management von diesen Reserves und die Reihenfolge, in der man die Karten spielt, ist wichtig, um besser in Forty-Five zu werden. %TODO Mehr dazu ist im Gamedesign.
Jede Bullet hat außerdem einen Damage-Wert, welcher angibt, wie viele Lebenspunkte dem Gegner durch die Karte abgezogen werden,
falls die Karte auf den Gegner geschossen wird. Fast alle Karten haben außerdem einen einzigartigen Effekt,
welcher angesehen werden kann, indem der Spieler mit der Maus über die gewünschte Bullet hovert.
In dem Popup, welches nach dem Hovern sichtbar ist, befinden sich die Infos zu dem Effekt der Karte,
dazu gehört fast immer ein Trigger und der Effekt selbst.
Der Trigger gibt an, wann sich der Effekt aktiviert. Es gibt verschiedene Trigger, wie zum Beispiel das Aktivieren des Effektes beim Spielen der Bullet. %TODO siehe mehr später bei trigger
Zusätzlich zu dem Effekt und dem Trigger steht bei vielen Karten außerdem noch ein Flavortext dabei.
Ein Flavortext ist ein Spruch, der der Bullet ein wenig Kontext hinzufügt.
Dies kann passieren durch einen dummen Spruch, einen Witz, eine Anspielung oder einen story- bzw. worldbuilding-relevanten Text.
Falls nötig werden relevante Infos zu dem Effekt in einer extra Box rechts oder links angezeigt.
Bullets werden, wenn sie gespielt werden in den Revolver geladen.

\subsection{Der Revolver}\label{der_revolver}
Der Revolver ist das Spielfeld von \FF, in weleches die Bullets platziert bzw "geladen" werden.Mithilfe von Drag and drop,
kann der PSieler die Karte aus seiner hand in den Revolver legen. Dies ist das bereits erwähnte spielen einer Karte, zu dem auch das Bezahlen dazugehört.
Der genaue ablauf des Spielen einer Karte ist wie Folgt:
Drag and drop in den gewünschten revolver platz > bezahlen der Reserves, abbruch falls nicht genug Reserves vorhanden sind >
platzieren der Bullet > womöglich aktivieren von Triggern, falls die Bullet einen Effekt hat der einen On-Placedown Trigger hat.
Der  Revolver besteht aus 5 kammern bzw felder in die Bullets geladen werden können. nummeriert sind sie von 1 -5. %TODO bild und noch umdrehen von 5 zu 1
Der revolver kann von dem Spieler geschossen werden. Wird geschossen wird die Bullet in der obersten chamber auf den Gegner geschossen,
verlässt sie den Rvolvr und der Gegner verliert Lebenspunkte in der Höhe der Dmg Value der gecshossenen Bullet.
Die Bullet wird unter das deck gelegt und der revolver dreht sich einmal nach rechts.
Die Revolverrotation ist ein wichtiger Teil von forty-five, da man das Placement der Bullets
im revolver sich genau überlegen muss um das meiste aus seinen Bullets raussuchen. %todo mehr in game design.
Es gibt außerdem einen Trigger, wlecher aktiviert wird wenn die Karte geschossen wird namens On-Shot.
Zusätzlich gibt es Karten mit Effekten, welche die Rvolverrotation verändern wie z.B Bewitched Bullet, welche den Revolver nach links statt nach rechts rotiert

Die Komplexität von \FF kommt von dem meistern des Revolvers. Das strategische platuieren von Bullets,
die reihenfolge der Bullets und das verstehen wie eine Revolver rotation sich auf Bullets, den Spieler oder den gegner auswirken.
Die gerade erwähnte Bewitched Bullet kann zum Biepiel dazu verwendet werden Karten , weleche davon profetieren lange im revolver zu bleiben,
wieder weiter von dem Schießen wegzuschibene. Bull et zum Beipeil fügt dem gegner Schaden zu wenn sich Bull et Rotiert im Revolver. %TODO BILDER!!!
Bewitched Bullet kann dazu verwendet werden zwei zusätzliche rotationen rauszuholen in dem man den revolver einmal nach links dreht bevor Bull et den Revolver verlässt.
Bewitched Bullet kann zumm beipeil auch zum einschieben von Bullets verwendet werden und so wieter udn so fort.
Die kombinationsmöglichkeiten sind endlos und fast jeder effekt einer Bullet kann mit dem einer anderen Bullet verknmüpft werden.



%slots rotationen usw karten weg wenn shot und karten unters deck gelegt wenn shot usw


\subsection{Zonen}\label{backpack_and_deck}
%probbaly unnötig

\subsection{Der Gegner}\label{der_gegner}
Der Gegner, auf welche die Bullets des Spielers geschossen werden, besteht aus mehreren Komponenten.
Genuase wie der Spieler, verfügt der Gegner über einen HP wert. Hat der Wert null erreicht, stirbt der Gegner.
Nachdem der Spieler End turn druückt, beendeter seinen zug und der Gegner ist dran.Beierite während dem Zug des Spielers zeigt der Gegner die AKTIOn an, welche er in seinem Zug ausführen wird.
Gegner aktionen werden "zufällig ausgewählt aus dem pool von Aktio  nen des Gegners.
Jedcoh wurden die aktionen angepasst um nicht nur fairness zu garantieren,
sondern auch eine balance zwischen zu leicht und zu schwer zu finden.
Mehr zu dem Balancen der Gegner kann in 07 Gamedesign nachgelsene werden.
Sobald der Gegner dran ist, wird die Aktion automatisch ausgeführt. Je nach Gegnertyp gibt es andere Aktionen welche ein Gegner ausführen kann. Universell ist jedoch die Aktion des Schaden zufügens.
Es gibt viele verschieden Arten von aggner aktionen, die dafür entwickelt werden, mehr Abwechslung beim Bekämpfen der Gegner zu besiegen.
Außerdem sollte noch erwähnt werden, dass es auch passieren aknn, dass der Spieler gegen mehr als einen Gegner kämpft. Ist das der fall, sucht sch der Spieler das Ziel der REvolverschüsse durch klickn auf den gwolten gegner aus.
Dargetsellt durch ein symbol über dem Kopf des Gegners, sind die Aktionen weniger eine Reaktion auf den Spieler, sondern eher wird erwartet, dass der Spieler die Aktion des Gegner in das palnnn seines Zuges miiteinbezieht.
Bei der eben ganananten Schadenaktion, kann der Spieler darauf reagieren und schaden verhindern indem er die Karte parried.

\subsection{parrying}\label{backpack_and_deck}
Das paarryien des Schadens der von dem Gegner verursachtet wird wird in  \FF durch Bullet gemacht.
Die Idee dahinter ist, dass die eigenen Bullet verwendet wird um den Angriff des Gegners abzuhalten, indem man seinen Bullet gegen die vom gegner schießt.
Der Spieler bekommt die Wahl ob er parrien möchte oder nicht, was durch ein eigenen popup während dem gegner zug geregelt wird.
entscheidet sich der Spieler dazu zu parrien, wir ddafür die Bullet im vordersten Slot verwnedet.
Dann wird der dmg wert des Gegners mit dem dr Bullet substituiert. Ist der dmg wert der Bullet größer als der des Gegners,
Geht kein Schaden durch. Ist es anders rum, bekommt der Spieler so viel Schaden wie die Differenz ausmacht.
Es ist wichtig zu erwähnen, das nicht geparried werden muss, soollte der Spieler die Bullet lieber behaltn wollen,
bzw wenn er der  Meinung ist, den Schaden einfach einzustecken zu können. Durch einen Parry geht die Bullet verloren,
es aktivieren keine effekte und sie wird danach wieder unters deck gelegt.

Als zusatz gibt es Bullets, die einen devensiven effejt haben, mit denenn sich der Spieler zusätzlich schützen kann.

\subsection{Spezifische Regeln}\label{spezifische_regeln}
%overkill dmg
% leer schiesen.

\subsection{Encounter Modifier}\label{backpack_and_deck}
%

% resets author
\renewcommand{\kapitelautor}{}
