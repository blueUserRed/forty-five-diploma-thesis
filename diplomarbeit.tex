
\documentclass[
    headings=optiontotocandhead,% Erweiterung für das optionale Argument der
                                % Gliederungsbefehle aktiviert.
    twoside,
    numbers=noenddot,% Keine Punkte am Ende der Gliederungsnummern und davon
                     % abgeleiteten Nummern
    toc=flat, %Flache TOC --- kann man anpassen (auskommentieren)
    12pt, % Schriftgröße
    titlepage, % es wird eine Titelseite verwendet
    parskip=full, % Abstand zwischen Absätzen (ganze Zeile)
    listof=totoc, % Verzeichnisse im Inhaltsverzeichnis aufführen
    listof=flat, % mehr Abstand für grosse Zahlen
    numbers=noenddot, % kein Punkt am Ende bei Nummern
    %%enlargefirstpage,% Gibt es bei scrartcl nicht!!!!
    bibliography=totoc, % Literaturverzeichnis im Inhaltsverzeichnis aufführen
    %index=totoc, % Index im Inhaltsverzeichnis aufführen
    %captions=tableheading, % Beschriftung von Tabellen für Ausgabe oberhalb
                           % der Tabelle formatieren
    %draft % Status des Dokuments (final/draft) draft hinzufügen zum anziegen
    %%der zeilen ende
    a4paper,DIV=14,
    BCOR=15mm,
% captions=tablesignature,
]{scrbook}


\setcounter{secnumdepth}{3}

%\usepackage[T1]{fontenc}
\usepackage[utf8]{inputenc}
\usepackage[english, ngerman]{babel, varioref} % Deutsch muss letztes sein
\usepackage{lastpage}
\usepackage{listings}
\usepackage{blindtext}
\usepackage[inline]{enumitem} %% Aufzählungen nicht so weit einrücken

% Listen etwas wenige einrücken, erfordert enumitem
\setitemize{leftmargin=*}

\usepackage{lmodern}
\usepackage{xspace}
\usepackage{graphicx}
\graphicspath{ {./images/} }
\usepackage{float}

\usepackage[hyphens]{url}

\usepackage{makeidx}
\makeindex

\usepackage{natbib}

\PassOptionsToPackage{normalem}{ulem}

\usepackage{ulem}
\usepackage{needspace}

\setlength\partopsep{0.5ex} % schoenere Listen

\usepackage[bottom]{footmisc} % fussnote ganz unten

\usepackage[]{microtype}
\UseMicrotypeSet[protrusion]{basicmath} % disable protrusion for tt fonts

%\usepackage{multirow}   % Allows table elements to span several rows.
%\usepackage{booktabs}   % Improves the typesettings of tables.
%\usepackage{subcaption} % Allows the use of subfigures and enables their referencing.
\usepackage[ruled,linesnumbered,algochapter]{algorithm2e} % Enables the writing of pseudo code.
\usepackage[usenames,dvipsnames,table]{xcolor} % Allows the definition and use of colors. This package has to be included before tikz.
\usepackage{nag}       % Issues warnings when best practices in writing LaTeX documents are violated.
%\usepackage{todonotes} % Provides tooltip-like todo notes.
\usepackage{color}
%\usepackage[binary-units]{siunitx}

% Definiert einen Bundsteg von 1.5cm
% NUR BEI BEIDSEITIGEN DRUCK!!
\usepackage{geometry}
\geometry{
    left = 2cm,
    right = 2.5cm,
    bindingoffset = 1.5cm,
}

%  Kopf und Fußzeilen -- links und rechts verschieden
\newcommand{\kopfseitenummer}{{\bfseries \thepage}}
\newcommand{\kopfkapl}{{\bfseries\leftmark}}
\newcommand{\kopfkapr}{{\bfseries\rightmark}}
\newcommand{\kopfbild}{\voffset7mm\includegraphics[width=25mm]{HTL3RLogoRGB}}
\newcommand{\kopfHTL}{Höhere Technische Bundeslehranstalt Wien 3, \\Rennweg 	Abteilung für Informationstechnologie}

\usepackage{fontspec}
\usepackage{scalefnt}

% setzt Schriftart für Fließtext
% muss installiert sein
\setmainfont{Aptos}

% setzt Schriftart für Code-Blöcke
% muss installiert sein
\newfontfamily\codefont{JetBrainsMono-Regular.ttf}[NFSSFamily=JetBrainsMonoFamily]



\usepackage[automark,headsepline,footsepline,plainfootsepline]{scrlayer-scrpage}
\setkomafont{pageheadfoot}{\normalcolor\footnotesize\scshape}
\setkomafont{pagenumber}{\normalfont\normalsize}
\clearpairofpagestyles
\ihead{\voffset7mm\includegraphics[width=35mm]{logo_red}}
%\ihead{\headmark}
\ohead{\kopfbild}
\ifoot{\kapitelautor}
\ofoot{\pagemark}
\ModifyLayer[addvoffset=-.6ex]{scrheadings.foot.above.line}% Linie verschieben
\ModifyLayer[addvoffset=-.6ex]{plain.scrheadings.foot.above.line}% Linie verschieben
\setlength{\headheight}{32pt}

% alle Seiten mit Kopfzeile
\renewcommand{\chapterpagestyle}{scrheadings}


\usepackage{minted}
\usemintedstyle{forty_five_style}
% Konfig für Code-Blöcke
\setminted{
    frame=lines,
    framesep=2mm,
    breaklines=true,
    fontfamily=JetBrainsMonoFamily,
    fontsize=\scriptsize
}

%\usepackage{awesomebox}
%\setlength{\aweboxleftmargin}{1pt}

%\usepackage{scrhack}

%% glossar
% kann man löschen falls kein Glossar gebraucht
%\usepackage[acronym, toc]{glossaries}
%\makeglossaries
%\input{text/glossar.tex}

\usepackage{tcolorbox}
\tcbuselibrary{xparse,skins,breakable}
\definecolor{htl3red}{RGB}{255,51,0}
\newtcolorbox{TitlePageBox}{%
    breakable,
    blanker,
    left=1em,
    borderline west={0.15cm}{3pt}{htl3red},
}

\usepackage[unicode=true,
    bookmarks=true,bookmarksnumbered=false,bookmarksopen=false,
    breaklinks=true,pdfborder={0 0 0},backref=false,colorlinks=false]
{hyperref}
\hypersetup{
    pdftitle={.Forty-Five},
    pdfauthor={Markus Böheim, Nils Hubmann, Philip Jankovic, Marvin Kurka, Felix Zwickelstorfer},
    pdfsubject={Diplomarbeit},
    pdfkeywords={Forty-Five, Card-Game, card game, wild west, Wild-West, open source, western game, bullets, revolver}
}
\urlstyle{same} % don't use monospace font for urls

% Auch Fußnoten bündig ausrichten
\deffootnote[]{1em}{1em}{\textsuperscript{\thefootnotemark\ }}
\sloppy % weniger Meldungen
\voffset7mm % etwas nach unten

%% schöner: 10000 -- gar keine, 1000 als Mittelweg
\clubpenalty = 10000 % Schusterjungen verhindern
\widowpenalty = 10000 % Hurenkinder verhindern
\displaywidowpenalty = 10000


%%%%%%%%%%%%%%%%%%%%%%%%%%%%%%%%%%%%%%%%%%%%%%%%%%%%%%%%%%%%%%%%%%%%%%%%%%%%%%%%%%
\begin{document}

\shorthandoff{"}
%% mit kapitelautor kann man den Autor festlegen oder auf leer setzen - steht dann in der Fußzeile.
%% bitte immer (gleich) nach der Überschrift setzen, nicht vorher -- sonst steht es bei Kapiteln eventuell eine Seite zu früh
\newcommand{\kapitelautor}{}

\newcommand{\bold}[1]{\textbf{#1}}
\newcommand{\italic}[1]{\emph{#1}}
\newcommand{\code}[1]{\texttt{#1}}

% einfaches "siehe ..." - das Ziel muss man markieren mit \label{name} -- macht pandoc automatisch
% einfache Variante
%\newcommand{\kap}[1]{Kapitel~\ref{#1}, Seite~\pageref{#1}}
%\newcommand{\siehe}[1]{siehe \kap{#1}}
%\newcommand{\abb}[1]{Abbildung~\ref{#1}, Seite~\pageref{#1}}
% bessere Variante - braucht varioref
\newcommand{\kap}[1]{Kapitel~\vref{#1}}
\newcommand{\siehe}[1]{siehe \kap{#1}}
\newcommand{\abb}[1]{Abbildung~\vref{#1}}


%% http://ieg.ifs.tuwien.ac.at/~aigner/download/tuwien.sty
%Div. Abkürzungen (in Anlehnung an Jochen Köpper, jkthesis):
%\RequirePackage{xspace}
\newcommand{\bzw}{bzw.\@\xspace}
\newcommand{\bzgl}{bzgl.\@\xspace}
\newcommand{\ca}{ca.\@\xspace}
\newcommand{\dah}{d.\thinspace{}h.\@\xspace}
\newcommand{\Dah}{D.\thinspace{}h.\@\xspace}
\newcommand{\ds}{d.\thinspace{}s.\@\xspace}
\newcommand{\evtl}{evtl.\@\xspace}
\newcommand{\ua}{u.\thinspace{}a.\@\xspace}
\newcommand{\Ua}{U.\thinspace{}a.\@\xspace}
\newcommand{\usw}{usw.\@\xspace}
\newcommand{\va}{v.\thinspace{}a.\@\xspace}
\newcommand{\vgl}{vgl.\@\xspace}
\newcommand{\zB}{z.\thinspace{}B.\@\xspace}
\newcommand{\ZB}{Zum Beispiel\xspace}
\newcommand{\FF}{.Forty-Five\@\xspace}
\newcommand{\ff}{.Forty-Five\@\xspace}

%% https://github.com/Digital-Media/HagenbergThesis
\newcommand{\latex}{La\-TeX\xspace} % kein schnoerkeliges LaTeX mehr
\newcommand{\tex}{TeX\xspace}       % kein schnoerkeliges TeX mehr
\newcommand{\bs}{\textbackslash}    % Backslash character
\newcommand{\obnh}{\hskip 0pt } %optional break without hyphen: e.g. PlugIn{\obnh}Filter

\newcommand{\sa}{s.\ auch\@\xspace}
\newcommand{\so}{s.\ oben\xspace}
\newcommand{\su}{s.\ unten\@\xspace}

\newcommand{\uae}{u.\thinspace{}\"A.\@\xspace}
\newcommand{\uva}{u.\thinspace{}v.\thinspace{}a.\@\xspace}
\newcommand{\uvm}{u.\thinspace{}v.\thinspace{}m.\@\xspace}

\newcommand{\inlineCode}[1]{\mintinline{text}{#1}}
\newcommand{\inlineKotlin}[1]{\mintinline{kotlin}{#1}}
\newcommand{\inlineOnj}[1]{\mintinline{onj}{#1}}
\newcommand{\inlineGlsl}[1]{\mintinline{glsl}{#1}}
\newcommand{\inlineJava}[1]{\mintinline{java}{#1}}
% \citeauthor \citeyear
%\newcommand{\zit}[1]{ (vgl. \cite{#1})}
\newcommand{\zit}[1]{ (vgl. \citeauthor{#1} \citeyear{#1} [online])}
\newcommand{\zitbuch}[1]{ (vgl. \citeauthor{#1} \citeyear{#1})}
%\newcommand{\zitt}[2]{(\cite{#1, #2})}
%\newcommand{\zittt}[3]{(\cite{#1, #2, #3})}
%\newcommand{\zitttt}[4]{(\cite{#1, #2, #3, #4})}

\newcommand{\quoted}[1]{\frqq#1\flqq}
\newenvironment{coolQuote}{\begin{quote}\itshape\frqq}{\flqq\end{quote}}

\newcommand{\zid}[1]{(\citeauthor{#1} \citeyear{#1} [online])}
\newcommand{\zidbuch}[1]{(\citeauthor{#1} \citeyear{#1})}

\newcommand{\codeblockCaption}[1]{#1} % might be changed later

\newenvironment{liste}{\begin{itemize}\setlength{\itemsep}{1pt}\setlength{\itemsep}{0pt}\setlength{\parsep}{0pt}}{\end{itemize}}

\newenvironment{infoBox}{\begin{awesomeblock}[blue]{3pt}{}{magenta}}{\end{awesomeblock}} % todo: make look good

\newenvironment{codeBlock}[2]
{\VerbatimEnvironment\begin{figure}[H]\def\myenvargumentII{#2}\centering\begin{minted}{#1}}
{\end{minted}\caption{\myenvargumentII}\end{figure}}



\frontmatter % Switches to roman numbering
\title{Diplomarbeit}

\begin{titlepage}
\begin{minipage}[b]{1\columnwidth}
\parbox[b]{99mm}{
\begin{TitlePageBox}
\footnotesize% klein
\textsf{% und ohne Rifen
\textbf{\textsc{Höhere Technische Bundeslehranstalt} Wien 3, Rennweg}\\
\\
Höhere Abteilung für Mechatronik\\
Höhere Abteilung für Informationstechnologie\\
Fachschule für Informationstechnik}
\end{TitlePageBox}
}\hfill\parbox[b]{50mm}{\includegraphics[width=51mm]{HTL3RLogoRGB}}
\mbox{}
\end{minipage}

\vspace{1cm}


\begin{center}

\textbf{\LARGE{}Diplomarbeit}{\large{}}\\ % todo: figure out what is going on with this line
{\large{}\vspace{15mm}
 }
\textbf{\large{}\FF}\\

 \vfill

 ausgeführt an der\\
 Höheren Abteilung für Informationstechnologie/Medientechnik\\
 der Höheren Technischen Lehranstalt Wien 3 Rennweg\\

 \vfill
 im Schuljahr 2023/2024\\

\vspace{1cm}

{
\renewcommand{\arraystretch}{1.8}
\begin{tabular}{l c r}
durch  & \hfill & unter Anleitung von \\
\textbf{\large{}Böheim Markus} && Nussbaumer Vincent \\
\textbf{\large{}Hubmann Nils} && Sturm Gerhard \\
\textbf{\large{}Jankovic Philip} && Nussbaumer Vincent \\
\textbf{\large{}Kurka Marvin} && Weiss Florian \\
\textbf{\large{}Zwickelstorfer Felix} && Weiss Florian \\
\end{tabular}
}

\vfill

Wien, \today
\par\end{center}

\end{titlepage}

\chapter*{Kurzfassung}

%
Da das Interesse an Videospielen stetig steigt, wachst der Videospielmarkt nach wie vor. Es ist nichtmehr nur eine Unterhaltungsmöglichkeit, sondern eine eigene Welt.
Es gibt eine Vielzahl an Menschen, welche täglich Videospiele spielen und darin eine Flucht aus der Realität sehen.

Ziel des Diplomarbeitsprojekts ist es, ein rogulite Cardgame zu entwickeln, welches über Steam gedownloadet werden kann und auf Windows Systemen spielbar ist.
Der Spieler durchlebt mehrere Runs und bekämpft Gegner mit der Hilfe von Spielkarten auf seinem Weg.
Die Map, die erkundet wird, ist geprägt von: Gegnern die zu bekämpfen sind, Shops in denen neue Bullets erworben werden, und Heilevents die den Spieler heilen.
Ziel ist es alle Kämpfe zu bestreiten und sich ein Deck zu bauen, dass allen Herausforderungen standhält.

Der Hintergedanke besteht darin Spieler für Kartenspiele zu begeistern und zu Menschen zu animieren ihre Ideen zu verfolgen. Es soll für Unterhaltung sorgen und die Genres Wilder Westen und Kartenspiele kombinieren.
%


\blindtext[1]


\chapter*{Abstract}
\selectlanguage{english}

Because the interest in video games is constantly rising, the video game market is growing as well.
Video games come in all kinds of different genres and are mainly used for entertainment.
The goal of the diploma project was to develop a virtual card game, which was published via Steam and can be played on
Windows Systems.
The player experiences multiple runs and fights enemies with the help of the playing cards, which they collected or bought
along the way.
The map, which can be explored by the player, is characterized by: Fights against various enemies, Shops that offer unique
cards or Events where the player can heal themselves.
The goal of the game is to win fight after fight and create a deck that withstands all enemies.
The primary focus of the game is entertainment, but also to further creative and logical thinking.

\selectlanguage{ngerman}

\chapter*{Ehrenwörtliche Erklärung}

Hiermit versichere ich, dass ich die vorliegende Arbeit selbstständig verfasst und keine anderen Hilfsmittel als die angegebenen benützt habe. Die Stellen, die anderen Werken (gilt ebenso für Werke aus elektronischen Datenbanken oder aus dem Internet) wörtlich oder sinngemäß entnommen sind, habe ich unter Angabe der Quelle und Einhaltung der Regeln wissenschaftlichen Zitierens kenntlich gemacht. Diese Versicherung umfasst auch in der Arbeit verwendete bildliche Darstellungen, Tabellen, Skizzen und Zeichnungen.

\begin{flushleft}
\bigskip{}
Wien, am \today \\
\newcommand{\namesigdate}[2][8cm]{
\vspace{2cm}~\newline
\parbox{#1}{\hrule\centering #2\Large\strut}
\hfill
}
%\namesigdate{Mitarbeiter:in Eins}
%\namesigdate{Mitarbeiter:in Zwei}
%\namesigdate{Mitarbeiter:in Drei}
<eigenhändige Unterschriften aller Teammitglieder>
\par\end{flushleft}



\chapter*{Präambel}
Die Inhalte dieser Diplomarbeit entsprechen § 7(1) und § 24 der Verordnung des Bundesministers für Bildung über die abschließenden Prüfungen in den berufsbildenden mittleren und höheren Schulen (Prüfungsordnung BMHS) vom 30.5.2012 (BGBl. Nr. II 177/2012) in der derzeit geltenden Fassung.

\textbf{Liste der betreuenden Lehrer}
Prof. Gerhard Sturm \\ % Hauptbetreuer>
Prof. Florian Weiss \\
Prof. Vincent Nussbaumer \\

%%%%%%%%%%%%%%%%%%%%%%%%%%%%%%%%%%%%%%%%%%%%%%%%%%%%%%%%%%%%%%%%%%%%%%%%%%%%%%%%%%%%%%%%
\cleardoublepage{}
\tableofcontents{}
\cleardoublepage{}
%\listoftables %todo maybe add again if needed
%\todo{kann entfallen falls (fast) leer}
%\cleardoublepage{}
\listoffigures


%hier geht es los mit dem Text - auf einer rechten Seite
\cleardoublepage{}
%\pagenumbering{arabic}
\mainmatter

\chapter{Ziele}
% \renewcommand{\kapitelautor}{}  % bleibt eventuell leer (gemeinsame Arbeit)

Das erste Kapitel stellt die Ziele der DA (inkl. individuelle Ziele
aller Mitarbeiter) dar.\todo{viel Text schreiben}

Mögliche Gliederung (nach~\cite{leitfaden})

\begin{itemize}
\item  Einleitung
\item   Zielsetzung und Aufgabenstellung des Gesamtprojekts
\item   individuelle Zielsetzung und Aufgabenstellung mit Terminplan der einzelnen Teammitglieder
\item   Grundlagen und Methoden (Ist-Situation, Lösungsansätze, konkrete Vorgehensweise)
\item   Bearbeitung der Aufgabenstellung (technische Beschreibungen, Berechnungen)
\item   Ergebnisse (Ergebnisdarstellung, kritische Gegenüberstellung mit der Zielsetzung
 und der gewählten Vorgehensweise)
\end{itemize}

Mögliche Variante: Ziele laut Antrag, mit Querverweisen zu den einzelnen Kapiteln.

%
\chapter{Projektmanagement}\label{ch:projektmanagement}


\section{Überblick}\label{sec:ueberblick}

\renewcommand{\kapitelautor}{Autor: Nils} % todo: replace

%
Beim Entwickeln eines Videospiels ist das Projektmanagement eine ausschlaggebende Rolle. Nicht nur sorgt man für eine zeitliche Planung, sondern man hat auch intern eine klare Struktur.
Dies kann einerseits die Herangehensweise, aber auch die Kommunikation im Projektteam immens beeinflussen. Es gibt verschiedenste Projektmanagementmethoden, die man verwenden kann, um sein Projekt zum Erfolg zu führen.
In unserem Fall haben wir uns zweier bedient.

Scrum ist ein Agile-Framework, das es Projekte und Teams bei einheitliche und geplanter Arbeit unterstützt und ihnen hilft, Aufgaben nach dringlichkeit zu priorisieren und zu erledigen.
Die Methode gibt Vorgaben mit Rollenverteilungen, Abläufen und Richtlinien an, um die Arbeit und Vorgangsweise von Projektteams durchgehend zu Verbessern und auf fortlaufende Änderung vorzubereiten.
Standardgemäß besteht Scrum aus zeitlich definierten Sprints, welche das Ziel verfolgen, nach jedem Sprint kleine Teilerfolge zu erzielen.

"Scrum is an empirical process, where decisions are based on observation, experience and experimentation. 
Scrum has three pillars: transparency, inspection and adaptation. This supports the concept of working iteratively. 
Think of Empiricism as working through small experiments, learning from that work and adapting both what you are doing and how you are doing it as needed."\cite{Scrum}


Das Framework arbeitet mit einem Board das eine Übersicht über alle Aufgaben verschafft und in drei Kategorien gegliedert ist.
Mit dem Status ToDo, in Progress, und Done wird der momentane Arbeitsstand der Tasks signalisiert. Das Board ist eine Art des visuellen Projektmanagements die einen einfacheren Überblick verschaffen soll.\cite{AsanaKanban}

 Die Entscheidung ist auf diese beiden Methoden, da es ermöglicht das Projekt in zwei Phasen zu unterteilen.
Zuerst in eine große Entwicklungsphase, wo die Großteile der Spielsysteme entstanden sind, sei es das Mapsystem, das Backpacksystem oder der Shop, aber auch Demo Sounds und die Steampage.
Als Zweites eine Testphase, wo ein größerer Fokus auf ständiges Testen und anpassen des Spiels gelegt wird.
Die Testphase lief auch primär in Tasks ab, da es sich oft um schnelle Änderungen und Anpassungen im Spiel bezog, da laufend neue Testdemos released werden.
Zusätzlich hat eine enge interne Kommunikation es ermöglicht schnellstmöglich auf jegliche Probleme beim Testen zu reagieren und ständig neue Anpassungen zu treffen.
%

% resets author
\renewcommand{\kapitelautor}{}


\section{Scrum}\label{sec:scrum}

\renewcommand{\kapitelautor}{Autor: Irgendwer} % todo: replace

\subsection{Rollen}\label{subsec:rollen}

%
\subsubsection{Scrum Master}\label{subsubsec:Scrum-Master}
%
Der Leiter eines Projektteams ist der Scrum Master. Er hat sich damit zu befassen, das das Projekt kontinuirliche Fortschritte erzielt und das durchgehend Verbesserung und Optimierungen am Entwicklungprozess geschehen.
Eine Weiter Tätigkeit mit der sich der Scrum Master befasst ist die Einhaltung des Projektablaufs zum Beispiel Sprints und Meetings.\cite{AsanaScrumMaster}
Im Rahmen des Diplomarbeitsprojekts hat der Scrum Master auch die Aufgaben eines Projektleiters übernommen, wie die Organisation und durchführung von Meetings und ist Ansprechperson für projektspezifische Angelegenheiten gewesen.
%
\subsubsection{Produkt Owner}\label{subsubsec:Product-Owner}
%
"Beim Product Owner handelt es sich um eine standardmäßige Rolle in Scrum-Teams, deren Fokus darauf gerichtet ist, das bestmögliche Produkt für Endnutzer abzuliefern.
Um dies zu erreichen, entwickelt der Product Owner eine Vision davon, wie das Produkt funktionieren soll, definiert spezifische Produktfunktionen und unterteilt diese in Product-Backlog-Elemente, an denen das Scrum-Team arbeiten kann. 
Der Product Owner trägt die Verantwortung für das fertiggestellte Produkt."\cite{AsanaProductOwner}

Im Rahmen des Diplomarbeitsprojekts hat der Product Owner auch die Aufgaben eines stellvertretenden Projektleiters übernommen, wie die Unterstützung und Hilfe bei Organisation von Meetings und interne Planung und Kommunikation.
%
\subsubsection{Development Team}\label{subsubsec:Development-Team}
%
Das Development Team ist mit der Produkt/Software entwicklung beschäftigt. Es arbeitet nach Vorgaben des Product Owners, unter der führung des Scrum Masters um sein Ziel zu erreichen.
Im Rahmen des Diplomarbeitsprojekts war das Development Team aus einem Leiter der sich auch mit Sound und Vertrieb beschäftigt hat, zwei Designern und 2 Programmierern zusammengesetzt.
%
\subsection{Product Backlog}\label{subsec:product-backlog}
%

Der Product Backlog wurde während der Sprints verwendet und vom Projektleiter gepflegt. Der Backlog zeigt eine Aufgabenübersicht zugeordent zu den verantwortlichen Teammitgliedern.
Die Aufgaben werden während Sprints in den jeweiligen Sprint Backlog gezogen, um diese nach und nach während eines Sprints zu erledigen.
Das Ziel ist die Tasks so aufzuspalten, dass man sie in einem Sprint erledigen kann und sie nur bei keinem Erfolg im nächsten Sprint weiter behandelt werden müssen.
Er wurde in der Endphase durch Tasks ersetzt da, das Backlog System während des Testens nicht passend ist.
%

\subsection{Sprints}\label{subsec:sprints}
%
Ein Sprint ist in der Projektmanagementmethode Scrum das Hauptwerkzeug für agiles Management. Sprints sind zeitlich vordefinierte Rahmen in denen das Development Team Zeit hat, um ihr Sprintgoal zu erfüllen.
Dies startet mit einer Planungsphase, dem Sprintplanning, geht über in die Entwicklungsphase, der Sprint, und endet mit einem Sprint Review, in dem die Erreichung des Sprint Goals geprüft wird.
Zuletzt wird ein Retrospective absolviert, um sich intern Feedback zu geben, über Positives und Negatives im Sprint.
%

% resets author
\renewcommand{\kapitelautor}{}


\section{Risiken}\label{sec:risiken}

\renewcommand{\kapitelautor}{Autor: Irgendwer} % todo: replace
%
Eine Risikoanalyse ist ein wichtiger Teil im Planungsprozess, um mögliche Herausforderungen zu berücksichtigen.
Mit einem klaren Überblick über bevorstehende Risiken, lassen sich Maßnahmen treffen um diese vorzubeugen oder eventuell zu vermeiden.\cite{AsanaRisiken}

Projekte sind immer stark von Risiken geprägt so auch .Forty-Five. Darum gab es auch eine Planungsphase in der wir uns mit den bevorstehenden Schwierigkeiten befasst haben.
Es gibt klassische Risiken wie eine zeitliche Verzögerung durch Ausfälle oder Schwierigkeiten beim Know-How in dem Fall der Steam Release.
Dieses Diplomarbeitsprojekt war jedoch von ganz anderen Herausforderungen gezeichnet. Es ist die Kunst andere zufrieden zu stellen und zu unterhalten die eine Herausferdeung stellten.
Das Ziel ist es Leute mit dem Videospiel zu unterhalten und dafür zu sorgen, dass es nicht eintönig oder gar langweilig wird.
Spieler sollen nicht nur Spaß am Spielen haben, sondern im besten Fall sogar ihren Freunden davon zu erzählen.
Das größte Risiko für ein Videospiel ist, dass es nicht unterhaltsam ist und keinen Zeitvertreib bietet. Darum haben wir uns auch am Schluss des Projekts so intensiv mit dem Unterhaltungsfaktor auseinandergesetzt.
Denn Videospiel Design besteht aus verschiedensten Aspekten, die alle auch ein Teil des Gamedesigns sind.
%

% resets author
\renewcommand{\kapitelautor}{}


\section{Tools}\label{sec:tools}

\renewcommand{\kapitelautor}{Autor: Irgendwer} % todo: replace

\subsection{Jira}\label{subsec:jira}

%
% text goes here
%

\subsection{Zeittracking}\label{subsec:zeittracking}

%
% text goes here
%

% resets author
\renewcommand{\kapitelautor}{}


%
\chapter{Gamedesign}\label{ch:gamedesign}


\section{Einleitung}\label{sec:einleitung}

\renewcommand{\kapitelautor}{Autor: Philip Jankovic}


\FF ist ein digitales Kartenspiel in einem Wild West Setting. Der spieler reist durch den Wilden Westen auf einer Map,
sammelt Karten und bekämpft damit Gegner um auf der Map fortzuschreiten. Karten sind Bullets und werden in das Spielfeld, also den Revolver geladen.
Jede Bullet hat einen eigenen Effekt, der von dem Spieler genützt werden kann um immer bessere Combos aufzubauen und damit immer stärkere Gegner zu bekämpfen.
Der Revolver wird dazu verwedet, Bullets auf den Gegner abzufeuern. Nachdem eine Bullet geschossen wurde, verlässt die
Bullet den Revolver und der Revolver sich eines nach rechts. Der Gegner kann verschiedene Gegneraktionen ausführen und damit dem Spieler schaden.
Durch die  immer besser werdenden Bullets des Spielers, ist es ihm möglich, die immer schwerern Gegner zu bezwingen und damit den Wilden Westen zu durchqueren.
Die Map enthält verschiedene Biome und Events, durch die der Spieler neue Karten erwirbt oder er sich heilen kann.



\subsection{Begriffserklärung}\label{begriffserklärung}
Mechanik: In \FF sind Mechaniken einzelne Komponente des Spieles, welche zusammgesetzt das Spiel ergeben.

Combo: In \FF wird eine Combo als eine Kombination von Karten mit guter Synergie zueinander bezeichnet.


Map: Die Karte von \FF, auf der sich der Spieler bewegt und diverse Events auswählt, wie \zB Kämpfe oder Shops.

Road: Eine Road in \FF ist eine zufällig generierter Abschnitt des Spieles. Eine Road befinndet sich immer zwischen zwei Areas.

Area: Nicht zufällig gnerierte Abschnitte der \FF Map.

Rogue-Like: Ein Genre von Videospiel, bei welchem der Fortschritt des Spielers verloren geht, sollte er streben/verlieren.

Run: Ein Run ist ein Durchlauf in einem Rogue-like oder Rogue-lite Spiel. \zit{zitatdeckbuilding}

Rogue-Lite: Ein Subgenre des Rogue-Likes, bei welchem Teile des Fortschrittes in den nächsten Run mitgenommen werden.
Es ist einfacher und weniger frustrierend als ein Rogue-Like

% text goes here
%

% resets author
\renewcommand{\kapitelautor}{}


\section{Kartenspiele/Spiele als abstrakte Konzepte}\label{sec:abstrakte-konzepte}

\renewcommand{\kapitelautor}{Autor: Irgendwer} % todo: replace

%
% text goes here
%

% resets author
\renewcommand{\kapitelautor}{}


\section{Flavour}\label{sec:flavour}

\renewcommand{\kapitelautor}{Autor: Irgendwer} % todo: replace

\subsection{Die Wichtigkeit des Settings und des Flavors für ein Kartenspiel}\label{subsec:wichtigkeit-des-flavours}

%
% text goes here
%

\subsection{Flavor durch Spielmechaniken und Spielmechaniken durch Flavor}\label{subsec:flavour-durch-mechaniken}

%
% text goes here
%

% resets author
\renewcommand{\kapitelautor}{}


\section{Geschichte hinter der Entwicklung des Gameplays}\label{sec:gameplay-geschichte}

\renewcommand{\kapitelautor}{Autor: Irgendwer} % todo: replace

%
% text goes here
%

% resets author
\renewcommand{\kapitelautor}{}


\section{Die verschiedenen version von .Forty-Five}\label{sec:verschiedene-versionen}

\renewcommand{\kapitelautor}{Autor: Irgendwer} % todo: replace

%
% text goes here
%

% resets author
\renewcommand{\kapitelautor}{}


\section{.Forty-Fives grundlegendste Regeln}\label{sec:grundlegenste-regeln}

\renewcommand{\kapitelautor}{Autor: Irgendwer} % todo: replace

%
.Forty-Five ist ein digitales Kartenspiel des Subgenres des Rogue-like Deckbuilders. %TODO: Quelle für Erklärung
Der Spieler bewegt sich über eine Map, die prozentual generiert ist, sammelt Karten und benutzt diese Karten,
um Gegner zu bekämpfen. Sollte der Spieler einen Kampf verlieren, stirbt er, seine gesammelten Karten
gehen verloren und er startet wieder am Anfang. Da es sich jedoch um ein "Rogue-Lite" handelt und nicht um ein "Rogue-Like",
gibt es eine Art von speicherbarem Fortschritt, den der Spieler in sein nächstes Leben mitnehmen kann.
Ein Durchlauf bzw. ein Leben des Spielers wird als "Run" bezeichnet. Ein Run endet mit dem Tod des Spielers.
Während der Entwicklung wurden jene Elemente, die in das nächste Leben übergehen, als "Rogue-Lite-Elemente" bezeichnet.


%zuerst erklären was roads und so sind????
\subsection{.Forty-Fives Rogue-Lite-Elemente}\label{rogue_lite_elemente}

Um das Spielerlebnis des Spielers nicht allzu frustrierend zu gestalten, wurden zwei verschiedene Mechaniken eingeführt.
Bei der ersten Mechanik handelt es sich um die Erhöhung der maximalen Lebenspunkte des Spielers bei Heilpunkten im Spiel. Der Spieler hat die Möglichkeit
sich zu entscheiden, ob er sich lieber nur für diesen Run heilt und dafür auch eine größere Menge an Lebenspunkten erhält, oder ob er
lieber in das Erhöhen seiner Lebenspunktkapazität investiert. Bei letzterem handelt es sich natürlich um eine kleinere Zahl als bei der anderen Wahl.
Dies dient dazu, dem Spieler die Entscheidung offen zu halten, entweder etwas Langwieriges aber Bleibendes zu investieren oder lieber noch einmal ordentlich
Lebenspunkte aufzutanken, was bei der zweiten Mechanik ins Spiel kommt. %todo quelle füg max hp und hp generell

Schafft der Spieler einen Abschnitt des Spiels, also eine sogenannte Road, werden seine bis zu dem Zeitpunkt gesammelten Karten
gespeichert und sind ab dann selbst nach einem Tod jederzeit zugänglich und spielbar. Verknüpft mit der ersten Mechanik
hat der Spieler die Möglichkeit, sich dazu zu entscheiden, seine Leben wieder aufzuheilen und das Ende der Road anzustreben.
Ist er der Meinung, es sowieso nicht mehr zu dem Abschnittsende zu schaffen, wählt er die Erhöhung der maximalen
Lebenspunkte, um so zumindest einen Vorteil in den nächsten Runs zu haben. Der Spieler muss also das Risiko und die
Belohnung abwägen und daran entscheiden, welche Wahl er trifft.

\subsection{Das Sammeln der Karten}\label{sammeln_der_Karten}

Nach jedem überlebten Kampf bekommt der Spieler eine neue Karte. Er bekommt drei verschiedene Karten zur Auswahl gezeigt und darf sich eine davon aussuchen.
Außerdem gibt es auf der Map verteilt noch zusätzliche Events, die der Spieler aufsuchen kann, um seine Sammlung zu erweitern.
Das bringt nicht nur wieder eine Entscheidung des Spielers mit sich, die den Verlauf des Runs ändert, sondern bringt auch mehr Abwechslung,
da der Spieler nicht immer nur die Karten nehmen kann, die er gerne hätte. Es bringt auch eine Art Entdeckerlust mit sich, da der Spieler
auf diese Weise natürlich die Karten erst nach und nach sieht und nicht alle gleich auf einmal (siehe z.B. Magic Arena, wo alle Karten sofort in einer eigenen Liste zugänglich sind). %todo: Quelle
Diese Art des Erwerbs von Karten wird "Draft" genannt %TODO Quelle
und zwingt den Spieler mit begrenzter Wahl und Ressourcen das Beste daraus zu machen. Diese Designpolitik wird beim Design der Karten beachtet und wird zu einem späteren Zeitpunkt genauer erläutert %TODO macht man das so?
Viele Rogue-like Deckbuilder haben ein ähnliches System. Inspiriert wurde das Sammeln der Karten in ".Forty-Five" von Spielen wie Inscription und Slay the Spire, %Todo Quelle
hat jedoch einen großen Unterschied. Anders als in z.B. Slay the Spire können Karten ganz einfach aus dem Kartendeck entfernt werden, nachdem sie einmal ausgewählt wurden.
Karten können aus dem Deck in den Backpack verschoben werden und ermöglichen dadurch ein flexibleres Spielerlebnis. %todo Quelle. Es gibt auch einen Deckbuilder, der das so macht, keine Ahnung wie der heißt.


\subsection{Der Backpag und das Deck}\label{backpack_and_deck}
\begin{infoBox}
\end{infoBox}
Note: Zu diesem Zeitpunkt geht es nicht um die Designprinzipien hinter dem Deck und den Karten,
sondern nur um die Erklärung der grundlegenden Mechaniken. Das Gamedesign wird zu einem späteren Zeitpunkt beschrieben.

Der Backbag und das Deck dienen beide als Speicherort für Karten, mit dem Unterschied, dass die Karten im Deck aktiv im
Kampf eingesetzt werden und die Karten im Backbag mehr als Reserve gelten.
Karten können frei zwischen den beiden verschoben werden, solange die festgelegte Mindestanzahl an Karten im Deck eingehalten wird.
Da Forty-Five viele verschiedene Strategien bietet, gibt es mehrere Decks, zwischen denen man einfach wechseln kann.
Das hat zur Folge, dass der Spieler nicht immer ein Deck zerlegen muss, um eine andere Strategie auszutesten.
Außerdem können Decks umbenannt werden, um besser wiedergefunden und erkannt zu werden.
Karten können im Backbag nach Kriterien wie Kosten oder Namen auf- und absteigend sortiert werden.
Sammelt der Spieler eine neue Karte, kann er sich entscheiden, ob er die Karte gleich seinem Deck oder doch eher seinem
Backbag hinzufügen möchte.
Der Backbag und das Deck ermöglichen es dem Spieler, Karten, die er gesammelt hat, aus dem Deck zu nehmen und damit
nicht zu verwenden, sowie Decks zu bauen, die zu einer Strategie passen.
Startet der Spieler nun einen Kampf, wird das zuletzt ausgewählte deck verwendet.


\subsection{Der Kampf}\label{backpack_and_deck}
%grundsätzliche sachen wie reserves, gegner und spieler, gewonnen wenn gegner tod usw karten am anfang ziehe zwei karten am anfang vom turn

Während der Spieler über die Map reist, sind Kämpfe unausweichlich.
Gekämpft wird mit den gesammelten Karten, mit dem Ziel, den oder die Gegner zu besiegen und dabei so wenig Lebenspunkte
wie möglich zu verlieren bzw. nicht zu sterben. Gewonnen hat der Spieler, wenn alle Gegner besiegt wurden.
Ein Kampf ist aufgeteilt in Züge. Immer abwechselnd ist entweder der Spieler oder der Gegner dran. Am Anfang des Kampfes
werden eine vordefinierte Anzahl an Karten vom Deck des Spielers gezogen. %todo die genaue Anzahl fehlt noch (5 oder 6?)
Anschließend hat der Spieler die Möglichkeit, nach Belieben seinen Zug auszuführen. Ein Spielerzug wird mit dem "End Turn"
Button beendet, und das Betätigen des Knopfes startet den automatischen Ablauf des Gegnerzuges. Der Gegner führt eine mehr oder weniger zufällige Aktion aus, und danach ist wieder der Zug des Spielers.
Am Start des Zuges des Spielers werden zwei Karten vom Deck gezogen. Pro Zug des Spielers stehen dem Spieler 4 "Reserves"
zur Verfügung. Diese werden jeweils auch am Anfang des Zuges wieder auf die maximale Anzahl aufgefüllt.
Reserves können dazu verwendet werden, Karten zu bezahlen und damit auch zu spielen.


\subsection{Karten}\label{Karten}
Cards in \FF are Bullets. Each Bullet has a cost that must be paid in order to play the card.
The reserves needed for the Bullet are paid automatically IF the player is able to afford the Bullet, the moment the Bullet is played.
Das Management von diesen Reserves und die Reihenfolge, in der man die Karten spielt, ist wichtig, um besser in Forty-Five zu werden. %TODO Mehr dazu ist im Gamedesign.
Jede Bullet hat außerdem einen Damage-Wert, welcher angibt, wie viele Lebenspunkte dem Gegner durch die Karte abgezogen werden,
falls die Karte auf den Gegner geschossen wird. Fast alle Karten haben außerdem einen einzigartigen Effekt,
welcher angesehen werden kann, indem der Spieler mit der Maus über die gewünschte Bullet hovert.
In dem Popup, welches nach dem Hovern sichtbar ist, befinden sich die Infos zu dem Effekt der Karte,
dazu gehört fast immer ein Trigger und der Effekt selbst.
Der Trigger gibt an, wann sich der Effekt aktiviert. Es gibt verschiedene Trigger, wie zum Beispiel das Aktivieren des Effektes beim Spielen der Bullet. %TODO siehe mehr später bei trigger
Zusätzlich zu dem Effekt und dem Trigger steht bei vielen Karten außerdem noch ein Flavortext dabei.
Ein Flavortext ist ein Spruch, der der Bullet ein wenig Kontext hinzufügt.
Dies kann passieren durch einen dummen Spruch, einen Witz, eine Anspielung oder einen story- bzw. worldbuilding-relevanten Text.
Falls nötig werden relevante Infos zu dem Effekt in einer extra Box rechts oder links angezeigt.
Bullets werden, wenn sie gespielt werden in den Revolver geladen.

\subsection{Der Revolver}\label{der_revolver}
Der Revolver ist das Spielfeld von \FF, in weleches die Bullets platziert bzw "geladen" werden.Mithilfe von Drag and drop,
kann der PSieler die Karte aus seiner hand in den Revolver legen. Dies ist das bereits erwähnte spielen einer Karte, zu dem auch das Bezahlen dazugehört.
Der genaue ablauf des Spielen einer Karte ist wie Folgt:
Drag and drop in den gewünschten revolver platz > bezahlen der Reserves, abbruch falls nicht genug Reserves vorhanden sind >
platzieren der Bullet > womöglich aktivieren von Triggern, falls die Bullet einen Effekt hat der einen On-Placedown Trigger hat.
Der  Revolver besteht aus 5 kammern bzw felder in die Bullets geladen werden können. nummeriert sind sie von 1 -5. %TODO bild und noch umdrehen von 5 zu 1
Der revolver kann von dem Spieler geschossen werden. Wird geschossen wird die Bullet in der obersten chamber auf den Gegner geschossen,
verlässt sie den Rvolvr und der Gegner verliert Lebenspunkte in der Höhe der Dmg Value der gecshossenen Bullet.
Die Bullet wird unter das deck gelegt und der revolver dreht sich einmal nach rechts.
Die Revolverrotation ist ein wichtiger Teil von forty-five, da man das Placement der Bullets
im revolver sich genau überlegen muss um das meiste aus seinen Bullets raussuchen. %todo mehr in game design.
Es gibt außerdem einen Trigger, wlecher aktiviert wird wenn die Karte geschossen wird namens On-Shot.
Zusätzlich gibt es Karten mit Effekten, welche die Rvolverrotation verändern wie z.B Bewitched Bullet, welche den Revolver nach links statt nach rechts rotiert

Die Komplexität von \FF kommt von dem meistern des Revolvers. Das strategische platuieren von Bullets,
die reihenfolge der Bullets und das verstehen wie eine Revolver rotation sich auf Bullets, den Spieler oder den gegner auswirken.
Die gerade erwähnte Bewitched Bullet kann zum Biepiel dazu verwendet werden Karten , weleche davon profetieren lange im revolver zu bleiben,
wieder weiter von dem Schießen wegzuschibene. Bull et zum Beipeil fügt dem gegner Schaden zu wenn sich Bull et Rotiert im Revolver. %TODO BILDER!!!
Bewitched Bullet kann dazu verwendet werden zwei zusätzliche rotationen rauszuholen in dem man den revolver einmal nach links dreht bevor Bull et den Revolver verlässt.
Bewitched Bullet kann zumm beipeil auch zum einschieben von Bullets verwendet werden und so wieter udn so fort.
Die kombinationsmöglichkeiten sind endlos und fast jeder effekt einer Bullet kann mit dem einer anderen Bullet verknmüpft werden.



%slots rotationen usw karten weg wenn shot und karten unters deck gelegt wenn shot usw


\subsection{Zonen}\label{backpack_and_deck}
%probbaly unnötig

\subsection{Der Gegner}\label{der_gegner}
Der Gegner, auf welche die Bullets des Spielers geschossen werden, besteht aus mehreren Komponenten.
Genuase wie der Spieler, verfügt der Gegner über einen HP wert. Hat der Wert null erreicht, stirbt der Gegner.
Nachdem der Spieler End turn druückt, beendeter seinen zug und der Gegner ist dran.Beierite während dem Zug des Spielers zeigt der Gegner die AKTIOn an, welche er in seinem Zug ausführen wird.
Gegner aktionen werden "zufällig ausgewählt aus dem pool von Aktionen des Gegners.
Jedcoh wurden die aktionen angepasst um nicht nur fairness zu garantieren,
sondern auch eine balance zwischen zu leicht und zu schwer zu finden.
Mehr zu dem Balancen der Gegner kann in 07 Gamedesign nachgelsene werden.
Sobald der Gegner dran ist, wird die Aktion automatisch ausgeführt. Je nach Gegnertyp gibt es andere Aktionen welche ein Gegner ausführen kann. Universell ist jedoch die Aktion des Schaden zufügens.
Dargetsellt durch ein symbol über dem Kopf des Gegners, sind die Aktionen weniger eine Reaktion auf den Spieler, sondern eher wird erwartet, dass der Spieler die Aktion des Gegner in das palnnen seines Zuges miiteinbezieht.
Bei der eben ganennanten Schadenaktion, kann der Spieler darauf reagieren und schaden verhindern indem er die KArte parried.

\subsection{parrying}\label{backpack_and_deck}
%slbsterklärend, wie es funktioniert warum usw

\subsection{Spezifische Regeln}\label{spezifische_regeln}
%overkill dmg
%

\subsection{Encounter Modifier}\label{backpack_and_deck}
%

% resets author
\renewcommand{\kapitelautor}{}

\input{text/08_gamedesign/07_gamedesign.tex}

\section{Der Revolver}\label{sec:der-revolver}

\renewcommand{\kapitelautor}{Autor: Irgendwer} % todo: replace

%
% text goes here
%

% resets author
\renewcommand{\kapitelautor}{}

\input{text/08_gamedesign/09_meet_the_bullets.tex}

\section{Bullets und Effekte}\label{sec:bullets-und-effekte}

\renewcommand{\kapitelautor}{Autor: Irgendwer} % todo: replace

\subsection{Trigger}\label{subsec:trigger}

%
% text goes here
%

\subsection{Statuseffekte}\label{subsec:statuseffekte}

%
% text goes here
%

\subsection{Traiteffekte}\label{subsec:traiteffekte}

%
% text goes here
%

% resets author
\renewcommand{\kapitelautor}{}


%
\chapter{Marketing}\label{ch:marketing}

\usepackage{hyperref}
\section{Zielgruppe}\label{sec:zielgruppe}

\renewcommand{\kapitelautor}{Autor: Nils Hubmann}

%
Die Einschätzung der Produktqualität und seines Erfolgs auf dem Markt hängt größtenteils vom Nutzen ab, den Kunden aus dem Produkt ziehen.
Es ist jedoch unerlässlich, dass das Produkt die Anforderungen, Wünsche und Erwartungen der Kunden optimal erfüllt, da sie letztendlich über den Erfolg eines Unternehmens entscheiden.
In diesem Zusammenhang sind Analysen wie die Definition der Zielgruppe äußerst hilfreich.
Unter dem Begriff Zielgruppenanalyse versteht man alle Aktivitäten, die damit einhergehen, zu verstehen, was die Konsumenten von einem Produkt erwarten, wie sie sich verhalten und welche Bedürfnisse sie haben.cite{Zielgruppe}


Die übliche Klassifizierung von Zielgruppen basiert oft auf äußeren Merkmalen, wie sozio-demografischen Daten oder finanziellen Aspekten.
Jedoch geht es bei der Definition einer Zielgruppe weit darüber hinaus.
Eine Zielgruppe zeichnet sich vor allem durch gemeinsame Wünsche, Bedürfnisse und Probleme aus.
Daher ist es wichtig, die Zielgruppe primär über ihre Bedürfnisse und Probleme zu definieren.
Erst danach können weitere übliche Kriterien wie Kaufkraft, Religion oder Alter zur weiteren Eingrenzung herangezogen werden. \zit{Zielgruppedef}

Dabei sind wir auf folgende Ergebnisse gekommen.

\subsubsection{Demographische Merkmale}\label{subsubsec:Demographische-Merkmale}

Die demografischen Merkmale beziehen sich auf das Alter, Geschlecht, Berufsstand und weiteren wichtigen Aspekte des täglichen Lebens.\zit{DemographischeMerkmale}

\begin{liste}
    \item Alter: 16-50 jährig (Das Spiel hat eine Altersbeschränkung von 16 Jahren)
    \item Geschlecht: Das Geschlecht ist nicht relevant
    \item Bildung: mittelmäßiges Bildungsniveau
    \item Beruf: Das Spiel ist für keine bestimmte Berufsgruppen interessant.
    \item Einkommen: Da das Spiel gratis sein wird, ist das Einkommen der Zielgruppe nicht relevant.
\end{liste}

\subsubsection{Geographische Merkmale}\label{subsubsec:Geographische-Merkmale}

„Auch diese Art von Merkmalen wird recht häufig genutzt, um zu erfahren, wo ihre Kunden wohnen, um so eine Werbung zu machen, welche auch die richtigen Leute am richtigen Ort trifft.“\zit{GeographischeMerkmale}
Die geografische Aufteilung nach Bundesländern oder Bezirken ist für dieses Computerspiel von geringer Bedeutung, da das Spiel global betrachtet wird und die Spielerbasis weltweit anspricht.
Die Kommunikation erfolgt hauptsächlich in englischer Sprache, um eine weitreichende Verständigung zu ermöglichen, insbesondere in englischsprachigen Ländern wie den USA, Großbritannien, Kanada und anderen.
Diese Länder werden aufgrund ihrer Muttersprache als Zielregionen betrachtet.

\subsubsection{Psychographische Merkmale}\label{subsubsec:Psychographische-Merkmale}

"Psychografische Faktoren sind beispielsweise Verhaltensmerkmale, Werte, Vorlieben und Charaktereigenschaften Ihrer Kunden und Kundinnen.
Welchen Lifestyle haben sie?
Was bewegt sie und warum?
Welchen Hobbys gehen sie in ihrer Freizeit nach?
Ist ein ausgeprägtes Gesundheits- und Umweltbewusstsein vorhanden?
Haben sie traditionelle Einstellungen oder
sind sie offen für Neues?
All diese Fragen helfen Ihnen, sich besser in Ihre Zielgruppe hineinzuversetzen." \zit{PsychographischeMerkmale}

\bold{Persönlichkeit:} Das Spiel ist hauptsächlich an Personen gerichtet, die sich für Karten- und Strategiespiele interessieren.

\bold{Freundesstand:} Da dieses Spiel allein spielbar ist, sind für die Nutzung keine weiteren Personen notwendig.

\bold{Hobby:} Computerspielen

Folgende Genres, die zur Bestimmung der Verhaltensweise wichtig sind:
\begin{liste}
    \item WildWest
    \item Strategie
    \item Indie
    \item Einzelspieler
    \item Roguelite
    \item Kostenlos
\end{liste}


\subsubsection{Zusammenfassung}\label{subsubsec:Zusammenfassung}
Das Spiel ist hauptsächlich für Hobbyspieler interessant, die nicht viel Geld in Spiele investieren wollen. Dabei hat \ff einen Vorteil gegenüber anderen Indiegames einen Vorteil,
da diese zwar nicht so viel kosten aber trotzdem nicht gratis sind.
Die meisten Indiegames kosten zwischen 5 und 25€ und sind daher Low Budget Games.\zit{IndiegamesPreis}
Des Weiteren wissen viele neue Spieler gar nicht, ob ihnen das Spiel gefällt, weshalb sie es, dadurch das es gratis ist, testen können.
Das kann wiederum dem Spiel helfen, mehr Aufmerksamkeit zu bekommen.


% resets author
\renewcommand{\kapitelautor}{}


\section{Social Media}\label{sec:social-media}

\renewcommand{\kapitelautor}{Autor: Irgendwer} % todo: replace

%
% text goes here
%

% resets author
\renewcommand{\kapitelautor}{}


\section{Website}\label{sec:website}
\renewcommand{\kapitelautor}{Autor: Nils Hubmann}

%
Die Webseite von \ff ist nicht nur ein digitaler Schauplatz und ausschlaggebend für die Bekanntmachung unseres Spiels

\bold{Promotion des Spiels:}
Eine Webseite ist von entscheidender Bedeutung, um das Spiel einem breiteren Publikum bekannt zu machen.
Durch die Präsentation von Screenshots, Trailern, Hintergrundgeschichten und Funktionen des Spiels können wir potenzielle Spieler ansprechen und ihr Interesse wecken.
Mit einer gut gestalteten und informativen Webseite können wir unsere Zielgruppe effektiv erreichen und das Spiel auf dem Markt positionieren.

\bold{Einblicke zum Spiel, dem Team und dem Projekt:}
Die Webseite bietet nicht nur Informationen über das Spiel selbst, sondern auch über das Team hinter dem Projekt.
Hierfür steht eine separate Projektseite zur Verfügung.
Indem wir Einblicke in das Projekt liefern, wollen wir die Seite als Möglichkeit nutzen und Leute auf die Idee aufmerksam machen.
Spieler können sich so besser mit dem Spiel identifizieren und eine emotionale Bindung dazu aufbauen.

\bold{Promotion der Steampage:}
Die Steampage ist das wichtigste Vertriebsmittel unseres Spiels und die Webseite dient als Sprungbrett, um Spieler dorthin zu führen.
Durch gezielte Verlinkungen und Trailer auf unserer Webseite
können wir die Neugier der Besucher wecken und sie dazu ermutigen, die Steampage zu besuchen.
Auf der Steampage finden sie dann detaillierte Informationen, Bilder und Videos, die ihr Interesse weiter vertiefen.
%

\vfill
\pagebreak

\subsection{Implementation}
\renewcommand{\kapitelautor}{Autor: Marvin Kurka}

Da es sich bei der Website um eine relativ simple, statische Website handelt, die kaum mit dem User interagiert.
die Entscheidung getroffen kein Framework wie \zB Vue oder React zu verwenden.
Allerdings wurde Sass verwendet, eine Css-Extension, die mehrere Features hat, die das Erstellen und Strukturieren
von großen Stylesheets vereinfachen.
Sass erlaubt \zB das Erstellen und Verwenden von Variablen, das Unterordnen von Selektoren oder das Aufteilen von
Stylesheets in mehrere Dateien.\zit{sassDoc}
Da ein relativ umfangreiches Stylesheet für die Website notwendig war, wurde die Entscheidung getroffen Sass zu
verwenden, um die Maintainability zu erhöhen.

Eine weitere Eigenschaft der \FF Website ist, dass besonders viele Bilder verwendet werden, um die Website dem Stil des
Spiels anzugleichen.

\begin{figure}[H]
    \centering
    \includegraphics[width=1.0\textwidth]{website.png}
    \caption{Screenshot der Website}
\end{figure}

Das wirkt sich negativ auf die Ladezeiten der Website aus.
Um diesen Effekt so gering wie möglich zu halten, wurde das von Google entwickelte webp Bild-Format verwendet, welches
eine deutlich bessere Komprimierung als vergleichbare Formate wie \zB jpg erzielt.
In einer vergleichenden Studie, welche von Google durchgeführt wurde, erreichte webp je nach verwendeter Methodik eine
Verkleinerung von 14\% gegenüber eines mit jpg komprimierten Bildes.\zit{googleWebpStudie}

% resets author
\renewcommand{\kapitelautor}{}


\section{Trailer}\label{sec:trailer}

\renewcommand{\kapitelautor}{Autor: Irgendwer} % todo: replace

%
% text goes here
%

% resets author
\renewcommand{\kapitelautor}{}

\section{Werbeplakate}\label{sec:trailer}

\renewcommand{\kapitelautor}{Autor: Markus Böheim}

Ein Plakat dient als Interessenwecker und Call-to-Action. Mit Aufforderungen wie beispielsweise
\quoted{Besuche unsere Website} oder \quoted{Play Now} wird das Engagement gefördert.
Beim Designen eines Plakats um die Veröffentlichung des Spiels aufzuhypen, müssen folgende Dinge beachtet werden und vorhanden sein: Name des Spiels, passendes Artwork zum Spiel, das Studio welches das Spiel entwickelt hat und das geplante Veröffentlichungsdatum. Optional ist eine kurze Beschreibung über das Spiel und die Mitglieder des Projekts.
Als Artwork für das Plakat bietet sich eine Kampfszene, einen sogenannten \quoted{Standoff} am besten an, da dieser
ein klassisches Bild des wilden Westens ist und der Kampf das Kernstück von \FF repräsentiert. Eine kalte Farbpalette
sorgt für einen dramatischen aufmerksamkeitserregenden Eindruck. Daher wurde für Schwarz, Weiß und ein blaustichiges Rot
entschieden. Damit der Hintergrund nicht nur Weiß ist, sondern auch interessanter
anzusehen ist, wurde eine lizenzfreie papierartige Textur von \url{https://texturelabs.org/?ct=666} verwendet. Da die
Karten in \FF eine wichtige Rolle spielen, mussten diese auf irgendeine Art und Weise in dem Plakat implementiert
werden. Zuerst wurden mit Adobe Xd viele Karten aneinandergelegt und als PNG exportiert. Dieses Bild wurde danach in
die Silhoutte der gezeichneten Hauptfigur auf dem Plakat eingefügt mithilfe einer Schnittmaske. Eine Schnittmaske ist
eine Gruppe von Ebenen, auf die eine Maske angewendet wird. Die sichtbaren Begrenzungen der gesamten Gruppe werden
durch die unterste Ebene (auch Basisebene genannt) bestimmt \zit{schnittMasken}. Als Anregung oder Aufforderung wurde
sich für den Text
\quoted{Will you take the Challenge?} auf dem Plakat entschieden. Dieser Text wurde genauso wie der Text
\quoted{.Forty-Five} texturiert um einen rustikalen Look zu verpassen. Da diese Schriftart keine Serifen aufweist und
daher modern wirkt \zit{serifen} , wurde mit einer Displacement-Map die Form rauer gemacht. Eine Displacement Map ist eine
Graustufenkopie (Schwarzweiß) eines Bildes. Indem die Farben, Lichter und Schatten des Bildes in Graustufen
vereinfacht werden, können neue Elemente hinzugefügt werden, die den Höhen und Tiefen des Originalbilds folgen \zit{
    dispMap}.
Das
Plakat soll einem Filmplakat ähneln, weshalb sich für eine Schriftart für den Text am Ende des Plakats entscheidet wird, die einer eines Filmplakats ähnelt. Der Rahmen sorgt dafür, dass sich das Plakat von Wänden abhebt. Beispielsweise würde bei einer weißen Wand der weiße Hintergrund des Plakats mit der Wand verfließen. Letztendlich verleihen
\quoted{Grunge} und \quoted{Ink/Paint} Texturen von Texturelabs dem Plakat den letzten Feinschliff. Das Plakat wird in Adobe Photoshop erstellt, da viele Pixelgrafiken und Effekte - die andere Programme von Adobe nicht erzielen können - verwendet werden müssen. Allerdings muss der Fakt, dass die exportiere Datei eine Pixelgrafik ist, mit einer dementsprechend hohen Auflösung von 7160x10399px kompensiert werden.

\begin{figure}[H]
    \centering
    \includegraphics[width=0.7\textwidth]{fortyfivePlakat.png}
    \caption{Das \FF Plakat}
\end{figure}

% resets author
\renewcommand{\kapitelautor}{}


\chapter{Kotlin}\label{ch:kotlin}


\section{Code Clarity}\label{sec:code-clarity}

\renewcommand{\kapitelautor}{Autor: Irgendwer} % todo: replace

%
% text goes here
%

% resets author
\renewcommand{\kapitelautor}{}


\section{Null Safety}\label{sec:null-safety}

\renewcommand{\kapitelautor}{Autor: Irgendwer} % todo: replace

%
% text goes here
%

% resets author
\renewcommand{\kapitelautor}{}


\section{Flow Typing}\label{sec:flow-typing}

\renewcommand{\kapitelautor}{Autor: Irgendwer} % todo: replace

%
% text goes here
%

% resets author
\renewcommand{\kapitelautor}{}


\section{Lambdas}\label{sec:lambdas}

\renewcommand{\kapitelautor}{Autor: Irgendwer} % todo: replace

%
% text goes here
%

% resets author
\renewcommand{\kapitelautor}{}


\section{Nothing Typ}\label{sec:nothing-type}

\renewcommand{\kapitelautor}{Autor: Marvin Kurka}

Der \inlineKotlin{kotlin.Nothing} Typ ist in der Kotlin Standard-Bibliothek definiert, er hat allerdings einige
spezielle Eigenschaften.
Zum Beispiel handelt sich bei Nothing um einen unified subtype (auch bottom Type genannt) für alle in Kotlin
definierten Typen.
Das heißt, eine Instanz von Nothing wäre auch eine Instanz jeder anderen Klasse, sei es \inlineKotlin{String},
\inlineKotlin{Int} oder \inlineKotlin{() -> Boolean}.
Da so ein Objekt einen logischen Widerspruch darstellt, handelt es sich bei Nothing auch um einen uninhabited Type.
Das heißt, dass niemals eine tatsächliche Instanz von Nothing existieren kann.

Auf den ersten Blick mag so ein Typ nicht sehr nützlich wirken, er erlaubt den Compiler aber einige nützliche
Annahmen über den Control Flow zu machen.
Sollte jemals eine Variable den Wert Nothing annehmen, was einen logischen Widerspruch darstellt, weiß der Compiler,
dass dieser Code niemals erreicht werden kann.\cite{kspeCFG,kspecNothing}

%! language = kotlin
\begin{minted}{kotlin}
fun getNothing(): Nothing {
    ...
}

fun test() {
    // Nothing ist ein Subtype von String, daher ist der folgende Code erlaubt
    val s: String = getNothing()

    // für diesen Code gibt der Compiler eine Warning aus, da er niemals erreicht werden kann
    println("Hello World")
}
\end{minted}
\codeblockCaption{
Dadurch das die Funktion \inlineKotlin{getNothing()} den Rückgabewert Nothing hat, der niemals existieren kann,
weiß der Compiler, dass diese Funktion niemals auf normale Art und Weise returnen kann und in einer Exception
resultieren muss.
}

Um den Nothing-Typen tatsächlich nützlich zu machen, sind in Kotlin Statements, die einen Control-Flow Transfer
verursachen eigentlich Expressions mit dem Rückgabewert Nothing.
Beispiele sind: \inlineKotlin{throw}, \inlineKotlin{return}, \inlineKotlin{break} und \inlineKotlin{continue}.
Da diese Expressions einen sofortigen Wechsel des Control-Flows verursachen hat das zur Folge, dass der nachfolgende
Code niemals ausgeführt wird.
Damit können diese Expressions den Wert Nothing haben.\cite{kspeCFG}

%! language = kotlin
\begin{minted}{kotlin}
operator fun Int?.plus(other: Int?): Int? {
    // return hat den Wert Nothing, was ein Subtype von Int ist,
    // daher ist die Verwendung nach dem elvis operator erlaubt
    val first = this ?: return null
    val second = other ?: return null
    return first + second
}
\end{minted}
\codeblockCaption{Praktisches Beispiel für die Verwendung des Nothing-Typen}

Dadurch das \inlineKotlin{return} in Kotlin eine Expression ist, ist auch der folgende Code gültiges Kotlin:

%! language = kotlin
\begin{minted}{kotlin}
fun test() {
    println("Hello")
    return return return return return
}
\end{minted}

Funktionen, die den Rückgabewert Nothing haben, müssen immer eine Exception werfen.

%! language = kotlin
\begin{minted}{kotlin}
fun crashProgram(cause: String): Nothing {
    MyLogger.log("Program crashed because of: $cause")
    throw RuntimeException(cause)
}

fun doSomething(input: String?) {
    otherFunction(input ?: crashProgram("input must not be null"))
}
\end{minted}

Ein Beispiel für eine Funktion aus der Kotlin Standard-Bibliothek die Nothing zurückgibt, ist die
\inlineKotlin{TODO()} Funktion.

%! language = kotlin
\begin{minted}{kotlin}
fun someFunction(): Int = TODO("not yet implemented")
\end{minted}

% resets author
\renewcommand{\kapitelautor}{}


\section{Extension Functions}\label{sec:extension-functions}

\renewcommand{\kapitelautor}{Autor: Irgendwer} % todo: replace

%
% text goes here
%

% resets author
\renewcommand{\kapitelautor}{}


\section{Inline Functions}\label{sec:inline-functions}

\renewcommand{\kapitelautor}{Autor: Irgendwer} % todo: replace

%
% text goes here
%

% resets author
\renewcommand{\kapitelautor}{}


\section{Getter und Setter}\label{sec:getter-und-setter}

\renewcommand{\kapitelautor}{Autor: Irgendwer} % todo: replace

%
% text goes here
%

% resets author
\renewcommand{\kapitelautor}{}


\section{Kotlin vs. Java}\label{sec:kotlin-vs-java}

\renewcommand{\kapitelautor}{Autor: Irgendwer} % todo: replace

%
% text goes here
%

% resets author
\renewcommand{\kapitelautor}{}


%
\chapter{Tools}\label{ch:tools}


\section{Git}\label{sec:git}

\renewcommand{\kapitelautor}{Autor: Irgendwer} % todo: replace

\subsection{Workflow}\label{subsec:workflow}

%
% text goes here
%

\subsection{Github}\label{subsec:github}

%
% text goes here
%

% resets author
\renewcommand{\kapitelautor}{}


\section{Onj}\label{sec:onj}

Onj ist eine selbst entwickelte Markupsprache, die Ähnlichkeiten zu Sprachen wie Json, Toml oder Yaml aufweist,
diese allerdings um einige Features erweitert.
Onj-Dateien sind dafür ausgelegt, von Hand geschrieben zu werden und haben daher einige Quality of Life Features, die
ähnliche Sprachen nicht aufweisen.
Außerdem soll Onj auch für größere Projekte geeignet sein, ohne das der Wartungsaufwand zu groß wird.
Daher hat Onj Mechanismen um repetitiven Code zu vermeiden, wie imports oder Variablen, die Möglichkeit Strukturen
mit Schemas zu validieren und die Fähigkeit mittels Namespaces mit Kotlin-Code zu interagieren.

Onj-Dateien können nicht nur von Programmen eingelesen werden, Onj-Strukturen können auch mittels der
\inlineKotlin{buildOnjObject}-Funktion im Programm gebaut und in Dateien geschrieben werden.
Dabei können die Strukturen nicht nur zu gültigen Onj, sondern auch zu Json serialisiert werden.


\subsection{Warum Onj?}\label{subsec:warum-onj}

\renewcommand{\kapitelautor}{Autor: Irgendwer} % todo: replace

%
% text goes here
%

% resets author
\renewcommand{\kapitelautor}{}


\subsection{Überblick Syntax}\label{subsec:ueberblick-syntax}

\renewcommand{\kapitelautor}{Autor: Irgendwer} % todo: replace

%
% text goes here
%

% resets author
\renewcommand{\kapitelautor}{}


\subsection{Variablen}\label{subsec:variablen}

\renewcommand{\kapitelautor}{Autor: Marvin Kurka}

Um das Wiederholen von Code so gering wie möglich zu halten, werden in Onj Variablen verwendet, um häufige Strukturen
zu extrahieren.
Mittels des Variablennamen kann dan auf diese zugegriffen werden, auch Verknüpfungen mit Punkten sind erlaubt.

%! language = Onj
\begin{codeBlock}{onj}{Beispiel: Variablen in Onj}
var number = 5;
favoriteNumber: number,
otherNumber: (number + 1) * number,

var colors = {
    red: "#ff0000",
    green: "#00ff00",
    blue: "#0000ff",
};
favoriteColor: colors.blue,

var fruits = ["apple", "banana", "grape"];
// auf arrays wird mittels des Indexes zugegriffen
favoriteFruit: fruits.0,
\end{codeBlock}

Es können auch drei Punkte verwendet werden, um den Inhalt einer Variable in die aktuelle Struktur zu übernehmen.

%! language = Onj
\begin{codeBlock}{onj}{Beispiel: Triple-Dot Syntax in Onj}
var fruits = ["apple", "pear"];
fruitSalad: [
    "banana",
    "grape",
    ...fruits
],
\end{codeBlock}

Statt bei einem Zugriff einfach den Key-Namen oder den Index anzugeben, kann auch in Klammern eine Expression angegeben
werden.
Diese wird evaluiert und das Resultat wird für den Zugriff verwendet.

%! language = Onj
\begin{codeBlock}{onj}{Demo: dynamische Zugriffe in Onj}
var index = 1;
var arr = [0, 1, 2, 3, 4, 5];
myValue: arr.(index + 2) // => 3
\end{codeBlock}

% resets author
\renewcommand{\kapitelautor}{}


\subsection{Imports}\label{subsec:imports}

\renewcommand{\kapitelautor}{Autor: Irgendwer} % todo: replace

%
% text goes here
%

% resets author
\renewcommand{\kapitelautor}{}


\subsection{Onj Schemas}\label{subsec:onj-schemas}

\renewcommand{\kapitelautor}{Autor: Marvin Kurka}

Wenn Configdateien vom Programm eingelesen werden, muss überprüft werden, dass diese im richtigen Format sind.
Das kann vom Programmierer manuell gemacht werden, dies ist aber sehr umständlich, vor allem großen/komplexen Strukturen.
Stattdessen werden Schemas verwendet, die definieren wie eine Onj-Datei strukturiert sein darf.
Diese Schema-Dateien können vom Programm eingelesen und mit den Onj-Strukturen verglichen werden.

Schemas werden in \inlineCode{.onjschema} Dateien definiert.
Diese haben prinzipiell einen ähnlichen Syntax zu normalen \inlineCode{.onj} Dateien, statt Werten werden jedoch
Typen angegeben und einige Features sind nicht supported.

Die konkreten Unterschiede sind:
\begin{liste}
    \item Werte (\zB Zahlen, Strings, Booleans, \ldots) sind nicht erlaubt, stattdessen werden Typen angegeben.
    \item Funktionen sind nicht erlaubt.
    \item Variablenzugriffe (\inlineOnj{color.green}) sind nicht erlaubt.
\end{liste}

Anders als normale Onj-Dateien stellen Onjschema-Dateien allerdings Syntax um Typen zu definieren zur Verfügung.
Neben simplen literals (\inlineOnj{int}, \inlineOnj{float}, \inlineOnj{boolean}, \inlineOnj{string}) können auch
Objekte definiert werden.
Der Syntax für Objekte ist ident zu normalen Onj-Dateien.

\begin{codeBlock}{onj}{Demo: Objekt in OnjSchema}
myObject: {
    myNumber: int,
    myString: string
}
\end{codeBlock}

Arrays können auf zwei Arten definiert werden.
Entweder als Literal, die den Typen an jeder Stelle definiert, oder als Typ gefolgt von eckigen Klammern.
Bei der zweiten Variante kann in den Klammern optional die Länge des Arrays definiert werden.

\begin{codeBlock}{onj}{Demo: Arrays in OnjSchema}
myArray: [string, boolean, float],
myNumbers: int[],
threeNumbers: int[3]
\end{codeBlock}

Um anzugeben, dass ein Typ null sein darf, wird ein Fragezeichen verwendet (\inlineOnj{int?}).
Um anzugeben, dass ein beliebiger Typ verwendet werden darf, wird ein Stern (\inlineOnj{*}) verwendet.

Wenn in einem Schema ein Objekt definiert wird, darf dieses nur die Keys haben, die tatsächlich im Schema angegeben
wurden.
Hat ein Objekt einen Key, der im Schema nicht zu finden ist, führt das zu einem Fehler.
Um anzugeben, dass ein Objekt auch andere Keys erlauben soll wird der \inlineOnj{...*}-Syntax verwendet.

% resets author
\renewcommand{\kapitelautor}{}


\subsection{Named Objects}\label{subsec:named-objects}

\renewcommand{\kapitelautor}{Autor: Marvin Kurka}

Bei der Validation von komplexen Onj-Strukturen kann es unter gewissen Umständen zu Problemen können, vor allem wenn
an einer Stelle mehrere verschiedene Objekte möglich sind.
Um das Problem besser erkenntlich zu machen, hier ein Beispiel:

%! language = Onj
\begin{codeBlock}{onj}{Beispiel: Schlechte Umsetzung einer UI-Struktur 1}
// onj struktur
uiElements: [
    {
        type: "label",
        text: "Hello World",
        font: "red wing",
    },
    {
        type: "image",
        path: "./some/image.png"
    }
]
\end{codeBlock}

\begin{codeBlock}{onj}{Beispiel: Schlechte Umsetzung einer UI-Struktur 2}
// onj schema
var uiElement = {
    type: string,
    ...*
};
uiElements: uiElement[]
\end{codeBlock}

Im Schema kann nur garantiert werden, das ein UI-Element einen \inlineOnj{type}-Key hat, alles andere wird offen
gelassen.
Das macht die Schema-Validierung in diesem Fall praktisch nutzlos.
Die Lösung, die Onj für dieses Problem bietet, sind Named Objects.
Im Schema kann eine Named Object Group definiert werden, die mehrere Named Objects enthält.
Jedes Named Object hat einen Namen, über den es identifiziert wird und kann beliebige Keys mit beliebigen Typen
definieren.
Im Schema kann dann der Name der Named Object Group als Typ verwendet werden, der alle Named Objects, die Teil der Group
sind, erlaubt.

Hier noch einmal das selbe Beispiel, implementiert mit Named Objects:

%! language = Onj
\begin{codeBlock}{onj}{Beispiel: Gute Umsetzung einer UI-Struktur 1}
// onj struktur
uiElements: [
    $Label {
        text: "Hello World",
        font: "red wing",
    },
    $Image {
        path: "./some/image.png"
    }
]
\end{codeBlock}

\begin{codeBlock}{onj}{Beispiel: Gute Umsetzung einer UI-Struktur 2}
// onj schema
$UiElement {
    $Label {
        text: string,
        font: string
    }
    $Image {
        path: string
    }
}
uiElements: $UiElement[]
\end{codeBlock}

% resets author
\renewcommand{\kapitelautor}{}


\subsection{Namespaces}\label{subsec:namespaces}

\renewcommand{\kapitelautor}{Autor: Marvin Kurka}

Namespaces erlauben es, Onj zu erweitern, indem eigene Funktionen, zusätzliche globale Variablen oder
eigene Datentypen definiert werden.
Solche Namespaces werden in Kotlin definiert und mit Annotations markiert.

Funktionen werden im Namespace definiert und mit der \inlineKotlin{@RegisterOnjFunction} markiert.
Diese Annotation nimmt ein String mit einem OnjSchema als Parameter, der die Signatur der Funktion beschreibt.
Das ist notwendig, da Onj das Überladen von Funktionen erlaubt, \dah mehrere Funktionen dürfen denselben Namen haben.
Um herauszufinden, welche Funktion tatsächlich aufgerufen werden soll, vergleicht der Onj-Parser die mitgegebenen
Parameter mit dem Schema und ruft die erste Funktion auf, bei der diese übereinstimmen.

%! language = Kotlin
\begin{codeBlock}{kotlin}{Demo: Deklaration einer Onj-Funktion in Kotlin}
@OnjNamespace
object MyNamespace {

    @RegisterOnjFunction(schema = "params: [string, int]")
    fun repeatString(s: OnjString, times: OnjInt) = OnjString(s.value.repeat(times.value))
}
\end{codeBlock}

Neben normalen Funktionen stellt Onj noch drei spezielle Arten von Funktionen zur Verfügung:

\begin{itemize}
    \item Conversions: Diese werden mit folgendem Syntax aufgerufen: \inlineOnj{value#function}.
        Solche Funktion werden zum Beispiel verwendet, um einen Wert von einem Datentypen zu einem anderen zu
        konvertieren.
    \item Infix Funktionen: Diese werden mit folgendem Syntax aufgerufen: \inlineOnj{value1 function value2}.
        Beispiele sind die pow-Funktion (\inlineOnj{10 pow 5}) oder die in-Funktion (\inlineOnj{3 in [1, 2, 3, 4]}).
    \item Operator Overloading: Solche Funktionen erlauben es zu definieren, wie sich Operatoren wie
        \inlineOnj{+} oder \inlineOnj{*} für eigene Datentypen verhalten.
\end{itemize}

Wird so eine spezielle Funktion verwendet, wird das in der \inlineKotlin{@RegisterOnjFunction} Annotation angegeben.

%! language = Kotlin
\begin{codeBlock}{kotlin}{Demo: Deklaration einer Onj-Conversion in Kotlin}
@OnjNamespace
object MyNamespace {

    @RegisterOnjFunction(schema = "params: [string]", type = OnjFunctionType.CONVERSION)
    fun greeting(name: OnjString) = OnjString("Hello, ${name.value}")
}
\end{codeBlock}

Weiters können Namespaces globale Variablen definieren.

%! language = Kotlin
\begin{codeBlock}{kotlin}{Demo: Deklaration von globalen Onj-Variablen in Kotlin}
@OnjNamespace
object MyNamespace {

    @OnjNamespaceVariables
    val variables: Map<String, OnjObject> = mapOf(
        "myGlobal" to OnjInt(5)
    )
}
\end{codeBlock}

In Onj kann ein use-Statement verwendet werden, um einen Namespace zu inkludieren.

%! language = Onj
\begin{codeBlock}{onj}{Demo: Verwendung eines Onj-Namespaces}
use MyNamespace;

global: myGlobal,
greeting: "Reader"#greeting,
string: repeatString("a", 5)
\end{codeBlock}

% resets author
\renewcommand{\kapitelautor}{}



\renewcommand{\kapitelautor}{Autor: Felix Zwickelstorfer}
\section{Gradle}\label{sec:gradle}

\renewcommand{\kapitelautor}{Autor: Felix Zwickelstorfer}

Gradle ist ein Build-Tool, welches benötigte Abhängigkeiten und Code-Bibliotheken automatisch herunterlädt.
Mit Gradle wird auch das Kompilieren von Projekten automatisiert.
Gradle verwendet die JVM (Java Virtual Maschine), wodurch es besonders kompatibel mit Programmiersprachen ist, die auch auf der JVM laufen.
Da Kotlin eine davon ist, ist Gradle ideal für Forty-Five.
Weiteres ist die Einbindung diverser IDEs mit Gradle gegeben, was die Verwendung vereinfacht.
Bei Gradle gibt es zwei wichtige Konfigurationspunkte für Forty-Five: Tasks und dependencies.

\textbf{Tasks}

Tasks sind ausführbare Funktionen in Gradle, welche meistens das eigentliche Programm starten oder builden.
Meistens schreibt man diese nicht selbst, sondern sie werden von einer Bibliothek mitgegeben.
Bei Forty-Five war dies hauptsächlich gdx.

\textbf{Dependencies}

Dependencies sind Verweise auf andere Codeblöcke, die Gradle benötigt zum Ausführen von Tasks.
Dabei gibt es zwei verschiedene Bereiche.
Es gibt Abhängigkeiten, die für Gradle sind, und die für das "fertige Programm".
Diese unterscheiden sich meistens dadurch, dass Gradle auch dependencies für das Testen hat, welche man bei einem build nicht benötigt.


%\begin{codeBlock}{gradle}{Beispiel: buildscript in Gradle}
%    buildscript {
%        ext.kotlinVersion = '1.7.0'
%        ext.gdxVersion = '1.11.0'
%    repositories {
%        mavenLocal()
%        // other repositories
%        jcenter()
%    }
%    dependencies {
%        classpath "org.jetbrains.kotlin:kotlin-gradle-plugin:$kotlinVersion"
%        classpath "com.badlogicgames.gdx:gdx-tools:$gdxVersion"
%        classpath 'edu.sc.seis.launch4j:launch4j:2.5.4'
%    }
%}
%\end{codeBlock}

\section{Onj}\label{sec:onj}


\subsection{Warum Libgdx?}\label{subsec:warum-libgdx}

\renewcommand{\kapitelautor}{Autor: Irgendwer} % todo: replace

%
% text goes here
%

% resets author
\renewcommand{\kapitelautor}{}


\subsection{OpenGL}\label{subsec:opengl}

\renewcommand{\kapitelautor}{Autor: Marvin Kurka}

\subsubsection{Renderpipeline}

OpenGL definiert eine Abfolge an Schritten, aus denen der Rendering-Prozess besteht.
Diese Abfolge wird als die OpenGL Renderpipeline bezeichnet.
Die OpenGL Renderpipeline ist programmierbar: Das heist es können Shaderprogramme in GLSL (der OpenGL Shading Language)
geschrieben werden, die beschreiben, wie sich bestimme Schritte der Pipeline verhalten.
Der Schritt im Rendering-Vorgang ist der Vertex Shader, der für jeden Vertex, den das Program spezifiziert hat,
ausgeführt wird.
Der Vertex Shader kann dann den jeweiligen Vertex verändern.
Das erlaubt die Implementation von Transformationen, wie Translation, Skalierung oder Rotation.
Weiters, wenn in einer 3D-Umgebung gearbeitet wird, findet hier die Projektion in den Screen-Space statt.
Optional können Tessellation und Geometry Shader definiert werden, welche OpenGL Primitives in kleinere Primitives
unterteilen oder zusätzliche Geometrie (= zusätzliche Vertexe) generieren.
Danach durchlaufen die Vertexe den Primitive Assembly Schritt, in dem aus den Vertexen Primitives wie Dreiecke,
Rechtecke oder Linien gebaut werden.
Diese Primitives werden anschließend gerastert, was heißt, dass für jeden Pixel, den das Primitive überlappt, ein
Fragment generiert wird.
Welche Farbe für jedes Fragment dann tatsächlich gerendert wird, wird vom Fragment Shader entschieden, welcher für jedes
Fragment ausgeführt wird.
Der Fragment Shader ist theoretisch optional, wird er jedoch weggelassen, sind die ausgegebenen Farben nicht definiert.\cite{openglRenderPipeline}

\subsubsection{Shader}

Shader erlauben es die OpenGL Renderpipeline zu skripten und werden in der OpenGL Shading Language, kurz GLSL
geschrieben.
Sowohl der Vertex Shader, als auch der Tessellation, Geometry und Fragment Shader sind Beispiele für den Einsatz von
Shadern in der oben beschriebenen Renderpipeline.
Eine Ausnahme stellen Compute Shader dar.
Diese sind nicht Teil der Renderpipeline und werden verwendet, um beliebige Operationen auf der GPU durchzuführen.
Häufig werden diese bei Aufgaben eingesetzt, die sich besonders gut parallelisieren, wie das Trainieren von KIs oder
das Berechnen von Hashes \zB beim Mining von Kryptowährung.
Da in \FF nur Vertex und Fragment Shader eingesetzt wurden, werden auch nur diese nun näher erläutert.\cite{openglRenderPipeline}

\subsubsection{Der Vertex Shader}

Der Vertex Shader wird für jeden Vertex eines Meshes ausgeführt, wobei openGL Rückgabewerte cashed und so die mehrfache
Ausführung eines Vertex Shaders für gleiche Vertexe verhindern kann.
Der Vertex Shader transformiert den Vertex, üblicherweise mittels einer Projektionsmatrize, die vom Programm übergeben
wird.
Unten ist die der Standard Vertex Shader von LibGdx zu sehen.

\begin{codeBlock}{glsl}{Standard Vertex Shader von LibGdx}
attribute vec4 a_position;
attribute vec4 a_color;
attribute vec2 a_texCoord0;
uniform mat4 u_projTrans;
varying vec4 v_color;
varying vec2 v_texCoords;

void main() {
    v_color = a_color;
    v_color.a = v_color.a * (255.0/254.0);
    v_texCoords = a_texCoord0;
    gl_Position = u_projTrans * a_position;
}
\end{codeBlock}

Die mit \inlineGlsl{attribute} markierten Variablen sind Werte, die vom Programm mitgegeben werden und für jede
Vertex Shader Instanz verschieden sind.
Das inkludiert \zB die Koordinaten des Vertex.
Mit \inlineGlsl{uniform} markierte Variablen werden ebenfalls vom Programm mitgegeben, sind aber für alle Shader Instanzen,
sowohl Vertex als auch Fragment Shader, gleich.
Mit \inlineGlsl{varying} markierte Variablen werden vom Vertex Shader berechnet und an den Fragment Shader weitergegeben.

Neben der vorher angesprochenen Transformation setzt der Vertex Shader auch einige Werte, die später vom Fragment
Shader verwendet werden können.
Dies inkludiert die Textur-Koordinate, die angibt, wo in der Textur der Fragment Shader samplen soll, und die Vertex
Farbe, die es erlaubt, die Farbe der Textur anzupassen.

Nun ist es aber so, das die Zahl der Fragment Shader Instanzen sich stark von der Zahl der Vertex Shader Instanzen
unterscheiden kann, da der Fragment Shader für jeden Pixel, den ein Primitive abdeckt, ausgeführt wird.
Deswegen stellt sich die Frage, wie die Ausgabewerte des Vertex Shader auf die Fragment Shader verteilt werden.
Wird im Shader nichts anderes spezifiziert, werden alle mit \inlineGlsl{varying} markierten Variablen zwischen den
Vertexen, aus denen das Primitive besteht, linear interpoliert.\cite{openglVertexShader, openglTypeQualifiers}

\begin{figure}[H]
    \centering
    \includegraphics[width=1.0\textwidth]{triangle_rendering.png}
    \caption{Beispiel für die Interpolation der varyings}
\end{figure}

Oben ist ein Beispiel für diese Interpolation zu sehen.
Jeder Vertex des Dreiecks definiert eine Farbe, die über eine \inlineGlsl{varying} Variable an den Fragment Shader
weitergegeben wird.
Dieser rendert einfach die Farbe die er erhalten hat, ohne sie zu ändern.
Die Farben der Vertexe sind (im Uhrzeigersinn, oben links beginnend): grün, rot, blau.
Wie in der Grafik zu sehen ist, werden die Farben zwischen den Vertexen interpoliert, wodurch Pixel, die nicht genau
an einem Vertex liegen, eine Mischfarbe haben.

\subsubsection{Der Fragment Shader}

Der Fragment Shader wird für jeden Pixel, den ein Primitiv überlappt ausgeführt und bestimmt, welche Farbe final
gerendert wird.
Meistens passiert das, indem der Fragment Shader eine Textur an der durch den Vertex Shader bestimmten Textur
Koordinate sampled.
Unten ist der Standard Fragment Shader von LibGdx zu sehen.

\begin{codeBlock}{glsl}{Standard Fragment Shader von LibGdx (gekürzt)}
varying vec4 v_color;
varying vec2 v_texCoords;
uniform sampler2D u_texture;

void main() {
    gl_FragColor = v_color * texture2D(u_texture, v_texCoords);
}
\end{codeBlock}

Zusätzlich wird der Wert aus der Textur mit der im Vertex Shader gesetzten Vertex Farbe multipliziert.\cite{openglFragmentShader}

\subsection{Anwendungen von Shadern}

In \FF wurden verschiedene Mechanismen für die Umsetzung von Animationen verwendet.
Einige wurden Frame-by-Frame animiert, bei anderen werden Attribute von Widgets über Zeit verändert.
Shader allerdings haben sich als ein extrem mächtiges Werkzeug zu Erstellung von Animationen herausgestellt, das es
ermöglicht komplexe grafische Effekte umzusetzen, die auch noch live auf das Spiel reagieren können.

\subsubsection{Rendering von Postprocessing Effekten}\label{subsubsec:postprocessing-effects}

Bei vielen Effekten, die mit Shadern erzielt werden, handelt es sich um Postprocessing-Effekte.
Das heißt, dass zuerst das gesamte Spiel normal gerendert wird, und das Resultat dann mittels eines Shaders
manipuliert wird.
Um das möglich zu machen, werden Framebuffer, auch Framebufferobject oder FBO genannt, eingesetzt.
Statt direkt auf den Bildschirm zu rendern, kann stattdessen zu einem Framebuffer gerendert werden, der dann das
Resultat speichert.\cite{openGlFbo}

Wenn also ein Postprocessing Effekt aktiv ist, wird das Spiel zuerst zu einem Framebuffer gerendert, und dieser dann
mit dem Postprocessing Shader zu dem Bildschirm.
Falls mehrere Postprocessing Effekte aktiv sind, ist ein Framebuffer nicht mehr ausreichend, da einen Framebuffer
zu sich selber zu rendern undefiniertes Verhalten auslöst.\cite{openGlFbo}
Stattdessen wird eine Technik names Ping-Pong-Rendering verwendet.\cite{pingPongRendering}
Bei dieser werden zwei Framebuffer verwendet, zwischen denen hin und her gerendert wird, wobei bei jedem Render-Pass ein
Postprocessing Effekt angewendet wird.
Bei dem letzten Effekt wird, statt zu einem Framebuffer, zu dem Bildschirm gerendert.

\subsubsection{Der Screenshake Postprocessor}

Der Screenshake-Postprocessor ist der einzige, der im Vertex Shader und nicht im Fragment Shader implementiert ist.
Er wendet eine Sinus-Funktion an, um jeden Vertex zu verschieben, was ein verzerrt aussehendes Bild bewirkt.
Schnelles Ein- und Ausschalten dieses Shaders bewirkt den Screenshake Effekt beim Schießen des Revolvers.

\subsubsection{Die Destroy-Animation}

Wenn eine Karte zerstört wird, wird eine Animation abgespielt, bei der die Karte ``zerfressen`` wird.
Dieser Effekt ist kein Postprocessing-Effekt, da er nur eine Karte und nicht den ganzen Bildschirm betrifft.

\begin{figure}[H]
    \centering
    \includegraphics[width=0.6\textwidth]{card_destroy.png}
    \caption{Bild der Animation, wenn eine Karte zerstört wird}
\end{figure}

Um diesen Effekt zu erzeugen, wird dem Shader nicht nur die Kartentextur, sondern auch eine Noise-Textur mitgegeben.
Die Verwendung einer Noise-Textur statt im Shader generierten Noise hat den Vorteil, das sie deutlich performanter ist,
da die Noise-Werte im schon im Vorhinein berechnet sind.
Sonst müssten diese beim Rendern, im Shader, für jeden Pixel, berechnet werden.
Vorgefertigte Texturen haben allerdings den Nachteil, dass die Auflösung limitiert ist, und das die Noise-Parameter
zu Runtime nicht geändert werden können.

\begin{figure}[H]
    \centering
    \includegraphics[width=0.3\textwidth]{perlin.png}
    \caption{Die verwendete Noise-Textur}
\end{figure}

Der Shader definiert zwei Thresholds, die mit der Noise-Textur abgeglichen werden.
Ab dem ersten Threshold wird statt der Kartentextur schwarz gerendert, ab dem zweiten transparent.
Ein stetiges Erhöhen der Thresholds erzeugt den Destroy-Effekt.

\subsubsection{Die Orb-Animation}

Die Orb-Animationen werden im Spiel verwendet, um zu signalisieren, wenn sich abstrakte Konzepte, die keine wirkliche
In-Game-Repräsentation hat, von einem Ort zu einem anderen bewegen.
Ein Beispiel dafür sind Reserves im Kampf oder das Cash am Ende des Kampfes.

\begin{figure}[H]
    \centering
    \includegraphics[width=1.0\textwidth]{orb_animation.png}
    \caption{Die Orb-Animation, wenn Reserves gezahlt wurden}
\end{figure}

Bei der Orb-Animation wird eine beliebige Texture von einem Ort zu einem anderen animiert, wobei sie dabei einen Schweif
hinterlässt.

Die Orb-Animation verwendet zwei Framebuffer, mit einer Ping-Pong-Strategie, wie im
Kapitel~\ref{subsubsec:postprocessing-effects} beschrieben.
Zusätzlich werden diese Framebuffer verwendet, um den Zustand des Schweifs über Frames hinweg zu speichern.
Das Rendern dieser Animation findet in zwei Stufen statt.
Zuerst wird der Framebuffer der Orb-Animation geupdated, danach wird die Animation auf den Bildschirm gerendert.
Der erste Schritt läuft folgendermaßen ab:

Zwei Framebuffer, hier A und B genannt, werden erschaffen, falls sie noch nicht existieren.
Beide Framebuffer haben nur die halbe Auflösung des Bildschirms, da das Renderzeit spart und durch den Blur-Effekt, der
im zweiten Schritt angewandt wird, nicht auffällt.
Zu Beginn enthält B den aktuellen Zustand des Schweifs.
Framebuffer B wird zu A gerendert, wobei ein spezieller Shader verwendet wird, der den Alpha-Kanal um einen festgelegten
Wert reduziert.
Bei dem Rendern des Framebuffers muss darauf geachtet werden, dass die OpenGL Blending Funktion richtig gesetzt ist,
um zu verhindern, dass der Inhalt der beiden Buffer anhand des Alpha-Werts geblendet wird.
Stattdessen soll der Farb-Wert des Shader-Outputs ohne Blending einfach übernommen werden.
Der Wert, um den der Alpha Kanal reduziert wird, wird mit der aktuellen Frametime multipliziert, um eventuelle
Lagspikes auszugleichen.
Diese Alpha-Reduktion wird durchgeführt, um dafür zu sorgen, dass der Schweif langsam ausbleicht.

Danach wird die Textur der Animation an der aktuellen Stelle zu Framebuffer A gerendert, um den Schweif um die passierte
Bewegung zu erweitern.
Optional kann die Textur auch mehrmals zwischen der letzten und der aktuellen Position gezeichnet werden, was vor allem
bei schnellen Animationen helfen kann, sichtbare Löcher im Schweif zu vermeiden.
Zum Schluss des ersten Schrittes werden die Framebuffer getauscht, was heißt, dass A zu B und B zu A wird.

Das Resultat dieses Schrittes ist der Framebuffer B, der in Abständen die Textur der Animation auf transparentem
Hintergrund enthält, wobei ältere Pixel einen niedrigeren Alpha-Wert haben.

\begin{figure}[H]
    \centering
    \includegraphics[width=1.0\textwidth]{orb_animation_fbo_B_low_segments.png}
    \caption{Inhalt des Framebuffer B, auf den Bildschirm gerendert, wobei eine Textur pro Frame gezeichnet wird }
\end{figure}

\begin{figure}[H]
    \centering
    \includegraphics[width=1.0\textwidth]{orb_animation_fbo_B_high_segments.png}
    \caption{Inhalt des Framebuffer B, auf den Bildschirm gerendert, wobei 20 Texturen pro Frame gezeichnet werden }
\end{figure}

Im zweitem Schritt wird die Animation tatsächlich zum Bildschirm gerendert.
Dabei wird zuerst der Schweif aus Framebuffer B gerendert, und dann die Textur der Animation an der aktuellen Position.
Zweiteres ist trivial, der erste Teil aber nicht.
Bevor der Schweif gerendert wird, muss zuerst ein Blur-Effekt angewendet werden, damit die Texturen die zu Framebuffer
B gerendert wurden miteinander und mit dem Hintergrund verschwimmen.

Für die Orb-Animation wird ein 2-Pass Gaussian Blur verwendet, da dieser oft bessere Ergebnisse liefert als \zB ein
Box-Blur.\cite{gaussianBlur}
Allerdings ist ein Gaussian-Bur auch teurer zu berechnen, was aber unter anderem durch die halbierte Auflösung keine
Probleme macht.
Ein Blur-Effekt funktioniert, in dem für jeden Pixel die Textur nicht nur an der Stelle des Pixels gesampled wird,
sondern auch an umliegenden Stellen.
Die finale Farbe des Pixels ist dann der Schnitt aus all diesen Samples.
Was den Gaussian Blur von \ua einem Box-Blur unterscheidet ist, das die Samples anhand einer gaußschen Verteilung
gewichtet werden, was dazu führt, dass Pixel, die näher an der Mitte liegen, einen größeren Einfluss auf das Ergebnis
haben.\cite{gaussianBlur}
Diese Gewichte sind in einem Look-up-Table definiert, müssen also nicht jedes Mal berechnet werden.

Eine besondere Herausforderung bei dem Gaussian-Blur Shader war der Umgang mit Transparenzen.
Ein transparenter Pixel hat, auch wenn er nicht dargestellt wird, eine Farbe (Im Falle der Orb-Animation ist das Schwarz).
Wenn nun ein Pixel am Rand der Textur ist, wird auch im transparenten Bereich gesampled,
wo der Shader Schwarz zurückbekommt.
Das führt zu einer deutlichen Verdunkelung der Pixel am Rand des Schweifs.
Um dieses Problem zu lösen, wird das gaußsche Gewicht zuerst mit dem Alpha-Wert multipliziert, sodass transparente Pixel
keine/weniger Einfluss auf das Resultat haben.
Die tatsächlich verwendeten Gewichte werden gemeinsam mit den Samples aufsummiert und dann dividiert, um den Schnitt zu
erhalten.

Die Orb-Animation verwendet eine 2-Pass Version des Gaussian Blur, bei der zuerst horizontal, dann vertikal geblurred
wird.
Das heist aber auch, dass für den ersten Pass wieder zu einem Framebuffer gerendert werden muss.
Hier kann aber einfach Framebuffer A verwendet werden, da dieser zu diesem Zeitpunkt nicht in Verwendung ist.
Zum Schluss des zweiten Schritts wird die Textur der Animation ohne Blur direkt auf den Bildschirm gezeichnet.

% resets author
\renewcommand{\kapitelautor}{}


\subsection{Pixmaps}\label{subsec:pixmaps}

\renewcommand{\kapitelautor}{Autor: Irgendwer} % todo: replace

%
% text goes here
%

% resets author
\renewcommand{\kapitelautor}{}


\subsection{Scene2D}\label{subsec:scene2D}

\renewcommand{\kapitelautor}{Autor: Irgendwer} % todo: replace

%
% text goes here
%

% resets author
\renewcommand{\kapitelautor}{}


\subsection{Memory Management}\label{subsec:memory-management}

\renewcommand{\kapitelautor}{Autor: Irgendwer} % todo: replace

%
% text goes here
%

% resets author
\renewcommand{\kapitelautor}{}



%
\chapter{Programmaufbau}\label{ch:programmaufbau}


\section{Screens}\label{sec:screens}

\renewcommand{\kapitelautor}{Autor: Irgendwer} % todo: replace

%
% text goes here
%

% resets author
\renewcommand{\kapitelautor}{}


\section{Advanced Text}\label{sec:advanced-text}

\renewcommand{\kapitelautor}{Autor: Irgendwer} % todo: replace

%
% text goes here
%

% resets author
\renewcommand{\kapitelautor}{}

%
\section{Kampfablauf}\label{sec:kampfablauf}

\renewcommand{\kapitelautor}{Autor: Irgendwer} % todo: replace

%
% text goes here
%

% resets author
\renewcommand{\kapitelautor}{}


\section{Ressourcen- und Speicher-Management}\label{sec:resourcenmanagement}

\renewcommand{\kapitelautor}{Autor: Marvin Kurka}

Kotlin, wie auch Java und andere JVM-basierende Sprachen verwendet einen Garbage Collector, kurz GC, um den
Speicherplatz von nicht mehr verwendetet Objekten freizugeben.\cite{oracleGC}

\subsection{Wie funktioniert Garbage Collection?}

Garbage Collection ist in zwei Phasen aufgeteilt.
In der ersten Phase, dem Marking, werden alle Objekte, für die noch eine Referenz existiert, markiert.
Dabei geht der GC von sogenannten Roots aus, was Objekte sind, auf die das Programm eine unmittelbare
Referenz hat.
Diese werden markiert und der Prozess wird rekursiv für alle Referenzen, die diese Objekte halten, wiederholt.
Am Ende sind alle Objekte, für die das Programm noch eine gültige Referenz hält markiert.
In der zweiten Phase werden alle Objekte, die nicht markiert wurden, und demnach nicht mehr erreichbar sind, gelöscht.
Zusätzlich kann hier eine Defragmentierung stattfinden.\cite{oracleGC}

Eine GC durchzuführen ist teuer.
Erst muss jedes Objekt in der Baumstruktur des Heaps durchgegangen und markiert werden, und dann beim
Löschen/Defragmentieren müssen große Teile des Heaps kopiert werden.
Dazu kommt, dass es sich bei Garbage Collections um "Stop the World Events" handelt.
Da große Teile des GC-Prozesses nicht parallel mit den Application Threads laufen können, muss das
Programm für die Dauer der Garbage Collection pausiert werden, was sowohl die Geschwindigkeit als auch die
Responsiveness des Programms beeinträchtigt.\cite{oracleGC}

Um dieses Problem wenigstens zu mindern, verwendet die JVM eine generational Garbage Collection.
Die meisten Objekte, die in der Laufzeit eines Programms allokiert werden, haben nur eine sehr geringe Lebenszeit,
während einige wenige Objekte signifikant längere Lebenszeiten haben.
Die JVM macht sich das zu Nutze, um den Heap in verschiedene Generation einzuteilen, je nachdem wie viele GCs
ein Objekt schon überlebt hat.
Bei einem Minor Garbage Collection Event werden dann nur die jüngeren Generationen durchsucht, während bei einem
Major Garbage Collection Event der gesamte heap durchsucht wird.\cite{oracleGC}

\subsection{Gargabe Collection in Videospielen}

GC funktioniert am besten mit kleinen Objekten mit kurzer Lebensdauer.
Diese überleben im Idealfall schon die erste minor GC nicht, und müssen deshalb nicht in den Bereich der older
Generation überführt werden.
Außerdem ist das allokieren/kopieren/löschen durch die kleine Größe wenig Aufwand.
Weiters sollten große Objekte im Idealfall nicht im stable State des Programmes allokiert werden, um die Zahl an major
GCs gering zu halten.\cite{infoqJavaPerformance}

Allerdings ist es gerade in Videospielen so, dass es besonders viele sehr große und langlebige Objekte gibt.
Texturen, Animationen und Sounds können oft Megabytes an RAM benötigen und werden oft im stable state des Programms
allokiert, \zB bei einer Screen-Transition.
In Videospielen spielt oft auch die Latenz, also wie schnell auf User-Input reagiert wird, eine wichtige Rolle.
Diese kann durch GC beeinträchtigt werden, da diese die Applikation Thread pausiert und diese dann nicht sofort auf
den Input reagieren könne.

Außerdem werden oft Ressourcen verwendet, die gar nicht von einer konventionellen GC verwaltet werden können, da sie zur
GPU hochgeladen werden müssen.
Das inkludiert \zB Texturen oder Shader.
Solche Ressourcen könnten theoretisch über einen Finalizer, der automatisch läuft, bevor die GC ein Objekt collected,
freigegeben werden, das würde die Performance des GC-Prozesses allerdings weiter verschlechtern.\cite{infoqJavaPerformance}

Ein weiterer Grund, warum man sich bei großen Objekten nicht auf den GC verlassen sollte, ist, dass oft eine feine
Kontrolle über große Ressourcen notwendig ist.
Sollte irgendwo im Programm noch eine einzelne Referenz zu einem Objekt existieren, wird dieses nicht vom GC
eingesammelt, auch wenn es vielleicht nicht mehr verwendet wird.
So können sehr einfach Memory Leaks entstehen, die auch noch sehr schwer zu debuggen sind.
All das macht GC besonders ungeeignet für Videospiele und ist er Grund, warum LibGdx einen anderen Ansatz für das
Managen von großen Ressourcen verfolgt.

\subsection{LibGdx und das Disposable-Interface}

In LibGdx werden große Ressourcen durch nativen Code allokiert und nicht vom GC verwaltet.
Klassen, die solche Ressourcen repräsentieren, implementieren das Disposable-Interface, das die
\inlineKotlin{Disposable::dispose} Funktion zur Verfügung stellt.
Diese muss am Ende des Lebenszyklus des Objektes aufgerufen werden, um den allokierten Speicher wieder freizugeben.\cite{libGdxMemoryManagement}

Diese Herangehensweise macht den effizienten Umgang mit Ressourcen möglich und gibt dem Programmierer Kontrolle über
die Lebenszeit von großen Objekten.
Allerdings führt das auch dazu, dass viele Fehlerquellen, die durch GC eliminiert wurden, wieder auftreten können.
Beispiele sind Memory Leaks, die auftreten, wenn eine Ressource nicht wieder freigegeben wird, oder eine
Use-after-Free-Situation, bei der eine Ressource verwendet wird, obwohl sie eigentlich schon freigegeben wurde, was
zu undefinierten Verhalten führt.\cite{libGdxMemoryManagement}

\subsection{Der ResourceManager}

Gutes Ressourcenmanagement ist in der Spielentwicklung extrem wichtig.
Schlechtes Management kann \zB zu langen Wartezeiten bei Screen-Transitions führen, und so die Immersion brechen.
Aber falscher Umgang mit Ressourcen kann nicht nur das Spielerlebnis verschlechtern, er kann das Spiel sogar
komplett unspielbar machen.
Die im vorherigen Absatz beschriebenen Bugs wie Use-after-Free oder Memory Leaks können zu Crashes oder anderen
Fehlverhalten des Spiels führen.
Deswegen war das Ressourcenmanagement bei der Entwicklung von \FF eine besonders große Herausforderung.
Um diese zu bewältigen wurde der ResourceManager entwickelt, eine Klasse, die Ressourcen über die gesamte
Applikation hinweg verwaltet.

\subsubsection{Grundfunktion des ResourceManager}

Klassen, die Ressourcen vom ResourceManager verwenden wollen, müssen das ResourceBorrower-Interface implementieren.
Dieses enthält keine Funktionen, es dient nur dem Zweck, dem Programmierer zu erinnern, dass diese Klasse Ressourcen
verwendet und diese auch zurückgeben muss.

Eine Klasse, die ResourceBorrower implementiert kann die \inlineKotlin{ResourceManager::borrow} Funktion verwenden,
um sich eine Ressource auszuborgen.
Die Ressource muss zu diesem Zeitpunkt nicht unbedingt geladen sein, sie wird nur als in Verwendung markiert.
Um die Ressource tatsächlich zu bekommen, wird die \inlineKotlin{ResourceManager::get} Funktion verwendet.
Spätestens zu diesem Zeitpunkt wird das Laden erzwungen.
Um die Ressource wieder zurückzugeben, wird die \inlineKotlin{ResourceManager::giveBack} Funktion verwendet.
Auch hier gilt wieder, dass ein Aufruf dieser Funktion nicht unbedingt zum Entladen der Ressource führen muss,
da auch ein zweiter Borrower sie noch verwenden könnte.

Diese zentrale Verwaltung ermöglicht effizientes Management der Ressourcen, da der ResourceManager zu jedem Zeitpunkt
über die Gesamtsituation im Programm Bescheid weiß.
Das ermöglicht es, dass doppelte Laden von Ressourcen oder das Entladen und sofortige neu Laden zu verhindern.

\subsubsection{Das asynchrone Laden von Ressourcen}

Das Laden von Ressourcen ist eine sehr teure Operation, da oft MB-große Dateien zuerst in dem RAM gelesen und
dann zur GPU hochgeladen werden müssen.
Würde das alles synchron mit der Render-Logik auf dem Main-Thread passieren, würde das zu merkbaren Verzögerungen
führen.
Daher war klar, dass das Laden von Ressourcen so weit wie möglich auf einem separaten Thread passieren sollte.
Allerdings war es in der Praxis nicht umsetzbar, alle Ressourcen vollständig von einem anderen Thread aus zu laden.
Grund dafür ist der OpenGL Kontext.
Nur ein Thread kann den OpenGL Kontext halten, und dieser Thread ist im Falle von LibGdx immer der Main-Thread.
Nur dieser kann dann Operation durchführen, die mit OpenGL zu tun haben, wie \zB Grafiken zu rendern, aber auch nur
dieser Thread kann Daten zur GPU hochladen.
Ein Aufruf von \zB dem \inlineKotlin{Texture} Konstruktor von einem Thread ohne OpenGL-Kontext würde zu nicht
definierten Verhalten führen.\cite{libGdxThreading}

Um dieses Problem so gut wie möglich zu umgehen, wurde für jeden möglichen Ressourcen-Typ eine Zwei-Schritt Lade Logik
definiert.
Der erste Schritt kann sicher auf jedem Thread ausgeführt werden, und lädt \zB in Falle einer Textur die Daten als
Pixmap in den RAM\@.
Im zweiten Schritt, der nur auf dem Main-Thread ausgeführt werden darf, passiert der tatsächliche Aufruf des
\inlineKotlin{Texture} Konstruktors und damit der Upload zur GPU\@.
Da auch der zweite Schritt Zeit in Anspruch nimmt, kann es hier zu Verzögerungen beim Rendering kommen, das hat sich
aber leider nicht vermeiden lassen.

\subsubsection{Der ServiceThread}

Der ServiceThread ist ein Thread der zu Beginn des Programms gestartet wird und dann parallel zu diesem läuft.
Er übernimmt mehrere Aufgaben, \zB das Zeichnen der Karten-Texturen, bei weitem die Größte ist aber das Laden von
Ressourcen.
Der ServiceThread kommuniziert mit dem Main-Thread über einen \inlineKotlin{Channel<ServiceThreadMessage>}, über
den der Main-Thread Nachrichten an den ServiceThread schicken kann.
Nach einem Event, bei dem viele Ressourcen ausgeborgt wurden (\zB einer Screen-Transition) wird die
\inlineKotlin{ServiceThreadMessage.PrepareResources} Nachricht and den ServiceThread geschickt.
Dieser startet dann eine Coroutine für jede ausgeborgte, aber nicht geladene Ressource, die diese vorbereitet.
Diese Coroutines werden von Kotlins IO-Dispatcher verwaltet, der diese auf mehrere Worker-Threads verteilt ausführt.

Im Main-Thread wird währenddessen das Laden der Ressource so lange wie möglich herausgezögert, um den ServiceThread
eine möglichst hohe Chance zu geben, die Ressource parallel zu laden.
Ein Image \zB lädt seine dazugehörige Textur erst beim ersten \inlineKotlin{draw()} Aufruf.
Wenn nun ein Zugriff stattfindet, tritt eine dieser Möglichkeiten ein:

\begin{itemize}
    \item Die Ressource war bereits geladen: Der best Case, die Ressource war beim \inlineKotlin{borrow} Aufruf bereits
        in Verwendung und muss nicht geladen werden.
    \item Die Ressource ist vorbereitet: Der ServiceThread hat die notwendigen Daten bereits geladen, nur noch der
        zweite Schritt des Ladevorgangs muss ausgeführt werden.
    \item Die Ressource ist gerade in Vorbereitung: Der Main-Thread muss auf den ServiceThread warten und führt dann
        den zweiten Schritt des Ladevorgangs aus.
        Der Main-Thread blockiert in diesem Zeitraum.
    \item Die Ressource ist nicht geladen: Der worst Case, der komplette Ladevorgang muss auf dem Main-Thread ausgeführt
        werden.
        Der Main-Thread blockiert in diesem Zeitraum.
\end{itemize}

\subsubsection{Das Laden von FrameAnimations}

Aufgrund ihrer besonders hohen Größe haben FrameAnimations ihre eigene Lade-Logik.
So kann verhindert werden, dass Animationen jemals auf dem Main-Thread geladen werden, was zu extremen Lag führen würde.

Zuerst werden die Frames der Animationen in einen Atlas verpackt.
Diese Atlase sind Texturen mit Größen von maximal 4096x4096 Pixeln, die mehrere Frames der Animationen enthalten.
Das minimiert die Anzahl der Texturen, die zur GPU hochgeladen werden müssen.
Das Verpacken der Atlase findet nicht zu Runtime statt, sondern zu Compile-Time, indem ein Gradle-Task ausgeführt wird.

\begin{figure}[H]
    \includegraphics[scale=0.5]{example_animation_atlas.png}
    \caption{Beispiel für Animations-Atlas in 20\% der Original-Auflösung}
\end{figure}

Die \inlineKotlin{DeferredFrameAnimation} Klasse implementiert die Lade-Logik für Animationen.
Beim Erstellen einer Instanz dieser Klasse wird eine Nachricht an der ServiceThread geschickt, die diesem befiehlt,
den Atlas mit den Frames er Animation zu laden.
Der ServiceThread startet eine Coroutine auf einem gesonderten Thread nur für Animation, um zu verhindern, dass das
Laden der Animation das Laden anderer Ressourcen, die zeitkritisch sein könnten, blockiert.
Während die eigentliche Animation geladen wird, rendert DeferredFameAnimation ein statisches Preview-Bild,
typischerweise der erste Frame der Animation, um die Ladezeit zu überbrücken.
Nachdem das asynchrone Laden des Atlases abgeschlossen ist, müssen die Texturen dieses Atlases
auf dem Main-Thread zur GPU hochgeladen werden.
Durch die Größe der Texturen führt das zu merkbaren Lag.
Um das zumindest weniger auffällig zu machen, passiert der Upload der Texturen zwischen Render-Zyklen und mit 100ms
Zeitverzögerung zwischen Uploads.

% resets author
\renewcommand{\kapitelautor}{}


\renewcommand{\kapitelautor}{Autor: Felix Zwickelstorfer}
\section{Savestates}\label{sec:savestates}
\renewcommand{\kapitelautor}{Autor: Felix Zwickelstorfer}

In Forty-Five gibt es drei verschiedene Savefiles, die unterschiedliche Daten speichern und diverse Zwecke erfüllen.
Diese sind die permanente und normale Speicherdatei, als auch die User-Präferenzen.
Jedes dieser drei hat eine default Version, welche beim erstmaligen Starten des Spiels geladen und verwendet werden.
Diese werden auch verwendet, falls ein Fehler mit einer aktuell benutzten Datei auftritt.
Die Unterschiede werden im Folgenden näher erklärt.


\subsection{Savefile}\label{subsec:savefile}

Das normale Savefile, das eigentlich nur als Savefile betitelt wird, ist jenes, welches sich über einen einzelnen Run verändert.
Das heißt, während man herumgeht und Karten sammelt, werden diese temporär in diese Datei geschrieben.
Es beinhaltet beispielsweise:
\begin{itemize}
    \item Karten: Welche Karten der Spieler aktuell besitzt.
     Dies beinhaltet auch alle Karten der Decks.
    \item Decks: Die 5 Konfigurierbaren Decks, also den Namen und die Karten mit den jeweiligen Positionen.
    \item Leben: Wie viele Leben der Spieler aktuell hat, als auch die maximale Anzahl der Leben.
    \item Statistiken: Diverse Daten wie \zB die Anzahl der gewonnenen Kämpfe oder der verbrauchten Reserves.
    \item Position: Auf welcher Map und welchem Feld der Spieler steht.
    Zusätzlich dazu auch das davor besuchte Feld, falls man auf einem Kampf steht, welcher einen nur in eine bestimmte Richtung gehen lässt, nämlich zurück.
\end{itemize}


\subsection{permanentes Savefile}\label{subsec:perma-savefile}
Das permanente Savefile wird, wie der Name sagt, eigentlich niemals zurückgesetzt.
Falls diese Datei beim Start des Spiels fehlt, wird es vom Programm wie eine neue Installation gewertet und auch die anderen beiden Speicherdateien werden neu überschrieben.
Es beinhaltet teilweise über scheidende Daten zur normalen Speicherdatei, wie die Karten die man besitzt.
Allerdings werden diese nur dann aktualisiert, falls man in eine neue Area (=Stadt) geht, die man noch nie davor besucht.
Die bereits gesehenen Areas werden auch in dem File gespeichert.
Zusätzlich dazu beinhaltet es noch Daten bezüglich des Tutorials, vor allem welche bereits durchgespielt worden sind.

\subsection{User-Präferenzen}\label{subsec:user-prefs}
Die User-Präferenzen werden ähnlich wie die permanente Speicherdatei eigentlich nur beim erstmaligen Start neu geladen, da man diese meistens nicht ändern will.
Sie beinhalten alles, was man in den Einstellungen ändern kann, also die gewünschte Lautstärke für sowohl die Musik als auch die Soundeffekte als auch auf welchem Screen man nach Starten des Spiels sieht.
Weiteres beinhaltet es die Option, ob der Screen bei bestimmten Aktionen Wackeln darf oder nicht.


\section{Effektsystem}\label{sec:effectsystem}

\renewcommand{\kapitelautor}{Autor: Irgendwer} % todo: replace

%
% text goes here
%

% resets author
\renewcommand{\kapitelautor}{}


\section{Key-Input System}\label{sec:key-input-system}

\renewcommand{\kapitelautor}{Autor: Irgendwer} % todo: replace

%
% text goes here
%

% resets author
\renewcommand{\kapitelautor}{}


\section{Utility Systeme}\label{sec:utility-systeme}


\subsection{Stylesystem}\label{subsec:stylesystem}

\renewcommand{\kapitelautor}{Autor: Marvin Kurka}

Wie im Kapitel\ref{sec:screens} erklärt, sind Screens in Onj-Dateien definiert.
Dabei war es uns wichtig, Widgets schnell und einfach designen und auch simple Animationen hinzufügen zu können.
Deswegen wurde ein Style-System entwickelt, das es ermöglicht gewisse Attribute von Widgets über die Screen-Dateien
zu ändern, und dass auch abhängig von Bedingungen und animiert.

Jedes Widget, dass über dieses System gestyled werden kann, implementiert das StyledActor-Interface und
bekommt einen StyleManager vom ScreenBuilder zugewiesen.
Der StyleManager speichert die StyleProperties des Widgets.
StyleProperties sind Aspekte des Widgets, die das StyleSystem ändern kann, \zB den Hintergrund.

Jede StyleProperty hat verschiedene StyleInstructions, die versuchen der Property einen konkreten Wert zuzuweisen,
\zB \inlineOnj{background: "black_texture"} wäre eine StyleInstrunction für die background-StyleProperty.
Jede Instruction kann auch eine Bedingung haben, die angibt, ob sie gerade aktiv ist, \zB könnte eine Instruction
den Hintergrund nur dann setzen, wenn über das Widget gehovert wird.
Kommen mehrere Instructions für eine Property in Frage, wird eine Priority verwendet, um zu entscheiden, welche
Instruction gewinnt.

Um simple Animationen zu ermöglichen, wird eine AnimatedStyleInstruction, eine Subklasse von StyleInstruction, verwendet.
Diese speichert den Zeitpunkt, an dem sie die Kontrolle erlangt hat, und interpoliert basieren darauf ihren Wert.
Das funktioniert nur für numerische Werte.

Hier ein Beispiel aus \inlineCode{choose_card_screen.onj}, dass das System in Verwendung zeigt.

%! language = Onj
\begin{minted}{onj}
styles: [
    {
        style_priority: 1, // setzt die priority für alle instructions in diesem style
        positionType: positionType.absolute,
        positionRight: 24#percent,
        width: 18#percent,
        height: 25.0#percent,
        alpha: 1.0,
    },
    {
        // reduziert den Alpha-Wert, wenn das Widget disabled ist
        style_priority: 2,
        style_condition: actorState("disabled"),
        alpha: 0.3,
    },
    // Die nächsten beiden Styles erzeugen eine animierte Bewegung, wenn ein anderes Widget über dises gedragged wird
    {
        style_priority: 2,
        positionBottom: -5#percent,
        style_animation: {
            duration: 0.1,
            interpolation: interpolation.linear
        },
    },
    {
        style_priority: 3,
        style_condition: actorState("draggedHover"),
        positionBottom: 0#percent,
        style_animation: {
            duration: 0.1,
            interpolation: interpolation.linear
        },
    },
]
\end{minted}

Die \inlineOnj{actorState}-Funktion fragt hier den Zustand des Widgets ab.
In welchen States ein Widget ist, wird vom Programm mittels den \inlineKotlin{StyledActor::enterActorState} und
\inlineKotlin{StyledActor::leaveActorState} Funktionen festgelegt.
Diese Bedingungen können auch verknüpft werden, \zB
\inlineOnj{(hover() and state("something")) or not(actorState("something else"))}.

Die hier verwendeten Funktionen und globalen Variablen sind in einem Namespace, nämlich dem
\inlineKotlin{StyleNamespace}, definiert.

% resets author
\renewcommand{\kapitelautor}{}


\subsection{Timelines}\label{subsec:timelines}

\renewcommand{\kapitelautor}{Autor: Marvin Kurka}

Ein Encounter in \FF ist zu einem großen Teil aus linearen Abfolgen von Aktionen aufgebaut.

Ein Beispiel: Der Spieler drückt die "shoot"-Taste:
\begin{itemize}
    \item Die Karte im Slot 5 des Revolvers wird entfernt
    \item Dem Gegner wird Schaden hinzugefügt
    \item Die Schadensanimation des Gegners wird gestartet (und läuft ab dann parallel und selbstständig)
    \item Es wird überprüft, ob die geschossene Karte Effekte mit dem Trigger "ON\_SHOT" hat
    \item Wenn ja, werden diese ausgeführt
    \item Die Rotations-Animation des Revolvers wird abgespielt
    \item Es wird darauf gewartet, dass die Rotations-Animation fertig ist, bevor weitere Aktionen durchgeführt werden
\end{itemize}

Dieses Beispiel ist etwas simplifiziert, in Wahrheit müssen auch Statuseffekte, Encounter Modifier und weitere Mechaniken
beachtet werden.

Um solche lineare Abfolgen einfach verwalten zu können, wurde die Timeline-Klasse entwickelt.
Diese Klasse speichert eine Serie an Aktionen und führt eine nach der anderen aus.
Eine Aktion kann dabei auch nachfolgende auch blockieren, zum Beispiel die Ausführung um eine Anzahl an
Millisekunden verzögern oder warten, bis eine Bedingung zutrifft.

Timelines können einfach mittels der \inlineKotlin{Timeline::timeline} Funktion erstellt werden.
Diese nimmt ein Builder-Lambda mit der \inlineKotlin{TimelineBuilderDSL} Klasse als Receiver, die diverse Funktionen
zur Verfügung stellt.

%! language = Kotlin
\begin{minted}{kotlin}
fun createTimeline() = Timeline.timeline {
    action {
        println("Hello")
    }
    delay(500)
    action {
        println("World")
    }
}
\end{minted}
\codeblockCaption{Dieser Code printet Hello, wartet eine halbe Sekunde und printet dann World}

Mittels der \inlineKotlin{TimelineBuilderDSL::include} Funktion können auch die Aktionen anderer Timelines inkludiert
werden.
Hängt eine Sub-Timeline von Werten ab, die während des Bauens noch nicht bekannt sind, kann die
\inlineKotlin{TimelineBuilderDSL::includeLater} Funktion verwendet werden, die mit dem Erstellen der Sub-Timeline
wartet, bis die Aktion tatsächlich an der Reihe ist.
Mit der \inlineKotlin{TimelineBuilderDSL::delayUntil} Funktion kann die Ausführung der Timeline verzögert werden, bis
eine Bedingung zutrifft.

In der GameController Klasse, die große Teile des Encounters kontrolliert, sind viele Funktionen so aufgebaut, dass
sie statt eine gewisse Aufgabe zu erfüllen eine Timeline zurückgeben, die diese Aufgabe erfüllt.
Ein Beispiel ist die \inlineKotlin{GameController::rotateRevolver} Funktion, die, wenn sie aufgerufen wird, erst einmal
nichts tut.
Stattdessen gibt sie eine Timeline zurück, die erst wenn sie gestartet wird den Revolver rotiert.
Funktionen wie die \inlineKotlin{GameController::shoot} Funktion binden diese Timeline dann in ihre eigene ein.
Das erlaubt es dem GameController nach einem Userinput dynamisch eine Timeline aufzubauen, die Schritt für Schritt alle
notwendigen Aktionen durchführt.

% resets author
\renewcommand{\kapitelautor}{}



\section{Road Generation}\label{sec:road-generation}

\renewcommand{\kapitelautor}{Autor: Irgendwer} % todo: replace

%
% text goes here
%

% resets author
\renewcommand{\kapitelautor}{}


\section{Tutorial}\label{sec:tutorial}

\renewcommand{\kapitelautor}{Autor: Marvin Kurka}

Das Tutorial ist als eine Road implementiert, jedoch mit der Besonderheit das diese nicht von dem Mapgenerator generiert
wird.
Stattdessen ist eine statische Definition der Tutorial-Road (auch Road zero genannt) vorhanden, die bei jedem Run-Reset
kopiert wird.
Die Encounter im Tutorial sind, anders als die Encounter im eigentlichen Spiel, gescriptet, was heist, dass das
Gegnerverhalten fix vorgegeben und immer gleich ist.
Das hilft sicherzustellen, dass der Spieler genau die Erfahrungen macht, die vorgesehen sind.
Weiters wird im Tutorial immer wieder Text eingeblendet der dem Spieler grundlegende Funktionen des Spieles erklärt.

% resets author
\renewcommand{\kapitelautor}{}


\section{Biome}\label{sec:biome}

\renewcommand{\kapitelautor}{Autor: Marvin Kurka}

Um den Spieler mehr Abwechslung zu bieten wurden Biome implementiert.
Das Biom beeinflusst mehrere Aspekte des Spiels:

\begin{itemize}
    \item Den Hintergrund der Map
    \item Die Dekorationen auf der Map
    \item Die Hintergrundgeräusche, die abgespielt werden
    \item Der Hintergrund im Encounter
    \item Die Encounter die auf der Road auftauchen
    \item Die Karten, die gekauft/gefunden werden können
\end{itemize}

Sämtliche Änderungen auf der Map selber können direkt in der Konfiguration des Map-Generators implementiert werden.
Selbiges gilt für die Screen-Hintergründe, diese können direkt in der Screen-Definition konfiguriert werden, indem
der Zustand des Programmes mittels der \inlineOnj{state()} Funktion abgefragt wird.

Karten sind in Pools aufgeteilt.
Das hilft sicherzustellen, dass der Spieler nicht alle Karten sofort bekommen kann, sondern, besonders Karten mit
komplizierten Effekte, erst später im Spiel erhält.
Das ist mit den Biomen verknüpft und verwendet das RandomCardSelection Objekt.
Auch dieses Objekt kann über eine onj-Datei konfiguriert werden und bestimmt welche Karten wann und mit welcher
Wahrscheinlichkeit im Spiel auftauchen.
Hier werden bestimmte Karten anhand der Road/des Biomes geblacklistet und sind somit nicht erhältlich.

% resets author
\renewcommand{\kapitelautor}{}


\section{Sounds}\label{sec:sounds}

\renewcommand{\kapitelautor}{Autor: Marvin Kurka}

In \FF werden alle Sounds von dem SoundPlayer Objekt verwaltet.
Stellen im Code, die einen Sound abspielen wollen, können die \inlineKotlin{SoundPlayer::situation} Funktion verwenden,
um ein SoundEvent auszulösen.
Um herauszufinden, welcher Sound tatsächlich abgespielt werden soll, verwendet der SoundManager die sounds.onj Datei.
Diese definiert für jedes Event welcher Sound mit welcher Lautstärke abgespielt werden soll.

Außerdem spielt der SoundManger ambient sounds und Musik ab.
Welche Musik abgespielt wird, hängt vom Screen ab, während die gespielten ambient Sounds von dem Biom abhängen.
Ambient Sounds werden zufällig nach links oder rechts gepaned.

\vfill
\pagebreak

% resets author
\renewcommand{\kapitelautor}{}


%
\chapter{Sounds}\label{ch:sounds}


\section{DAW}\label{sec:daw}

\renewcommand{\kapitelautor}{Autor: Irgendwer} % todo: replace

%
% text goes here
%

% resets author
\renewcommand{\kapitelautor}{}


\section{Prozess}\label{sec:prozess}

\renewcommand{\kapitelautor}{Autor: Irgendwer} % todo: replace

%
% text goes here
%

% resets author
\renewcommand{\kapitelautor}{}


\section{Hintergrundmusik}\label{sec:hintergrundmusik}

\renewcommand{\kapitelautor}{Autor: Nils Hubmann} % todo: replace

%
In \ff ist Hintergrundmusik ein wichtiges Thema. Sie soll das Spielerlebnis anregen und den Spieler mit Emotionen passend zu der Spielsituation erfüllen.
Durch den Einsatz von Musik, die während des gesamten Spiels zu hören ist, wollen wir den Hörsinn der Spieler gezielt ansprechen und so ein deutlich umfassenderes Spielerlebnis kreieren.
Unsere Sounds sind auf die individuellen Teile des Spiels angepasst und sollen auch einerseits ruhige Stimmungen als auch Aufregung und Spannung vermitteln.
Es gibt 3 Teile der Hintergrundmusik die einerseits für den Titelbildschirm, für die Map die der Spieler erkundet, als auch für den Kampf, in dem hitzige Gefechte stattfinden.

\bold{Splice:}
Eine große Unterstützung im Projekt war die Plattform Splice. Diese bietet Effekt- und Soundsamples an, welche erworben werden können. Durch ein Abonnement wurden Credits gesammelt, die für passende Melodie Samples verwendet wurden.
Durch die große Auswahl an Sounds wurden passende Melodie Samples gesucht, die unseren Kriterien entsprachen.

\bold{Entstehung der Musikstücke}
Die Musik ist in Kooperation mit einer Externen Person namens Nils Jandrasits entstanden. Dieser hat das Team unterstützt und zusammen mit uns an der Hintergrundmusik gearbeitet.
Hierfür wurden die bereits erwähnten Samples verwendet und darauf aufgebaut. Als Digital Audio Workstation wurde Fl Studio verwendet, da Nils Jandrasits bereits über umfangreiche Kenntnisse in diesem Programm verfügt.
Nils Hubmann und Nils Jandrasits arbeiteten in mehreren Sessions an den Musikstücken und passten diese mit verschiedensten Instrumenten im Programm wie Trommeln und Gitarren an.
Ein wichtiger Faktor war hierbei nicht nur das Wild West Thema zu treffen, sondern auch die passende Situation und Stimmung zu vermitteln. Außerdem finden sich auch in den Musikstücken Effekt Sounds wieder.
Dazu gehören Geräusche wie das Traben von Pferden, die Rufe eines Uhus oder das Klappern einer Schlange.
Dies diente dazu gezielt Geräusche zu verwenden die in der Natur und im Wilden Westen vertreten sind.

\bold{Sound Testing}
Das Testen der Hintergrundmusik war ein wichtiger Part. Denn nicht nur die vermittelte Stimmung ist wichtig, sondern auch, dass die Musik nicht störend, bei mehrmaligem Hören, auf den Spieler wirkt.
Darum entschieden wir uns dazu lange unterschiedliche Musikparts zu designen, um den Abstand zwischen den gleichen Teilstücken der Melodie zu maximieren.
Außerdem wurden mehrmals Rhythmusänderungen in bestimmten Teilen vorgenommen, um die Spannung mancher Situation klarer definieren und steuern zu können.

%

% resets author
\renewcommand{\kapitelautor}{}


\section{UI-sounds}\label{sec:ui-sounds}

\renewcommand{\kapitelautor}{Autor: Irgendwer} % todo: replace

%
% text goes here
%

% resets author
\renewcommand{\kapitelautor}{}


%
\chapter{Grafisches Design}\label{ch:grafikdesign}

\section{Grafische Tools/Programme}

Für ein Projekt dieser Größe
ist ein einheitliches Design und eine konsistente Arbeitsweise wichtig, um das gewünschte Ergebnis zu erzielen.
Die Programme, die verwendet werden, um alle Grafiken und Assets zu erstellen, spielen daher eine große Rolle.
Die HTL Rennweg ermöglicht Schülerinnen und Schülern einen Zugang zu der Adobe Creative Cloud Suite, welche die größten und wichtigsten Design-Programme zur Verfügung stellt. Da das Diplomarbeitsteam bereits in den letzten vier Jahren der HTL einigermaßen Erfahrung mit der Adobe Creative Cloud gesammelt hat, ist diese und ihre Programme die richtige Wahl für \FF.

\subsection{Adobe XD}

Adobe Xd ist ein Programm, welches für das Designen von User Interfaces, Wireframes und Applikationen von Adobe entwickelt und 2015 veröffentlicht wurde (vgl. Wikipedia, 2024 [online]). Adobe Xd erlaubt es einem, schnell Designs auf Zeichenflächen zu erstellen,
die anschließend mit der \quoted{Prototyp} Funktion miteinander verknüpft und interagierbar gemacht werden.
Da dieses Programm teil der Adobe Creative Cloud ist, lässt es sich sehr einfach mit Adobe Photoshop und Illustrator nutzen, welche unter anderem verwendet wurden, um die Assets und Grafiken für \FF zu erstellen. Ein weiterer
\quoted{Selling-Point}
für das UI/UX Programm von Adobe, ist die Echtzeit Co-Editing Funktion, welche es dem Team ermöglicht, gleichzeitig an dem Mock Up von \FF zu arbeiten ohne Verzögerungen oder Einsatz eines Versionierungssystems.
Für \FF wurde Adobe Xd hauptsächlich für das Gestalten des Game User Interfaces verwendet, als auch für die Webseite, sowie sämtliche Grafiken für Steam, Plakate oder Social Media Posts, da es ein vektorbasiertes Programm ist. Andere Programme wie Adobe Photoshop oder Illustrator wären auch gut geeignet für alle Grafiken außerhalb des Spiels, besitzen allerdings keine Echtzeit Co-Editing Funktion und sind daher nicht die optimale Wahl für den Zweck der Diplomarbeit (vgl. Adobe, 2024 [online]).

\subsection{Adobe Photoshop}

Adobe Photoshop ist ein Bildbearbeitungsprogramm für Rastergrafiken des Herstellers Adobe. Das Programm ist heutzutage ein Industriestandard und der weltweite Marktführer in der Design Branche und wird von Fotografen, Künstlern und Grafikdesignern verwendet. Adobe Photoshop ist Teil der Adobe Creative Cloud und stellt daher eine gute Verbindung zu den anderen Programmen der Creative Cloud wie Illustrator oder Premiere Pro her. Tools wie Masken, Auswahlwerkzeuge, Zeichenwerkzeuge und komplexe Filter ermöglichen es einem, jeden kreativen Gedanken beim Erstellen des \FF Mockups umzusetzen, wie beispielsweise Hintergründe, Buttons, User Interface Elemente oder Icons. (vgl. Wikipedia, 2024 [online])
Für \FF ist die Kombination von Photoshop und Adobe Xd besonders praktisch, da die Funktion
\quoted{in Photoshop bearbeiten} in Adobe Xd,
Photoshop öffnet und direkte Veränderungen an dem ausgewählten Bild in Xd übertragen kann. Dadurch erspart man sich das langwierige, mehrmalige Exportieren von Assets, die jedes Mal neu in eine Zeichenfläche eingefügt werden müssten.
Mit Pinseln ist es möglich sich vielseitig auszudrücken beim Erstellen von Assets für Videospiele. Mit Pinselsets ist es möglich, den Stil von verschiedenen Werkzeugen wie Öl Pinsel, Tinte oder einen Bleistift zu replizieren. Adobe Photoshop bietet ein vorinstalliertes Pinselset von Kyle T. Webster an, welches in \FF verwendet wurde, um Icons wie zum Beispiel das Kampf-Symbol zu erstellen.

\begin{figure}[H]
    \centering
    \includegraphics[width=0.4\textwidth]{kampfSymbol.png}
    \caption{Kampf Symbol auf der Map}
\end{figure}

Dieses Pinselset beinhaltet 20 verschiedene Pinsel, wobei hauptsächlich \quoted{Klassischer Cartoonist} verwendet wurde. Mit diesem Pinsel ist es einfach Tinte zu replizieren, was bei \FF hilft, den wilden Westen Stil zu verpassen. Unter anderem wurde auch ein zweites Pinselset von ArtistMef verwendet, welches lizenzfrei ist und eine Vielfalt an Pinseln bietet (vgl. ArtistMef, 2024 [online]). Beispielsweise wurde die Form eines Buttons des Statistik Bildschirmes zuerst mit dem
\quoted{Sketching Brush} Pinsel erstellt und anschließend mit dem \quoted{Spray Brush} Pinsel texturiert, um Tiefe und Detail in den Button zu bringen.

\begin{figure}[H]
    \centering
    \includegraphics[width=0.4\textwidth]{tryAgainButton.png}
    \caption{'Try Again' Button im Statistik Fenster}
\end{figure}

Diese Pinsel können jeweils noch weiterhin angepasst werden unter den Pinseleinstellungen mit Optionen wie der Streuung, die bestimmt, wie breit der Pinsel aufgetragen wird oder Einstellungen zur Struktur des Pinselauftrags selbst. Je nach Einstellungen ist es einem möglich, seinen Assets einen einzigartigen Look zu verpassen und ein passendes, einheitliches und wiedererkennbares Design zu kreieren.

\begin{figure}[H]
    \centering
    \includegraphics[width=0.4\textwidth]{pinselStruktur.png}
    \caption{Struktur Einstellungen eines Pinsels in Photoshop}
\end{figure}

Eine weitere Methode um den Elementen von \FF den rustikalen western Look zu verpassen ist mit Texturing. Mit den Mischmodi von Photoshop ist es möglich einem beliebigen Objekt beispielsweise eine papier-artige Textur zu geben, indem eine weitere Ebene erstellt wird mit einem Bild von einem alten Stück Papier und einen passenden Mischmodus auswählt. Der Mischmodus bestimmt, wie sich ein Mal- bzw. Bearbeitungswerkzeug auf die Pixel im Bild auswirkt (Adobe, 2024). Zusätzlich kann mit einer Schnittmaske die Bildtextur auf die Pixel der darunterliegenden Ebene zugeschnitten werden. Diese Technik wird bei jedem Asset, welches als Hintergrund für Pop-Up Fenster oder ähnlichen dient, angewendet.

\subsection{Adobe Illustrator}

Illustrator ist Adobes größtes vektorbasierte Computerprogramm welches 1987 entwickelt wurde. Im Gegensatz zu klassischen Bildbearbeitungsprogrammen, werden die Objekte nicht in Form von Pixeln gespeichert, sondern als Vektoren. Illustrator ist wie das Gegenstück Photoshop, ein weltweiter Marktführer in der Designbranche und wird genutzt für die Erstellung von Logos, Symbolen, Skizzen, Zeichnungen, Typografie und Illustrationen. (Wikipedia, 2024) Der Vorteil von Vektorgrafiken, ist dass diese keine feste Auflösung haben wie Rastergrafiken, sondern von der Auflösung unabhängig sind und mit Pfaden und Kurven beschrieben werden. Das gängigste Vektorformat ist
\quoted{.svg},
welches die Pfade mithilfe von XML-Tags und Attributen beschreibt.

\begin{figure}[H]
    \centering
    \includegraphics[width=0.4\textwidth]{rasterVsVektor.png}
    \caption{Vergleich Rastergrafik Vektorgrafik}
%    \caption*{
%        \url{Quelle: https://helpcenter.shirtigo.de/wiki/druckmotiv/vektorgrafik/#:~:text=Grundlage\%20einer\%20Vektorgrafik\%20bilden\%20die,wie\%20Kurven\%2C\%20Kreise\%20oder\%20Polygone}
%    }
\end{figure}

Illustrator wurde in \FF für die Erstellung von Icons, Symbole, Sticker und Plakate verwendet. Beispielseise ist es für Sticker besonders wichtig in einem vektorbasierten Programm zu arbeiten, da mit Photoshop ein Qualitätsverlust je nach Skalierung auftreten kann, was insbesondere für Druck ungeeignet ist. Mit Illustrator ist es möglich in einem CMYK-Farbmodus zu arbeiten, welcher für Druck üblicherweise verwendet wird, da Drucker mit einem CMYK-Farbprofil arbeiten, im Vergleich zu Bildschirmen im RGB-Farbprofil.

\section{Grafik und Design}

„Design unterscheidet sich von Kunst in einem zentralen Punkt: es muss einen Zweck haben.“ (Reid, 2024) Mockups helfen, um die Funktionalität, Ästhetik und Benutzerfreundlichkeit eines Spiels zu visualisieren und zu testen. Es gibt viele Möglichkeiten, um ein Mockup zu erstellen, wie beispielsweise mit Papier und Bleistift, um einen groben Überblick zu erschaffen. Für große Applikationen werden allerdings Programme wie Adobe Xd oder Figma verwendet, da somit eine konsistente und dynamischere Arbeitsweise erzielt. Für \FF ist es wichtig, dass ein Programm verwendet wird, welches interaktive Seiten bauen lässt, die den Ablauf des Spiels simulieren.

\subsection{Navigation Bar}

Jedes moderne Videospiel hat ein Stück UI, welches immer sichtbar ist, um die wichtigsten Informationen, die für den Spieler relevant sind, anzuzeigen. Diese Grafik soll nicht zu viel Aufmerksamkeit zu sich ziehen, jedoch immer schnell sichtbar sein.
In \FF existiert eine Navigation Bar, bestehend aus einer schwarzen länglichen Leiste und drei Buttons, die zu den wichtigsten Menüs im Spiel führen, dem Hauptmenü, dem Rucksack und den Einstellungen. Die mit Photoshop Pinseln selbst gezeichneten Icons, helfen dem Spieler dabei noch schneller zu erkennen, welche Aufgabe der jeweilige Button hat, und merkt sich diese dadurch besser. In der schwarzen Leiste befinden sich die wichtigsten Informationen die für \FF relevant sind: die Lebensanzeige, den Ort, in dem sich der Spieler aufhält und wie viel Geld der Spieler bereits gesammelt hat. Beide Elemente haben eine raue Oberfläche, welche mit Texturen von
\quoted{Texturelabs.com}
texturiert wurden. Die Website bietet kommerziell kostenfreie Grafiken und Texturen, die über bereits existierende Grafiken gelegt werden können. Zusätzlich hat die schwarze Leiste als auch die Buttons eine raue Silhouette, um den Effekt zu verstärken.

\begin{figure}[H]
    \centering
    \includegraphics[width=1.0\textwidth]{navigationsleiste.png}
    \caption{Navigationsleiste Grafik in \FF}
\end{figure}

Die Navigation Bar ist immer sichtbar, außer – wenn der Spieler verloren hat – im Death-Screen und während eines Kampfes. Während einem Kampf ist eine leicht abgeänderte, kleinere Variante der Navigation Bar zu sehen, da beispielsweise die Option
\quoted{zum Titel zurückzukehren}
nicht mehr verfügbar ist. Allerdings werden weiterhin die Lebenspunkte, das Deck und das bereits gesammelte Geld, angezeigt.
Sogenannte \quoted{Hover-} und \quoted{Active-States} geben dem Spieler ein Feedback, wenn sie mit der Navigationsleiste interagieren. Beispielsweise wird der Backpack-Button heller, sobald man mit dem Mauszeiger auf ihn fährt, um den Spieler mitzuteilen, dass dieser klickbar ist und eine Aktion auslöst. Sobald der User auf den Knopf drückt, erscheint einerseits das Rucksack UI, aber auch wird der Backpack-Button hinausgezogen. Das visualisiert, dass der Rucksack gerade offen ist, beziehungsweise dieser Knopf ausgewählt wurde.

\subsection{Map Mockup}

Angefangen mit einem simplen Wireframe, bestand das Mockup vorerst aus Events – das sind kleine Punkte auf der Karte – welche mit Strichen verbunden wurden, die einen Weg durch verschiedene Landschaften bilden. Es wurde mit Placeholder Texturen gearbeitet, um ein grobes Bild des Gameplays zu erstellen. Das UI hatte zuerst einige Elemente mehr, wie beispielweise eine Legende, die beinhält, die alle Events auf der Karte erklärt oder eine Minimap, die einen Überblick über die ganze Karte zeigt. Die Elemente wurden im Laufe der Zeit weggelassen, da diese Grafiken immer sichtbar wären und dadurch das UI an Informationen überfluten, die nicht immer benötigt werden würden.

\begin{figure}[H]
    \centering
    \includegraphics[width=0.8\textwidth]{altesMockup.png}
    \caption{Alte Version des Map Mockups}
\end{figure}

Es wurden über die Zeit viele Versionen von Texturen erstellt, da wir erst im Laufe der Diplomarbeit den passenden Stil entwickelt und gefunden haben. Beispielsweise ist in der aktuellen Version des \FF Mockups eine Textur für den Boden gezeichnet worden, die einen nahtlosen Übergang hat, damit diese unendlich oft aneinandergelegt werden kann. Da sich \FF einerseits in einer Wüstenlandschaft abspielt, mussten Kakteen, Skelette oder Häuser gezeichnet werden, da sie in einer Wüste üblich sind.

\begin{figure}[H]
    \centering
    \includegraphics[width=0.8\textwidth]{mapAusschnitt.png}
    \caption{Ausschnitt der aktuellen Map}
\end{figure}

Das Event Popup ist das wichtigste Element, welches auf der Map sichtbar ist. Das Popup ist jedes Mal sichtbar, wenn der Spieler sich auf ein Event bewegt. Nehmen wir als Beispiel einen Kampf: Sobald der Spieler sich auf eines der kleinen Roten Symbole bewegt, welche mit einem Revolver Icon gekennzeichnet sind, erscheint auf der rechten Seite des Bildschirms das Showdown Fenster, welches dem Spieler die Informationen anzeigt, die für den Kampf relevant sind. Die folgenden Bilder zeigen die Entwicklung des Zeichenstils des User Interfaces.

\begin{figure}[H]
    \centering
    \includegraphics[width=1.0\textwidth]{showdown.png}
    \caption{Entwicklung des Showdown Fensters}
\end{figure}

Die Gestaltungsprinzipien sind in jeder Grafik in \FF wiederzufinden. In dem Popup ist erkennbar, dass die Information in der weißen Box zusammengehört, aufgrund des Gesetzes der Nähe. Mit der Trennlinie wird weiters betont, dass das Revolver Icon und das Wort Showdown der Titel für dieses Event sind. Auch das Gesetz der Ähnlichkeit zeigt sich in der schwarzen Fläche, der weißen Box, als auch dem Button, da diese alle denselben rauen Zeichenstil haben. Der Fight-Button hat sich mit der Zeit der Entwicklung vergrößert, da ein Button, der oft von dem Spieler betätigt wird, schnell und einfach zugängig sein sollte. Ebenfalls wurde ein Hover State erstellt, damit der Spieler schnelles Feedback zu seinen Aktionen erhält. Der einfachste weg einen Hover Zustand zu erstellen ist, indem die Farben einfach getauscht werden, da diese weiterhin dem Farbschema treu bleiben und sich stark genug von dem originalen Knopf ohne Hover Status abheben. (SpeedUX, 2024)

\begin{figure}[H]
    \centering
    \includegraphics[width=1.0\textwidth]{fightButton.png}
    \caption{Hover Zustand des Fight Buttons}
\end{figure}

\subsection{Encounter Mockup}

Das Layout der Buttons und des Revolvers spielen eine wichtige Rolle in \FF. Das Encounter Mockup, auch Kampf, Showdown oder Fight UI genannt, wurde wie das Event Popup im Laufe der Diplomarbeit mehrmals von Grund auf neu gebaut, aufgrund Stil- und Game-Design Änderungen. Insgesamt wurde drei Mal das Encounter UI neugestaltet. Nach vielem Experimentieren kam es zur Conclusio, dass der Revolver – das Spielfeld – in der Mitte des Bildschirms sein muss, da dieser am meisten verwendet wird.

\begin{figure}[H]
    \centering
    \includegraphics[width=1.0\textwidth]{altVsNeuEncounter.png}
    \caption{Altes versus verbessertes Encounter Mockup}
\end{figure}

Die Karten spielen eine genauso große Rolle. Für Kartenspiele – ob in Realität oder in Videospielen - ist es üblich, dass die Karten in einer Hand sind. Aus diesem Grund ist es die optimale Wahl, die Karten links und rechts von dem Revolver zu platzieren. Indem die Karten näher aneinander sind und sich überlappen, wirkt es mehr wie eine Hand, die die Karten hält. Damit sich die Karten auch gut voneinander abheben und genug Kontrast entsteht, wird ihnen ein Schlagschatten zugewiesen.

\begin{figure}[H]
    \centering
    \includegraphics[width=0.6\textwidth]{finalerEncounter.jpeg}
    \caption{Finale Version des Encounter Mockups}
\end{figure}

Die alten Versionen des Encounter Mockups weisen wenige Texturen auf, weshalb dadurch nicht dem rustikalen western Stil von \FF entsprechen. Es wurde sich dafür entschieden, dass die Reserves- und Deck-Inseln getrennt von der Holzleiste im unteren Bereich des Bildschirms sind, da ansonsten die Holzleiste zu viel Fläche des Bildschirms einnehmen würde und das Hovern von Karten manchmal die Reserves verdecken würden. Mit einer Animation, die die Inseln schweben lässt, bringen diese zusätzliche Dynamik für das Auge.

Die Shoot und Holster Buttons, oder auch End Turn Button genannt, waren in den vergangenen Versionen des Mockups ungleichmäßig platziert. Da beide Aktionen eine wichtige Rolle spielen, wurde sich dafür entschieden, dass die neu entworfenen Buttons seitlich aus dem Revolver stehen sollten, sodass sie einfach zugängig und groß genug seien können.
Das \quoted{Parry} Mockup ist Teil des Encounter Mockups. Wenn ein Gegner am Anfang seines Spielzuges dem Spieler schaden zufügen möchte, hat die Spieler die Option den Angriff zu blockieren. Um diese Aktion für den Spieler zu intuitiv wie möglich zu gestalten, tauschen sich die vorherigen Buttons, die aus dem Revolver standen, mit einer Animation. Um die Spieler weiter aufmerksam auf diese Situation zu machen, wird mit einem Ton und Vignetten der Effekt verstärkt. Das baut Spannung auf und teilt dem Spielenden mit, dass dieser gerade vor einer wichtigen Entscheidung steht. Zusätzlich befindet sich in der Mitte des Bildschirms ein Erklärungstext zu der Aktion.

\begin{figure}[H]
    \centering
    \includegraphics[width=0.8\textwidth]{parryMockup.png}
    \caption{Parry Mockup}
\end{figure}

Auch das Gesetz der Ähnlichkeit wurde beim Entwerfen des Mockups angewandt. In der linken oberen Ecke wird die Lebensanzeige und das Geld auf demselben schwarzen Balken angezeigt, der in jeder anderen Szene im Spiel zu sehen ist, um Spielern mitzuteilen, dass es hierbei um dieselbe Lebensanzeige handelt, welche für den Kampf und das strategische Handeln wichtig ist. Durch das Gesetz der Ähnlichkeit zeigt sich, dass die kürzere Leiste im Kampf dieselbe ist, wie die große Leiste auf der Map.

\begin{figure}[H]
    \centering
    \includegraphics[width=0.8\textwidth]{navigationbarEncounter.png}
    \caption{Status Bar in kurzer Variante während des Kampfes}
\end{figure}

Statuseffekte werden durch Karten oder Gegner-Angriffe dem Spieler hinzugefügt. Beispielsweise zeigt folgendes Bild die Statuseffekte Bewitched, Poison, Burning und die Anzahl an Schild am oberen Rand des Bildschirms während dem Kampf. Es wurde sich dafür entschieden eine helle Hintergrundfläche im Stil eines Papiers hier zu verwenden, da ein dunkler Hintergrund die Komposition einengen und einen Rahmen setzen würde, was in diesem Fall unpassend wäre, da der Hintergrund der Elemente verhältnismäßig hell ist. Rechts im Bild zu sehen sind die
\quoted{Encounter Modifier}.
Das Sind Effekte, die den gesamten Kampf beeinflussen können. Diese werden auf derselben Papiertextur dargestellt, allerdings zeigt sich die Beschreibung der einzelnen Encounter Modifier erst, wenn der Spieler mit dem Mauszeiger über die Grafik fährt. Im Normalzustand sind nur die Icons zu sehen, um so wenig Platz wie möglich aufzubrauchen, da sonst das User Interface zu überladen und eventuell sogar überfordernd wirkt für die Spielenden.

\begin{figure}[H]
    \centering
    \includegraphics[width=0.9\textwidth]{statusEffekte.png}
    \caption{Statuseffekte und Encounter Modifier}
\end{figure}

\subsection{Backpack Mockup}

Das Backpack beinhält die Karten, die der Spieler über die Zeit gesammelt hat. Es wird mittels dem Backpack Button geöffnet, welcher mit einer Ease In und Ease Out Animation in das Bild gezogen wird. Wenn eine Animation erhebliche Bildschirmänderungen mit sich bringt, beispielsweise wenn ein modales Fenster angezeigt wird, kann eine Dauer von 200–300 ms angemessen sein. Je weiter sich ein Element bewegen muss, desto wichtiger ist es, dass dies reibungslos und ohne Erschütterungen geschieht (besonders für bewegungsempfindliche Menschen, wie z. B. Benutzer mit Epilepsie oder Gleichgewichtsstörungen) (Laubheimer, 2024). Aus diesem Grund ist es wichtig, dass Animationen in \FF minimalistisch gehalten werden.

Eine simple Box um mehrere Objekte, zeigt unserem Gehirn, dass die Information darin zusammengehört. Genau dasselbe wird beim Backpack UI in \FF angewendet. Mit dem braunen, lederartigen, Rucksack-ähnlichem Asset wird das Gesetz der Geschlossen übermittelt. Dieses Konzept möge hier angewendet werden, da unser Gehirn in der Wahrnehmung Formen ergänzt, sodass geschlossene Figuren entstehen  (IWMedien, 2024). Der Name der Decks ist änderbar. Aus Usability technischen Gründen befindet sich der Knopf zur Bearbeitung direkt neben dem Namen, um es für den Spieler zu schnell zugänglich zu machen wie möglich, genauso wie alle anderen wichtigen Funktionen des Backpacks, wie die Sortierfunktion und die Wechselfunktion zwischen den verschiedenen Decks.

\begin{figure}[H]
    \centering
    \includegraphics[width=1.0\textwidth]{backpack.png}
    \caption{Screenshot des Backpack UIs}
\end{figure}

\subsection{Dialog Mockup}

Bei vielen Videospielen ist es üblich, dass sich auf dem Dialog Screen einerseits ein Textfeld und die Person mit der interagiert wird, befinden, wie beispielsweise auf
\quoted{Game UI Database} – eine Website die User Interface Screenshots von Videospielen zur Schau stellt – zu erkennen ist. Für die User Experience ist zu beachten, dass die Texte von den NPCs nicht zu lang sein dürfen, was auf einigen Screenshots der Website zu erkennen ist. Ein bis Zwei Zeilen sollten maximal pro Klick angezeigt werden. Dadurch wird die Gefahr verringert, dass Spieler den Text überfliegen. Weiters ist das Namenschild des NPCs ist auf derselben Seite wie der NPCs. Dadurch wird besser übermittelt, dass beispielsweise im folgenden Screenshot die Hexe den Namen
\quoted{Evil Witch} im Spiel trägt.

\begin{figure}[H]
    \centering
    \includegraphics[width=1.0\textwidth]{witchInteraktion.png}
    \caption{Interaktion mit dem NPC "Evil Witch"}
\end{figure}

\subsection{Title Screen und Death Screen}

Ein Videospiel ohne Titelbildschirm ist wie ein Buch ohne Cover. Der Erste Eindruck eines Spiels ist der wichtigste. Mit dem Title Screen kann eine gewisse Professionalität übermittelt werden und dem Marketing dienen, wenn dieser gut gelingt. Dieser wird sehr simpel gehalten und zeigt keine Besonderheiten, außer den Namen des Spiels und einem Text wie etwa
\quoted{Press to Start}.
Jedoch gibt es noch eine andere Art von Title Screen, nämlich das Menü-Basierte, wobei der Spieler unmittelbar nach dem Starten mehrere Optionen zur Auswahl hat, wie
\quoted{Credits} oder \quoted{Settings}.  Der Title Screen von \FF besteht
aus dem \FF Logo und den Optionen Continue, Abandon Run, Reset Game, View Credits und Quit. Die Reihenfolge dieser Buttons ist ein großer Bestandteil der User Experience. In anderen Videospielen ist sichtbar, dass die erste Option dafür da ist, um das Spiel zu starten, beziehungsweise um in \FF auf die Map zu kommen, während die letzte Option in der Auflistung dazu dient, das Spiel zu schließen, da ein Spielentwickler möchte, dass der User das Spiel so lange wie möglich beschäftigt ist und nicht auf die Idee kommt das Spiel zu schließen.

\begin{figure}[H]
    \centering
    \includegraphics[width=1.0\textwidth]{titlescreen.png}
    \caption{Title Screen von \FF}
\end{figure}

Der Hintergrund für den Title Screen wurde in Adobe Photoshop erstellt, und die Grafiken digital per Hand gezeichnet. Die Idee war es, ein Mural nachzustellen, angepasst auf \FF. Ein Mural ist eine Art der Wandmalerei, bei der das Bild auf Wänden oder Decken gemalt wird. (Wikipedia, Wandmalerei, 2024) Die Textur, die über das gesamte Artwork gelegt wurde, unterstützt diesen Effekt im Title Screen. Das Spiel dreht sich rund um die Bullets. Daher ist es wichtig, dass die Bullets gut im Title Screen vertreten sind und präsentiert werden. Der Titelbildschirm steht auch als Gegenstück zu dem Death Screen. In dem Start Screen wird ein Sonnenaufgang gezeigt, während andererseits ein Spieler im Death Screen – wenn dieser verloren hat – sich in einem Loch befindet, wie folgender Ausschnitt aus dem Spiel zeigt. Dadurch wird Anfang und Ende des Spiels besser präsentiert. Der Death Screen soll zeigen, wie der Spieler gerade begraben wird, nachdem ihm die Leben in einem Kampf ausgegangen sind. Da hauptsächlich nur Rot in diesem Bild vorkommt, wird dem Spielendem schnell übermittelt, dass es sich hierbei um eine Niederlage handelt, da die Farbe Rot oft mit dem Negativen verbunden wird. Es wurde kein auffälliger Button eingefügt, um aus dem Death Screen zu kommen, sondern nur ein kleiner Schriftzug, der dem Spieler hinweist, dass er mit einem Mausklick weiterkommt. Der Haupt Fokus soll auf dem großen Schriftzug
\quoted{A Terrible Fate} und die Zeichnung im Hintergrund liegen.

\begin{figure}[H]
    \centering
    \includegraphics[width=1.0\textwidth]{deathscreen.png}
    \caption{Death Screen von \FF}
\end{figure}

\subsection{Map Event Mock-ups}

Auf der Map findet man verschiedene Events. Einen Shop, interagier bare NPCs und Orte, an denen sich der Spieler heilen oder eine neue Karte auswählen kann. Ähnlich zu dem Kampf, erscheint auf der rechten Seite des Screens ein Pop-up Fenster, indem man dieses Event starten, beziehungsweise den Ort betreten kann. Den einzelnen Ereignissen auf der Karte wurden Farben zugewiesen, damit der Spieler diese schneller erkennen kann. Die Icons der Events spielen eine große Rolle, da ein Spieler beim ersten Blick sofort erkennen muss, was er sich bei dem Ereignis ungefähr erwarten kann. Die Symbole wurden mit demselben Pinsel gezeichnet, wie die UI Elemente, um für Einheit im Design zu sorgen. Je simpler Icons sind, desto besser wird die Information übermittelt.

\begin{figure}[H]
    \centering
    \includegraphics[width=1.0\textwidth]{ereignisseMap.png}
    \caption{Nahansicht der Ereignisse auf der Map}
\end{figure}

Der Shop zeigt auf der linken Seite den Händler, der einem die Ware verkauft, während sich auf der rechten Seite die Karten befinden, welche auf einer schwarzen Fläche hinterlegt sind. Um die Bullets von anderen Texten die Informationen zum Shop geben zu trennen, wird eine weiße Box hinter diese gelegt. Weiteres ist für die User Experience wichtig, dem Spieler mitzuteilen, dass dieser die Karten ziehen muss, um sie in den Rucksack oder das Deck zu legen. Daher befindet sich unten am Bild ein Informationstext, der die Funktionalität dem Spieler erklärt.

Bei einem weiteren Ereignis auf der Map, kann sich der Spieler eine von drei neuen Karten aussuchen, die er noch nicht besitzt. Das gleiche Event ist auch zusehen, nachdem der Spieler einen Kampf gewonnen hat. Damit der User weniger denken muss, wird hier dasselbe Prinzip verwendet wie beim Shop, als auch beim Rucksack, nämlich muss der Spieler seine ausgewählte Karte in sein Deck oder den Rucksack ziehen. Weiters besteht die Option auch keine Karte auszuwählen, wird dem Spieler allerdings nicht empfohlen, daher ist der Button für diese Entscheidung etwas versteckt und rot markiert, um zu signalisieren, dass dies vermutlich keine gute Entscheidung sei.

Sehr ähnlich aufgebaut ist das Heil Event. Im gleichen Stil aufgebaut ist ein Fenster zu sehen, nachdem man das Ereignis betreten hat, indem sich der Spieler zwischen zwei Heilungsoptionen entscheiden kann. Allerdings zieht hier der Spieler nichts in sein Deck oder in den Rucksack, sondern muss sich zwischen zwei Auswahlmöglichkeiten entscheiden. Daher ist es Usability freundlicher, einen Button zu haben, der die Auswahl bestätigt, den Spieler heilt und ihn wieder zurück zur Map bringt. Dem Spieler muss mitgeteilt werden, dass er bei diesem Event die zwei Heilungsoptionen anklicken kann, um eines davon auszuwählen. Daher ist ein
\quoted{Hover}- und \quoted{Active} Zustand vorhanden.

Das Letzte Event ist auch ein Heilevent, allerdings hat der Spieler hier keine mehreren Auswahlmöglichkeiten, sondern nur eine Option. Deswegen hat der Spieler hier keine Möglichkeit etwas auszuwählen, sondern nur einen Button, um sich zu heilen und zur Map zurückzukehren.

\begin{figure}[H]
    \centering
    \includegraphics[width=1.0\textwidth]{mapEventsUI.png}
    \caption{Screenshot aus dem Mockup der Map Events}
\end{figure}

Für einen Wiedererkennungswert ist es wichtig für Einheit im Design zu sorgen. Die Hintergrundtexturen werden weiterverwendet und die Buttons sind wie im Screenshot zu sehen die gleichen. Weiters wird der Hintergrund von jedem Map Event abgedunkelt, damit der Fokus auf das Pop Up liegt, und nicht mit dem Hintergrund verschwimmt. Auch Erklärungstexte oder zusätzliche Informationen zu den Jeweiligen Events sind immer am Ende der Pop Ups zu finden. Alle diese kleinen Merkmale sorgen dafür, dass der Spieler nicht nachdenken muss und sofort erkennen kann, was passiert.

\subsection{Popups}

Der Begriff Popup stammt aus der Programmierung von Computersoftware. Ein \quoted{Pop-up} ist letztendlich ein Fenster oder eine grafische Benutzeroberfläche, die plötzlich auf dem Bildschirm erscheint. Diese Pop-ups können verschiedene Zwecke haben, wie beispielsweise Eingabeaufforderungen, Benachrichtigungen, Warnungen oder für sonstige Informationen. Meistens haben diese nur die Optionen
\quoted{Ja} und \quoted{Nein},
oder wenn das Pop-up nur eine Mittelung sei, nur einen
\quoted{Ok} Button (Wikipedia, Dialog (Benutzeroberfläche), 2024).

Auch in Videospielen ist dieser Begriff vertreten. In \FF gibt es Pop-ups im Tutorial des Spiels, bei sämtlichen Events auf der Map und nach Kämpfen. Beispielsweise ist im folgenden Screenshot das \quoted{Card Extraction Pop-up} zusehen. Dieses Pop-Up teilt dem Spieler mit, dass die in ein der vorherigen Road gesammelten Karten gespeichert wurden, da der Spieler es in die nächste Area geschafft hat, gespeichert wurden. Mit dem Symbol oben auf dem Pop-Up versteht der User schneller was für einen Screen er gerade sieht, und merkt sich für die Zukunft, dass dieser blaue Farbton für das Speichern der Karten in \FF verwendet wird.

Grundsätzlich ist jedes Pop-Up gleich aufgebaut. Ein schwarzer Texturierter Hintergrund, welcher mit Photoshop erstellt wurde, einem Titel, einem kurzen Erklärungstext auf der Unterseite, und einem Button, um das Fenster wieder zu schließen. Die Form der drei Flächen – schwarz, blau und weiß – wurden in Photoshop mit dem \quoted{Sketching Brush} Pinsel erstellt und danach mit dem
\quoted{Mad Ink} Pinsel in der Farbe Weiß texturiert. Dafür müssen die Pixel der Ebene in Photoshop fixiert werden, damit man nur innerhalb der Pixel der Ebene malt und nicht außerhalb. Danach stellt man den Pinsel auf 3\\% Deckkraft und 70\\% Fluss und fährt mit ihm einmal über die Fläche. Diese Methode bei der Erstellung von jedem Pop-Up angewendet. Der blauen Fläche wurde noch ein blau leuchtender Schlagschatten hinzugefügt, damit das Fenster etwas mehr an Tiefe bekommt und nicht mehr so flach und leblos wirkt.

\begin{figure}[H]
    \centering
    \includegraphics[width=0.7\textwidth]{cardExtraction.png}
    \caption{Card Extraction Pop Up}
\end{figure}

Eine weitere Art von Pop-Ups in \FF ist ein Fenster, in das der Spieler Karten ziehen muss, da dieser bereits zu viele Karten auf der Hand hat und welche ablegen, beziehungsweise zurück ins Deck legen muss. Da der Spieler vom Kampf es gewöhnt ist, die Karten in den Revolver zu ziehen, ist es hier am intuitivsten, den Spieler die Karten, die abzugeben sind, in eine Fläche zu ziehen. Dieser wird wie auf Folgendem Screenshot aus dem Mock-Up als Deckstapel angezeigt, damit der Spieler weiß, wohin die Karten gehen. Im Fall, dass der Spieler trotzdem nicht weiß was zu tun ist, stehen kurze Erklärungstexte auf dem Pop-Up. Letztendlich gibt es rechts unten einen Button, um die Auswahl an zurückgelegten Karten zu bestätigen, damit der Spieler nicht aus Versehen eine Karte abgibt, die er eigentlich noch behalten wollte.

\begin{figure}[H]
    \centering
    \includegraphics[width=0.6\textwidth]{encounterPopup.png}
    \caption{Pop-Up während des Kampfes}
\end{figure}

Ein Beispiel für ein Pop-Up, welches nur für Informationen dient, sind die sogenannten \quoted{Hover Details}. Für \FF ist es essenziell zu wissen, wie die Karten funktionieren und was ihre Effekte sind, daher müssen diese Informationen schnell zugänglich sein. Wenn der Spieler mit seinem Mauszeiger über eine Karte fährt, erscheint nach kurzer Zeit ein Pop-Up, welches nur aus Text besteht, welches die Karte beschreibt und dessen Effekte.

\begin{figure}[H]
    \centering
    \includegraphics[width=0.6\textwidth]{hoverdetails.png}
    \caption{Beschreibung der Karte während Hover-Zustand}
\end{figure}

Während einem Kampf kommt es auch zu gegnerischen Angriffen, die besondere Auswirkungen auf den Kampf haben können, wie beispielsweise die Hexe, die die Revolver Drehrichtung des Spielers für die nächsten \quoted{x} Rotationen ändert. Damit der User das mitbekommt, erscheinen kleine Zeichnungen nacheinander, die im Comic Stil die Animation der Hexe zeigen. Dieses Pop-Up kommt zusätzlich zu den Grafiken mit einer Fläche, die die Aktion des Gegners erklärt und dessen Dauer anzeigt. Die Animation verschwindet nicht von selbst, sondern verschwindet erst nach einem Mausklick, um sicherzustellen, dass der Spielende die Information, die für den Kampf wichtig ist, mitnimmt.

\begin{figure}[H]
    \centering
    \includegraphics[width=0.6\textwidth]{enemyAction.png}
    \caption{Animation eines gegnerischen Angriffs}
\end{figure}

\subsection{Settings Menu}

Sobald ein Spieler ein Spiel gekauft hat, ist es wahrscheinlich, dass er seine ersten Erfahrungen in den Einstellungen macht, bevor er überhaupt spielen wird. Für PC-Spieler ist die Überprüfung der Einstellungen, um sicherzustellen, dass das Spiel ordnungsgemäß läuft, wahrscheinlich der erste Schritt, bevor das neue Spiel gestartet wird. Ebenso können Streamer und YouTuber das Spiel im Voraus vorbereiten, einschließlich Lautstärke, um sicherzustellen, dass ihr Publikum den besten ersten Eindruck von dem Spiel bekommt (Giguere, 2024). Während der Entwicklung von \FF war es noch unklar, wie groß das Spiel letztendlich wird. Davon ist auch die Größe der Einstellungen abhängig. Daher wurde ein Einstellungen Menu mit Ankerpunkten vorausgeplant, falls die Einstellungen doch größer werden sollten. Das System von Ankerpunkten in den Einstellungen ist in vielen Spielen vertreten und hilft dem Spieler, schneller durch die Optionen zu navigieren. Beispielsweise verwendet das Videospiel \quoted{Armored Core IV: Fires of Rubicon} das selbe Layout wie das Einstellungen Menu von \FF.

\begin{figure}[H]
    \centering
    \includegraphics[width=0.8\textwidth]{settingsMenu.png}
    \caption{Settings Menu mit Ankerpunkten}
\end{figure}

Das gleiche Layout wurde in \FF auch implementiert, allerdings im Stil aller anderen UI-Elemente und beschränkt sich auf die Einstellungen, die programmiertechnisch bereits implementiert sind. Das Einstellungen Menu beinhaltet Audio Slider, welche Gesamtlautstärke, Musiklautstärke und Soundeffekte regulieren. Zusätzlich gibt es noch QOL (Quality of Life) Einstellungen, wie der \quoted{Show Screen Shake} Effekt bei einem Revolverschuss und eine Auswahl mit welchem Fenster das Spiel gestartet werden soll.

\begin{figure}[H]
    \centering
    \includegraphics[width=0.8\textwidth]{settingsMockup.png}
    \caption{Settings Menu Mock-Up}
\end{figure}

\section{Bullet Designs}

Für ein Kartenspiel sei es nicht gewöhnlich, dass die Spielkarten, Kugeln sind und überhaupt, dass diese Karten in einen Revolver geladen werden. Das muss alles beim Entwerfen der Spielkarten von \FF berücksichtigt und im Hinterkopf behalten werden. Da \FF ein Setting im wilden Westen hat, wurde sich von alten Spielkarten aus dem wilden Westen Inspiration genommen. Im Laufe der Diplomarbeit wurde ein Guidelines Dokument erstellt, welches diese Inspiration erfasst. Beispielsweise zeigt folgende Collage an Bildern eine Idee, wie die Bullets in \FF gestaltet wurden.

\begin{figure}[H]
    \centering
    \includegraphics[width=0.8\textwidth]{bulletMoodboard.png}
    \caption{Moodboard Ausschnitt der Bullet Guidelines}
\end{figure}

Spielkarten weisen oft die Symbole \quoted{Pik} und \quoted{Karo} auf. Diese gehören zu den vier Farbsymbolen, die in Kartenspielen vorkommen können. In Kartenspielen werden diese Symbole verwendet, um die verschiedenen Karten zu kategorisieren und einzuteilen. Diese Symbole hatten damals bei der Entwicklungsphase von \FF noch eine Bedeutung. Es waren verschiedene Kategorien an Karten geplant, wurde jedoch die Bedeutung der Symbole aufgrund dem Game Design ausgelassen. Die Bullets weisen verhältnismäßig große abgerundete Kanten auf, um den Comic Stil des Spiels zu betonen. Daher wurden auch alle Bullets in einem Comic Stil gezeichnet. Das bedeutet, dass schwarze Outlines auf dem Design der Bullet an sich selbst sind, und die Bullet Designs hauptsächlich simpel gehalten sind. Letztendlich wurde sich für folgendes Design entschieden, als Vorlage für die nachkommenden Karten.

\begin{figure}[H]
    \centering
    \includegraphics[width=0.4\textwidth]{bullet.png}
    \caption{Artwork der normalen Bullet}
\end{figure}

\begin{figure}[H]
    \centering
    \includegraphics[width=0.4\textwidth]{finalBullet.png}
    \caption{Artwork der Final Bullet}
\end{figure}

\subsection{Bullet Photoshop Template}

Für ein Kartenspiel ist es üblich, dass dies hunderte, wenn nicht sogar tausende von Karten beinhalten. Diese zu designen, hat einen extrem hohen Zeitaufwand. Daher ist es besonders wichtig, einen guten Workflow zu haben und dafür ist ein Template essenziell. Einerseits wurde sich eine sinnvolle Ordnerstruktur für die hunderten Designs der Karten überlegt, welche dabei hilft, Ordnung zu halten, um schnell auf Bullets zugreifen zu können und um diese zu exportieren. Das Photoshop Template verfügt über mehrere Ordner, welche die Bereiche einer Karte zuständig sind. Beispielweise sind alle Texte, die auf einer Karte vorkommen, in dem grünen Ordner \quoted{Card Name} oder sind mit der Farbe Grün gekennzeichnet. Der Frame beziehungsweise die Form der Karte wurde mittels einer Quadrat Form mit abgerundeten Ecken erstellt. Diese Sorgt dafür, dass man nur innerhalb des Rahmens beziehungsweise der Karte zeichnen kann, und nicht aus Versehen rauszeichnet. Für den Namen der Bullets wurde eine Schriftart ausgewählt, welche einen rustikalen westlichen Touch hat. Serifen passen dazu besonders gut, da Serifen hauptsächlich in alten Schriftfamilien vorkommen und auch damals schon im wilden Westen verwendet wurden. Letztendlich wurde sich für die Schriftart
\quoted{Card Characters} entschieden, welche kostenlos und lizenzfrei im Internet zu finden ist, wie auf der Seite \url{https://www.fonts4free.net/}. Wenn Zeichen zu nah aneinander liegen, können Texte aus weiterer Entfernung oder niedriger Auflösung schwerer lesbar sein. Da die Spielkarten – wenn sie in der Hand während eines Kampfes sind – relativ klein sind, musste der Zeichenabstand als auch der Zeilenabstand auf die Größe der Karten angepasst werden, damit die Texte lesbarer sind. Für das Erstellen der meisten Bullets, wurden in dem Template die Ordner Frame, Symbols und Background nicht geändert. Einzelne Karten mit Besonderheiten, wie beispielsweise besonderen Effekten oder Karten, die erst spät im Spiel freigeschaltet werden können, erhalten zum Beispiel einen anderen Hintergrund, damit diese sich von den anderen Bullets abheben.

\begin{figure}[H]
    \centering
    \includegraphics[width=0.6\textwidth]{bulletTemplate.png}
    \caption{Ebenen Struktur der Bullet Template}
\end{figure}

\subsection{Designvorgang einer Bullet}

Wenn für eine neue Karte nur die Farbe oder die Textur der Bullet geändert werden muss, erfolgt das mit dem Template sehr schnell. Es wird im Ordner \quoted{Bullet} die Ebene \quoted{Body} ausgewählt – zuständig für den Körper der Bullet Grafik – und ändert den Farbton der Bullet. Entweder mit dem Tastenkürzel
\quoted{STRG+U} oder über manuelle Suche das Fenster
\quoted{Farbton/Sättigung öffnen} geöffnet, worin mit einem Regler der Farbton geändert werden kann. Abschließend wird noch der Name der Karte
im \quoted{Card Name} Ordner geändert.

\begin{figure}[H]
    \centering
    \includegraphics[width=0.6\textwidth]{farbtonRegler.png}
    \caption{Farbton/Sättigung Regler in Photoshop}
\end{figure}

Jedoch ist nicht jede Bullet einfach zu designen. Beispielsweise ist die \quoted{Eclipse Bullet} bei weitem komplizierter zu erstellen als eine andere Bullet, bei der sich mehr oder weniger nur die Farbe ändert. Eine Eclipse ist im deutschen übersetzt eine Sonnenfinsternis. Das Ziel war es, eine Bullet zu erstellen, die wie eine Sonnenfinsternis aussieht.  Als erstes wird eine Kopie des Templates in einem neuen Ordner
\quoted{Eclipse Bullet} erstellt. Danach wird ein passender Hintergrund für diese Karte ausgesucht. Um eine Sonnenfinsternis zu imitieren, muss die Karte Schwarz sein. Mit den Sternen im Hintergrund der Karte wird noch einmal betont, dass der Hintergrund das Weltall repräsentieren soll. Da diese Bullet einen besonderen Effekt haben wird, ist es gerechtfertigt, dass diese einen einzigartigen Hintergrund bekommt und somit von den anderen Karten hervorgehoben wird und als etwas besonderes erscheint. Für die Bullet an sich selbst, wurde zuerst eine schwarze Silhouette der Bullet erstellt, indem die
\quoted{Body}
Ebene ausgewählt wird und schwarz angemalt wird. Daraufhin wird eine Kopie dieser Ebene erstellt, welche direkt darüber liegt. Diese Ebene bekommt einen orangenen direktionalen Schlagschatten, welches über das Fülloptionen Menü in Photoshop und einer Ebenenmaske mit einem Verlauf von Schwarz auf Weiß erfolgt. Das bedeutet, dass der Schlagschatten an einem Ende der Bullet abnimmt, wie auf der zweiten Abbildung zu sehen ist. Im dritten Schritt wird ein innerer Schatten der Bullet hinzugefügt, was auch über die Fülloptionen mit der Funktion
\quoted{Innerer Schatten}
funktioniert. Das sorgt für einen realistischeren Glow Effekt auf der Bullet. Im letzten Schritt wird ein Linseneffekt eingebaut, welcher die Sonne, die hinter der Bullet ist, repräsentiert. In Photoshop kann unter dem Reiter
\quoted{Filter} und \quoted{Renderfilter} einBlendenfleck hinzugefügt werden. Dort sind verschiedene Objektivarten auszuwählen, die verschiedene Blendenflecke erzeugen.
Die Option \quoted{35mm}
sieht am ehesten aus wie eine Sonne. Dieser Linseneffekt wurde noch an den Rand der Bullet platziert, was es so wirken lässt, als würde eine Sonne hinter der Bullet leuchten. Anschließend wird die Bullet in der Auflösung 500x500 Pixel als PNG (Portable Network Graphics) exportiert, damit die abgerundeten Ecken erhalten bleiben.

\begin{figure}[H]
    \centering
    \includegraphics[width=1.0\textwidth]{eclipseBullet.png}
    \caption{Aufbau der Eclipse Bullet in vier Schritten}
\end{figure}












\section{Artstyle}\label{sec:artstyle}

\renewcommand{\kapitelautor}{Autor: Philip Jankovic}

\FF in game visuals sind handgezeichnet und oft statisch. Beim Zeichnen wurde zuerst eine "Reference" gesucht.


\subsection{References}\label{subsec:references}
References beim Zeichnen sind Bilder oder andere Artworks an denen sich orientiert wird. Vorallem am Anfang ist das
Arbeiten mit References fast ein Muss, da das einfache Vorstellen der Zeichnung im Kopf und das anschließende Zeichnen davon höchst anspruchsvoll ist.
Die Perspektive, Kleidung und Körperposen können von der Reference inspieriert werden.
Jedoch wird versucht die Reference nicht 100\% abzuzeichnen, sondern auch den eigenen Stil hinzufügen. \zit{referance}


Für \FF wurden References von "Pinterest", jedoch auch Inspirationen aus diversen Filmen und anderen Medien gesucht. \zit{pinterest}


Ein Beispiel für die Nutzung verschiedenster References ist der Manga \quoted{JOJO's Bizzzar Adventure} von Hirohiko Araki\zit{jojo},
welcher auch eine Inspiration für einige Teile von \FF ist. Nicht nur im Bezug auf die Zeichnungen, sondern auch auf die von dem Autor benützen Krativfindungsmethoden.
Mehr dazu kann in dem Kapitel \ref{methoden} gelesen werden.
Araki verwendet eine große Auswahl als References für seine Zeichnungen, vorallem aus der Welt
der Fashion, was auch in seinem Artstyle zu sehen ist.


\FF bezieht euch ein paar References von "JOJO's Bizzzar Adventure", jedoch ist die Auswahl der benützen References groß und abweichslungsreich.
Auch eigene selbst aufgenomme Reference-Bilder wurden verwenedet für \zB den Main-Charackter des Spieles.



Bei dem Nutzen einer Reference muss jedoch darauf geachtet werden, das Bild nicht nur abzuzeichnen. Je nach Zeichenstil
oder wie stilisiert ein Zeichenstil ist, besteht eine Gefahr, ein Bild nur abzuzeichnen, anstatt es als Reference zu verwenden.
Das Bewahrens eines eigenen Zeichenstiles ist dabei besonders wichtig.


\subsection{Artsytle von \FF}\label{subsec:artsytle}

Der Artstyle von \FF ist im Comic-stil gehalten, mit groben, Bleistift Outlines und Coloring auf der Ebene darunter.
Coloring wird mit einem Farbpinsel auf 100\% Fluss gemacht, damit die Kanten schön hart sind und es sich besser von dem Hintergrund abheben.
Auch wenn die Zeichnungen selber nicht realistisch gezeichnet sind, halten sie sich an realistische Posen und Propertionen, sind also nicht wirklich stilisiert.
Das Spiel nimmt sich dadurch nicht zu ernst, passt aber gut zu dem rauen Wild-West Setting und schlägt für den Betrachter trotz all dieser komischen Bullets
einen Anker in der Realität.

\begin{figure}[H]
    \centering
    \includegraphics[width=0.8\textwidth]{artstylepic.jpg}
    \caption{Beispiel: Artstyle von \FF}
\end{figure}

\begin{figure}[H]
    \centering
    \includegraphics[width=0.8\textwidth]{outline.jpg}
    \caption{Beispiel: Verwendeter Ouline Brush}
\end{figure}



Um den Zeichnungen mehr Tiefe zu geben wird über der Color-Ebene eine Muliplate-Ebene auf 50\% Oppacity verwendet um Schatten
darzustellen. Geshaded wird mit dem selben Pinsel mit welchem Gecolored wird. Das Shading bleibt bei diesem einen grauen ton,
härtere shadows werden durch "Crosshatching" mit dem Outline-Bleistift gemacht.\zit{crosshatching}

\begin{figure}[H]
    \includegraphics[width=0.4\textwidth]{crossha.jpg}
    \caption{Beispiel: Crosshatching in \FF}
\end{figure}

\renewcommand{\kapitelautor}{}

%
\chapter{Steam Release}\label{ch:steamrelease}


\appendix



\chapter{Anhang 1 - Erklärung Bytecode Instruktionen}\label{ch:anhang-1}

\renewcommand{\kapitelautor}{Autor: Marvin Kurka}

Zuerst, hier nochmal der Code-Block:

%! language = kotlin
\begin{minted}{kotlin}
fun runLambda(lambda: () -> Unit) = lambda()
inline fun runLambdaInline(lambda: () -> Unit) = lambda()

fun test() {
    runLambda {
        println("runLambda")
    }
    runLambdaInline {
        println("runLambdaInline")
    }
}
\end{minted}

Und der Byte-Code:

\begin{minted}{text}
 0: getstatic     #34                 // Field TestKt$test$1.INSTANCE:LTestKt$test$1;
 3: checkcast     #18                 // class kotlin/jvm/functions/Function0
 6: invokestatic  #36                 // Method runLambda:(Lkotlin/jvm/functions/Function0;)V
 9: iconst_0
10: istore_0
11: iconst_0
12: istore_1
13: ldc           #37                 // String runLambdaInline
15: getstatic     #43                 // Field java/lang/System.out:Ljava/io/PrintStream;
18: swap
19: invokevirtual #49                 // Method java/io/PrintStream.println:(Ljava/lang/Object;)V
22: nop
23: nop
24: return
\end{minted}

Zwei Stellen im Bytecode ergeben auf den ersten Blick nicht wirklich Sinn, nämlich 9--12 und 22--23.
Diese sind hier erklärt.

\section{Stelle 1: 9-12}

\begin{minted}{text}
 9: iconst_0
10: istore_0
11: iconst_0
12: istore_1
\end{minted}

Hier schreibt der Compiler 0 in die Register 0 und 1, liest die Werte aber nie aus.
Ein Blick in den LVT der Funktion liefert einen Hinweis:

\begin{minted}{text}
LocalVariableTable:
    Start  Length  Slot  Name   Signature
    13      10     1 $i$a$-runLambdaInline-TestKt$test$2   I
    11      13     0 $i$f$runLambdaInline   I
\end{minted}

Der LVT oder Local Variable Table speichert Informationen zu den lokalen Variablen einer Funktion, die normalerweise
bei der Kompilation verloren gehen würden.
Er ist optional und nicht für die korrekte Ausführung des Codes notwendig, kann aber von
Debuggern verwendet werden.\cite{jvmspecLVT}

Da durch das inlinen der Funktion kein Frame am Callstack generiert wird, geht nützliche Information im Stacktrace
verloren.
Diese lokalen Variablen speichern Informationen zu aufgerufenen inline-Funktionen in ihren Namen, die dann von
Debugging-Tools ausgelesen werden können.\cite{youTrackFakeVariables}

\section{Stelle 2: 22-23}

Hier generiert der Compiler zwei nicht notwendige nop-Instruktionen.
Wie sich herausstellt, hat auch dieser Bytecode mit inline-Funktionen zu tun.
Unter gewissen Umständen, \zB wenn eine inline-Funktion nur eine andere inline-Funktion aufruft,
wird für die erste Funktion kein Bytecode generiert.
Das führt zu mehreren Problemen, zum Beispiel kann dadurch keine Assoziation zu einer Zeilennummer hergestellt werden,
da Informationen im LineNumberTable immer mit einem Bytecode-Offset verknüpft sind.
Außerdem führt das zu Problemen beim debuggen, da auch Breakpoints immer auf eine Instruktion im Bytecode verweisen.
Deswegen wird für eine inline-Funktion immer ein \inlineCode{nop} ausgegeben, um zu garantieren, dass
zumindest eine Instruktion für jede inline-Funktion ausgegeben wird.
Im oberen Code sind zwei \inlineCode{nop} Instruktionen vorhanden, da in Kotlin \inlineKotlin{println} auch eine
inline-Funktion ist.\cite{youTrackNops}

% resets author
\renewcommand{\kapitelautor}{}



%was auch immer: technische Dokumentationen etc.
%
%Zusätzlich sollte es geben:
%\begin{itemize}
%\item Abkürzungsverzeichnis
%\item Quellenverzeichnis (hier: Bibtex im Stil plaindin)
%\item optional: Akronyme und Glossar
%\end{itemize}

%% optional: Akronyme und Glossar
% kann man löschen falls kein Glossar gebraucht
%\printglossary[type=\acronymtype, title=Abkürzungsverzeichnis, toctitle=Abkürzungsverzeichnis]
%\printglossary[type=main, title=Glossar, toctitle=Glossar]

%\printindex{}

%% Flattersatz -- damit werden die langen URLs besser umgebrochen
\raggedright %% eventuell auskommentieren
%\bibliographystyle{plaindin}%Alternative unsrtdin - Nummern im Text aufsteigend
\bibliographystyle{abbrvnat}
\bibliography{diplom}


\cleardoublepage
\newcommand{\Messbox}[2]{%Parameters: #1=Breite, #2=Hoehe
\setlength{\unitlength}{1.0mm}%
\begin{picture}(#1,#2)%
\linethickness{0.05mm}%
\put(0,0){\dashbox{0.2}(#1,#2)%
{\parbox{#1mm}{%
\centering\footnotesize
%{\bf MESSBOX}\\
% if \textrm fails use \rm
Breite $ = #1 {\textrm\ mm}$\\
Höhe $ = #2 {\textrm\ mm}$
}}}\end{picture}
}

\end{document}
