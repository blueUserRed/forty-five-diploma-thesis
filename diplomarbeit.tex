
\documentclass[
    headings=optiontotocandhead,% Erweiterung für das optionale Argument der
                                % Gliederungsbefehle aktiviert.
    twoside,
    numbers=noenddot,% Keine Punkte am Ende der Gliederungsnummern und davon
                     % abgeleiteten Nummern
    toc=flat, %Flache TOC --- kann man anpassen (auskommentieren)
    12pt, % Schriftgröße
    titlepage, % es wird eine Titelseite verwendet
    parskip=full, % Abstand zwischen Absätzen (ganze Zeile)
    listof=totoc, % Verzeichnisse im Inhaltsverzeichnis aufführen
    listof=flat, % mehr Abstand für grosse Zahlen
    numbers=noenddot, % kein Punkt am Ende bei Nummern
    %%enlargefirstpage,% Gibt es bei scrartcl nicht!!!!
    bibliography=totoc, % Literaturverzeichnis im Inhaltsverzeichnis aufführen
    %index=totoc, % Index im Inhaltsverzeichnis aufführen
    %captions=tableheading, % Beschriftung von Tabellen für Ausgabe oberhalb
                           % der Tabelle formatieren
    %draft % Status des Dokuments (final/draft) draft hinzufügen zum anziegen
    %%der zeilen ende
    a4paper,DIV=14,
    BCOR=15mm,
% captions=tablesignature,
]{scrbook}


\setcounter{secnumdepth}{3}

%\usepackage[T1]{fontenc}
\usepackage[utf8]{inputenc}
\usepackage[english, ngerman]{babel, varioref} % Deutsch muss letztes sein
\usepackage{lastpage}
\usepackage{listings}
\usepackage{blindtext}
\usepackage[inline]{enumitem} %% Aufzählungen nicht so weit einrücken

% Listen etwas wenige einrücken, erfordert enumitem
\setitemize{leftmargin=*}

\usepackage{lmodern}
\usepackage{xspace}
\usepackage{graphicx}
\graphicspath{ {./images/} }
\usepackage{float}

\usepackage[hyphens]{url}

\usepackage{makeidx}
\makeindex

\usepackage{natbib}

\PassOptionsToPackage{normalem}{ulem}

\usepackage{ulem}
\usepackage{needspace}

\setlength\partopsep{0.5ex} % schoenere Listen

\usepackage[bottom]{footmisc} % fussnote ganz unten

\usepackage[]{microtype}
\UseMicrotypeSet[protrusion]{basicmath} % disable protrusion for tt fonts

%\usepackage{multirow}   % Allows table elements to span several rows.
%\usepackage{booktabs}   % Improves the typesettings of tables.
%\usepackage{subcaption} % Allows the use of subfigures and enables their referencing.
\usepackage[ruled,linesnumbered,algochapter]{algorithm2e} % Enables the writing of pseudo code.
\usepackage[usenames,dvipsnames,table]{xcolor} % Allows the definition and use of colors. This package has to be included before tikz.
\usepackage{nag}       % Issues warnings when best practices in writing LaTeX documents are violated.
%\usepackage{todonotes} % Provides tooltip-like todo notes.
\usepackage{color}
%\usepackage[binary-units]{siunitx}

% Definiert einen Bundsteg von 1.5cm
% NUR BEI BEIDSEITIGEN DRUCK!!
\usepackage{geometry}
\geometry{
    left = 2cm,
    right = 2.5cm,
    bindingoffset = 1.5cm,
}

%  Kopf und Fußzeilen -- links und rechts verschieden
\newcommand{\kopfseitenummer}{{\bfseries \thepage}}
\newcommand{\kopfkapl}{{\bfseries\leftmark}}
\newcommand{\kopfkapr}{{\bfseries\rightmark}}
\newcommand{\kopfbild}{\voffset7mm\includegraphics[width=25mm]{HTL3RLogoRGB}}
\newcommand{\kopfHTL}{Höhere Technische Bundeslehranstalt Wien 3, \\Rennweg 	Abteilung für Informationstechnologie}

\usepackage{fontspec}
\usepackage{scalefnt}

% setzt Schriftart für Fließtext
% muss installiert sein
\setmainfont{Aptos}

% setzt Schriftart für Code-Blöcke
% muss installiert sein
\newfontfamily\codefont{JetBrainsMono-Regular.ttf}[NFSSFamily=JetBrainsMonoFamily]



\usepackage[automark,headsepline,footsepline,plainfootsepline]{scrlayer-scrpage}
\setkomafont{pageheadfoot}{\normalcolor\footnotesize\scshape}
\setkomafont{pagenumber}{\normalfont\normalsize}
\clearpairofpagestyles
\ihead{\voffset7mm\includegraphics[width=35mm]{logo_red}}
%\ihead{\headmark}
\ohead{\kopfbild}
\ifoot{\kapitelautor}
\ofoot{\pagemark}
\ModifyLayer[addvoffset=-.6ex]{scrheadings.foot.above.line}% Linie verschieben
\ModifyLayer[addvoffset=-.6ex]{plain.scrheadings.foot.above.line}% Linie verschieben
\setlength{\headheight}{32pt}

% alle Seiten mit Kopfzeile
\renewcommand{\chapterpagestyle}{scrheadings}


\usepackage{minted}
\usemintedstyle{forty_five_style}
% Konfig für Code-Blöcke
\setminted{
    frame=lines,
    framesep=2mm,
    breaklines=true,
    fontfamily=JetBrainsMonoFamily,
    fontsize=\scriptsize
}

%\usepackage{awesomebox}
%\setlength{\aweboxleftmargin}{1pt}

%\usepackage{scrhack}

%% glossar
% kann man löschen falls kein Glossar gebraucht
%\usepackage[acronym, toc]{glossaries}
%\makeglossaries
%\input{text/glossar.tex}

\usepackage{tcolorbox}
\tcbuselibrary{xparse,skins,breakable}
\definecolor{htl3red}{RGB}{255,51,0}
\newtcolorbox{TitlePageBox}{%
    breakable,
    blanker,
    left=1em,
    borderline west={0.15cm}{3pt}{htl3red},
}

\usepackage[unicode=true,
    bookmarks=true,bookmarksnumbered=false,bookmarksopen=false,
    breaklinks=true,pdfborder={0 0 0},backref=false,colorlinks=false]
{hyperref}
\hypersetup{
    pdftitle={.Forty-Five},
    pdfauthor={Markus Böheim, Nils Hubmann, Philip Jankovic, Marvin Kurka, Felix Zwickelstorfer},
    pdfsubject={Diplomarbeit},
    pdfkeywords={Forty-Five, Card-Game, card game, wild west, Wild-West, open source, western game, bullets, revolver}
}
\urlstyle{same} % don't use monospace font for urls

% Auch Fußnoten bündig ausrichten
\deffootnote[]{1em}{1em}{\textsuperscript{\thefootnotemark\ }}
\sloppy % weniger Meldungen
\voffset7mm % etwas nach unten

%% schöner: 10000 -- gar keine, 1000 als Mittelweg
\clubpenalty = 10000 % Schusterjungen verhindern
\widowpenalty = 10000 % Hurenkinder verhindern
\displaywidowpenalty = 10000


%%%%%%%%%%%%%%%%%%%%%%%%%%%%%%%%%%%%%%%%%%%%%%%%%%%%%%%%%%%%%%%%%%%%%%%%%%%%%%%%%%
\begin{document}

\shorthandoff{"}
%% mit kapitelautor kann man den Autor festlegen oder auf leer setzen - steht dann in der Fußzeile.
%% bitte immer (gleich) nach der Überschrift setzen, nicht vorher -- sonst steht es bei Kapiteln eventuell eine Seite zu früh
\newcommand{\kapitelautor}{}

\newcommand{\bold}[1]{\textbf{#1}}
\newcommand{\italic}[1]{\emph{#1}}
\newcommand{\code}[1]{\texttt{#1}}

% einfaches "siehe ..." - das Ziel muss man markieren mit \label{name} -- macht pandoc automatisch
% einfache Variante
%\newcommand{\kap}[1]{Kapitel~\ref{#1}, Seite~\pageref{#1}}
%\newcommand{\siehe}[1]{siehe \kap{#1}}
%\newcommand{\abb}[1]{Abbildung~\ref{#1}, Seite~\pageref{#1}}
% bessere Variante - braucht varioref
\newcommand{\kap}[1]{Kapitel~\vref{#1}}
\newcommand{\siehe}[1]{siehe \kap{#1}}
\newcommand{\abb}[1]{Abbildung~\vref{#1}}


%% http://ieg.ifs.tuwien.ac.at/~aigner/download/tuwien.sty
%Div. Abkürzungen (in Anlehnung an Jochen Köpper, jkthesis):
%\RequirePackage{xspace}
\newcommand{\bzw}{bzw.\@\xspace}
\newcommand{\bzgl}{bzgl.\@\xspace}
\newcommand{\ca}{ca.\@\xspace}
\newcommand{\dah}{d.\thinspace{}h.\@\xspace}
\newcommand{\Dah}{D.\thinspace{}h.\@\xspace}
\newcommand{\ds}{d.\thinspace{}s.\@\xspace}
\newcommand{\evtl}{evtl.\@\xspace}
\newcommand{\ua}{u.\thinspace{}a.\@\xspace}
\newcommand{\Ua}{U.\thinspace{}a.\@\xspace}
\newcommand{\usw}{usw.\@\xspace}
\newcommand{\va}{v.\thinspace{}a.\@\xspace}
\newcommand{\vgl}{vgl.\@\xspace}
\newcommand{\zB}{z.\thinspace{}B.\@\xspace}
\newcommand{\ZB}{Zum Beispiel\xspace}
\newcommand{\FF}{.Forty-Five\@\xspace}
\newcommand{\ff}{.Forty-Five\@\xspace}

%% https://github.com/Digital-Media/HagenbergThesis
\newcommand{\latex}{La\-TeX\xspace} % kein schnoerkeliges LaTeX mehr
\newcommand{\tex}{TeX\xspace}       % kein schnoerkeliges TeX mehr
\newcommand{\bs}{\textbackslash}    % Backslash character
\newcommand{\obnh}{\hskip 0pt } %optional break without hyphen: e.g. PlugIn{\obnh}Filter

\newcommand{\sa}{s.\ auch\@\xspace}
\newcommand{\so}{s.\ oben\xspace}
\newcommand{\su}{s.\ unten\@\xspace}

\newcommand{\uae}{u.\thinspace{}\"A.\@\xspace}
\newcommand{\uva}{u.\thinspace{}v.\thinspace{}a.\@\xspace}
\newcommand{\uvm}{u.\thinspace{}v.\thinspace{}m.\@\xspace}

\newcommand{\inlineCode}[1]{\mintinline{text}{#1}}
\newcommand{\inlineKotlin}[1]{\mintinline{kotlin}{#1}}
\newcommand{\inlineOnj}[1]{\mintinline{onj}{#1}}
\newcommand{\inlineGlsl}[1]{\mintinline{glsl}{#1}}
\newcommand{\inlineJava}[1]{\mintinline{java}{#1}}
% \citeauthor \citeyear
%\newcommand{\zit}[1]{ (vgl. \cite{#1})}
\newcommand{\zit}[1]{ (vgl. \citeauthor{#1} \citeyear{#1} [online])}
\newcommand{\zitbuch}[1]{ (vgl. \citeauthor{#1} \citeyear{#1})}
%\newcommand{\zitt}[2]{(\cite{#1, #2})}
%\newcommand{\zittt}[3]{(\cite{#1, #2, #3})}
%\newcommand{\zitttt}[4]{(\cite{#1, #2, #3, #4})}

\newcommand{\quoted}[1]{\frqq#1\flqq}
\newenvironment{coolQuote}{\begin{quote}\itshape\frqq}{\flqq\end{quote}}

\newcommand{\zid}[1]{(\citeauthor{#1} \citeyear{#1} [online])}
\newcommand{\zidbuch}[1]{(\citeauthor{#1} \citeyear{#1})}

\newcommand{\codeblockCaption}[1]{#1} % might be changed later

\newenvironment{liste}{\begin{itemize}\setlength{\itemsep}{1pt}\setlength{\itemsep}{0pt}\setlength{\parsep}{0pt}}{\end{itemize}}

\newenvironment{infoBox}{\begin{awesomeblock}[blue]{3pt}{}{magenta}}{\end{awesomeblock}} % todo: make look good

\newenvironment{codeBlock}[2]
{\VerbatimEnvironment\begin{figure}[H]\def\myenvargumentII{#2}\centering\begin{minted}{#1}}
{\end{minted}\caption{\myenvargumentII}\end{figure}}



\frontmatter % Switches to roman numbering
\title{Diplomarbeit}

\begin{titlepage}
\begin{minipage}[b]{1\columnwidth}
\parbox[b]{99mm}{
\begin{TitlePageBox}
\footnotesize% klein
\textsf{% und ohne Rifen
\textbf{\textsc{Höhere Technische Bundeslehranstalt} Wien 3, Rennweg}\\
\\
Höhere Abteilung für Mechatronik\\
Höhere Abteilung für Informationstechnologie\\
Fachschule für Informationstechnik}
\end{TitlePageBox}
}\hfill\parbox[b]{50mm}{\includegraphics[width=51mm]{HTL3RLogoRGB}}
\mbox{}
\end{minipage}

\vspace{1cm}


\begin{center}

\textbf{\LARGE{}Diplomarbeit}{\large{}}\\ % todo: figure out what is going on with this line
{\large{}\vspace{15mm}
 }\textbf{\large{}\todo{TODO}eventuell KURZTITEL}\\
\textbf{\large{}Ausgeschriebener Titel der Diplomarbeit}\\

 \vfill

 ausgeführt an der\\
 Höheren Abteilung für Informationstechnologie/Medientechnik\\
 der Höheren Technischen Lehranstalt Wien 3 Rennweg\\

 \vfill
 im Schuljahr 2023/2024\\

\vspace{1cm}

{
\renewcommand{\arraystretch}{1.8}
\begin{tabular}{l c r}
durch  & \hfill & unter Anleitung von \\
\textbf{\large{}Böheim Markus} && Nussbaumer Vincent \\
\textbf{\large{}Hubmann Nils} && Sturm Gerhard \\
\textbf{\large{}Jankovic Philip} && Nussbaumer Vincent \\
\textbf{\large{}Kurka Marvin} && Weiss Florian \\
\textbf{\large{}Zwickelstorfer Felix} && Weiss Florian \\
\end{tabular}
}

\vfill

Wien, \today
\par\end{center}

\end{titlepage}

\chapter*{Kurzfassung}

%
Da das Interesse an Videospielen stetig steigt, wachst der Videospielmarkt nach wie vor. Es ist nichtmehr nur eine Unterhaltungsmöglichkeit, sondern eine eigene Welt.
Es gibt eine Vielzahl an Menschen, welche täglich Videospiele spielen und darin eine Flucht aus der Realität sehen.

Ziel des Diplomarbeitsprojekts ist es, ein rogulite Cardgame zu entwickeln, welches über Steam gedownloadet werden kann und auf Windows Systemen spielbar ist.
Der Spieler durchlebt mehrere Runs und bekämpft Gegner mit der Hilfe von Spielkarten auf seinem Weg.
Die Map, die erkundet wird, ist geprägt von: Gegnern die zu bekämpfen sind, Shops in denen neue Bullets erworben werden, und Heilevents die den Spieler heilen.
Ziel ist es alle Kämpfe zu bestreiten und sich ein Deck zu bauen, dass allen Herausforderungen standhält.

Der Hintergedanke besteht darin Spieler für Kartenspiele zu begeistern und zu Menschen zu animieren ihre Ideen zu verfolgen. Es soll für Unterhaltung sorgen und die Genres Wilder Westen und Kartenspiele kombinieren.
%


\blindtext[1]


\chapter*{Abstract}
\selectlanguage{english}

Because the interest in video games is constantly rising, the video game market is growing as well.
Video games come in all kinds of different genres and are mainly used for entertainment.
The goal of the diploma project was to develop a virtual card game, which was published via Steam and can be played on
Windows Systems.
The player experiences multiple runs and fights enemies with the help of the playing cards, which they collected or bought
along the way.
The map, which can be explored by the player, is characterized by: Fights against various enemies, Shops that offer unique
cards or Events where the player can heal themselves.
The goal of the game is to win fight after fight and create a deck that withstands all enemies.
The primary focus of the game is entertainment, but also to further creative and logical thinking.

\selectlanguage{ngerman}

\chapter*{Ehrenwörtliche Erklärung}

Hiermit versichere ich, dass ich die vorliegende Arbeit selbstständig verfasst und keine anderen Hilfsmittel als die angegebenen benützt habe. Die Stellen, die anderen Werken (gilt ebenso für Werke aus elektronischen Datenbanken oder aus dem Internet) wörtlich oder sinngemäß entnommen sind, habe ich unter Angabe der Quelle und Einhaltung der Regeln wissenschaftlichen Zitierens kenntlich gemacht. Diese Versicherung umfasst auch in der Arbeit verwendete bildliche Darstellungen, Tabellen, Skizzen und Zeichnungen.

\begin{flushleft}
\bigskip{}
Wien, am \today \\
\newcommand{\namesigdate}[2][8cm]{
\vspace{2cm}~\newline
\parbox{#1}{\hrule\centering #2\Large\strut}
\hfill
}
%\namesigdate{Mitarbeiter:in Eins}
%\namesigdate{Mitarbeiter:in Zwei}
%\namesigdate{Mitarbeiter:in Drei}
<eigenhändige Unterschriften aller Teammitglieder>
\par\end{flushleft}



%%%%%%%%%%%%%%%%%%%%%%%%%%%%%%%%%%%%%%%%%%%%%%%%%%%%%%%%%%%%%%%%%%%%%%%%%%%%%%%%%%%%%%%%
\cleardoublepage{}
\tableofcontents{}
\cleardoublepage{}
\listoftables
\todo{kann entfallen falls (fast) leer}
\cleardoublepage{}
\listoffigures


%hier geht es los mit dem Text - auf einer rechten Seite
\cleardoublepage{}
%\pagenumbering{arabic}
\mainmatter

\chapter{Ziele}
% \renewcommand{\kapitelautor}{}  % bleibt eventuell leer (gemeinsame Arbeit)

Das erste Kapitel stellt die Ziele der DA (inkl. individuelle Ziele
aller Mitarbeiter) dar.\todo{viel Text schreiben}

Mögliche Gliederung (nach \cite{leitfaden})

\begin{itemize}
\item  Einleitung
\item   Zielsetzung und Aufgabenstellung des Gesamtprojekts
\item   individuelle Zielsetzung und Aufgabenstellung mit Terminplan der einzelnen Teammitglieder
\item   Grundlagen und Methoden (Ist-Situation, Lösungsansätze, konkrete Vorgehensweise)
\item   Bearbeitung der Aufgabenstellung (technische Beschreibungen, Berechnungen)
\item   Ergebnisse (Ergebnisdarstellung, kritische Gegenüberstellung mit der Zielsetzung
 und der gewählten Vorgehensweise)
\end{itemize}

Mögliche Variante: Ziele laut Antrag, mit Querverweisen zu den einzelnen Kapiteln.

\chapter{Formatierung}

% wer hat diese Kapitel geschrieben oder leer
\renewcommand{\kapitelautor}{Autor: Hans Huber}

\input{text/demo_files/demo_kap1.tex}


\chapter{Code-Blöcke Test}\label{ch:code-blocke-test}

\renewcommand{\kapitelautor}{Autor: Marvin}

Das in einem Wild-West Stil gehaltene Spiel ist ein neuartiger Spin von dem klassischen Genre des "Rogue Lites".
Der Spieler erkundet eine Welt die aus verschiedenen Landschaften wie einem verzauberten Wald oder einer trockenen Wüste,
sammelt Karten in Form von Bullets, baut Decks und bekämpft Gegner in spannenden Kämpfen.
Die Karten/Bullets können in den Revolver geladen und abgefeuert werden.
Je nachdem, wie die Karten im Revolver platziert werden, können die einzigartigen Fähigkeiten jeder Karte sowie die
Drehung der Revolvertrommel dazu verwendet werden, immer stärkere Combos zu spielen und dem Gegner Effeckte und Schaden zuzufügen.
Die verschiedenen computergesteuerten Gegnertypen verfügen jedoch über ihre eigenen Tricks und setzen den Spieler mit Attacken unter Druck.

\begin{minted}{kotlin}
    /**
     * creates a new instance of the card named [name] and puts it in the hand of the player
     */
    @AllThreadsAllowed
    fun tryToPutCardsInHandTimeline(name: String, amount: Int = 1): Timeline = Timeline.timeline {
        var cardsToDraw = 0
        action {
            val maxCards = hardMaxCards - cardHand.cards.size
            cardsToDraw = min(maxCards, amount)
        }
        action {
            if (cardsToDraw == 0) return@action
            val cardProto = cardPrototypes
                .firstOrNull { it.name == name }
                ?: throw RuntimeException("unknown card: $name")
            repeat(cardsToDraw) {
                cardHand.addCard(cardProto.create())
            }
            FortyFiveLogger.debug(logTag, "card $name entered hand")
            checkCardMaximums()
        }
    }
\end{minted}

\begin{minted}{onj}
import "imports/colors.onj" as color;
import "screens/input_maps.onj" as inputMaps;
import "screens/shared_components.onj" as sharedComponents;

use Common;
use Screen;
use Style;

var worldWidth = 1600.0;
var worldHeight = 900.0;

options: {
    transitionAwayTime: 1.0,
    background: "hover_detail_background",
    inputMap: [
        ...(inputMaps.defaultInputMap),
        ...(inputMaps.healOrMaxInputMap),
        ...(inputMaps.addMaxHPInputMap),
    ],
    screenController: $HealOrMaxHPScreenController {
        addLifeActorName: "add_lives_option",
    }
},
styles: [
    {
        width: 40#percent,
        height: 83#percent,
        flexDirection: flexDirection.column,
        alignItems: align.center,
        justifyContent: justify.center,
        touchable: touchable.enabled,
        background: "heal_or_max_selector_background",
        marginRight: 0#percent,
        positionRight: 0#percent,
    },
    {
        style_priority: 2,
        background: "heal_or_max_selector_background_selected",
        width: 70#percent,
        height: 125#percent,
        style_condition: actorState("selected"),
        marginRight: -30#percent,
        positionRight: 22#percent,
    },
],
root $Box {
} chilren [

]
\end{minted}

\Blindtext

%%%%%%%%%%%%%%%%%%%%%%%%%%%%%%%%%%%%%%%%%%%%%%%%%%%%%%%%%%%%%%%%%%%%%%%%%%%%%%%%%%%%%%%%%%
% wer hat diese Kapitel geschrieben oder leer
\renewcommand{\kapitelautor}{}


\chapter{Planung}

% Das komplette nächste Kapitel wird in der externen Datei diplomarbeit2.tex gespeichert.
% Es wird an dieser Stelle im Dokument eingebaut.
% Damit ist es möglich, mehrere Personen an diversen Teilen der Diplomarbeit arbeiten zu lassen.

%\input{markdown/diplomarbeit2.md.tex}

%\input{markdown/einstellungen.md.tex}

\input{text/demo_files/demo_fuelltext.tex}


\appendix

\chapter{Anhang 1\label{chap:Anhang-1}}

was auch immer: technische Dokumentationen etc.

Zusätzlich sollte es geben:
\begin{itemize}
\item Abkürzungsverzeichnis
\item Quellenverzeichnis (hier: Bibtex im Stil plaindin)
\item optional: Akronyme und Glossar
\end{itemize}

%% optional: Akronyme und Glossar
% kann man löschen falls kein Glossar gebraucht
\printglossary[type=\acronymtype, title=Abkürzungsverzeichnis, toctitle=Abkürzungsverzeichnis]
\printglossary[type=main, title=Glossar, toctitle=Glossar]

\printindex{}

%% Flattersatz -- damit werden die langen URLs besser umgebrochen
\raggedright %% eventuell auskommentieren
%\bibliographystyle{plaindin}%Alternative unsrtdin - Nummern im Text aufsteigend
\bibliographystyle{alphadin}
\bibliography{diplom}


\cleardoublepage
\newcommand{\Messbox}[2]{%Parameters: #1=Breite, #2=Hoehe
\setlength{\unitlength}{1.0mm}%
\begin{picture}(#1,#2)%
\linethickness{0.05mm}%
\put(0,0){\dashbox{0.2}(#1,#2)%
{\parbox{#1mm}{%
\centering\footnotesize
%{\bf MESSBOX}\\
% if \textrm fails use \rm
Breite $ = #1 {\textrm\ mm}$\\
Höhe $ = #2 {\textrm\ mm}$
}}}\end{picture}
}
\begin{center} {\Large --- Druckgröße kontrollieren! ---}
\bigskip

\Messbox{100}{50} % Angabe der Breite/Hoehe in mm
\bigskip

{\Large --- Diese Seite nach dem (Probe-)Druck entfernen! ---}
\todo{Diese Seite nach dem  (Probe-)Druck entfernen!
Nicht notwendig wenn Druck+Binden extern passiert.}
\end{center}
\end{document}
